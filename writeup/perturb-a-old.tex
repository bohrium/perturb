
%==============================================================================
%    LATEX PREAMBLE  
%==============================================================================

\documentclass{article}
\usepackage[T1]{fontenc}
\usepackage{microtype}
\usepackage{graphicx}
\usepackage{subfigure}
\usepackage{booktabs} % for professional tables
\usepackage{hyperref, xcolor}
\usepackage{amsmath, amssymb, amsthm, hanging, graphicx, txfonts, ifthen}
        
\usepackage[percent]{overpic}

\newcommand{\theHalgorithm}{\arabic{algorithm}}
\newtheorem{klem}{Key Lemma}
\newtheorem{thm}{Theorem}
\newtheorem{cor}{Corollary}
\newtheorem{prop}{Proposition}
\newtheorem{quest}{Question}

\usepackage{icml2019}
%\usepackage[accepted]{icml2019}

\usepackage{array}   % for \newcolumntype macro
\newcolumntype{L}{>{$}l<{$}}

\definecolor{moor}{rgb}{0.8,0.2,0.2}
\definecolor{moog}{rgb}{0.2,0.8,0.2}
\definecolor{moob}{rgb}{0.2,0.2,0.8}

\renewcommand*{\thefootnote}{\color{red}\fnsymbol{footnote}} 

\newcommand{\Free}{\mathcal{F}}
\newcommand{\Forg}{\mathcal{G}}
\newcommand{\Mod}{\mathcal{M}}
\newcommand{\Hom}{\text{\textnormal{Hom}}}
\newcommand{\image}{\text{\textnormal{im}}}
\newcommand{\Ss}{\mathcal{S}}
\newcommand{\Pp}{\mathcal{P}}
\newcommand{\Nn}{\mathcal{N}}
\newcommand{\SGD}{\text{\textnormal{SGD}}}
\newcommand{\RR}{\mathbb{R}}
\newcommand{\expc}{\mathbb{E}}
\newcommand{\expct}[1]{\mathbb{E}\left[#1\right]}
\newcommand{\wrap}[1]{\left(#1\right)}
\newcommand{\wasq}[1]{\left[#1\right]}
\newcommand{\wang}[1]{\left\langle#1\right\rangle}
\newcommand{\wive}[1]{\left\llbracket#1\right\rrbracket}
\newcommand{\worm}[1]{\left\|#1\right\|}
\newcommand{\wabs}[1]{\left|#1\right|}
\newcommand{\wurl}[1]{\left\{#1\right\}}
\newcommand{\partbox}[1]{
    \text{
        \fboxsep=0.5pt
        \tiny
        \fbox{#1}
    }
}
\DeclareMathOperator*{\Avg}{\text{\sffamily A}}

\newcommand{\plotplace}[3]{
    \begin{overpic}[width=#2, height=#3]{../plots/blank.png}
        \put( 5, 85){
            \begin{tabular}{p{#2-1.0cm}}
                #1
            \end{tabular}
        }
    \end{overpic}
}
\newcommand{\plotmoo}[3]{
    \includegraphics[width=#2           ]{../#1.png}
}

\newcommand{\bdia}[1]{\begin{gathered}\includegraphics[scale=0.22]{../diagrams/#1.png}\end{gathered}}
\newcommand{\dia} [1]{\begin{gathered}\includegraphics[scale=0.18]{../diagrams/#1.png}\end{gathered}}
\newcommand{\mdia}[1]{\begin{gathered}\includegraphics[scale=0.14]{../diagrams/#1.png}\end{gathered}}
\newcommand{\sdia}[1]{\begin{gathered}\includegraphics[scale=0.10]{../diagrams/#1.png}\end{gathered}}

\newcommand{\half}{\frac{1}{2}}
\newcommand{\sixth}{\frac{1}{6}}

\newcommand{\ofsix}[1]{
    {\tiny $\substack{
        \ifthenelse{\equal{#1}{0}}{\blacksquare}{\square}
        \ifthenelse{\equal{#1}{1}}{\blacksquare}{\square} \\
        \ifthenelse{\equal{#1}{2}}{\blacksquare}{\square} 
        \ifthenelse{\equal{#1}{3}}{\blacksquare}{\square} \\
        \ifthenelse{\equal{#1}{4}}{\blacksquare}{\square}
        \ifthenelse{\equal{#1}{5}}{\blacksquare}{\square}
    }$}
}


\newcommand{\lorem}[1]{
    Lorem ipsum dolor sit amet, consectetur adipiscing elit...\\
    \nopagebreak\vspace{#1cm} \ \\
    ...sunt in culpa qui officia deserunt mollit anim id est laborum.
}


\begin{document}

%==============================================================================
%       TITLE AND AUTHOR
%==============================================================================

\icmltitlerunning{Descent as Scattering}

\twocolumn[
    \icmltitle{A Space-Time Approach to Analyzing Stochastic Gradient Descent}
    
    \begin{icmlauthorlist}
        \icmlauthor{Samuel C.~Tenka}{mit}
    \end{icmlauthorlist}
    \icmlaffiliation{mit}{
        Computer Science and Artificial Intelligence Lab,
        Massachusetts Institute of Technology,
        Cambridge, Massachusetts, USA
    }
    \icmlcorrespondingauthor{Samuel C.~Tenka}{coli@mit.edu}
    
    \icmlkeywords{Machine Learning, SGD, ICML}
    
    \vskip 0.3in
]
\printAffiliationsAndNotice{}

%==============================================================================
%       ABSTRACT        
%==============================================================================

\begin{abstract}
    We adapt Feynman Diagrams to reason about the behavior, at small learning
    rates, of SGD and its variants.  We interpret the diagrams as histories of
    scattering events, thus offering a new physical picture of descent.

    Illustrating this technique, we
        construct a regularizing term that causes large-batch GD to emulate
        small-batch SGD,
        %
        exhibit a counter-intuitive loss landscape on which SGD eternally moves
        uphill despite well-behaved boundary conditions and zero curvature in
        the direction of motion,
        %
        generalize the Akaike Information Criterion to a smooth,
        optimization-aware quantity liable to descent, and
        %
        quantify the differences --- due to time-discretization, inter-epoch
        correlations, and non-gaussian noise --- between SGD and the popular
        approximation SDE. 
\end{abstract}

%==============================================================================
%       INTRODUCTION    
%==============================================================================

\section{Introduction}

    %--------------------------------------------------------------------------
    %           Orienting Invitation 
    %--------------------------------------------------------------------------

    Stochastic gradient descent (SGD) decreases an unknown objective $l$ by
    performing discrete-time ``steepest''\footnote{
        The concept of ``steepness'' depends on the choice of metric for the
        space on which the real function $l$ is defined.  Recognizing this
        non-naturality, we consider all (flat) metrics at once.  Specifically,
        we will Taylor expand an (inverse) metric $\eta^{\mu\nu}$ around $0$.
    } descent on noisy estimates of $l$.  A key question is how such noise
    affects the final objective value.  We connect SGD dynamics to physical
    scattering theory, thus providing a quantitative and qualitative toolkit
    for answering this question.

    %--------------------------------------------------------------------------
    %           Killer Applications  
    %--------------------------------------------------------------------------

    Specifically, we derive a diagram-based formalism for reasoning about SGD
    via a path integral over possible interactions between weights and data.
    The formalism permits perturbative analysis, leading to predictions of
    learning curves for small $\eta$.  Unlike the continuous-time limits of
    previous work, this framework models discrete time, and with it, the
    potential {\bf non-Gaussianity} of noise.  We thus obtain new results
    quantifying the {\bf effect of epoch number, batch size, and momentum} on
    SGD test loss.  We also contrast SGD against popular continuous-time
    approximations such as ordinary or stochastic differential equations (ODE,
    SDE): our work gives the finite-$N$, finite-$\eta^{-1}$ corrections to 
    these approximations.
    
    %--------------------------------------------------------------------------
    %           Connection to Physics
    %--------------------------------------------------------------------------

    Path integrals offer not only quantitative predictions but also an exciting
    new viewpoint --- that of iterative optimization as a {\bf scattering
    process}.  Much as individual Feynman diagrams (see \citet{dy49a}) depict
    how local particle interactions compose into global outcomes, our diagrams
    depict how individual SGD updates influence each other before affecting a
    final test loss.  In fact, we import from physics tools such as {\bf
    crossing symmetries} (see \citet{dy49b}) and {\bf re-normalization} (see
    \citet{ge54}) to simplify our calculations and refine our estimates.  The
    diagrams' combinatorial properties yield precise qualitative conclusions as
    well, for instance that to order $\eta^3$, {\bf inter-epoch} shuffling does
    not affect expected test loss.

%==============================================================================
%       BACKGROUND AND NOTATION
%==============================================================================

\section{Background and Notation}

    %--------------------------------------------------------------------------
    %           Tensor Conventions
    %--------------------------------------------------------------------------

    \subsection{A Smooth Stage: Tensor Conventions}
        We adopt summation notation for Greek, suppressing indices when clear.
        To expedite dimensional analysis, we regard the learning rate as an
        inverse metric $\eta^{\mu\nu}$ that converts a gradient into a
        displacement (\cite{bo13}).  We use $\eta$ to raise indices; for
        example, with $C$ denoting the covariance of gradients, its ``trace''
        will be $C^{\mu}_{\mu} = \eta^{\mu\nu} C_{\mu\nu}$.  Standard syntactic
        constraints make manifest which expressions transform naturally with
        respect to optimization dynamics.
        
        We assume that all polynomials of the $0$th and higher derivatives of
        the losses $l_n$, considered as random functions on weight space, have
        infinitely differentiable expectations.
        
        Kol\'{a}\u{r} gives a careful introduction to these differential
        geometric ideas \yrcite{ko93}.

    %--------------------------------------------------------------------------
    %           Combinatorial Structures
    %--------------------------------------------------------------------------
    
    \subsection{Combinatorial Costumes: Structure Sets}
        We make use of \emph{structure sets}, i.e. sets $S$ equipped with a
        preorder $\leq$ and an equivalence relation $\sim$.  The morphisms of
        structure sets are strictly increasing maps that preserve $\sim$ and
        its negation.  A structure set is \emph{pointed} if it has a unique
        maximum element and this element forms a singleton $\sim$-class.  The
        categories $\Ss$ of structure sets and $\Pp$ of pointed structure sets
        enjoy a free-forgetful adjunction $\Free, \Forg$.
    
        A \emph{diagram} is a rooted tree equipped with an equivalence relation
        $\sim$ on nodes.  We draw the tree with thin edges, with the root at
        the far right, and we indicate $\sim$ with fuzzy ties.  By reading the
        tree as a Hasse graph, we see that each diagram $D$ induces a structure
        set, by abuse of notation also named $D$.  An $\Ss$-map from $D$ to
        $[P]=(\Forg\circ\Free)^P(\text{empty set})$ is an \emph{ordering} of
        $D$, where $P$ counts $D$'s equivalence classes.  Let $e(D)$ and $o(D)$
        count edges and orderings of $D$.
    
        Fong gives a swift introduction to these category theoretic and
        diagrammatic ideas \yrcite{fo19}.
            
    %--------------------------------------------------------------------------
    %           Forms of SGD          
    %--------------------------------------------------------------------------

    \subsection{The Parameterized \emph{Personae}: Forms of SGD}
        SGD decreases an objective $l$ by updating on smooth, unbiased i.i.d.
        estimates $(l_n: 0\leq n<N)$ of $l$.  The pattern of updates is
        determined by a structure set $S$ whose preorder is a total preorder
        with element $i$ inside strongly connected component $C(i)$: for a map
        $\pi:S\to [N]$ that induces $\sim$, we define SGD inductively as
        $\text{SGD}_{S}(\theta) = \theta$ when $S$ is empty and otherwise
        $$
            \SGD_S(\theta) =
                \SGD_{S\setminus M}(\theta^\mu - \eta^{\mu\nu} \nabla_\nu l_{M}(\theta))
        $$
        where $M = \min S \subseteq S$ specifies a batch and $l_M = \sum_{m\in
        M} l_{\pi(m)} / \wabs{M}$ is a batch average.  Since the distribution
        of $l_n$ is permutation invariant, the non-canonical choice of $\pi$
        does not affect the distribution of output $\theta$s.
    
        Of special interest are structure sets that divide sequentially into
        $M\times B$ many \emph{epochs} each with $N/B$ many disjoint
        \emph{batches} of size $B$.  An SGD instance is then determined by $N,
        B, M$, and an \emph{inter-epoch shuffling scheme}.  The cases $B=1$ and
        $B=N$ we call \emph{pure SGD} and \emph{pure GD}.  The $M=1$ case of
        pure SGD we call \emph{vanilla SGD}.

%==============================================================================
%    DIAGRAM CALCULUS FOR SGD
%==============================================================================

\section{Diagram Calculus for SGD}

    %--------------------------------------------------------------------------
    %           Role of Diagrams      
    %--------------------------------------------------------------------------

    \subsection{The Role of Diagrams}
        Suppose $s$ is smooth on weight space; e.g. $s$ may be a test loss.  We
        track $s(\theta)$ as $\theta$ evolves by SGD:
        \begin{klem}
            The Maclaurin series of $s(\theta_T)$ with respect to $\eta$ is:
            \begin{equation}\label{eq:dyson}
                \sum_{(d_t: 0\leq t<T)}
                (-\eta)^{\sum_t d_t}
                \left(
                    \prod_{0 \leq t < T}
                        \left.  \frac{(g \nabla)^{d_t}}{d_t!} \right|_{g=\nabla l_t(\theta)}
                \right)
                (s) (\theta_0)
            \end{equation}
        \end{klem}
        In averaging over training sets $(l_t: 0\leq t<T)$ we may factor the
        expectation of the above product according to independence relations
        between the $l_t$.  We view various training procedures (e.g. pure GD,
        pure SGD) as {\bf prescribing different independence relations} that
        lead to different factorizations and hence to potentially different
        generalization behavior at each order of $\eta$.
    
        An instance of the above product (for $s=l_a$ drawn from a test set and
        $0\leq c\leq b<T$) is $-\eta^3 (\nabla l_c \nabla)^2 (\nabla l_b
        \nabla) l_a$, which is
        {\small
        \begin{align*}
            - (\nabla^\lambda l_c) (\nabla^\mu l_c) (\nabla_\lambda \nabla_\mu \nabla^\nu l_b) (\nabla_\nu l_a)   
            - (\nabla^\lambda l_c) (\nabla^\mu l_c) (\nabla_\lambda \nabla^\nu l_b) (\nabla_\mu \nabla_\nu l_a) \\
            - (\nabla^\lambda l_c) (\nabla^\mu l_c) (\nabla_\mu \nabla^\nu l_b) (\nabla_\lambda \nabla_\nu l_a)   
            - (\nabla^\lambda l_c) (\nabla^\mu l_c) (\nabla^\nu l_b) (\nabla_\lambda \nabla_\mu \nabla_\nu l_a)
        \end{align*}
        }
        To reduce clutter, we adapt the string notation of \citet{pe71}.  Then,
        in expectation over $(l_c, l_b, l_a)$ drawn i.i.d.:
        \begin{align}
            \cdots
            &= 
                 \sdia{(01-2-3)(02-12-23)}
                +\sdia{(01-2-3)(02-13-23)}
                +\sdia{(01-2-3)(03-12-23)}
                +\sdia{(01-2-3)(03-13-23)} \\
                \label{eq:simpl}
            &=
                \underbrace{2\sdia{(01-2-3)(02-12-23)}}_{
                   -2~\expct{{\color{moor}(\nabla l)(\nabla l)}}~\expct{{\color{moog}\nabla\nabla\nabla l}}~\expct{{\color{moob} \nabla l}}
                }
                +
                \underbrace{2\sdia{(01-2-3)(02-13-23)}}_{
                   -2~\expct{{\color{moor}(\nabla l)(\nabla l)}}~\expct{{\color{moog}\nabla \nabla l}}~\expct{{\color{moob}\nabla \nabla l}}
                }
        \end{align}
        Above, each node corresponds to an $l_n$ (here, red for $l_c$, green
        for $l_b$, blue for $l_a$), differentiated $g$ times for a degree-$g$
        node (for instance, $l_b$ is differentiated thrice in the first diagram
        and twice in the second).  Thin \emph{edges} mark contractions by
        $-\eta$.  Fuzzy \emph{ties} denote correlations by connecting identical
        loss functions (here, $l_c$ with $l_c$).  The colors are redundant with
        the fuzzy ties.  The value $v(D)$ of a diagram $D$ is the expected
        value of the corresponding tensor expression.
        
        Crucially, for a fixed, i.i.d.  distribution over $(l_c, l_b, l_a)$,
        {\bf the topology of a diagram determines its value}.  For instance,
        $\sdia{(01-2-3)(02-12-23)} = \sdia{(01-2-3)(03-13-23)}$.  Thus follows
        the simplification of equation \ref{eq:simpl}.  We may convert back to
        explicit tensor expressions, invoking independence between untied nodes
        to factor the expression.  However, as we will see, the diagrams offer
        physical intuition, streamline computations, and determine useful
        unbiased estimators of the statistics they represent.  
    
        We define a diagram with fuzzy outlines to be the difference between
        the fuzzy tied and untied versions : $\sdia{c(01-2)(01-12)} =
        \sdia{(01-2)(01-12)}-\sdia{(0-1-2)(01-12)}$. 
        
        The recipes for writing down test (or train) losses of SGD and its
        variants are straight-forward in the diagram notation because they
        reduce the problem of evaluating the previous dynamical expressions to
        the problem of counting graph embeddings.  The more complicated the
        direct computation, the greater the savings of using diagrams.  An
        appendix provides details and proofs for a variety of situations.  For
        now, we focus on the test loss of SGD.
    
    %--------------------------------------------------------------------------
    %           Recipe for Test Loss and Generalization Gap
    %--------------------------------------------------------------------------

    \subsection{Recipe for the Test Loss and Generalization Gap of SGD}

        Our results all follow from this theorem and its analogues.
        Throughout, the $d$-edged diagrams give the order $\eta^d$ terms.  We
        will only consider SGD variants with time-invariant batch size $B$. 
        \begin{thm} \label{thm:test}
            SGD's expected test loss has a Maclaurin series
            given as a weighted sum of diagrams:
            \begin{equation}\label{eq:sgdcoef}
                \sum_{D \in \image(\Free)}
                \left(
                    \sum_{f: D\to\Free(S)}
                    \prod_{i\in S} \frac{1}{|f^{-1}(i)|!}
                \right)
                \frac{v(D)}{B^{e(D)}}
            \end{equation}
            Here, $D$ is (an isomorphism class of) a diagram of form
            $\Free(T)$ and $f$ is a morphism in $\Pp$.
        \end{thm}
        \begin{thm}
            SGD's expected generalization gap (test loss minus train loss) is
            formula \ref{eq:sgdcoef} with $v(D)$ replaced by
            %\begin{equation}
            $
                \sum_{p \in D/\sim_D} v(D_p)/N
            $.
            %\end{equation}
            Here, $p$ ranges through equivalance classes of $D$, and $D_p$ is $D$
            with a fuzzy outline connecting $D$'s maximal node to $p$,
            e.g. $(\sdia{(0-1)(01)})_{p=\sdia{(0)()}} = \sdia{c(01)(01)}$.
        \end{thm}
    
        In the special case of $B=1, M=1$:
        \begin{prop}\label{prop:vanilla}
            The order $\eta^d$ contribution to the expected test loss of one-epoch
            SGD with singleton batches is:
            \begin{equation}\label{eq:sgdbasiccoef}
                \frac{(-1)^d}{d!} \sum_{D\in \image(\Free)} 
                o(D) {N \choose P-1} {d \choose d_0,\cdots,d_{P-1}}
                v(D)
            \end{equation}
            where $D$ ranges over $d$-edged diagrams whose equivalence classes
            have sizes $d_p: 0\leq p\leq P$, with $d_P=1$
            and, without loss, are each antichains.
        \end{prop}
        A $P$-part, $d$-edged diagram then contributes $\Theta\left((\eta N)^d
        N^{P-d-1}\right)$ to the loss.  For example, there are six diagrams to
        third order, and they have $(4+2)+(2+2+3)+(1)$ many orderings --- see
        Table \ref{tab:scatthree}.  Intuitively, $\eta N$ measures the {\bf
        physical time} of descent, and $1/N$ measures {\bf coarseness} of time
        discretization.  So we have a double-series in $(\eta N)^d N^{P-d-1}$,
        where $d$ counts thin edges and $d+1-P$ counts fuzzy ties; the $P=d+1$
        terms correspond to a discretization-agnostic (hence continuous-time,
        noiseless) ODE approximation to SGD, while $P\leq d$ gives correction
        terms modeling time-discretization and hence noise.  

    %--------------------------------------------------------------------------
    %           Theoretical Corollaries                     
    %--------------------------------------------------------------------------

    \subsection{Consequences of the Recipe}

        \begin{cor}
            For one-epoch SGD on singleton batches through fixed physical time
            $T$: the order $N^{-1}$ deviation of SGD's test loss from ODE's is
            $
                ({{T^2 N^{-1}}/{2}}) \sdia{c(01-2)(02-12)}
            $.
            The order $N^{-2}$ deviation of SGD's test loss due to non-gaussian
            noise is
            $
                -({{T^3 N^{-2}}/{6}}) (\sdia{c(012-3)(03-13-23)} - 3 \sdia{c(01-2-3)(03-13-23)})
            $.
        \end{cor}
        For finite $N$, these effects make SDE different from SGD.  SDE also
        fails to model the correlations between updates in multiepoch SGD.  On
        the other hand, in the $N=\infty$ limit for which SDE matches SGD,
        optimization and generalization become computationally intractable and
        trivial, respectively. 
    
        \begin{table}[h!]
            \centering 
            \resizebox{\columnwidth}{!}{%
            \begin{tabular}{c|c|c}
                {\LARGE $\Theta\left((\eta N)^3 N^{-0}\right)$} &
                {\LARGE $\Theta\left((\eta N)^3 N^{-1}\right)$} &
                {\LARGE $\Theta\left((\eta N)^3 N^{-2}\right)$} \\ \hline
                \begin{tabular}{c}
                    \begin{tabular}{LL}
                        \bdia{(0-1-2-3)(01-12-23)} & \bdia{(0-1-2-3)(01-13-23)}
                    \end{tabular} \\
                    \begin{tabular}{LL}
                        \bdia{(0-1-2-3)(02-13-23)} & \bdia{(0-1-2-3)(03-12-23)}
                    \end{tabular} \\ \hline
                    \begin{tabular}{LL}
                        \bdia{(0-1-2-3)(03-13-23)} & \bdia{(0-1-2-3)(02-12-23)}
                    \end{tabular}
                \end{tabular}
                &
                \begin{tabular}{c}
                    \begin{tabular}{LL}
                        \bdia{(01-2-3)(02-13-23)} & \bdia{(01-2-3)(03-12-23)}
                    \end{tabular} \\ \hline
                    \begin{tabular}{LL}
                        \bdia{(0-12-3)(01-13-23)} & \bdia{(0-12-3)(02-13-23)}
                    \end{tabular} \\ \hline
                    \begin{tabular}{LLL}
                        \bdia{(01-2-3)(03-13-23)} & \bdia{(0-12-3)(03-13-23)} & \bdia{(01-2-3)(02-12-23)} 
                    \end{tabular}
                \end{tabular}
                &
                \begin{tabular}{c}
                    \begin{tabular}{L}
                        \bdia{(012-3)(03-13-23)}
                    \end{tabular}
                \end{tabular}
            \end{tabular}
            }
            \caption{
                Degree-$3$ scattering diagrams for $B=M=1$ SGD's test loss.
                {\bf Left:} $(d, P) = (3, 3)$.  Diagrams for ODE behavior.
                {\bf Center:} $(d, P) = (3, 2)$.  $1$st order deviation of SGD away from ODE.
                {\bf Right:} $(d, P) = (3, 1)$.  $2$nd order deviation of SGD from ODE with appearance of
                non-Gaussian statistics.
            }
            \label{tab:scatthree}
        \end{table}
    
        A quick combinatorial argument shows:
        \begin{cor}
            To order $\eta^3$, inter-epoch shuffling doesn't affect SGD's
            expected test loss.
        \end{cor}
        Indeed, for any inter-epoch shuffling scheme: 
        \begin{prop}\label{prop:ordtwo}
            To order $\eta^2$, the test loss of SGD --- on $N$
            samples for $M$ epochs with batch size $B$ dividing $N$ and with any
            shuffling scheme --- has expectation
            {\small
            \begin{align*}
                                                        \mdia{(0)()}
                &+ MN                                   \mdia{(0-1)(01)}
                 + MN\wrap{MN - \frac{1}{2}}            \mdia{(0-1-2)(01-12)} \\
                &+ MN\wrap{\frac{M}{2}}                 \mdia{c(01-2)(02-12)}  
                 + MN\wrap{\frac{M-\frac{1}{B}}{2}}     \mdia{c(01-2)(01-12)}
            \end{align*}
            }
        \end{prop}
    
        \begin{cor}
            To order $\eta^2$, one-epoch SGD has 
            $
                 \wrap{\frac{M-1}{M}}\wrap{\frac{B+1}{B}}\wrap{\frac{N}{2}} \sdia{c(01-2)(01-12)}
            $
            less test loss than $M$-epoch SGD with learning rate $\eta/M$. 
        \end{cor}
    
        Given an unbiased estimator $\hat{C}$ of gradient covariance, we may get
        GD to mimic SGD:
        \begin{cor}
            The expected test loss of pure SGD is, to order $\eta^2$,
            less than that of pure GD by
            $
                  \wrap{\frac{M(N-1)}{2}} \sdia{c(01-2)(01-12)}
            $.
            Moreover, GD on a modified loss 
            $
                \tilde l_n = l_n + \wrap{\frac{N-1}{4N}} \hat{C}_\nu^\nu(\theta)
            $
            has an expected test loss that agrees with SGD's to second order.
        \end{cor}
    
        \begin{prop}
            When initialized at a test minimum, vanilla SGD's weight moves as ODE's
            weight plus some gradient $\nabla\phi$ plus, to order $\eta^3$: 
            $
                \sdia{c(01-2-3)(02-12-23)}
            $.
        \end{prop}
    
    %--------------------------------------------------------------------------
    %           Descent as Scattering                       
    %--------------------------------------------------------------------------

    \subsection{Descent as Scattering}
        In sum, SGD's test loss is a weighted sum of diagrams, each $d$-edged
        diagram contributing to order $\eta^d$.  We depict the $\Ss$-maps of
        $f:D\to S$ Theorem \ref{thm:test} as an embedding of the graph $D$ into
        the structure set or ``spacetime'' $S$.  Thus, SGD's test loss is a sum
        over all embeddings in spacetime; see Figure (\ref{fig:spacetime}).
        This loss depends on the shape of spacetime, which in encodes
        correlations between updates.  To compute a test loss, we simply count
        embeddings.  For instance, the order $\eta^2$ diagrams in
        (\ref{fig:spacetime}) all contribute the same amount to test loss, so
        we just need to find out how many such embedded diagrams there are.   
        \begin{figure}[h!] \label{fig:spacetime}
            \centering  
            \plotmoo{diagrams/spacetime}{\columnwidth}{3.0cm}  
            \caption{
                Some diagrams embedded in spacetime.  The left four diagrams
                give order $\eta^1$, $\eta^1$, $\eta^3$, and $\eta^5$
                contributions to pure GD's test loss.  The right four each
                contribute $\eta^2 \expct{\nabla^2} \expct{\nabla}^2$; their
                equivalence demonstrates crossing symmetry.  Only diagrams
                whose nodes fall within shaded diagonals contribute to pure SGD's
                test loss (with an extra factor $N$ per edge). 
            }
        \end{figure}
        \begin{figure}[h!] \label{fig:vsmulti}
            \centering  
            \plotmoo{diagrams/spacetime-b}{\columnwidth}{3.0cm}
            \caption{
                Comparison of pure GD's vs pure SGD's test loss.  We may
                normalize almost every order $\eta^2$ GD diagram to an
                equivalent SGD diagram by horizontal or vertical shifts (see
                left ten diagrams).  By contrast, $MN{N\choose 2}$ many
                $\sdia{(01-2)(01-12)}$s turn into $\sdia{(0-1-2)(01-12)}$s (see
                right two diagrams).  So pure GD's test loss exceeds pure SGD's
                test loss by $M ((N-1)/2) \sdia{c(01-2)(01-12)}$.
            }
        \end{figure}
        Likewise, as shown in Figure (\ref{fig:vsmulti}), the order $\eta^2$
        diagrams of pure GD and pure SGD are nearly in correspondence, except
        for a discrepancy that shows the two test losses differ by $M ((N-1)/2)
        \sdia{c(01-2)(01-12)}$.
        %This argument generalizes to yield proposition \ref{prop:ordtwo}.
    
    %--------------------------------------------------------------------------
    %           Toward Effective Theories                   
    %--------------------------------------------------------------------------

    \subsection{Effective Theories}
        An important idea is that of \emph{renormalization}, i.e. the
        summarization of myriad small-scale interactions into an effective
        large-scale theory.  We can use this to refine our $d$th order
        computations for fixed $d$.  In fact, for some large-$\eta T$ limits
        in a positive-hessian setting, the unrenormalized theory does not
        converge while the renormalized theory does.

        For example, consider the diagrams that are uncorrelated chains ---
        $\sdia{(0-1)(01)}, \sdia{(0-1-2)(01-12)},
        \sdia{(0-1-2-3)(01-12-23)}, \cdots$.  When embedded with initial and
        final nodes separated by duration $t$, this series of diagrams in sum
        contributes
        $
            G (I-\eta H)^{t-1} \eta G
            \approx
            G \exp(-\eta T H) \eta G
            +
            o(\eta)
        $.
        We may thus organize diagrams together by the homeomorphism classes of
        their \emph{geometric realizations}; each class yields a sum over
        durations.  For example, the above chains contribute the following to
        test loss for vanilla SGD:
        \begin{align*}
            G \sum_{0\leq t<T} (I+\eta H)^{T-t-1} \eta G
            &\approx
            G \wrap{\int_t \exp(-\eta (T-t) H)} \eta G\\
            &=
            G \wrap{\frac{I - \exp(-\eta T H)}{H}} G
        \end{align*}
        This 
        $
            \wrap{I - \exp(-\eta T H)}/H
        $
        is thus an ``effective propagator''.

        In general, to use the renormalized recipe, one lists all the diagrams
        that are \emph{irreducible} in the sense that they contain no singleton
        degree-two nodes (such as the middle node of $\sdia{(0-1-2)(01-12)}$).
        The contribution of a diagram, as usual, will be a sum over all its
        embeddings into spacetime.  This time, though, the integrand will be
        exponentially weighted by $\exp(-\eta \Delta T H)$ for each edge of
        duration $\Delta T$.  One thus gets an integral of exponentials.

        The renormalized theory predicts large-$\eta T$ phenomena:
        \begin{prop}[nonconservative entropic force]\label{prop:entropic}
            When initialized at a test minimum, vanilla SGD's weight moves as
            ODE's weight plus some gradient $\nabla\phi$ plus, to order
            $\eta^2$, at a rate of 
            $$
                C
                \frac{I\otimes I - \exp(-\eta T (H\otimes I + I\otimes H))}{H\otimes I + I\otimes H}
                \eta^2 J
                \frac{I - \exp(-\eta T H)}{H}
                \eta
            $$
            per timestep.  For large   
        \end{prop}

        \begin{prop}[flat and sharp minima overfit less]\label{prop:overfit}
            When initialized at a test minimum, pure GD's test loss is, to
            order $\eta$, 
            $$
                \frac{1}{N} C \frac{\wrap{I - \exp(-\eta T H)}^2}{H} \eta^2
            $$
        \end{prop}

%==============================================================================
%    EXPERIMENTS AND APPLICATIONS
%==============================================================================

\section{Experiments and Applications}

    %--------------------------------------------------------------------------
    %           Vanilla SGD                                 
    %--------------------------------------------------------------------------

    \subsection{Vanilla SGD}
        We here test Proposition \ref{prop:vanilla}. 
        \begin{figure}[h!]
            \centering
            \plotmoo{plots/test-vanilla-fashion}{0.48\columnwidth}{3.0cm} 
            \plotmoo{plots/gen-cifar}{0.48\columnwidth}{3.0cm}
            %\plotplace{test loss decrease near minimum}{0.98\columnwidth}{4.0cm} \\
            %\plotmoo{plots/gen-gap}{0.98\columnwidth}{4.0cm}
            %\plotplace{nongaussian example}{4.0cm}{4.0cm}
            \caption{\lorem{2}}
        \end{figure}
    
    %--------------------------------------------------------------------------
    %           Toward Effective Theories                   
    %--------------------------------------------------------------------------

    \subsection{Emulating Small Batches with Large Ones}
        We here test Proposition \ref{prop:vanilla}. 
        \begin{figure}[h!]
            \centering
            %\plotmoo{plots/batch-match}{0.48\columnwidth}{4.0cm}
            \plotmoo{plots/big-bm}{0.48\columnwidth}{4.0cm}
            \plotmoo{plots/multi-fashion-logistic-0}{0.48\columnwidth}{4.0cm}
            %\plotmoo{plots/test-loss-bm}{0.98\columnwidth}{4.0cm}
            %\\
            %\plotplace{???}{0.98\columnwidth}{4.0cm}
            %\plotmoo{plots/test-opt-lenet-covreg-long-small-2n-05}{0.98\columnwidth}{4.0cm}
            \caption{\lorem{2}}
        \end{figure}
    
    %%--------------------------------------------------------------------------
    %%           Toward Effective Theories                   
    %%--------------------------------------------------------------------------

    %\subsection{Analyzing Second Order Methods}
    %    We demonstrate how our approach extends to more sophisticated
    %    optimizers by analyzing a momentum-based and a hessian-based method. 
    %
    %    \lorem{3}
    %    momentum
    %
    %    Now consider a hessian-based update parameterized by a scalar $\lambda$: 
    %    $$
    %        \theta \leftsquigarrow
    %        \theta -
    %        (\eta^{-1} + \lambda \nabla \nabla l_t(\theta))^{-1} \nabla l_t(\theta)
    %    $$
    %
    %    invhess
    %    \begin{figure}[h!]
    %        \centering
    %        \plotplace{momentum}{4.0cm}{4.0cm} 
    %        \plotplace{invhess}{4.0cm}{4.0cm}
    %        \caption{\lorem{2}}
    %    \end{figure}
    
    %--------------------------------------------------------------------------
    %           Toward Effective Theories                   
    %--------------------------------------------------------------------------

    \subsection{Epochs and Overfitting}
        \lorem{3}
        %\begin{figure}[h!]
        %    \centering
        %    \plotmoo{plots/multi-fashion-logistic-0}{0.98\columnwidth}{4.0cm}
        %    %\plotplace{multiepoch vs gd limit}{4.0cm}{4.0cm}
        %    \caption{\lorem{2}}
        %\end{figure}
    
    %--------------------------------------------------------------------------
    %           Toward Effective Theories                   
    %--------------------------------------------------------------------------

    \subsection{Comparison to Continuous Time}
        \lorem{3}
        %Also, sgd interepoch correlations
        %\begin{figure}[h!]
        %    \centering
        %    %\plotplace{distinguishing landscape}{4.0cm}{4.0cm}
        %    \plotmoo{plots/vs-sde}{0.48\columnwidth}{4.0cm}
        %    %\plotplace{loss landscape: mean and covariance}{4.0cm}{4.0cm}
        %    \plotmoo{plots/thermo-linear-screw}{0.48\columnwidth}{4.0cm}
        %    %\caption{\lorem{2}}
        %    \caption{\lorem{2}}
        %\end{figure}
        As a simplest example where the loss landscape is not a Gaussian
        Process, consider fitting a centered normal $\Nn(0, \sigma^2)$ to some
        one-dimensional data.  We parameterize the landscape by
        $h=\log(\sigma^2)$.  The gradient at sample $x$ and weight $\sigma$ is
        then $g_x(h) = (1-x^2\exp(-h))/2$.  If $x\sim \Nn(0, 1)$ is standard
        normal, then $g_x(h)$ will be affinely related to a chi-squared, and in
        particular non-gaussian.  At $h=1$, the expected gradient vanishes, and
        the test loss only involves diagrams with no singleton leaves; to third
        order, it is
        $
            \sdia{(0)()}
            +\frac{N}{2} \sdia{c(01-2)(02-12)}
            +{N\choose 2} \sdia{c(03-1-2)(01-12-23)}
            +\frac{N}{6} \sdia{c(012-3)(03-13-23)}
        $

    %--------------------------------------------------------------------------
    %           Thermodynamic Engine                        
    %--------------------------------------------------------------------------

    \subsection{A Nonconservative Entropic Force}
        To test Proposition \ref{prop:entropic}'s predicted force, 
        we construct a counter-intuitive loss landscape wherein, for
        arbitrarily small learning rates, SGD steadily increases the weight's
        z component despite 0 test gradient in that direction.
        Our mechanism differs from that discovered by \citet{ch18}.
        Specifically, because in this landscape the force is
        $\eta$-perpendicular to the image of $\eta C$, that work predicts an
        entropic force of $0$.  This disagreement in predictions is possible
        because our analysis does not make any assumptions of equilibrium,
        conservatism, or assumptions.
        Moreover, by modeling arbitrary covariances, we refine the
        David-Mingwei result that SGD seeks minima flat with respect to
        $\eta^{-1}$ to the statement that SGD seeks minima that are flat with
        respect to $C$.

        Intuitively, the presence of the term
        $
            \sdia{c(01-2-3)(02-12-23)}
        $
        in our test loss expansion indicates that 
        {\bf SGD descends on a covariance-smoothed landscape}.
        So, even in a valley of global minima, SGD will move away from minima
        whose hessian aligns with the current covariance.  However, by the time
        it moves, the new covariance might differ from the old one, and SGD will
        be repelled by different hessians than before.  By setting the
        covariance to lag the hessian by a rotational phase, we construct
        a landscape in which this entropic force occurs forever. 
        The landscape is defined for
        three-dimensional $w\in \RR^3$ (initialized at the origin) and
        one-dimensional $x \sim \Nn(0, 1))$:
        $$
            l_x(w) = \frac{1}{2} H(z)(w, w) + x \cdot S(z)(w)  
        $$
        Here, $H(z)(w, w) = \|w\|^2 + (\cos(z) w_x + \sin(z) w_y)^2$
        and   $S(z)(w)    = \cos(z-\pi/4) w_x + \sin(z-\pi/4) w_y$.
        We see that there is a valley of global minima defined by $x=y=0$. 
        If SGD is initialized there, then to leading order in $\eta$ and for
        large $T$, the renormalized theory predicts a $z$-speed of $\eta^2/6$ 
        per timestep.  We see that this prediction, unlike the
        un-renormalized prediction, agrees with experiment.

        %\begin{figure}[h!]
        %    \centering
        %\end{figure}
        \begin{figure}[h!]
            \centering
            %\plotplace{distinguishing landscape}{4.0cm}{4.0cm}
            \plotmoo{plots/vs-sde}{0.48\columnwidth}{4.0cm}
            %\plotplace{loss landscape: mean and covariance}{4.0cm}{4.0cm}
            \plotmoo{plots/thermo-linear-screw}{0.48\columnwidth}{4.0cm}
            %\caption{\lorem{2}}
            \caption{\lorem{2}}
        \end{figure}

        Observe that the net displacement well exceeds the $z$-periodicity of
        $2\pi$. 

        By stitching copies of this example with itself, we may cause SGD to
        traverse through minimum-valleys of any sufficiently smooth shape,
        including closed loops and unbounded curves. 

    %--------------------------------------------------------------------------
    %           Toward Effective Theories                   
    %--------------------------------------------------------------------------

    \subsection{Sharp and Flat Minima Both Overfit Less}
        Prior work has varyingly found that \emph{sharp} minima overfit less
        (after all, $l^2$ regularization increases curvature) or that
        \emph{flat} minima overfit less (after all, flat minima are more
        robust to small displacements in weight space).  Our proposition
        \label{prop:overfit} reconciles these competing intuitions by showing
        how the relationship of generalization and curvature depends on the
        learning task's noise structure.  We emphasize the role of a metric in
        controlling this distinction.
        
        \begin{figure}[h!]
            \centering
            \plotmoo{plots/tak}{0.48\columnwidth}{4.0cm}
            \plotmoo{plots/tak-reg}{0.48\columnwidth}{4.0cm}
            %\caption{\lorem{2}}
        \end{figure}

        In proving Proposition \ref{prop:overfit}, we offer a modern
        derivation of the Takeuchi Information Criterion (TIC), a
        generalization of the Akaike Information Criterion (AIC) that to our
        knowledge has not been derived in the English language literature.
        Because the TIC estimates a smooth hypothesis class's generalization
        gap, it is tempting to use it as an additive regularization term.
        However, the TIC is singular where the hessian is, and as such gives
        insensible results for over-parameterized models.  We explain these
        singularities and explain how the implicit regularization of gradient
        descent both demands and enables a singularity-removing correction to
        the TIC.  The resulting \emph{Stabilized TIC} (STIC) uses the metric
        implicit in gradient descent to threshold flat from sharp minima.  It
        thus offers a principled method for optimizer-aware model selection
        easily compatible with automatic differentiation systems.  By
        descending on STIC, we may tune smooth hyperparameters such as $l_2$
        coefficients.


%==============================================================================
%    RELATED WORK    
%==============================================================================

\section{Related Work}

    %--------------------------------------------------------------------------
    %           History of SGD
    %--------------------------------------------------------------------------

    It was \citet{ki52} who, in uniting gradient descent \citep{ca47} with
    stochastic approximation \citep{ro51}, invented SGD.  Since the development
    of back-propagation for efficient differentiation \citep{we74}, SGD
    has been used to train connectionist models including neural
    networks \citep{bo91}, in recent years to remarkable success \citep{le15}.

    %--------------------------------------------------------------------------
    %           Analyzing Overfitting; Relevance of Optimization; SDE Errs  
    %--------------------------------------------------------------------------

    Several lines of work quantify the overfitting of SGD-trained networks
    \citep{ne17a}.  For instance, \citet{ba17} controls the Rademacher
    complexity of deep hypothesis classes, leading to generalization bounds
    that are optimizer-agnostic.  However, since networks trained via SGD
    generalize despite their seeming ability to shatter large sets
    \citep{zh17}, one infers that generalization arises from the aptness to
    data of not only architecture but also optimization \citep{ne17b}.  Others
    have focused on the implicit regularization of SGD itself, for instance by
    modeling descent via stochastic differential equations (SDEs) (e.g.
    \citet{ch18}).  However, per \citet{ya19}, such continuous-time analyses
    cannot treat covariance correctly, and so they err when interpreting
    results about SDEs as results about SGD for finite trainsets.

    %--------------------------------------------------------------------------
    %           We Extend Dan's Approach                     
    %--------------------------------------------------------------------------

    Following \citet{li17} and \citet{ro18}, we avoid making a continuous-time
    approximation by instead Taylor-expanding around the learning rate
    $\eta=0$.  In fact, we develop a diagrammatic method for evaluating each
    Taylor term that is inspired by the field theory methods popularized by
    \citet{dy49a}.  Using this technique, we quantify the overfitting effects
    of batch size and epoch number, and based on this analysis, propose a
    regularizing term that causes large-batch GD to emulate small-batch SGD,
    thus establishing a precise version of the relationship --- between
    covariance, batch size, and generalization --- conjectured in
    \citet{ja18}.  
    
    %--------------------------------------------------------------------------
    %           Phemenology of Rademacher Correlates such as Hessians
    %--------------------------------------------------------------------------

    While we make rigorous, architecture-agnostic predictions of learning
    curves, these predictions become vacuous for large $\eta$.  In particular,
    while our work does not assume convexity of the loss landscape, before
    renormalization it also is blind to large-$\eta T$ convergence of SGD.
    Other discrete-time dynamical analyses allow large $\eta$ by treating deep
    generalization phenomenologically, whether by fitting to an
    empirically-determined correlate of Rademacher bounds \citep{li18}, by
    exhibiting generalization of local minima {\bf flat} with respect to the
    standard metric (see \citet{ho17}, \citet{ke17}, citet{wa18}), or by
    exhibiting generalization of local minima {\bf sharp} with respect to the
    standard metric (see \citet{st56}, \citet{di17}, \citet{wu18}).  Our work,
    which makes explicit the dependence of generalization on the underlying
    metric and on the form of gradient noise, reconciles those latter,
    seemingly clashing claims.
    
    %--------------------------------------------------------------------------
    %           Our Work vs Other Perturbative Approaches            
    %--------------------------------------------------------------------------

    Others have imported the perturbative methods of physics to analyze descent
    dynamics:  \citet{dy19} perturb in inverse network width, employing 't
    Hooft diagrams to compute deviations of a specific class of deep
    architectures from Gaussian processes.  Meanwhile, \cite{ch18} and
    \citet{li17} perturb in learning rate to second order by approximating
    noise between updates as gaussian and uncorrelated.  This approach does not
    generalize to higher orders, and, because correlations and heavy tails are
    essential obstacles to concentration of measure and hence of
    generalization, it does not model the generalization behavior of SGD.  By
    contrast, we use Penrose diagrams to compute test and train losses to
    arbitrary order in learning rate, quantifying the effect of non-gaussian
    and correlated noise.  Our method accounts for optimization and applies to
    any smooth architecture.  We hence extend \citet{ro18} beyond leading order
    and beyond $2$ time steps, allowing us to compare, for instance, the
    expected test losses of multi-epoch and single-epoch SGD.

%==============================================================================
%    CONCLUSION      
%==============================================================================

\section{Conclusion}

    %--------------------------------------------------------------------------
    %           Summarize Contributions                     
    %--------------------------------------------------------------------------

    We developed a novel diagrammatic method for analyzing gradient descent.
    We introduced a novel regularizing term, thus showing that {\bf large-batch
    GD can be made to emulate small-batch SGD} and completing a project
    suggested by \citet{ro18}.  This is significant because, while small batch
    sizes can lead to better generalization \citep{bo91}, modern infrastructure
    increasingly rewards large batch sizes \citep{go18}.  We showed also that
    in multi-epoch SGD, inter-epoch shuffling induces no $3$rd order effect
    on test loss.  In other words, we proved that {\bf nongaussian effects
    matter more than shuffling order} for fixed finite $N$ and small $\eta$.

    %--------------------------------------------------------------------------
    %           Ask Questions                               
    %--------------------------------------------------------------------------

    The diagram method also enriches our intuitions with physical analogies.
    For example, it offers a clearer understanding of the empirically verified
    limit cycles found in Chaudhari.  As our physical analogy emphasizes the
    underlying metric, it reconciles competing views of whether sharp or flat
    minima generalize.  This bridge to physics is but a beginning.  With the
    diagram method established, it is especially interesting to further develop
    toward Lagrangian methods and curved backgrounds: 
    \begin{quest}
        Do Lagrangians govern effective theories of SGD and its variants, or
        is there a fundamental obstacle to such a representation?
    \end{quest}
        More than prior connections to Hamiltonian formalisms \cite{ch14}, a
        full Langrangian formalism would be valuable for organizing the
        potential diversity of interactions in SGD variants and for extracting
        hidden conservation and scaling laws.  Given a symplectic structure, it
        is routine to translate between Hamiltonians and Lagrangians.  However,
        SGD lacks such structure.  Much physics stems from adding novel terms
        to a well understood base Lagrangian, and it would be pleasing to have
        an analogous workflow for optimization dynamics.  Because diagrams in
        physics have historically arisen from path integrals over exponentiated
        Lagrangians, our current work may serve as a conceptual bridge toward
        that question.  For example, we have found that some {\bf higher-order}
        methods --- such as the hessian-based update
        $
            \theta \leftsquigarrow
            \theta -
            (\eta^{-1} + \lambda \nabla \nabla l_t(\theta))^{-1} \nabla l_t(\theta)
        $
        parameterized by small $\eta, \lambda$ --- are amenable to diagrammatic
        analysis.  Intuitively, the new hessian term introduces a new type of
        data-weight interaction, and with it, a new type of diagram vertex.
        Though diagrams suffice for computations, it is Lagrangians that offer
        the deepest non-perturbative insight.
    \begin{quest}
        How does curvature of weight space affect the generalization behavior
        of gradient descent?
    \end{quest}
        Throughout this work, we have assumed a flat metric $\eta^{\mu\nu}$.
        However, the general case of curved spaces have concrete applications
        in the \emph{\bf learning on manifolds} paradigm of \citep{bo13}, notably
        specialized to \citep{am98}'s \emph{natural gradient descent} and
        \citep{ni17}'s \emph{hyperbolic embeddings}.  The diagram formalism may
        be adapted to curved weight spaces: one may represent the affine
        connection as a node, thus giving rise to non-tensorial and hence
        gauge-dependent diagrams.  A promising line of work is thus to adapt
        our current results to the curved case, especially to extract
        qualitative results about the role of curvature in generalization.

%==============================================================================
%    ACKNOWLEDGEMENTS
%==============================================================================

\subsection{Acknowledgements}
    We thank Dan A. Roberts and Sho Yaida for introductions to their work and
    for posing several of the questions we answer here.  We feel deeply
    grateful to Sho Yaida and Josh B. Tenenbaum for their compassionate
    guidance.  We appreciate the generosity of
        Andrzej Banburski
        and
        Wenli Zhao
    in offering crucial feedback on writing.

%==============================================================================
%    REFERENCES      
%==============================================================================

%\section*{References}
    \bibliography{perturb}
    \bibliographystyle{icml2019}

%==============================================================================
%    APPENDICES      
%==============================================================================

\section*{A. Derivation of Diagram Rules}

\subsection*{Dyson Series for Iterative Optimizers}
    If a density $\rho$ governs a point $\theta$ in weight space, then after a
    sequence of updates $\theta \mapsto \theta - \eta^{\mu\nu} \nabla_\mu
    l(\theta)$ on losses $(l_t: 0\leq t < T)$, the following density (up to an
    error term whose Maclaurin series vanishes; all perturbative results will
    implicitly carry such terms) will govern the new point:
    \begin{equation}\label{eq:descexp}
        \exp\left(+\eta^{\mu\nu} \nabla_\mu l_{T-1}(\theta) \nabla_\nu\right)
        \cdots
        \exp\left(+\eta^{\mu\nu} \nabla_\mu l_0(\theta) \nabla_\nu\right)
        \rho
    \end{equation}
    We write 
    $
        \prod \exp\left(+\eta \nabla l \nabla\right) \rho
    $
    for short.
    The exponent above is a linear operator that acts on a space of
    sufficiently smooth maps; in particular, the $\nabla_\nu$ does not act on
    the $\nabla_\mu l(\theta)$ with which it pairs.  Integrating by parts, we
    write the expectation over initial values after $T$ steps of a function $s$
    of weight space (e.g. $s$ may be test loss) as:
    \begin{align}\label{eq:contraexp}
        \int_\theta \rho(\theta) \left(
            \prod_{0 \leq t \leq T} \exp\left(
                -\eta^{\mu\nu} \nabla_\mu l(\theta) \nabla_\nu
            \right) s
        \right)(\theta)
    \end{align}
    Since the exponentials above might not commute, we may not compose
    the product of exponentials into an exponential of a sum.  We instead
    compute an expansion in powers of $\eta$.  Setting the initialization
    $\rho(\theta) = \delta(\theta-\theta_0)$ to be deterministic, and labeling
    as $\theta_t$ the weight after $t$ steps, we find:
    \begin{equation}\label{eq:dyson}
        s(\theta_T) =
        \sum_{0\leq d < \infty} (-\eta)^d
        \sum_{\substack{(d_t: 0\leq t<T) \\ \sum_t d_t = d}}
        \left(
            \prod_{0 \leq t < T} \left.
                \frac{(g \nabla)^{d_t}}{d_t!}
            \right|_{g=\nabla l_t(\theta)}
        \right) s (\theta_0)
    \end{equation}

\subsection*{Correspondence between Terms and Diagram Embeddings}
\subsection*{Error Rates for the Renormalized Theory }

\section*{B. Tutorial on Diagram Rules}
    After reviewing how diagrams correspond to specific landscape statistics,
    we will work through four examples in using diagrams to analyze descent.

    \subsection*{Evaluating a Diagram}
        
    \subsection*{Integrating a Diagram over Spacetime}
        
    \subsection*{The Chladni Repulsion: When and Why does SGD outperform GD?}
        
    \subsection*{The 3rd Order Curl: Which Minima Does SGD Prefer?}
        
    \subsection*{The Roles of Covariance: the Generalization Gap of SGD}
        
    %\subsection*{An Adaptable Framework: Analyzing a Second Order Optimizer}

\section*{C. Derivations of Perturbative Results}

    \lorem{3}
    \lorem{3}
    \lorem{3}
    \lorem{3}
    \lorem{3}

\section*{D. Diagram Rules vs Direct Perturbation}
    Diagram methods from Stueckelberg to Peierls have flourished in physics
    because they enable swift computations and offer immediate intuition that
    would otherwise require laborious algebraic manipulation.  We demonstrate
    how our diagram formalism likewise streamlines analysis of descent by
    comparing direct perturbation to the new formalism on sample problems.

    \subsection*{Effect of Batch Size}
        We compare the test losses of pure SGD and pure GD.  Because pure
        SGD and pure GD differ in how samples are correlated, their test loss
        difference involves a covariance and hence occurs at order $\eta^2$.  

        \subsubsection*{Diagram Method}
        Since SGD and GD agree on noiseless landscapes, we consider only
        diagrams with fuzzy ties.  Since we are working to second order, we
        consider only two-edged diagrams.  There are only two such diagrams,
        $\sdia{(01-2)(02-12)}$ and
        $\sdia{(01-2)(01-12)}$.
        The first diagram, $\sdia{(01-2)(02-12)}$, embeds in GD's space time in
        $N^2$ as many ways as it embeds in SGD's spacetime, due to horizontal
        shifts.
        Likewise, there are $N^2$ times as many embeddings of
        $\sdia{(01-2)(02-12)}$ in distinct epochs of GD's spacetime as there
        are in distinct epochs of SGD's spacetime.
        However, each same-epoch embedding of $\sdia{(01-2)(01-12)}$ within 
        any one epoch of GD's spacetime corresponds by vertical shifts to
        an embedding of $\sdia{(0-1-2)(01-12)}$ in SGD.  There are $MN{N\choose 2}$ 
        many such embeddings in GD's spacetime, so GD's test loss exceeds SGD's
        by 
        $$
            \eta^2 \frac{MN{N\choose 2}}{N^2} (\sdia{(01-2)(01-12)} - \sdia{(0-1-2)(01-12)}) 
            =
            \eta^2 \frac{M(N-1)}{2} \sdia{c(01-2)(01-12)}
        $$
        Since $(\nabla^2 l) (\nabla l) = \nabla((\nabla l)^2)/2$, we can 
        summarize this difference as
        $$
            \eta^2 \frac{M(N-1)}{4}
            G \nabla C 
        $$

        See Figure FILL IN for a visualization.

        \subsubsection*{Direct Perturbation} 
        We compute the displacement $\theta_T-\theta_0$ to order $\eta^2$ 
        for pure SGD and separately for pure GD.  Expanding
        $
            \theta_t \in \theta_0 + \eta a(t) + \eta^2 b(t) + o(\eta^2)
        $, we find:
        \begin{align*}
            \theta_{t+1} &=     \theta_t - \eta \nabla l_{n_t} (\theta_t) \\
                         &\in       \theta_0
                                +   \eta a(t) + \eta^2 b(t)
                                -   \eta (
                                            \nabla l_{n_t}
                                        +   \eta \nabla^2 l_{n_t} a(t) 
                                    )
                                +   o(\eta^2) \\
                         &=     \theta_0
                            +   \eta (a(t) - \nabla l_{n_t})
                            +   \eta^2 (b(t) - \nabla^2 l_{n_t} a(t)) 
                            +   o(\eta^2)
        \end{align*}
        To save space, we write $l_{n_t}$ for $l_{n_t}(\theta_0)$.
        It's enough to solve the recurrence $a(t+1) = a(t) - \nabla l_{n_t}$
        and $b(t+1) = b(t) - \nabla^2 l_{n_t} a(t)$.  Since $a(0), b(0)$
        vanish, we have $a(t) =-\sum_{0\leq t<T} \nabla l_{n_t}$ and $b(t) =
        \sum_{0\leq t_0 < t_1 < T} \nabla^2 l_{n_{t_1}} \nabla l_{n_{t_0}}$.
        We now expand $l$:
        \begin{align*}
            l(\theta_T) \in    l   &+   (\nabla l) (\eta a(T) + \eta^2 b(T)) \\
                                   &+   \frac{1}{2} (\nabla^2 l) (\eta a(T) + \eta^2 b(T))^2
                                    +   o(\eta^2) \\
                        =      l   &+   \eta ((\nabla l) a(T))
                                    +   \eta^2 ((\nabla l) b(T) + \frac{1}{2} (\nabla^2 l) a(T)^2 )
                                    +   o(\eta^2)
        \end{align*}
        Then $\expct{a(T)} = -MN(\nabla l)$ and, since the $N$ many
        singleton batches in each of $M$ many epochs are pairwise independent,
        \begin{align*}
            \expct{(a(T))^2}
            ~&=
            \sum_{0\leq t<T} \sum_{0\leq s<T} \nabla l_{n_t} \nabla l_{n_s} \\
            ~&= 
            M^2N(N-1)   \expct{\nabla l}^2 +
            M^2N        \expct{(\nabla l)^2}
        \end{align*}
        Likewise, 
        \begin{align*}
            \expct{b(T)}
            = 
            ~&\sum_{0\leq t_0 < t_1 < T} \nabla^2 l_{n_{t_1}} \nabla l_{n_{t_0}} \\
            =
            ~&\frac{M^2N(N-1)}{2} \expct{\nabla^2 l} \expct{\nabla l} + \\
            ~&\frac{M(M-1)N}{2}  \expct{(\nabla^2 l) (\nabla l)} 
        \end{align*}

        Similarly, for pure GD, we may demand that $a, b$ obey recurrence
        relations $a(t+1) = a(t) - \sum_n \nabla l_n/N$ and
        $b(t+1) = b(t) - \sum_n \nabla^2 l_n a(t)/N$, meaning that
        $a(t) = -t \sum_n \nabla l_n/N$ and
        $b(t) = {t \choose 2} \sum_{n_0} \sum_{n_1} \nabla^2 l_{n_0} \nabla l_{n_1}/N^2$.
        So $\expct{a(T)} = -MN(\nabla l)$ and
        \begin{align*}
            \expct{(a(T))^2}
            ~&=
            M^2 
            \sum_{n_0} \sum_{n_1} \nabla l_{n_0} \nabla l_{n_1} \\
            ~&= 
            M^2 N(N-1)  \expct{\nabla l}^2 + 
            M^2 N       \expct{(\nabla l)^2}
        \end{align*}
        and
        \begin{align*}
            \expct{b(T)}
            = 
            ~&{MN \choose 2}\frac{1}{N^2}
            \sum_{n_0} \sum_{n_1} \nabla^2 l_{n_0} \nabla l_{n_1} \\
            =
            ~&\frac{M(MN-1)(N-1)}{2} \expct{\nabla^2 l} \expct{\nabla l} + \\
            ~&\frac{M(MN-1)}{2}      \expct{(\nabla^2 l) (\nabla l)} 
        \end{align*}
        We see that the expectations for $a$ and $a^2$ agree between pure SGD and
        pure GD.  So only $b$ contributes.  We conclude that pure GD's test loss
        exceeds pure SGD's by
        \begin{align*}
               ~&\eta^2
                \wrap{\frac{M(MN-1)(N-1)}{2}  - \frac{M^2N(N-1)}{2}}
                \expct{\nabla^2 l} \expct{\nabla l}^2 \\
            +   ~&\eta^2 
                \wrap{\frac{M(MN-1)N}{2} - \frac{M(M-1)N}{2}}
                \expct{(\nabla^2 l) (\nabla l)} \expct{\nabla l} \\
            = 
                ~&\eta^2     \frac{M(N-1)}{2}
            \expct{\nabla l} \wrap{
                  \expct{(\nabla^2 l) (\nabla l)}
                - \expct{\nabla^2 l} \expct{\nabla l}
            }
        \end{align*}
        Since $(\nabla^2 l) (\nabla l) = \nabla((\nabla l)^2)/2$, we can 
        summarize this difference as
        $$
            \eta^2 \frac{M(N-1)}{4}
            G \nabla C 
        $$



    \subsection*{Effect of Nongaussian Noise at a Minimum}
        We consider pure SGD initialized at a local minimum of the test loss.
        One expects $\theta$ to diffuse around that minimum according to
        gradient noise.  We compute the effect on test loss of nongaussian
        diffusion.  Specifically, we compare SGD test loss on the loss
        landscape to SGD test loss on a different loss landscape defined as a
        Gaussian process whose every covariance agrees with the original
        landscape's.  We work to order $\eta^3$ because at lower orders,
        gaussian and nongaussian landscapes will by definition match. 

        \subsubsection*{Diagram Method}
            Because $\expct{\nabla l}$ vanishes at the initialization, all
            diagrams with a degree-one vertex that is a singleton vanish.
            Because we work at order $\eta^3$, we consider $3$-edged diagrams.
            Finally, because all expectatations and covariances of gradients
            match the two landscapes, we consider only diagrams with at least
            one partition of size at least $3$.  The only such test diagram is
            $\sdia{(012-3)(03-13-23)}$.

            \subsubsection*{Direct Perturbation}
            We compute the displacement $\theta_T-\theta_0$ to order $\eta^2$ 
            for pure SGD.  Expanding
            $
                \theta_t \in \theta_0 + \eta a_t + \eta^2 b_t + \eta^3 c_t 
                + o(\eta^3)
            $, we find:
            \begin{align*}
                \theta_{t+1}
                =
                \theta_t    &-  \eta \nabla l_{n_t} (\theta_t) \\
                \in\theta_0 &+  \eta a_t + \eta^2 b_t + \eta^3 c_t \\
                            &-  \eta \wrap{
                                     \nabla l_{n_t}
                                    +\nabla^2 l_{n_t} (\eta a_t + \eta^2 b_t)
                                    +\frac{1}{2} \nabla^3 l_{n_t} (\eta a_t)^2
                                }
                             +  o(\eta^3) \\
                =
                \theta_0    &+   \eta   \wrap{a_t - \nabla l_{n_t}} \\
                            &+   \eta^2 \wrap{b_t - \nabla^2 l_{n_t} a_t} \\ 
                            &+   \eta^3 \wrap{
                                     c_t
                                    -\nabla^2 l_{n_t} b_t
                                    -\frac{1}{2} \nabla^3 l_{n_t} a_t^2
                                 }
                             +   o(\eta^3)
            \end{align*}
            We thus have the recurrences
            $
                a_{t+1} = a_t - \nabla l_{n_t}
            $,
            $
                b_{t+1} = b_t - \nabla^2 l_{n_t} a_t
            $, and
            $
                c_{t+1} = c_t -\nabla^2 l_{n_t} b_t 
                              -\frac{1}{2} \nabla^3 l_{n_t} a_t^2
            $
            with solutions:
            \begin{align*}
                \eta a_t = &-\eta \sum_{t} \nabla l_{n_t}
                \\ 
                \eta^2 b_t = &+\eta^2 \sum_{t_0 < t_1} \nabla^2 l_{n_{t_1}} \nabla l_{n_{t_0}}
                \\
                \eta^3 c_t^\mu =
                    &-\sum_{t_0 < t_1 < t_2} 
                        \nabla^\mu \nabla_\nu l_{n_{t_2}}
                        \nabla^\nu \nabla_\sigma l_{n_{t_1}} \nabla^\sigma l_{n_{t_0}} \\
                    &-\frac{1}{2}
                        \sum_{t_a, t_b < t}
                        \nabla^\mu \nabla^\nu \nabla^\sigma l_{n_t}
                        \nabla_\nu l_{n_{t_a}}
                        \nabla_\sigma l_{n_{t_b}}
            \end{align*}
            We use tensor indices above because the contraction pattern would
            otherwise be ambiguous.

            Since the test loss of SGD is
            \begin{align*}
                l(\theta_T)
                \in
                        l(\theta_0)
                &+     (\nabla   l)(\eta a_T + \eta^2 b_T + \eta^3 c_T)                                 \\
                &+\frac{\nabla^2 l}{2}(\eta a_T + \eta^2 b_T             )^2                            \\
                &+\frac{\nabla^3 l}{6}(\eta a_T                          )^3 
                 +o(\eta)^3                                                                             \\
                =
                    l(\theta_0)
                &+  \eta       \wrap{(\nabla l) a_T                               }                     \\
                &+  \eta^2     \wrap{(\nabla l) b_T + \frac{\nabla^2 l}{2} a_T^2  }                     \\
                &+  \eta^3     \wrap{(\nabla l) c_T + (\nabla^2 l) a_T b_T + \frac{\nabla^3}{6} a_T^3}
                 +o(\eta)^3                                                                             
            \end{align*}
            the third order terms are $c_T$, $a_T b_T$, and $a_T^3$.  So let us
            compute their expectations:
            \begin{align*}
                \expct{c_T} =
                   &-\sum_{t_0 < t_1 < t_2} 
                     \expct{\nabla^2 l_{n_{t_2}} \nabla^2 l_{n_{t_1}} \nabla l_{n_{t_0}}}
                   \\
                   &-\frac{1}{2} \sum_{t_a, t_b < t}
                     \expct{\nabla^3 l_{n_t} \nabla l_{n_{t_a}} \nabla l_{n_{t_b}}}
            \end{align*}


    \subsection*{Effect of Sample Size on Generalization Gap}
        \subsubsection*{Diagram Method}
            
        \subsubsection*{Direct Perturbation}

    \subsection*{Time Discretization}
        \subsubsection*{Diagram Method}
        \subsubsection*{Direct Perturbation}


\section*{E. Bessel Factors for Estimating Multipoint Correlators from Data}
    Given samples from a joint probability space $\prod_{0\leq d<D} X_d$, we seek
    unbiased estimates of multipoint correlators (i.e. products of expectations
    of products) such as $\wang{x_0 x_1 x_2}\wang{x_3}$.  For example, say $D=2$
    and from $2S$ samples we'd like to estimate $\wang{x_0 x_1}$.  Most simply,
    we could use $\Avg_{0\leq s<2S} x_0^{(s)} x_1^{(s)}$, where $\Avg$ denotes
    averaging.  In fact, the following also works:
    \begin{equation} \label{eq:bessel}
        S
        \wrap{\Avg_{0\leq s< S} x_0^{(s)}}
        \wrap{\Avg_{0\leq s< S} x_1^{(s)}}
        +
        (1-S)
        \wrap{\Avg_{0\leq s< S} x_0^{(s)}}
        \wrap{\Avg_{S\leq s<2S} x_1^{(s)}}
    \end{equation}
    When multiplication is expensive (e.g. when each $x_d^{(s)}$ is a tensor and
    multiplication is tensor contraction), we prefer the latter, since it uses
    $O(1)$ rahther than $O(S)$ multiplications.  This in turn allows more efficient 
    use of large-batch computations on GPUs.  We now generalize this estimator
    to higher-point correlators (and $D\cdot S$ samples).

    For uniform notation, we assume without loss that each of the $D$ factors
    appears exactly once in the multipoint expression of interest; such expressions
    then correspond to partitions on $D$ elements, which we represent as maps
    $\mu:\wasq{D}\to \wasq{D}$ with $\mu(d)\leq d$ and $\mu\circ \mu=\mu$. 
    Note that $\wabs{\mu} \coloneqq \wabs{im(\mu)}$ counts $\mu$'s parts.
    We then define the statistic
    $$
        \wurl{x}_\mu
        \coloneqq
        \prod_{0\leq d<D} \Avg_{0\leq s<S} x_d^{(\mu(d)\cdot S + s)}
    $$
    and the correlator $\wang{x}_\mu$ we define to be the expectation of 
    $\wurl{x}_\mu$ when $S=1$.  In this notation, \ref{eq:bessel} says: 
    $$
        \wang{x}_{\partbox{0}\partbox{1}}
        =
        \expct{
            S       \cdot \wurl{x}_{\partbox{0 1}} +
            (1-S)   \cdot \wurl{x}_{\partbox{0}\partbox{1}}
        }
    $$
    Here, the boxes indicate partitions of $\wasq{D}=\wasq{2}=\{0,1\}$.
    Now, for general $\mu$, we have:
    \begin{equation} \label{eq:newbessel}
        \expct{S^D \wurl{x}_\mu}
        =
        \sum_{\tau\leq \mu} \wrap{
            \prod_{0\leq d<D}
                \frac{S!}{\wrap{S-\wabs{\tau(\mu^{-1}(d))}}!}
        }
        \wang{x}_\tau
    \end{equation}
    where `$\tau \leq \mu$' ranges through partitions \emph{finer} than 
    $\mu$, i.e. maps $\tau$ through which $\mu$ factors.   
    In smaller steps, \ref{eq:newbessel} holds because
    \begin{align*}
        \expct{S^D \wurl{x}_\mu}
        &=
        \expct{
            \sum_{(0\leq s_d<S) \in \wasq{S}^D}
            \prod_{0\leq d<D}
            x_d^{\wrap{\mu(d)\cdot S + s_d}}
        }\\
        &=
        \sum_{\substack{(0\leq s_d<S) \\ \in \wasq{S}^D}}
        \expct{
            \prod_{0\leq d<D}
            x_d^{\wrap{\min \wurl{
                \tilde{d}~:~\mu(\tilde{d})\cdot S+s_{\tilde{d}} = \mu(d)\cdot S+s_d
            }}}
        }\\
        &=
        \sum_{\tau} \wabs{\wurl{\substack{
            (0\leq s_d<S)~\in~[S]^D~: \\
            \wrap{\substack{
                \mu(d)=\mu(\tilde{d}) \\
                \wedge~s_d=s_{\tilde{d}}
            }}
            \Leftrightarrow
            \tau(d)=\tau(\tilde{d})
        }}}
        \wang{x}_\tau \\
        &=
        \sum_{\tau\leq \mu} \wrap{
            \prod_{0\leq d<D}
                \frac{S!}{\wrap{S-\wabs{\tau(\mu^{-1}(d))}}!}
        }
        \wang{x}_\tau
    \end{align*}

    Solving \ref{eq:newbessel} for $\wang{x}_\mu$, we find:
    \begin{equation*}
        \text{\fbox{$
        \wang{x}_\mu
        =
        \frac{S^D}{S^{\wabs{\mu}}}
        \expct{
            \wurl{x}_\mu
        }
        -
        \sum_{\tau < \mu} \wrap{
            \prod_{d\in im(\mu)}
            \frac{\wrap{S-1}!}{\wrap{S-\wabs{\tau(\mu^{-1}(d))}}!}
        }
        \wang{x}_\tau
        $}}
    \end{equation*}
    This expresses $\wang{x}_\mu$ in terms of the batch-friendly estimator
    $\wurl{x}_\mu$ as well as correlators $\wang{x}_\tau$ for $\tau$ 
    \emph{strictly} finer than $\mu$.  We may thus (use dynamic programming to)
    obtain unbiased estimators $\wang{x}_\mu$ for all partitions $\mu$. 
    Symmetries of the joint distribution and of the multilinear multiplication
    may further streamline estimation by turning a sum over $\tau$ into a
    multiplication by a combinatorial factor.  For example, with complete
    symmetry:
    $$
        \wang{x}_{\partbox{012}}
        =
        S^2
        \wurl{x}_{\partbox{012}}
        -
        \frac{(S-1)!}{(S-3)!}
        \wurl{x}_{\partbox{0}\partbox{1}\partbox{2}}
        -
        3\frac{(S-1)!}{(S-2)!}
        \wurl{x}_{\partbox{0}\partbox{12}}
    $$
    We use such expressions throughout our experiments to estimate the
    (expected) values of diagrams.
\end{document}


    %\subsection*{Generalization Gap vs Curvature and Covariance}
    %    We relate the generalization gap of vanilla SGD to the ambient curvature
    %    and gradient noise.  Prior work has alternatingly claimed that sharp
    %    minima generalize better (after all, $l^2$ regularization increases
    %    curvature) and that flat minima generalize better (after all, small
    %    displacements have smaller costs near flat minima). 
    %    We resolve this seeming conflict.

    %    \subsubsection*{Diagram Method}
    %        The generalization gap of vanilla SGD is 
    %        \begin{align*}
    %            \frac{1}{N} (
    %            \eta N \sdia{c(01)(01)} 
    %            &-
    %            \eta^2 {N \choose 2} \wrap{
    %                \sdia{c(0-12)(01-12)} + 
    %                \sdia{c(02-1)(01-12)} + 
    %                \sdia{c(0-12)(02-12)} + 
    %                \sdia{c(02-1)(02-12)}   
    %            } \\
    %            &- 
    %            \eta^2 N \frac{1}{2} \wrap{
    %                \sdia{(012)(02-12)} -
    %                \sdia{(01-2)(02-12)}
    %            })
    %            + o(\eta^2) \\
    %            =
    %            \eta \sdia{c(01)(01)} 
    %            &-
    %            \eta^2 \frac{N-1}{2} \wrap{
    %                3\sdia{c(0-12)(01-12)} + 
    %                \sdia{c(02-1)(01-12)} 
    %            } \\
    %            &- 
    %            \eta^2 \frac{1}{2} \wrap{
    %                \sdia{(012)(02-12)} -
    %                \sdia{(01-2)(02-12)}
    %            }
    %            + o(\eta^2)
    %        \end{align*}

    %    \subsubsection*{Direct Perturbation}

%    \subsection*{Vanilla SGD Test Loss to 3rd Order}
%        For single-epoch SGD with singleton batches, we sum all relevant
%        diagrams through order $3$; the coefficients $4, 2; 2, 2, 3; 1$ come
%        from counting the elements of Table \ref{tab:scatthree}, and the other
%        coefficients come from analogous tables.  This yields:
%        \begin{align*}
%                &\mathcal{L}^{\text{SGD}}_\text{test}(T, \eta) \in                   
%            \\ 
%                   \sdia{(0)()}
%                &+ \frac{1}{1!} {T \choose 1} \wrap{\sdia{(0-1)(01)}}
%            \\
%                &+ \frac{1}{1!1!} {T \choose 2} \wrap{2 \sdia{(0-1-2)(01-12)}} 
%                 + \frac{1}{2!} {T \choose 1} \wrap{\sdia{(01-2)(02-12)}}
%            \\
%                &+ \frac{1}{1!1!1!} {T \choose 3} \wrap{
%                           4    \sdia{(0-1-2-3)(01-12-23)}+
%                           2    \sdia{(0-1-2-3)(03-13-23)}
%                       }
%            \\
%                &+ \frac{1}{2!1!} {T \choose 2} \wrap{
%                           2    \sdia{(01-2-3)(03-12-23)}+
%                           2    \sdia{(0-12-3)(02-13-23)}+
%                           3    \sdia{(01-2-3)(02-12-23)}
%                       }
%            \\
%                &+ \frac{1}{3!} {T \choose 1} \wrap{
%                           1    \sdia{(012-3)(03-13-23)}
%                }
%                + o(\eta^3)\footnotemark
%        \end{align*}
%        \footnotetext{
%            We use little-$o(\eta^d)$ instead of big-$O(\eta^{d+1})$ to avoid
%            specializing to analytic loss landscapes.  Error terms depend on
%            the loss landscape and on $T$.  When gradients are uniformly
%            bounded, the $T$ dependence is at most linear.
%        }
%
%
%    \subsection*{Generalization Gap of a Hessian-Based Method}
%
%    \subsection*{Test Loss of SGD vs GD}
%        The generalization gap $\mathcal{L}^{\text{SGD}}_\text{gen} =
%        \mathcal{L}^{\text{SGD}}_\text{test} -
%        \mathcal{L}^{\text{SGD}}_\text{train}$ is suppressed by a factor $1/N$ ($N
%        \leq T$):
%        \begin{align*}
%            &N \cdot \mathcal{L}^{\text{SGD}}_\text{gen}(T, \eta) \in
%            \\
%            &+ \eta   {T \choose 1} \sdia{c(01)(01)} 
%            + \eta^2 {T \choose 2} \wrap{\sdia{c(01-2)(01-12)} + \sdia{c(02-1)(01-12)}} \\
%            &+ \frac{\eta^2}{2!} {T \choose 1} \wrap{\sdia{(012)(02-12)} - \sdia{(01-2)(02-12)}} 
%             + o(\eta^2) 
%            \\
%            =&+ \eta^{\lambda \mu} {T \choose 1} C_{\lambda \mu} 
%            + \eta^{\lambda \mu} \eta^{\nu \xi} {T \choose 2} \wrap{ \frac{1}{2} G_\lambda \nabla_\mu C_{\nu \xi} + C_{\mu \nu} H_{\xi \lambda} }
%            \\
%            &+ \frac{\eta^{\lambda \mu} \eta^{\nu \xi}}{2!} {T \choose 1} \wrap{ \wang{l_\lambda l_\nu l_{\mu \xi}} - (G_{\mu}G_{\nu} + C_{\mu\nu})H_{\xi \lambda} }
%             + o(\eta^2)
%        \end{align*}
%        The leading order term is $N \cdot \mathcal{L}^{\text{SGD}}_\text{gen}(T, \eta) \approx \eta T \wrap{\sdia{(01)(01)} - \sdia{(0-1)(01)}} = T \cdot \eta^{\lambda\mu} C_{\lambda\mu}$,
%        where $C$ is the covariance of gradients.  We thus recover a main result of \citet{ro18}.
%
%
%
%\section*{E. The $\eta$-Series' Domain of Convergence}
%    \lorem{3}
%    \lorem{3}
%    \lorem{3}
%
%\section*{F. Autonomous ODE Fitting}
%    We fit a .
%    These have the benefit 
%    In particular:
%    \begin{align}
%        y^\prime(t) =               a     &\to&   y(t) =  y(0) + a t                      \\
%        y^\prime(t) =         b y + a     &\to&   y(t) = (y(0) - (a/b)) \exp(b t) + (a/b) \\
%        y^\prime(t) = c y^2 + b y + a     &\to&                                             
%    \end{align}
%
%    \lorem{3}
%
