%   author: samtenka
%   change: 2020-06-03
%   create: 2020-05-29
%   descrp: LaTeX source for perturb project
%   to use: compile along with perturb.bib and diagram and plot directories

%==============================================================================
%=====  LATEX PREAMBLE  =======================================================
%==============================================================================

%~~~~~~~~~~~~~~~~~~~~~~~~~~~~~~~~~~~~~~~~~~~~~~~~~~~~~~~~~~~~~~~~~~~~~~~~~~~~~~
%~~~~~~~~~~~~~  Document Styling  ~~~~~~~~~~~~~~~~~~~~~~~~~~~~~~~~~~~~~~~~~~~~~

\documentclass{article}
\usepackage[utf8]{inputenc}
\usepackage[T1]{fontenc}
\usepackage{microtype}      

\usepackage{natbib}
%\usepackage{neurips_2020}
\usepackage[final]{neurips_2020}

%---------------------  mathematics  ------------------------------------------

\usepackage{amsmath, amssymb, amsthm, amsfonts}
\usepackage{mathtools, nicefrac, xstring, enumitem}

\usepackage{setspace}
%---------------------  tables  ----------------------------------------------- 

\usepackage{booktabs}
\usepackage{array}
\newcolumntype{L}{>{$}l<{$}}

%---------------------  graphics and figures  ---------------------------------

\usepackage{graphicx}
\usepackage{wrapfig, float, subfigure}
\usepackage{hanging, txfonts, ifthen}

\newcommand{\ofsix}[1]{
    {\tiny \raisebox{0.04cm}{$\substack{
        \ifthenelse{\equal{#1}{0}}{{\color{moor}\blacksquare}}{\square}
        \ifthenelse{\equal{#1}{2}}{{\color{moor}\blacksquare}}{\square}    
        \ifthenelse{\equal{#1}{4}}{{\color{moor}\blacksquare}}{\square} \\
        \ifthenelse{\equal{#1}{1}}{{\color{moor}\blacksquare}}{\square}    
        \ifthenelse{\equal{#1}{3}}{{\color{moor}\blacksquare}}{\square}
        \ifthenelse{\equal{#1}{5}}{{\color{moor}\blacksquare}}{\square}
    }$}}%
}

\newcommand{\offive}[1]{
    {\tiny
        \raisebox{-0.025cm}{\color{gray}\scalebox{2.5}{$\substack{
            \ifthenelse{\equal{#1}{0}}{{\color{moor}\blacksquare}}{\square} 
        }$}}%
        \raisebox{0.04cm}{$\substack{
            \IfSubStr{#1}{1}{{\color{moor}\blacksquare}}{\square}   
            \IfSubStr{#1}{1}{{\color{moor}\blacksquare}}{\square} \\
            \IfSubStr{#1}{2}{{\color{moor}\blacksquare}}{\square}    
            \IfSubStr{#1}{2}{{\color{moor}\blacksquare}}{\square}    
        }$}%
    }%
}

\newcommand{\ofthree}[1]{
    {\tiny \raisebox{0.04cm}{$
        \ifthenelse{\equal{#1}{0}}{{\color{moor}\blacksquare}}{\square}
        \ifthenelse{\equal{#1}{1}}{{\color{moor}\blacksquare}}{\square}    
        \ifthenelse{\equal{#1}{2}}{{\color{moor}\blacksquare}}{\square}
    $}}%
}

%---------------------  colors  -----------------------------------------------

\usepackage{xcolor, framed}
\definecolor{moolime}{rgb}{0.90,1.00,0.90}
\definecolor{moosky}{rgb}{0.90,0.90,1.00}
\definecolor{moopink}{rgb}{1.00,0.90,0.90}
\definecolor{moor}{rgb}{0.8,0.2,0.2}
\definecolor{moog}{rgb}{0.2,0.8,0.2}
\definecolor{moob}{rgb}{0.2,0.2,0.8}
\definecolor{mooteal}{rgb}{0.1,0.6,0.4}

%---------------------  intertext: footnotes and hyperlinks  ------------------ 

\usepackage[perpage]{footmisc}
\renewcommand*{\thefootnote}{
    \color{red}
    \arabic{footnote}
    %\fnsymbol{footnote}
} 

\usepackage{hyperref}

%~~~~~~~~~~~~~~~~~~~~~~~~~~~~~~~~~~~~~~~~~~~~~~~~~~~~~~~~~~~~~~~~~~~~~~~~~~~~~~
%~~~~~~~~~~~~~  Theorem Environments  ~~~~~~~~~~~~~~~~~~~~~~~~~~~~~~~~~~~~~~~~~

%---------------------  mathematical results  ---------------------------------

\theoremstyle{plain}
    \newtheorem*{klem*}{Key Lemma}
    \newtheorem{thm}{Theorem}
    \newtheorem*{thm*}{Theorem}
    \newtheorem{cor}{Corollary}
    \newtheorem{prop}{Proposition}

%---------------------  mathematical questions  -------------------------------

    \newtheorem{conj}{Conjecture}
    \newtheorem{quest}{Question}
    \newtheorem*{quest*}{Question}
    \newtheorem*{quests*}{Questions}

%---------------------  definitions, answers, remarks  ------------------------

\theoremstyle{definition}
    \newtheorem{defn}{Definition}
    \newtheorem*{answ*}{Answer}
    \newtheorem{rmk}{Remark}
    \newtheorem*{midea*}{Main Idea}
    \newtheorem*{rmk*}{Remark}
    \newtheorem{exm}{Example}

%~~~~~~~~~~~~~~~~~~~~~~~~~~~~~~~~~~~~~~~~~~~~~~~~~~~~~~~~~~~~~~~~~~~~~~~~~~~~~~
%~~~~~~~~~~~~~  Custom Math Commands  ~~~~~~~~~~~~~~~~~~~~~~~~~~~~~~~~~~~~~~~~~

%---------------------  expanding containers  ---------------------------------

\newcommand{\wrap}[1]{\left(#1\right)}
\newcommand{\wasq}[1]{\left[#1\right]}
\newcommand{\wang}[1]{\left\langle#1\right\rangle}
\newcommand{\wive}[1]{\left\llbracket#1\right\rrbracket}
\newcommand{\worm}[1]{\left\|#1\right\|}
\newcommand{\wabs}[1]{\left|#1\right|}
\newcommand{\wurl}[1]{\left\{#1\right\}}

\newcommand{\partitionbox}[1]{
    \text{
        \fboxsep=0.5pt
        \tiny
        \fbox{#1}
    }
}

%---------------------  special named objects  --------------------------------

\newcommand{\Free}{\mathcal{F}}
\newcommand{\Forg}{\mathcal{G}}
\newcommand{\Mod}{\mathcal{M}}
\newcommand{\Hom}{\text{\textnormal{Hom}}}
\newcommand{\Aut}{\text{\textnormal{Aut}}}
\newcommand{\image}{\text{\textnormal{im}}}
\newcommand{\uvalue}{\text{\textnormal{uvalue}}}
\newcommand{\rvalue}{\text{\textnormal{rvalue}}}
\newcommand{\edges}{\text{\textnormal{edges}}}
\newcommand{\ords}{\text{\textnormal{ords}}}
\newcommand{\parts}{\text{\textnormal{parts}}}
\newcommand{\SGD}{\text{\textnormal{SGD}}}
\DeclareMathOperator*{\Avg}{\text{\sffamily A}}
\newcommand{\expc}{\mathbb{E}}
\newcommand{\expct}[1]{\mathbb{E}\left[#1\right]}

%---------------------  fancy letters  ----------------------------------------

\newcommand{\Aa}{\mathcal{A}}
\newcommand{\Bb}{\mathcal{B}}
\newcommand{\Cc}{\mathcal{C}}   \newcommand{\CC}{\mathbb{C}}
\newcommand{\Dd}{\mathcal{D}}
\newcommand{\Ee}{\mathcal{E}}
\newcommand{\Ff}{\mathcal{F}}
\newcommand{\Gg}{\mathcal{G}}
\newcommand{\Hh}{\mathcal{H}}
\newcommand{\Ll}{\mathcal{L}}
\newcommand{\Mm}{\mathcal{M}}
\newcommand{\Nn}{\mathcal{N}}   \newcommand{\NN}{\mathbb{N}}
\newcommand{\Oo}{\mathcal{O}}
\newcommand{\Pp}{\mathcal{P}}
\newcommand{\Qq}{\mathcal{Q}}   \newcommand{\QQ}{\mathbb{Q}}
\newcommand{\Rr}{\mathcal{R}}   \newcommand{\RR}{\mathbb{R}}
\newcommand{\Ss}{\mathcal{S}}
\newcommand{\Tt}{\mathcal{T}}
\newcommand{\Uu}{\mathcal{U}}
\newcommand{\Vv}{\mathcal{V}}
\newcommand{\Ww}{\mathcal{W}}
\newcommand{\Xx}{\mathcal{X}}
\newcommand{\Yy}{\mathcal{Y}}
\newcommand{\Zz}{\mathcal{Z}}   \newcommand{\ZZ}{\mathbb{Z}}

%~~~~~~~~~~~~~~~~~~~~~~~~~~~~~~~~~~~~~~~~~~~~~~~~~~~~~~~~~~~~~~~~~~~~~~~~~~~~~~
%~~~~~~~~~~~~~  Pictures  ~~~~~~~~~~~~~~~~~~~~~~~~~~~~~~~~~~~~~~~~~~~~~~~~~~~~~

%---------------------  pictures with specified width or height  --------------

\newcommand{\plotmoow}[3]{\includegraphics[width=#2          ]{../#1}}
\newcommand{\plotmooh}[3]{\includegraphics[         height=#3]{../#1}}
\newcommand{\pmoo}[2]{\includegraphics[height=#1]{../plots/#2}}

%---------------------  inline diagrams of various sizes  ---------------------

\newcommand{\sizeddia}[2]{
    \begin{gathered}
        \includegraphics[scale=#2]{../diagrams/#1.png}
    \end{gathered}
}
\newcommand{\bdia}[1]{\protect \sizeddia{#1}{0.22}}
\newcommand{\dia} [1]{\protect \sizeddia{#1}{0.18}}
\newcommand{\mdia}[1]{\protect \sizeddia{#1}{0.14}}
\newcommand{\sdia}[1]{\protect \sizeddia{#1}{0.10}}

\newcommand{\mend}{\hfill $\Diamond$}

\newcommand{\Gauss}{\textsc{Gauss}}
\newcommand{\Archimedes}{\textsc{Archimedes}}
\newcommand{\MeanEstimation}{\textsc{Mean Estimation}}

%==============================================================================
%=====  FRONT MATTER  =========================================================
%==============================================================================

%~~~~~~~~~~~~~~~~~~~~~~~~~~~~~~~~~~~~~~~~~~~~~~~~~~~~~~~~~~~~~~~~~~~~~~~~~~~~~~
%~~~~~~~~~~~~~  Title and Author  ~~~~~~~~~~~~~~~~~~~~~~~~~~~~~~~~~~~~~~~~~~~~~

\title{%
    A Perturbative Analysis of Stochastic Descent
}

\newcommand{\tin}{\kern 0.12em}
\author{%
    \textbf{Samuel C.~Tenka} \\
    Computer Science and AI Lab \\
    Massachusetts Institute of Technology \\
    Cambridge, MA 02139 \\
    \texttt{c\,o\,l\,i\!{\tin\tiny@\tin}m\,i\,t.edu}
}

\begin{document}

    \maketitle

    
    %~~~~~~~~~~~~~~~~~~~~~~~~~~~~~~~~~~~~~~~~~~~~~~~~~~~~~~~~~~~~~~~~~~~~~~~~~~
    %~~~~~~~~~  Abstract  ~~~~~~~~~~~~~~~~~~~~~~~~~~~~~~~~~~~~~~~~~~~~~~~~~~~~~
    
    \begin{abstract}
        %
        %-------------  hammer and general nail  ------------------------------
        %
        We analyze stochastic gradient descent (SGD) at small learning rates.
        Unlike prior analyses based on stochastic differential equations, our
        theory models discrete time and hence non-Gaussian noise.
        %
        %-------------  applications  -----------------------------------------
        %
        We prove that gradient noise systematically pushes SGD toward flatter
        minima.  We characterize when and why flat minima overfit less than
        other minima.  We generalize the Akaike information criterion (AIC) to
        a smooth estimator of overfitting, hence enabling gradient-based model
        selection.  We show how non-stochastic GD with a modified loss function
        may emulate SGD.
        %
        %-------------  mention of experiments  -------------------------------
        %
        We verify our predictions on convnets for CIFAR-10 and Fashion-MNIST.
    \end{abstract}
    
%==============================================================================
%=====  INTRODUCTION  =========================================================
%==============================================================================

\section{Introduction}

    %~~~~~~~~~~~~~~~~~~~~~~~~~~~~~~~~~~~~~~~~~~~~~~~~~~~~~~~~~~~~~~~~~~~~~~~~~~
    %~~~~~~~~~  Orienting Invitation  ~~~~~~~~~~~~~~~~~~~~~~~~~~~~~~~~~~~~~~~~~

    %-----------------  object of study  --------------------------------------

    Practitioners benefit from the intuition that stochastic gradient descent
    (SGD) approximates noiseless gradient descent (GD) \citep{bo91}.  In this
    paper, we refine that intuition by showing how gradient noise biases
    learning toward certain areas of weight space.
    %
    %-----------------  vs ODE and SDE  ---------------------------------------
    %
    Departing from prior work, we model discrete time and hence non-Gaussian
    noise.  Indeed, we derive corrections to continuous-time, Gaussian-noise
    approximations such as ordinary and stochastic differential equations (ODE,
    SDE).
    For example, we construct a loss landscape on which SGD eternally cycles
    counterclockwise, a phenomenon impossible with ODEs. 
    %
    %-----------------  organization Plan  ------------------------------------
    %
    Leaving the rigorous development of the general theory to
    \S\ref{appendix:math}, our paper body highlights our theory's intuition and
    main corollaries.

    %~~~~~~~~~~~~~~~~~~~~~~~~~~~~~~~~~~~~~~~~~~~~~~~~~~~~~~~~~~~~~~~~~~~~~~~~~~
    %~~~~~~~~~  Soft Benefits: Physical Intuition and Further Applications  ~~~

    %-----------------  retrospective  ----------------------------------------
    %
    Our analysis offers a novel interpretation of SGD as a sum of many
    concurrent interactions between weights and data.  Diagrams such as
    $\sdia{c(01-2-3)(02-12-23)}$, analogous to those of \cite{fe49} and
    \cite{pe71}, depict these interactions. 
    %
    %-----------------  prospective  ------------------------------------------
    %
    \S\ref{appendix:future} discusses this bridge to physics --- and its
    relation to Hessian methods and natural GD --- as topics for future
    research.  We also discuss how this work may lessen the energy footprint
    required to train machine learning models.  More broadly, our work adds to
    the body of theory on optimization in the face of uncertainty,
    theory that may one day inform solutions to emerging issues in user privacy
    and pedestrian safety.

    \subsection{Example of diagram-based computation of SGD's test loss} \label{subsect:example}

        \newcommand{\nb} { \nabla }
        \newcommand{\lx} { l_x(\theta) }
        \newcommand{\teq} { \triangleq }
        \newcommand{\ex}[1] { \expc_x \wasq{#1} }

        If we run SGD for $T$ gradient steps with learning rate $\eta$ starting
        at weight $\theta_0$, then by Taylor expansion we may express the
        expected test loss of the final weight $\theta_T$ in terms of
        statistics of the loss landscape evaluated at $\theta_0$.
        As is, this Taylor series is unwieldy to write and interpret.
        Our technical contribution is to organize the computation of this
        Taylor series via combinatorial objects we call
        \emph{diagrams}:
        \begin{midea*}[Informal]
            We may enumerate the diagrams, and we may assign to
            each diagram a number that depends on $\eta, T$, such that
            summing those numbers over all diagrams yields SGD's expected test
            loss.  Restricting to the finitely many diagrams with $\leq d$
            edges leads to $o(\eta^d)$ error.
        \end{midea*}

        Deferring details, we illustrate the Main Idea by deriving a new result
        (Example \ref{exm:first}).  This shows our formalism's work flow,
        but only in later sections will we explain the mathematics.

        First, let $l_x(\theta)$ be weight
        $\theta$'s loss on datapoint $x$.  We define a dictionary between (a)
        tensors relating to this loss landscape and (b) diagram fragments that
        we will soon assemble:
        \begin{center}
            \begin{tabular}{ll}
                $G \teq \ex{\nb\lx}       \teq \mdia{MOO(0)(0)}     $ &                                                             \\
                $H \teq \ex{\nb\nb\lx}    \teq \mdia{MOO(0)(0-0)}   $ & $ C \teq \ex{(\nb\lx - G)^2} \teq \mdia{MOOc(01)(0-1)}    $ \\
                $J \teq \ex{\nb\nb\nb\lx} \teq \mdia{MOO(0)(0-0-0)} $ & $ S \teq \ex{(\nb\lx - G)^3} \teq \mdia{MOOc(012)(0-1-2)} $ 
            \end{tabular}
        \end{center}
        Here, $G, H, J$ denote the loss's derivatives with respect to 
        $\theta$, and $G, C, S$ denote the gradient's 
        cumulants with respect to the randomness in $x$.
        There are infinitely many analogues (with more edges), but they will
        not play a role in our our leading order results.  Each $\nabla^d l_x$
        corresponds to a degree-$d$ node, and fuzzy outlines group nodes that
        occur within the same expectation.  

        We obtain \emph{diagrams} by pairing together the loose ends of the
        above fragments.\footnote{
            A diagram's colors and geometric layout lack meaning: we
            {\color{moor} color} only for convenient reference, e.g.\ to
            a diagram's ``green nodes''.  Only the topology of a diagram
            --- not its size or angles --- appear in our theory.
        }
        For instance, we may join
        $
            C = \sdia{MOOc(01)(0-1)}
        $
        with
        $
            H = \sdia{MOO(0)(0-0)}
        $
        to get
        $
            \sdia{c(01-2)(02-12)}
        $.
        As another example, we may join two copies of
        $
            G = \sdia{MOO(0)(0)}
        $
        with two copies of
        $
            H = \sdia{MOO(0)(0-0)}
        $
        to get
        $
            \sdia{c(0-1-2-3)(01-12-23)} 
        $.
        Intuitively, each diagram represents the interaction of its components:
        of gradients ($G$), noise ($C, S, \cdots$) and curvature ($H, J,
        \cdots$).  In fact, \S\ref{appendix:interpret-diagrams} physically
        interprets edges as carrying information between updates and toward the
        test measurement.
        %
        \begin{exm} \label{exm:first}
            Does non-Gaussian noise affect SGD?  Specifically,
            let's compute how the \emph{skewness} $S$ affects SGD's test loss. The recipe is to identify
            the fewest-edged diagrams containing $S = \sdia{MOOc(012)(0-1-2)}$.
            In this case, there is one fewest-edged diagram ---
            $\sdia{c(012-3)(03-13-23)}$; it results from joining $S$ with
            $J=\sdia{MOO(0)(0-0-0)}$.  To evaluate a diagram, we multiply its
            components (here, $S, J$) with exponentiated $\eta H$'s, one for
            each edge (here, there are three edges).  The result is easiest
            to write in terms of an eigenbasis of $\eta H$:
            \begin{align*} %\label{eqn:nongauss}
                -\frac{\eta^3}{3!}
                \sum_{\mu\nu\lambda}
                    S_{\mu\nu\lambda}
                    \frac{
                        1 - \exp(-T\eta (H_{\mu\mu} + H_{\nu\nu} + H_{\lambda\lambda}))
                    }{
                        \eta (H_{\mu\mu} + H_{\nu\nu} + H_{\lambda\lambda})
                    }
                    J_{\mu\nu\lambda}
            \end{align*}
            This is leading order contribution of skewed noise ($S$) to SGD's
            test loss.
        \end{exm}
        \begin{rmk}
            To understand Example \ref{exm:first}'s result, we specialize
            to isotropic curvature ($\eta H = \|\eta H\|_2 I$) and take $T\to
            \infty$, obtaining:
            $
                - (\eta^3/3!)
                \sum_{\mu\nu\lambda}
                    S_{\mu\nu\lambda} J_{\mu\nu\lambda} / 3 \|\eta H\|_2
            $.
            Since $J = \nabla H$, $J / \|\eta H\|_2$ measures the relative
            change in the curvature, $H$, with respect to $\theta$.  So skewed
            noise affects SGD in proportion to the logarithmic derivative of
            curvature.  Gaussian approximations (e.g.\ SDE) miss this effect. 
        \end{rmk}

%==============================================================================
%=====  BACKGROUND AND NOTATION  ==============================================
%==============================================================================

\subsection{Background, notation, and assumptions} \label{sect:background}
       
    %~~~~~~~~~~~~~~~~~~~~~~~~~~~~~~~~~~~~~~~~~~~~~~~~~~~~~~~~~~~~~~~~~~~~~~~~~~
    %~~~~~~~~~  Tensor Conventions  ~~~~~~~~~~~~~~~~~~~~~~~~~~~~~~~~~~~~~~~~~~~

    %\subsection{Tensor conventions}
        Let $G, H, J; C, S$ be as in \S \ref{subsect:example}.  They are
        tensors with $1, 2, 3; 2, 3$ indices, respectively.
        We may implicitly sum repeated Greek indices: if a covector $A$
        and a vector $B$\footnote{
            Vectors/covectors are also called column/row vectors. 
            %Their coordinates transform differently, so they represent
            %distinct geometric types.
        } have coefficients $A_\mu, B^\mu$, then 
        $
            A_\mu B^\mu
            \triangleq
            \sum_\mu A_\mu \cdot B^\mu
        $.
        %To expedite dimensional analysis,
        We regard the learning rate as an
        inverse metric $\eta^{\mu\nu}$ that converts gradient covectors to
        displacement vectors \citep{bo13}.  We use the learning rate
        $\eta$ to raise indices; thus,
        $
            H^{\mu}_{\lambda}
            \triangleq
            \sum_{\nu} 
            \eta^{\mu\nu} H_{\nu\lambda}
        $ and
        $
            C^{\mu}_{\mu}
            \triangleq
            \sum_{\mu \nu} \eta^{\mu\nu} \cdot C_{\nu\mu}
        $.
        Though $\eta$ is a tensor, we may still define $o(\eta^d)$: a quantity
        $q$ \emph{vanishes to order $\eta^d$} when $\lim_{\eta\to 0} q/p(\eta)
        = 0$ for some homogeneous degree-$d$ polynomial $p$.

    %~~~~~~~~~~~~~~~~~~~~~~~~~~~~~~~~~~~~~~~~~~~~~~~~~~~~~~~~~~~~~~~~~~~~~~~~~~
    %~~~~~~~~~  The Loss Landscape  ~~~~~~~~~~~~~~~~~~~~~~~~~~~~~~~~~~~~~~~~~~~

    %\subsection{Loss landscape}

        %-------------  the landscape  ----------------------------------------

        We fix a loss function $l:\Mm\to\RR$ on a space $\Mm$ of weights.  We
        fix a distribution $\Dd$ from which unbiased estimates of $l$ are
        drawn.  We write $l_x$ for a generic sample from $\Dd$ and $(l_n: 0\leq
        n<N)$ for a training sequence drawn i.i.d.\ from $\Dd$.  We refer both
        to $n$ and to $l_n$ as \emph{training points}.  We assume
        \S\ref{appendix:assumptions}'s hypotheses, e.g.\ that $l, l_x$ are
        analytic and that all moments exist.
        %
        %-------------  specialization to a common case  ----------------------
        %
        For instance, our theory models $\tanh$ networks with cross entropy
        loss on bounded data --- and with weight sharing, skip connections,
        soft attention, dropout, and weight decay.  But it does not model
        $\text{ReLU}$ networks.
        
    %~~~~~~~~~~~~~~~~~~~~~~~~~~~~~~~~~~~~~~~~~~~~~~~~~~~~~~~~~~~~~~~~~~~~~~~~~~
    %~~~~~~~~~  Names of SGD Parameters  ~~~~~~~~~~~~~~~~~~~~~~~~~~~~~~~~~~~~~~

    %\subsection{SGD terminology}
        %SGD performs $\eta$-steepest descent on the estimates $l_n$.
        %
        Our general theory describes SGD with any number
             $N$ of training points,
             $T$ of updates, and 
             $B$ of points per batch.
        SGD then runs $T$ many updates (i.e. $E=TB/N$ epochs, i.e. $M=T/N$
        updates per point) of the form
        $
            \theta^\mu
            \coloneqq
            \theta^\mu -
            \eta^{\mu\nu} \nabla_\nu
                \sum_{n\in \Bb_t} l_n(\theta) / B
        $,
        where in each epoch, $\Bb_t$, the $t$th batch, is sampled without
        replacement from the training set.
        For simplicity, our paper body (but not the appendices) will assume
        unless otherwise stated that \textbf{SGD has $\mathbf{E=B=1}$ and GD
        has $\mathbf{T=B=N}$}.

%==============================================================================
%    RELATED WORK    
%==============================================================================

\subsection{Related work} \label{sect:related}

    It was \cite{ki52} who, in uniting gradient descent \citep{ca47} with
    stochastic approximation \citep{ro51}, invented SGD.  Since the development
    of back-propagation for efficient differentiation \citep{we74}, SGD has
    been used to train connectionist models, e.g.\ neural networks
    \citep{bo91}, recently to remarkable success \citep{le15}.

    %--------------------------------------------------------------------------
    %           Analyzing Overfitting; Relevance of Optimization; SDE Errs  
    %--------------------------------------------------------------------------

    Several lines of work treat the overfitting of SGD-trained networks
    \citep{ne17a}.  For example, \cite{ba17} controls the Rademacher complexity
    of deep hypothesis classes, leading to optimizer-agnostic generalization
    bounds.  Yet SGD-trained networks generalize despite their ability to
    shatter large sets \citep{zh17}, so generalization must arise from not only
    architecture but also optimization \citep{ne17b}.  Others approximate SGD
    by SDE to analyze implicit regularization (e.g.\ \cite{ch18}), but, per
    \cite{ya19a}, such continuous-time analyses cannot treat SGD noise
    correctly.
    %
    %%--------------------------------------------------------------------------
    %%           We Extend Dan's Approach                     
    %%--------------------------------------------------------------------------
    %
    We avoid these pitfalls by Taylor expanding around $\eta=0$ as in
    \cite{ro18}; unlike that work, we generalize beyond order $\eta^1$ and
    $T=2$.
    
    %--------------------------------------------------------------------------
    %           Phenomenology of Rademacher Correlates such as Hessians
    %--------------------------------------------------------------------------

    Our predictions are vacuous for large $\eta$.  Other analyses treat
    large-$\eta$ learning phenomenologically, whether by finding empirical
    correlates of gen.\ gap \citep{li18}, by showing that \emph{flat} minima
    generalize (\cite{ho17}, \cite{ke17}, \cite{wa18}), or by showing that
    \emph{sharp} minima generalize (\cite{st56}, \cite{di17}, \cite{wu18}).
    At least for small $\eta$, our theory reconciles these clashing claims.
    
    %--------------------------------------------------------------------------
    %           Our Work vs Other Perturbative Approaches            
    %--------------------------------------------------------------------------

    Prior work analyzes SGD perturbatively: \cite{dy19} perturb in inverse
    network width, using 't Hooft diagrams to correct the Gaussian Process
    approximation for specific deep nets.  Perturbing to order $\eta^2$,
    \cite{ch18} and \cite{li17} are forced to assume uncorrelated Gaussian
    noise.  By contrast, we use Penrose diagrams to compute test losses to
    arbitrary order in $\eta$.  We allow correlated, non-Gaussian
    noise and thus \emph{any} smooth architecture.  For instance, we do not
    assume information-geometric relationships between $C$ and $H$,%
    \footnote{
        Disagreement of $C$ and $H$ is typical in modern learning \citep{ro12,
        ku19}.
    }
    so we may model VAEs. 

%==============================================================================
%=====  DIAGRAM CALCULUS FOR SGD  =============================================
%==============================================================================

\section{Theory, specialized to $E=B=1$ SGD's test loss} \label{sect:calculus}

    %\subsection{Diagrams, embeddings, and re-summed values}
        \begin{wraptable}{r}{5cm}
            \begin{spacing}{0.5}
            \begin{tabular}{p{5cm}}
                \textbf{Examples}:
                The diagrams
                $\sdia{c(0-1)(01)}$, $\sdia{c(012-3)(03-13-23)}$ each have $2$
                parts; $\sdia{c(0-12-3)(03-13-23)}$, $\sdia{c(01-2-3)(02-12-23)}$
                have $3$.
                %
                Corollaries \ref{cor:overfit}, \ref{cor:epochs},
                \ref{cor:batch} have $E\neq 1 \neq B$, so they feature
                $\sdia{c(01)(01)}$ and $\sdia{c(01-2)(01-12)}$, generalized
                diagrams that violate the path condition. 
                %
                Diagrams $\sdia{c(0-1)(01)}$, $\sdia{c(0-1-2)(02-12)}$ 
                are irreducible; due to their green nodes,
                $\sdia{c(0-1-2)(01-12)}$, $\sdia{c(01-2-3)(03-12-23)}$ are not.
                %
                For all $f$,
                $|\Aut_f(\sdia{c(01-2-3)(03-12-23)})|=1$ and
                $|\Aut_f(\sdia{c(01-2-3)(02-12-23)})|=2$.
            \end{tabular}
            \end{spacing}
        \end{wraptable}

        A \emph{diagram} is a finite rooted tree equipped with a partition
        of its nodes that obeys the \emph{path condition}: no path from leaf to
        root may encounter any part more than once.
        We specify the root by drawing it rightmost.  We draw the parts of 
        the partition by grouping each part's nodes inside fuzzy outlines. 
        %
        A diagram is \emph{irreducible} when each of its degree-$2$ nodes is in
        a part of size one.
        %
        An \emph{embedding} $f$ of a diagram $D$ is an injection from
        $D$'s parts to (integer) times $0 \leq t \leq T$ that sends the
        root to $T$ and s.t., for each path from leaf to root, the
        corresponding sequence of times increases.  So $f$ might
        send $\sdia{c(01-2-3)(03-12-23)}$'s red part to $t=3$ and its green
        part to $t=4$, but --- because the green node has a red child ---
        not vice versa.
        %
        Let $\wabs{\Aut_f(D)}$ count automorphisms of $D$ that preserve $f$.
        %%%%%%%%%
        Up to unbiasing terms,\footnote{
            For example, we actually define $\sdia{MOOc(01)(0-1)}$ to be the
            cumulant $C = \expc\,[(\nb\lx - G)^2]$, not the moment
            $\expc\,[(\nb\lx)^2]$.  This centering is routine (see \S
            \ref{appendix:mobius}), tedious to notate, and un-germane, so we
            ignore it in the paper body.
        }
        we construct the \emph{re-summed value} $\rvalue_f(D)$ as follows:
        %
        \par\textbf{Node rule}: insert a factor a $\nabla^d l_x$for each degree $d$
        node. 
        %
        \par\textbf{Outline rule}: group each part's nodes within brackets $\expc_x []$.
        %
        \par\textbf{Edge rule}: if $f$ sends an edge's endpoints to times $t,
        t^\prime$, insert a factor of $K^{\wabs{t^\prime-t}-1} \eta$, where $K
        \triangleq (I-\eta H)$.
        %
        \par So if $f$ maps $\sdia{c(012-3)(03-13-23)}$'s red part to time $t =
        T-\Delta t$, then (the red part gives $S$; the green part, $J$):
        $$
            \rvalue_f\wrap{\sdia{c(012-3)(03-13-23)}} = 
            S_{\mu\lambda\rho}
                (K^{\Delta t-1}\eta)^{\mu\nu}
                (K^{\Delta t-1}\eta)^{\lambda\sigma}
                (K^{\Delta t-1}\eta)^{\rho\pi}
            J_{\nu\sigma\pi}
        $$
        In fact, we may integrate this expression per Remark
        \ref{rmk:integrate} to recover Example \ref{exm:first}.

    \subsection{Main result}

    %\subsection{Recipe for SGD's expected test loss}
        %~~~~~~~~~~~~~~~~~~~~~~~~~~~~~~~~~~~~~~~~~~~~~~~~~~~~~~~~~~~~~~~~~~~~~~
        %~~~~~  Recipe for Test Loss  ~~~~~~~~~~~~~~~~~~~~~~~~~~~~~~~~~~~~~~~~~
        Theorem \ref{thm:resum} expresses SGD's test loss as a sum over
        diagrams.  A diagram with $d$ edges scales as $O(\eta^d)$, so the
        following is a series in $\eta$.  We later truncate the series to small
        $d$, thus focusing on few-edged diagrams and simplifying the
        combinatorics of embeddings.
        \begin{thm}[Special case of $E=B=1$] \label{thm:resum}
            For any $T$: for $\eta$ small enough, SGD has expected test loss
            \begin{equation*} \label{eq:resum}
                \sum_{\substack{D~\text{an irreduc-} \\ \text{-ible diagram}}}
                ~
                \sum_{\substack{f~\text{an embed-} \\ \text{-ding of}~D}}
                ~
                \frac{(-1)^{|\edges(D)|}}{\wabs{\Aut_f(D)}}
                \,
                {\rvalue_f}(D)
            \end{equation*}
        \end{thm}

        %~~~~~~~~~~~~~~~~~~~~~~~~~~~~~~~~~~~~~~~~~~~~~~~~~~~~~~~~~~~~~~~~~~~~~~
        %~~~~~  Simplifications  ~~~~~~~~~~~~~~~~~~~~~~~~~~~~~~~~~~~~~~~~~~~~~~
 
        \begin{rmk} \label{rmk:integrate}
            In practice, we approximate sums over embeddings by integrals over
            times and $(I-\eta H)^t$ by $\exp(- \eta H t)$, reducing to a
            routine integration of exponentials at the cost of an error factor
            $1 + o(\eta)$.
        \end{rmk}

        %~~~~~~~~~~~~~~~~~~~~~~~~~~~~~~~~~~~~~~~~~~~~~~~~~~~~~~~~~~~~~~~~~~~~~~
        %~~~~~  Convergence  ~~~~~~~~~~~~~~~~~~~~~~~~~~~~~~~~~~~~~~~~~~~~~~~~~~
 
        \begin{thm} \label{thm:converge}
            If $\theta_\star$ is a local minimum of $l$ and $H(\theta_\star)$
            is strictly positive, then for SGD initialized sufficiently close
            to $\theta_\star$, the $d$th-order truncation of Theorem
            \ref{thm:resum} converges as $T\to \infty$.
        \end{thm}

        Caution: the $T\to \infty$ limit in Theorem \ref{thm:converge} might
        not measure any well-defined limit of SGD, since the limit might not
        commute with the infinite sum.  We have not seen such pathologies in
        practice, so we will freely speak of ``SGD in the large-$T$ limit'' as
        informal shorthand when referencing this Theorem.
      
    %\subsection{Insights from the formalism}
    
        \subsection{SGD descends on a $C$-smoothed landscape and prefers
        minima flat w.r.t.\ $C$.}
    
            \begin{cor}[Computed from $\sdia{c(01-2-3)(02-12-23)}$]
                \label{cor:entropic}
                Run SGD for $T \gg 1/\eta H$ from a non-degenerate test
                minimum.  Written in an eigenbasis of $\eta H$, $\theta$ has an
                expected displacement of
                $$
                    - \frac{\eta^3}{2}
                    \sum_{\mu\nu}
                        C_{\mu\nu}
                        \frac{1}{\eta (H_{\mu\mu} + H_{\nu\nu})}
                        J_{\mu\nu\lambda}
                        \frac{1}{H_{\lambda\lambda}}
                    + o(\eta^2)
                $$
            \end{cor}

            Intuitively, $D = \sdia{c(01-2-3)(02-12-23)}$ connects the
            subdiagram $\sdia{c(01-2)(02-12)} \propto CH$, via an extra edge on
            the green node (an extra $\nabla$ on $H$), to $D$'s degree-$1$
            root, $G$.  By l'H\^opital,\footnote{
                %
                Roughly:
                if a displacement $\Delta\theta$ grows loss by $G C\nabla H$
                nats, and by $G$ nats per foot, then $\Delta \theta$ is
                $C\nabla H$ foot.
                %
            } the displacement is $\propto -C\nabla H$.  That is, SGD moves
            toward minima that are flat \emph{with respect to} $C$ (Figure
            \ref{fig:cubicandspring}\offive{0}).
            %
            Taking limits to drop the non-degeneracy hypothesis, we expect
            \emph{sustained} motion toward flat regions in a valley of minima.
            By avoiding \cite{we19b}'s assumptions of constant $C$, we find
            that SGD's velocity field is typically non-conservative, i.e.\ has
            curl (\S\ref{subsect:entropic}).  Indeed, $\nabla(CH)$ is a total
            derivative but $C\nabla H$ is not.  Since, by low-pass
            filter theory, $CH/2+o(C)$ is the loss increase upon convolving $l$
            with a $C$-shaped Gaussian, we say that SGD descends on a
            $C$-smoothed landscape that changes as $C$ does.
            %
            Our $T\gg 1$ result is $\Theta(\eta^2)$, while \cite{ya19b}'s
            similar $T=2$ result is $\Theta(\eta^3)$.  Indeed, our analysis
            integrates the noise over many updates, hence amplifying $C$'s 
            effect.
            Experiments verify our law.
      
        \subsection{Both flat and sharp minima overfit less}
            \label{subsect:curvature-and-overfitting}%

            Intuitively, sharp minima are robust to slight changes in the
            average \emph{gradient} and flat minima are robust to slight
            \emph{displacements} in weight space (Figure
            \ref{fig:cubicandspring}\protect\offive{12}).  However, as SGD by
            definition equates displacements with gradients, it may be unclear
            how to reason about overfitting in the presence of curvature.
            %
            Our theory, by (automatically) accounting for the implicit
            regularization of fixed-$T$ descent, shows that both effects play
            a role.  In fact, by routine calculus on the left hand side of
            Corollary \ref{cor:overfit}, overfitting is maximized for medium
            minima with curvature $H \sim (\eta T)^{-1}$.
            %
            \begin{cor}[from $\sdia{c(01-2)(02-12)}$, $\sdia{c(01)(01)}$]\label{cor:overfit}
                Initialize GD at a non-degenerate test minimum $\theta_\star$.
                The overfitting (test loss minus $l(\theta_\star)$) and gen.\
                gap (test minus train loss) due to training are:
                $$
                    \wrap{\frac{C/N}{2H}}_{\mu\nu}^{\rho\lambda} ~
                        \wrap{(I - \exp(-\eta T H))^{\otimes 2}}^{\mu\nu}_{\rho\lambda}
                        + o(\eta^2)
                    ~~~~~ ; ~~~~~
                    \wrap{\frac{C/N}{H}}_{\mu\nu}^{\mu\lambda} ~
                        \wrap{I - \exp(-\eta T H)}^{\nu}_{\lambda}
                        + o(\eta)
                $$
            \end{cor}
            The gen.\ gap tends  
            to $C_{\mu\nu}(H^{-1})^{\mu\nu}/N$ as $T\to\infty$.  For maximum
            likelihood (ML) estimation in well-specified models near the ``true''
            minimum, $C=H$ is the Fisher metric, so we recover AIC:
            $(\textnormal{model dimension})/N$.  Unlike AIC, our more general
            expression is descendably smooth, may be used with MAP or ELBO tasks
            instead of just ML, and does not assume a well-specified model.
    
            \begin{figure}[h!]
                \centering
                \plotmooh{diagrams/entropic-force-diagram}{}{0.32\columnwidth} 
                \plotmooh{diagrams/sharp}{}{0.31\columnwidth}
                \caption{%
                    \textbf{Geometric intuition for curvature-noise interactions.}
                    \textbf{Left}:
                        Gradient noise pushes SGD toward minima that are flat
                        \emph{with respect to the covariance} (Corollary
                        \ref{cor:entropic}).  The red densities show the 
                        typical $\theta$s, perturbed from the
                        minimum due to noise $C$, in two cross sections of the
                        loss valley.  $J = \nabla H$ measures
                        how curvature changes across the valley.  Our theory
                        does not assume separation between ``fast'' and
                        ``slow'' modes, but we label them in the picture to
                        ease comparison with \cite{we19b}.
                    \textbf{\bf Right}:
                        Both curvature and the structure of noise affect
                        overfitting.  In each of the four subplots, the  
                        $\leftrightarrow$ axis represents weight space and the
                        $\updownarrow$ axis represents loss.
                        \protect\offive{1}:
                        \emph{covector}-perturbed landscapes favor large $H$s.
                        \protect\offive{2}:
                        \emph{vector}-perturbed landscapes favor small $H$s.
                        SGD's implicit regularization interpolates between
                        these rows (Corollary \ref{cor:overfit}).
                }
                \label{fig:cubicandspring}
            \end{figure}
    
        \subsection{High-$C$ regions repel small-$(E,B)$ SGD more than large-$(E,B)$ SGD}
            \label{subsect:epochs-batch}

            \begin{wrapfigure}[14]{r}{0.25\textwidth} 
                \centering
                \pmoo{3.5cm}{chladni}
                \caption{
                    \textbf{Chladni plate}. 
                    Grains of sand on a vibrating plate tend toward
                    stationary regions.
                }
                \label{fig:chladni}
            \end{wrapfigure}
            Physical intuition (\S\ref{appendix:interpret-diagrams}) suggests
            that noise repels SGD.  
            In particular, if two neighboring regions
            of weight space have high and low levels of gradient noise,
            respectively, then we expect the rate at which $\theta$ jumps from
            the former to the latter to exceed the opposite rate.  There is 
            thus a net movement toward regions of small $C$! 
            This mechanism parallels the Chladni effect \cite{ch87}
             (Figure \ref{fig:chladni}).\footnote{
                From Pierre Dragicevic and Yvonne Jansen's
                \href{http://www.dataphys.org/list/gallery/}{data physicalization project}, Creative
                Commons BY-SA 3.0. 
            }
            %
            Our theory makes this intuition precise; the drift is in the
            direction of $-\nabla C$, the effect is strongest when gradient
            noise is not averaged out by large batch sizes.
            \begin{cor}[$\sdia{c(01-2)(01-12)}$] \label{cor:batch}
                SGD avoids high-$C$ regions more than GD:
                $
                    l_{C}
                        \triangleq
                    \frac{N-1}{4 N}
                    \nabla^\mu C^{\nu}_{\nu}
                        =
                    \expct{\theta_{GD} - \theta_{SGD}}^\mu - o(\eta^2)
                $.
                If $\hat{l_c}$ is a smooth unbiased estimator of $l_c$, then GD
                on $l + \hat{l_c}$ has an expected test loss that agrees with
                SGD's to order $\eta^2$.  We call this method GDC.
            \end{cor}

            An analogous form of averaging occurs over multiple epochs.  For a
            tight comparison, we scale the learning rates appropriately so
            that, to leading order, few-epoch and many-epoch SGD agree.  Then
            few and many- epoch SGD differ, to leading order, in their
            sensitivity to $\nabla C$:
            \begin{cor}[$\sdia{c(01-2)(01-12)}$] \label{cor:epochs}
                SGD with $M=1$ and $\eta=\eta_0$ avoids high-$C$ regions more
                than SGD with $M=M_0$ and $\eta=\eta_0/M_0$.  Precisely:
                $
                    \expct{\theta_{M=M_0} - \theta_{M=1}}^\mu
                        =
                    \wrap{\frac{M_0-1}{M_0}} N
                    \wrap{\nabla^\mu C^{\nu}_{\nu}}
                    + o(\eta^2)
                $.
            \end{cor}
    
        \subsection{Non-Gaussian noise affects SGD but not SDE}
    
            Stochastic differential equations (SDE: see \cite{li18}) are a
            popular theoretical approximation of SGD, but SDE and SGD differ in
            several ways.  For instance, the inter-epoch noise correlations in
            multi-epoch SGD measurably affect SGD's final test loss (Corollary
            \ref{cor:epochs}), but SDE assumes uncorrelated gradient updates.
            Even if we restrict to single-epoch SDE, differences arise due to
            time discretization and non-Gaussian noise.  Intuitively, SGD and
            SDE respond differently to changes in curvature:
            %
            \begin{cor}[$\sdia{c(01-2)(02-12)}$, $\sdia{c(012-3)(03-13-23)}$] \label{cor:vsode}
                SGD's test loss is
                $
                    \frac{T}{2} C_{\mu\nu} H^{\mu\nu} + o(\eta^2)
                $
                more than ODE's and SDE's.
                The deviation from SDE due to skewed noise is
                $
                    - \frac{T}{6} S_{\mu\nu\lambda} J^{\mu\nu\lambda} 
                    + o(\eta^3)
                $.\footnote{
                    This approximation of Example \ref{exm:first}'s more exact
                    expression agrees with the latter to leading order in
                    $\eta$.
                }
            \end{cor}

%==============================================================================
%    EXPERIMENTS and CONCLUSION  
%==============================================================================

\section{Experiments}

    Despite the convergence results in Theorems \ref{thm:resum} and
    \ref{thm:converge}, we have no theoretical bounds for the domain and
    \emph{rate} of convergence.  Instead, we test our predictions by
    experiment.  We perceive support for our theory in drastic rejections of
    the null hypothesis.  For instance, in Figure \ref{fig:vanilla}\ofsix{4},
    \citep{ch18} predicts a velocity of $0$ while we predict a velocity of
    $\eta^2/6$
    %
    Here, \texttt{I} bars, \texttt{+} signs, and shaded regions all mark $95\%$
    confidence intervals based on the standard error of the mean.
    \S\ref{appendix:experiments} describes neural architectures, the definitions
    of artificial landscapes, sample sizes, and further plots.

    \subsection{Training time, epochs, and batch size; $C$ repels SGD more than GD}
        %----------------------------------------------------------------------
        %       Vanilla SGD                                 
        %----------------------------------------------------------------------
        We test Theorem \ref{thm:resum}'s order $\eta^3$ truncation on smooth
        convnets for CIFAR-10 and Fashion-MNIST.  Theory agrees with experiment
        through timescales long enough for accuracy to increase by $0.5\%$
        (Figure \ref{fig:vanilla}\ofsix{0},\ofsix{1}).
        %----------------------------------------------------------------------
        %       Epochs and Overfitting                      
        %----------------------------------------------------------------------
        \S\ref{appendix:figures} supports Corollary \ref{cor:epochs}'s
        predictions about epoch number.
        %----------------------------------------------------------------------
        %       Emulating Small Batches with Large Ones     
        %----------------------------------------------------------------------
        Figure \ref{fig:vanilla}\ofsix{2} tests Corollary \ref{cor:batch}'s
        claim that, relative to GD, high-$C$ regions \emph{repel} SGD.  This is
        significant because $C$ controls the rate at which the gen.\ gap (test
        minus train loss) grows (Corollary \ref{cor:overfit}, Figure
        \ref{fig:vanilla}\ofsix{3}).

        \begin{figure}[h!] 
            \centering
            \pmoo{3.5cm}{new-test-0}        \pmoo{3.5cm}{new-big-bm-new}          \pmoo{3.5cm}{new-thermo-linear-screw}
            \pmoo{3.5cm}{rebut-test-1-T100} \pmoo{3.5cm}{rebut-gen-cifar-lenet-4} \pmoo{3.5cm}{new-tak}
            \caption{
                {\bf Left: Perturbation models SGD for small $\eta T$.}
                Fashion-MNIST convnet's test loss vs learning rate.  In this
                small $T$ setting, we choose to use our theory's simpler 
                un-resummed values (\ref{appendix:evaluate-embeddings})
                instead of the more precise $\rvalue$s.
                %
                \protect\ofsix{0}: For all init.s tested ($1$ shown,
                $11$ unshown), the order $3$ prediction agrees with experiment
                through $\eta T \approx 10^0$, corresponding to a decrease
                in $0\mbox{-}1$ error of $\approx 10^{-3}$.
                %
                \protect\ofsix{1}: For large $\eta T$, our predictions
                break down.  Here, the order-$3$ prediction holds until the
                $0\mbox{-}1$ error improves by $5\cdot 10^{-3}$.
                Beyond this, $2$nd order agreement with experiment is
                coincidental.  
                %%%
                %%%
                \newline
                {\bf Center: $C$ controls gen.\ gap and distinguishes GD
                from SGD.}
                %
                With equal-scaled axes, \protect\ofsix{2} shows that
                GDC matches SGD (small vertical varianec) better than GD
                matches SGD (large horizontal variance) in test loss for a
                range of $\eta$ ($\approx 10^{-3}-10^{-1}$) and
                init.s\ (zero and several Xavier-Glorot trials) for
                logistic regression and convnets.  Here, $T=10$. 
                %
                \protect\ofsix{3}: CIFAR-10 generalization gaps.  For all
                init.s tested ($1$ shown, $11$ unshown), the
                degree-$2$ prediction agrees with experiment through $\eta T
                \approx 5\cdot 10^{-1}$.
                %%%
                %%%
                \newline
                {\bf Right: Predictions near minima excel for large $\eta T$.}%
                \protect\ofsix{4}: SGD travels \Archimedes' valley of global
                minima in the positive $z$ direction.  Note: $H$ and $C$ are
                bounded across the valley, we see drift for all small $\eta$,
                and we see displacement exceeding the landscape's period of
                $2\pi$.  So: the drift is not a pathology of well-chosen
                $\eta$, of divergent noise, or of ephemeral initial conditions.
                %
                \protect\ofsix{5}: For \MeanEstimation\, with fixed $C$ and a
                range of $H$s, initialized at the truth, the test losses after
                fixed-$T$ GD are smallest for very sharp and very flat $H$.
                Near $H=0$, our predictions improve on Takeuchi information
                \citep{di18} and thus on AIC.
                %%%
                %%%
                \newline
                \textbf{Throughout}:the \emph{re-normed} predictions refer to
                those of Theorem \ref{thm:resum}, approximated as in Remark
                \ref{rmk:integrate}.  Predictions not labeled as
                \emph{re-normed} are more drastic (namely, polynomial)
                approximations of Theorem \ref{thm:resum}'s result.
                We the authors should have written ``re-summed'' instead of
                ``re-normed'', and ``single-epoch, singleton-batch'' instead of
                ``vanilla''.
            }
            \label{fig:vanilla}
        %    \label{fig:batchandgen}
        %    \label{fig:thermoandtak}
        \end{figure}


    %--------------------------------------------------------------------------
    %           Thermodynamic Engine                        
    %--------------------------------------------------------------------------

    \subsection{Minima that are flat \emph{with respect to} $C$ attract SGD}
        %
        \begin{wrapfigure}[17]{r}{0.48\textwidth} 
            \centering
            \vspace{-15pt}
            \pmoo{3.0cm}{from-above}
            \pmoo{3.0cm}{from-side}
            \caption{
                \textbf{Two views of \Archimedes.}
                Green: a level surface of $l$ twisting around a valley of minima
                ($z$ axis) at its center; $l$ is large outside this surface.
                Purple tubes: $\theta$s typical due to non-isotropic noise.
                %
                The typical locations of $\theta$ are pulled
                toward lower loss (steepest descent toward the level surface)
                and so toward larger $z$.
                %
                The $z$ axis points into the page (\textbf{left}) or upward
                (\textbf{right}).
            }
            \label{fig:landscapes}
        \end{wrapfigure}
        %
        \label{subsect:entropic}
        To test the claimed dependence on $C$, \S\ref{appendix:artificial}
        constructs a landscape, \Archimedes, with non-constant $C$ throughout
        its valley of global minima.  Figure \ref{fig:landscapes} 
        depicts \Archimedes' chiral shape.\footnote{
            We made these plots with
            the help of Paul Seeburger's online applet,
            \href{https://www.monroecc.edu/faculty/paulseeburger/calcnsf/CalcPlot3D/}{CalcPlot3D}.
        }  As in
        Archimedes' screw or Rock-Paper-Scissors, each point $\theta$ has a
        neighbor that, from $C(\theta)$'s perspective but not absolutely, is
        flatter.  This permits eternal cyclic motion.
        %
        Indeed, Corollary \ref{cor:entropic} predicts 
        a $z$-velocity of $+\eta^2/6$ per timestep, while \cite{ch18}'s
        SDE-based analysis predicts a constant velocity of $0$.\footnote{
            Indeed, \Archimedes' velocity is $\eta$-perpendicular to the image
            of $(\eta C)^\mu_\nu$ in tangent space.
        }
        Our prediction agrees with experiment, even for large $T$ (Figure
        \ref{fig:vanilla}\ofsix{4}).
        %
        Because SGD's motion depends smoothly on the landscape, the special
        case of \Archimedes\ implies that non-conservativity is typical.
        %
        We may have sought an ``effective loss'' so that, up to $\sqrt{T}$
        diffusion terms, SGD on the old loss is like ODE on the new loss.  The
        non-conservativity of SGD's velocity shows that there is no such
        ``effective loss''.
        %


    %--------------------------------------------------------------------------
    %           Sharp vs Flat Minima                        
    %--------------------------------------------------------------------------

    \subsection{Sharp and flat minima both overfit less than medium minima} \label{subsect:overfit}

        Prior work (\S\ref{sect:related}) finds both that \emph{sharp} minima
        overfit less (for, $l^2$ regularization sharpens minima) or that
        \emph{flat} minima overfit less (for, flat minima are robust to small
        displacements).  In fact, both phenomena occur, and noise structure
        determines which dominates (Corollary \ref{cor:overfit}).  This effect
        appears even in \MeanEstimation\, (\S\ref{appendix:artificial}): Figure
        \ref{fig:vanilla}\ofsix{5}.
        %
        To combat overfitting, we may add Corollary \ref{cor:overfit}'s
        expression for gen.\ gap to $l$.  By descending on this regularized
        loss, we may tune smooth hyperparameters such as $l_2$ regularization
        coefficients for small datasets ($H \ll C/N$)
        (\S\ref{appendix:figures}).  Since matrix exponentiation takes time
        cubic in dimension, this regularizer is most useful for small models.

%==============================================================================
%    CONCLUSION      
%==============================================================================

\section{Conclusion: implications for practice} \label{sect:concl}

    %~~~~~~~~~~~~~~~~~~~~~~~~~~~~~~~~~~~~~~~~~~~~~~~~~~~~~~~~~~~~~~~~~~~~~~~~~~
    %~~~~~~~~~  Summarize Contributions  ~~~~~~~~~~~~~~~~~~~~~~~~~~~~~~~~~~~~~~

    We presented a diagram-based method for studying stochastic optimization on
    short timescales or near minima.
        Corollaries \ref{cor:entropic} and \ref{cor:overfit} together offer
        insight into SGD's success in training deep networks: SGD avoids
        curvature and noise, and curvature and noise control generalization.

    Analyzing $\sdia{c(01-2)(02-12)}$, we proved that \textbf{flat and sharp
    minima both overfit less} than medium minima.  Intuitively, flat minima are
    robust to vector noise, sharp minima are robust to covector noise, and
    medium minima robust to neither.  We thus proposed a regularizer enabling
    gradient-based hyperparameter tuning.
    %
    Inspecting $\sdia{c(01-2-3)(02-12-23)}$, we extended \cite{we19b} to
    nonconstant, nonisotropic covariance to reveal that \textbf{SGD descends on
    a landscape smoothed by the current covariance $C$}.
    As $C$ evolves, the
    smoothed landscape evolves, resulting in non-conservative dynamics.
    %
    Examining $\sdia{c(01-2)(01-12)}$, we showed that \textbf{GD may emulate
    SGD}, as conjectured by \cite{ro18}.  This is significant because, while
    small batch sizes can lead to better generalization \citep{bo91}, modern
    infrastructure increasingly rewards large batch sizes \citep{go18}.  

    %~~~~~~~~~~~~~~~~~~~~~~~~~~~~~~~~~~~~~~~~~~~~~~~~~~~~~~~~~~~~~~~~~~~~~~~~~~
    %~~~~~~~~~  Anticipate Criticism of Limitations  ~~~~~~~~~~~~~~~~~~~~~~~~~~

    %\subsection{Consequences}

        Since our predictions depend only on loss data near initialization,
        they break down after the weight moves far from initialization.  Our
        theory thus best applies to small-movement contexts, whether for long
        times (large $\eta T$) near an isolated minimum or for short times
        (small $\eta T$) in general.  Thus, the theory might help to analyze
        meta-learners based on fine-tuning (e.g.\ \cite{fi17}'s MAML).

        Much as meteorologists understand how warm and cold fronts interact
        despite long-term forecasting's intractability, we quantify how
        curvature and noise contribute to counter-intuitive dynamics governing
        each short-term interval of SGD's trajectory.  Equipped with our
        theory, practitioners may now refine intuitions --- e.g.\ that SGD
        descends on the training loss --- to account for noise.
       
%==============================================================================
%    CODA               
%==============================================================================

%~~~~~~~~~~~~~~~~~~~~~~~~~~~~~~~~~~~~~~~~~~~~~~~~~~~~~~~~~~~~~~~~~~~~~~~~~~~~~~
%~~~~~~~~~~~~~  Broader Impacts  ~~~~~~~~~~~~~~~~~~~~~~~~~~~~~~~~~~~~~~~~~~~~~~

\section*{Broader impacts}

    Though machine learning has the long-term potential for vast improvements
    in world-wide quality of life, it is today a source of enormous carbon
    emissions \citep{st19}.  Our analysis of SGD may lead to a reduced carbon
    footprint in three ways. 
     
    \textbf{First}, \S\ref{subsect:epochs-batch} shows how to modify the loss
    landscape so that large-batch GD enjoys the stochastic regularizing
    properties of small-batch SGD, or (symmetrically) so that small-batch SGD
    enjoys the stability of large-batch GD.  By unchaining the effective batch
    size from the actual batch size, we raise the possibility of training
    neural networks on a wider range of hardware than currently practical.  For
    example, asynchronous concurrent SGD (e.g. \cite{ni11}) might
    require less inter-device communication and therefore less power.
    %
    \textbf{Second}, \S\ref{sect:concl} discusses an application to meta-learning,
    which has the potential to decrease the per-task sample complexity and
    hence carbon footprint of modern machine learning.
    %
    \textbf{Third}, the modification of AIC developed in
    \S\ref{subsect:curvature-and-overfitting} and \S\ref{subsect:overfit}
    permits certain forms of model selection by gradient descent rather than
    brute force search.  This might drastically reduce the energy consumed
    during model selection.

    More broadly, this paper analyzes optimization in the face of uncertainty.
    As machine learning systems deployed today must increasingly address user
    privacy, pedestrian safety, and dataset diversity, it becomes important to
    recognize that test sets and training sets differ.  Toward this end,
    theoretical work relating to non-Gaussian noise may assist practitioners in
    building provably non-discriminatory, safe, or private models (e.g.
    \cite{dw06}).  By quantifying how correlated, non-Gaussian gradient noise
    affects descent-based learning, this paper contributes to such broader
    theory.

    That said, insofar as our theory furthers practice, it may instead
    contribute to the rapidly growing popularity of GPU-intensive learning,
    thus negating the aforementioned benefits and accelerating climate change.
    %
    Perhaps, then, it makes sense to examine the research goals that so often
    lead to massive computational costs.  For instance, only recently have
    these authors examined their routine assumption that smaller test losses
    are worth seeking; \emph{was this assumption obvious because it is true or
    merely because it is familiar}?
    %
    In fact, even ``pure'' theory hides a humbling host of assumptions
    ingrained and un-examined.  For example, we began the work reported here by
    asking, \textit{which minima does SGD prefer}?  Only after careful analysis
    did we realize that the question is ill-founded, for there is no absolute
    metric that SGD minimizes (\S\ref{subsect:entropic})! 
    %
    Zooming out: the authors have long prized sample efficiency --- and yet,
    should efficiency always be our goal?  Reflecting on optimization in
    general, \cite{ar19} suggests:
    perhaps it is not efficiency we should seek, but rather delight, surprise,
    and beauty.

%~~~~~~~~~~~~~~~~~~~~~~~~~~~~~~~~~~~~~~~~~~~~~~~~~~~~~~~~~~~~~~~~~~~~~~~~~~~~~~
%~~~~~~~~~~~~~  Acknowledgements  ~~~~~~~~~~~~~~~~~~~~~~~~~~~~~~~~~~~~~~~~~~~~~

\section*{Acknowledgements}

    %\begin{ack}
        We would like to thank
            \textsc{Sho Yaida},
            \textsc{Dan A.\ Roberts}, and
            \textsc{Josh Tenenbaum}
        for their patient guidance.
        %
        Through their incisive questions, \textsc{Dan A.\ Roberts}, \textsc{Ben
        R.\ Bray}, \textsc{Sasha Rakhlin}, \textsc{Satinder Singh}, \textsc{Sho
        Yaida}, and \textsc{Wenli Zhao} led us to the questions we answer here.
        %
        We appreciate the time that
            \textsc{Andy Banburski},
            \textsc{Ben R.\ Bray},
            \textsc{Jeff Lagarias}, and
            \textsc{Wenli Zhao}
        took to critique our drafts.
        %
        Without the encouragement of
            \textsc{Jason Corso},
            \textsc{Chloe Kleinfeldt},
            \textsc{Alex Lew}, 
            \textsc{Ari Morcos}, and
            \textsc{David Schwab},
        this paper would not be.
        %
        Finally, we thank our anonymous reviewers for inspiring an improved
        presentation.
        %
        This work was funded in part by MIT's Jacobs Presidential Fellowship.
    %\end{ack}
        
%~~~~~~~~~~~~~~~~~~~~~~~~~~~~~~~~~~~~~~~~~~~~~~~~~~~~~~~~~~~~~~~~~~~~~~~~~~~~~~
%~~~~~~~~~~~~~  References  ~~~~~~~~~~~~~~~~~~~~~~~~~~~~~~~~~~~~~~~~~~~~~~~~~~~

\newpage
%\section*{References}
    \bibliographystyle{plainnat}
    \bibliography{perturb}

%==============================================================================
%    APPENDICES      
%==============================================================================

\clearpage
\newpage
\renewcommand{\thesection}{\Alph{section}}
\setcounter{section}{0}

\section*{Organization of the appendices}
    The following three appendices serve three respective functions:
    \setlist{nolistsep}
    \begin{itemize}[noitemsep]
        \item to explain how to calculate using diagrams;
        \item to precisely state and prove our results, then pose a conjecture;
        \item to specify our experimental methods and results.
    \end{itemize}
    In more detail, we organize the appendices as follows.\\

    {\bf
    \par\noindent A ~ How to calculate test losses: a practical guide}      \hfill {\bf page \pageref{appendix:tutorial}}
    \par\indent     A.1 ~~ An example calculation                           \hfill \pageref{appendix:example}
    \par\indent     A.2 ~~ How to identify the relevant space-time          \hfill \pageref{appendix:draw-spacetime} 
    \par\indent     A.3 ~~ How to identify the relevant diagram embeddings  \hfill \pageref{appendix:draw-embeddings}
    \par\indent     A.4 ~~ How to evaluate each embedding                   \hfill \pageref{appendix:evaluate-embeddings}
    \par\indent     A.5 ~~ How to sum the embeddings' values                \hfill \pageref{appendix:sum-embeddings}
    \par\indent     A.6 ~~ Interpreting diagrams to build intuition         \hfill \pageref{appendix:interpret-diagrams}
    \par\indent     A.7 ~~ How to solve variant problems                    \hfill \pageref{appendix:solve-variants}
    \par\indent     A.8 ~~ Do diagrams streamline computation?              \hfill \pageref{appendix:diagrams-streamline}

    {\bf
    \par\noindent B ~ Mathematics of the theory}                            \hfill {\bf page \pageref{appendix:math}}
    \par\indent     B.1 ~~ Assumptions and definitions                      \hfill \pageref{appendix:assumptions}
    \par\indent     B.2 ~~ A key lemma \`a la Dyson                         \hfill \pageref{appendix:key-lemma}
    \par\indent     B.3 ~~ From Dyson to diagrams                           \hfill \pageref{appendix:toward-diagrams}
    \par\indent     B.4 ~~ Interlude: a review of M\"obius inversion        \hfill \pageref{appendix:mobius}
    \par\indent     B.5 ~~ Theorems \ref{thm:resum} and \ref{thm:converge}  \hfill \pageref{appendix:resum}
    \par\indent     B.6 ~~ How to modify proofs to handle variants          \hfill \pageref{appendix:prove-variants}
    \par\indent     B.7 ~~ Proofs of corollaries                            \hfill \pageref{appendix:corollaries}
    \par\indent     B.8 ~~ Future topics                                    \hfill \pageref{appendix:future}

    {\bf
    \par\noindent C ~ Experimental methods}                                 \hfill {\bf page \pageref{appendix:experiments}}
    \par\indent     C.1 ~~ What artificial landscapes did we use?           \hfill \pageref{appendix:artificial}  
    \par\indent     C.2 ~~ What image-classification landscapes did we use? \hfill \pageref{appendix:natural}
    \par\indent     C.3 ~~ Measurement process                              \hfill \pageref{appendix:measure}
    \par\indent     C.4 ~~ Implementing optimizers                          \hfill \pageref{appendix:optimizers}
    \par\indent     C.5 ~~ Software frameworks and hardware                 \hfill \pageref{appendix:frameworks}
    \par\indent     C.6 ~~ Unbiased estimators of landscape statistics      \hfill \pageref{appendix:bessel}
    \par\indent     C.7 ~~ Additional figures                               \hfill \pageref{appendix:figures}
    
    %{\bf
    %\par\noindent D ~ History of SGD}                                       \hfill {\bf page \pageref{appendix:history}}

%~~~~~~~~~~~~~~~~~~~~~~~~~~~~~~~~~~~~~~~~~~~~~~~~~~~~~~~~~~~~~~~~~~~~~~~~~~~~~~
%~~~~~~~~~~~~~  Tutorial  ~~~~~~~~~~~~~~~~~~~~~~~~~~~~~~~~~~~~~~~~~~~~~~~~~~~~~

\newpage
\section{How to calculate test losses: a practical guide}\label{appendix:tutorial}
    Our work introduces a novel technique for calculating the expected learning
    curves of SGD in terms of statistics of the loss landscape near
    initialization.  Here, we explain this technique.  Note that a new
    combinatorial object object --- \emph{spacetime} --- arises as we relax the
    paper body's assumption that $E=B=1$.  This, too, we will explain.  We note
    for now that there are are {\bf four steps} to computing the expected test
    loss, or other quantities of interest, after a specific number of gradient
    updates: 
    \begin{itemize}
        \item {\bf Draw the spacetime grid} that encodes our chosen SGD
            hyperparameters (namely, batch size, training set size, and number
            of epochs).
        \item {\bf Draw embeddings}, of diagrams into the
            spacetime, as needed for our desired precision.
        \item {\bf Evaluate each diagram embedding}, whether exactly
            (via $\rvalue$s) or roughly (via $\uvalue$s).
        \item {\bf Sum the embeddings' values} to obtain the quantity of
              interest as a function of $\eta$.
    \end{itemize}

    After presenting a small, complete example calculation that follows these
    four steps, we explain each of the respective steps in its own sub-section.
    We then discuss how diagrams often offer intuition as well as calculational
    help.  Though we focus on the computation of expected test losses, we
    describe how a small change in the above four steps allows for the
    computation also of variances (instead of expectations), of train losses
    (instead of test losses), or of weight displacements (instead of losses).
    We conclude by noting that our mathematical theory may be phrased without
    reference to diagrams; we compare such direct calculation to the diagram
    method, pointing out when and why diagrams streamline computation.

    \subsection{An example calculation}                             \label{appendix:example}
    \subsection{How to identify the relevant space-time}            \label{appendix:draw-spacetime}
    \subsection{How to identify the relevant diagram embeddings}    \label{appendix:draw-embeddings}
    \subsection{How to evaluate each embedding}                     \label{appendix:evaluate-embeddings}
        Make sure to also discuss un-re-summed values!!!
    \subsection{How to sum the embeddings' values}                  \label{appendix:sum-embeddings}
    \subsection{Interpreting diagrams to build intuition}           \label{appendix:interpret-diagrams}
        Make sure to also discuss Chladni effect \cite{ch87}!!!
    \subsection{How to solve variant problems}                      \label{appendix:solve-variants}
    \subsection{Do diagrams streamline computation?}                \label{appendix:diagrams-streamline}

%~~~~~~~~~~~~~~~~~~~~~~~~~~~~~~~~~~~~~~~~~~~~~~~~~~~~~~~~~~~~~~~~~~~~~~~~~~~~~~
%~~~~~~~~~~~~~  Math  ~~~~~~~~~~~~~~~~~~~~~~~~~~~~~~~~~~~~~~~~~~~~~~~~~~~~~~~~~

\newpage
\section{Mathematics of the theory}\label{appendix:math}
    \subsection{Assumptions and definitions}                        \label{appendix:assumptions}

    \subsection{A key lemma \`a la Dyson}                           \label{appendix:key-lemma}

        Suppose $s$ is an analytic function defined on the space of weights.
        The following Lemma, reminiscent of \cite{dy49a}, helps us track
        $s(\theta)$ as SGD updates $\theta$:
        \begin{klem*} \label{lem:dyson}
            For all $T$: for $\eta$ sufficiently small, $s(\theta_T)$ is a sum
            over tuples of natural numbers:
            \begin{equation}\label{eq:dyson}
                \sum_{(d_t: 0\leq t<T) \in \NN^T}
                (-\eta)^{\sum_t d_t}
                \wrap{
                    \prod_{0 \leq t < T}
                        \wrap{\left.
                            \frac{(g \nabla)^{d_t}}{d_t!}
                        \right|_{g = \sum_{n\in \Bb_t} \nabla l_n(\theta) / B}}
                }(s) (\theta_0)
            \end{equation}
            Moreover, the expectation symbol (over training sets) commutes with
            the sum over $d$s.
        \end{klem*}
        Here, we consider each $(g \nabla)^{d_t}$ as a higher order function
        that takes in a function $f$ defined on weight space and outputs a
        function equal to the $d_t$th derivative of $f$, times $g^{d_t}$.
        The above product then indicates composition of $(g \nabla)^{d_t}$'s
        across the different $t$'s.  In total, that product takes the function
        $s$ as input and outputs a function equal to some polynomial of $s$'s
        derivatives.

        \begin{proof}[Proof of the Key Lemma]%
            We work in a neighborhood of the initialization so that the tangent
            space of weight space is a trivial bundle.  For convenience, we fix
            a  coordinate system, and with it the induced flat,
            non-degenerate inverse metric $\tilde\eta$; the benefit is that we
            may compare our varying $\eta$ against one fixed $\tilde\eta$.
            Henceforth, a ``ball'' unless otherwise specified will mean a ball
            with respect to $\tilde\eta$ around the initialization $\theta_0$.
            Since $s$ is analytic, its Taylor series converges to $s$ within
            some positive radius $\rho$ ball.  By assumption, every $l_t$ is
            also analytic with radius of convergence around $\theta_0$ at least
            some $\rho>0$.  Since gradients are $x$-uniformly
            bounded by a continuous function of $\theta$, and since in finite
            dimensions the closed $\rho$-ball is compact, we have a strict
            gradient bound $b$ uniform in both $x$ and $\theta$ on gradient
            norms within that closed ball.  When
            \begin{equation} \label{eq:smalleta}
                2 \eta T b < \rho \tilde\eta
            \end{equation}
            as norms, SGD after $T$ steps on any train set
            will necessarily stay within the $\rho$-ball.\footnote{
                In fact, the factor of $2$ helps ensure that SGD initialized at
                any point within a $\rho/2$ ball will necessarily stay within
                the $\rho$-ball.
            } We note that the above condition on $\eta$ is weak enough to
            permit all $\eta$ within some open neighborhood of $\eta=0$.  

            Condition \ref{eq:smalleta} together with analyticity of $s$ then
            implies that
            $
                \wrap{\exp(-\eta g \nabla) s}(\theta) = s(\theta - \eta g)
            $
            when $\theta$ lies in the $\tilde\eta$ ball (of radius $\rho$) and
            its $\eta$-distance from that $\tilde\eta$ ball's boundary exceeds
            $b$, and that both sides are analytic in $\eta, \theta$ on the same
            domain --- and \emph{a fortiori} when $\theta$ lies in the ball of
            radius $\rho (1 - 1/(2T))$.  Likewise, a routine induction through
            $T$ gives the value of $s$ (after doing $T$ gradient steps from an
            initialization $\theta$) as
            $$
                \wrap{
                    \prod_{0\leq t<T}
                        \left.
                            \exp(-\eta g \nabla)
                        \right|_{g=\nabla l_t(\theta)}
                }
                (s)(\theta)
            $$
            for any $\theta$ in the $\rho (1-T/(2T)$-ball (that is, the
            $\rho/2$-ball), and that both sides are analytic in $\eta, \theta$
            on that same domain.  Note that in each exponential, the
            $\nabla_\nu$ does not act on the $\nabla_\mu l(\theta)$ with which
            it pairs.  

            Now we use the standard expansion of $\exp$.  Because (by
            analyticity) the order $d$ coeffients of $l_t, s$ are bounded by
            some exponential decay in $d$ that has by assumption an $x$-uniform
            rate, we have absolute convergence and may rearrange sums.  We
            choose to group by total degree:
            \begin{equation} \label{eq:expansion}
                \cdots 
                =
                \sum_{0\leq d < \infty} (-\eta)^d
                \sum_{\substack{(d_t: 0\leq t<T) \\ \sum_t d_t = d}}
                \wrap{
                    \prod_{0 \leq t < T} \left.
                        \frac{(g \nabla)^{d_t}}{d_t!}
                    \right|_{g=\nabla l_t(\theta)}
                } s (\theta)
            \end{equation}
            The first part of the Key Lemma is proved.  It remains to show that
            expectations over train sets commute with the above summation.

            We will apply Fubini's Theorem.  To do so, it suffices to show that   
            $$
                \wabs{c_d((l_t: 0\leq t<T))} 
                \triangleq
                \wabs{
                    \sum_{\substack{(d_t: 0\leq t<T) \\ \sum_t d_t = d}}
                    \wrap{
                        \prod_{0 \leq t < T} \left.
                            \frac{(g \nabla)^{d_t}}{d_t!}
                        \right|_{g=\nabla l_t(\theta)}
                    } s (\theta)
                }
            $$
            has an expectation that decays exponentially with $d$.  The symbol
            $c_d$ we introduce purely for convenience; that its value depends
            on the train set we emphasize using function application
            notation.  Crucially, no matter the train set, we have shown
            that the expansion \ref{eq:expansion} (that features $c_d$ appear
            as coefficients) converges to an analytic function for all $\eta$
            bounded as in condition \ref{eq:smalleta}.  The uniformity of this
            demanded bound on $\eta$ implies by the standard relation between
            radii of convergence and decay of coefficients that $\wabs{c_d}$
            decays exponentially in $d$ at a rate uniform over train sets.
            If the expectation of $\wabs{c_d}$ exists at all, then, it will
            likewise decay at that same shared rate.
            
            Finally, $\wabs{c_d}$ indeed has a well-defined expected value, for
            $\wabs{c_d}$ is a bounded continuous function of a
            (finite-dimensional) space of $T$-tuples (each of whose entries can
            specify the first $d$ derivatives of an $l_t$) and because the
            latter space enjoys a joint distribution.  So Fubini's Theorem
            applies.  The Key Lemma follows.   
        \end{proof}

    \subsection{From Dyson to diagrams}                             \label{appendix:toward-diagrams}

        We now describe the terms that appear in the Key Lemma.  The following
        result looks like Theorem \ref{thm:resum}, except it has $\uvalue(D)$
        instead of $\uvalue_f(D)$, and the sum is over all diagrams, not just
        irreducible ones.  In fact, we will use Theorem \ref{thm:pathint} to
        prove Theorem \ref{thm:resum}.

        \begin{thm}[Test Loss as a Path Integral] \label{thm:pathint}
            For all $T$: for $\eta$ sufficiently small, SGD's expected test
            loss is
            \begin{equation*}\label{eq:sgdcoef}
                \sum_{D}
                %\wrap{
                    \sum_{\text{embeddings}~f}
                    \frac{1}{\wabs{\Aut_f(D)}}
                %}
                \frac{\uvalue(D)}{(-B)^{|\edges(D)|}}
            \end{equation*}
            Here, $D$ is a diagram whose root $r$ does not participate in
            any fuzzy edge, $f$ is an embedding of $D$ into spacetime, and
            $\wabs{\Aut_f(D)}$ counts the graph-automorphisms of $D$ that
            preserve $f$'s assignment of nodes to cells.
            %
            If we replace $D$ by 
            $
                \wrap{-\sum_{p \in \parts(D)} (D_{rp} - D)/N}
            $, where $r$ is $D$'s root,
            we obtain the expected generalization gap (test minus train loss).
        \end{thm}

        Theorem \ref{thm:pathint} describe the terms that appear in the Key
        Lemma by matching each term to an embedding of a diagram in spacetime,
        so that the infinite sum becomes a sum over all diagram spacetime
        configurations.  The main idea is that the combinatorics of diagrams
        parallels the combinatorics of repeated applications of the product
        rule for derivatives applied to the expression in the Key Lemma.
        Balancing against this combinatorial explosion are factorial-style
        denominators, again from the Key Lemma, that we summarize in terms of
        the sizes of automorphism groups.

        \begin{proof}[Proof of Theorem \ref{thm:pathint}]
            We first prove the statement about test losses.
            Due to the analyticity property established in our proof of the
            Key Lemma, it suffices to show agreement at each degree $d$ and
            train set individually.  That is, it suffices to show --- for
            each train set $(l_n: 0\leq n<N)$, spacetime $S$, function $\pi:
            S\to [N]$ that induces $\sim$, and natural $d$ --- that
            \begin{align} \label{eq:toprove}
                (-\eta)^d
                \sum_{\substack{
                    (d_t: 0\leq t<T) \\
                    \sum_t d_t = d
                }}
                \wrap{
                    \prod_{0 \leq t < T} \left.
                        \frac{(g \nabla)^{d_t}}{d_t!}
                    \right|_{g=\nabla l_t(\theta)}
                } l (\theta)
                = \nonumber \\
                \sum_{\substack{
                    D \in \image(\Free) \\
                    \textnormal{with $d$ edges}
                }}
                \wrap{
                    \sum_{f: D\to\Free(S)}
                    \frac{1}{\wabs{\Aut_f(D)}}
                }
                \frac{\uvalue_\pi(D, f)}{B^{d}}
            \end{align}
            Here, $\uvalue_\pi$ is the value of a diagram embedding before
            taking expectations over train sets.  We have for all $f$ that
            $\expct{\uvalue_\pi(D, f)} = \uvalue(D)$.
            Observe that both sides of \ref{eq:toprove} are finitary sums.

            \begin{rmk}[Differentiating Products] \label{rmk:leibniz}
                The product rule of Leibniz easily generalizes to higher
                derivatives of finitary products:
                $$
                    \nabla^{\wabs{M}} \prod_{k \in K} p_k
                    = 
                    \sum_{\nu:M\to K} \prod_{k\in K} \wrap{
                        \nabla^{\wabs{\nu^{-1}(k)}} p_k
                    }
                $$
                The above has $\wabs{K}^{\wabs{M}}$ many term indexed by
                functions to $K$ from $M$.
            \end{rmk}

            We proceed by joint induction on $d$ and $S$.  The base cases
            wherein $S$ is empty or $d=0$ both follow immediately from the Key
            Lemma, for then the only embedding is the unique embedding of the
            one-node diagram $\sdia{(0)()}$.  For the induction step, suppose
            $S$ is a sequence of $\Mm = \min S \subseteq S$ followed by a
            strictly smaller $S$ and that the result is proven for $(\tilde d,
            \tilde S)$ for every $\tilde d \leq d$.  Let us group by $d_0$ the
            terms on the left hand side of desideratum \ref{eq:toprove}.
            Applying the induction hypothesis with $\tilde d = d - d_0$, we
            find that that left hand side is:
            \begin{align*}
                \sum_{\substack{
                    0 \leq d_0 \leq d
                }}
                \sum_{\substack{
                    \tilde D \in \image(\Free) \\
                    \textnormal{with $d-d_0$ edges}
                }}
                \frac{1}{d_0!}
                \sum_{\tilde f: \tilde D\to\Free(\tilde S)} \wrap{
                    \frac{1}{\wabs{\Aut_{\tilde f}(\tilde D)}}
                }
                ~\cdot~
                \\ %---------------------------------------------
                (-\eta)^{d_0}
                \left.
                    (g \nabla)^{d_0}
                \right|_{g=\nabla l_0(\theta)}
                \frac{\uvalue_\pi(\tilde D, \tilde f)}{B^{d-d_0}}
            \end{align*}
            Since $\uvalue_\pi(\tilde D, \tilde f)$ is a multilinear product of
            $d-d_0+1$ many tensors, the product rule for derivatives tells us
            that $(g \nabla)^{d_0}$ acts on $\uvalue_\pi(\tilde D, \tilde f)$
            to produce $(d-d_0+1)^{d_0}$ terms.  In fact,
            $
                g = \sum_{m\in \Mm} \nabla l_m(\theta) / B
            $ 
            expands to
            $B^{d_0}(d-d_0+1)^{d_0}$ terms, each conveniently indexed
            by a pair of functions $\beta:[d_0]\to \Mm$ and $\nu:[d_0]\to
            \tilde D$.  The $(\beta, \nu)$-term corresponds to an embedding
            $f$ of a larger diagram $D$ in the sense that it contributes
            $\uvalue_\pi(D, f)/B^{d_0}$ to the sum.  Here, $(f, D)$ is $(\tilde
            f, \tilde D)$ with $\wabs{\wrap{\beta \times \nu}^{-1}(n, v)}$ many
            additional edges from the cell of datapoint $n$ at time $0$ to the
            $v$th node of $\tilde D$ as embedded by $\tilde f$.

            By the Leibniz rule of Remark \label{rmk:leibniz}, this $(\beta,
            \nu)$-indexed sum by corresponds to a sum over embeddings $f$ that
            restrict to $\tilde f$, whose terms are multiples of the value of
            the corresponding embedding of $D$.  Together with the sum over
            $\tilde f$, this gives a sum over all embeddings $f$.  So we now
            only need to check that the coefficients for each $f:D\to S$ are as
            claimed.

            We note that the $(\beta, \nu)$ diagram (and its value) agrees with
            the $(\beta \circ \sigma, \nu \circ \sigma)$ diagram (and its
            value) for any permutation $\sigma$ of $[d_0]$.  The corresponding
            orbit has size
            \begin{align*}
                \frac{d_0!}{
                    \prod_{(m, i) \in \Mm \times \tilde D}
                        \wabs{(\beta \times \nu)^{-1}(m, i)}!
                }
            \end{align*}
            by the Orbit Stabilizer Theorem of elementary group theory.   

            It is thus enough to show that
            \begin{align*} \label{eqn:countclaim}
                \wabs{\Aut_f(D)} = 
                \wabs{\Aut_{\tilde f}(D)}
                \prod_{(m, i) \in \Mm \times \tilde D}
                    \wabs{(\beta \times \nu)^{-1}(m, i)}!
            \end{align*}
            We will show this by a direct bijection.  First, observe that
            $
                f = \beta \sqcup \tilde f:
                    [d_0] \sqcup \tilde D \to \Mm \sqcup \tilde S
            $. 
            So each automorphism $\phi: D\to D$ that commutes with $f$ induces
            both an automorphism
            $
                \Aa = \phi|_{\tilde D}: \tilde D\to \tilde D
            $
            that commutes with $\tilde f$ together with the data of a map
            $
                \Bb = \phi_{[d_0]}: [d_0] \to [d_0] 
            $
            that both commutes with $\beta$.  However, not every such pair of
            maps arises from a $\phi$.  For, in order for $\Aa \sqcup \Bb: D
            \to D$ to be an automorphism, it must respect the order structure
            of $D$.  In particular, if $x\leq_D y$ with $x \in [d_0]$ and $y
            \in \tilde D$, then we need
            $$
                \Bb(x) \leq_D \Aa(y)
            $$
            as well.  The
            pairs $(\Aa, \Bb)$ that thusly preserve order are in bijection with
            the $\phi \in \Aut_f(D)$.  There are $\wabs{\Aut_{\tilde f}(\tilde
            D)}$ many $\Aa$.  For each $\Aa$, there are as many $\Bb$ as there
            are sequences $(\sigma_i: i \in \tilde D)$ of permutations on
            $
                \{j\in [d_0]: j\leq_D i\} \subseteq [d_0]
            $ 
            that commute with $\Bb$.  These permutations may be chosen
            independently; there are 
            $
                \prod_{m\in \Mm}
                    \wabs{(\beta \times \nu)^{-1}(m, i)}!
            $
            many choices for $\sigma_i$.  Claim \ref{eqn:countclaim} follows,
            and with it the correctness of coefficients.
 
            The argument for generalization gaps parallels the above when we
            use $l-\sum_n l_n/N$ instead of $l$ as the value for $s$. 
            Theorem \ref{thm:pathint} is proved.
        \end{proof}

        \begin{rmk}[The Case of $E=B=1$ SGD]
            The spacetime of $E=B=1$ SGD permits all and only those
            embeddings that assign to each part of a diagram's partition  a
            distinct cell.  Such embeddings factor through a diagram
            ordering and are thus easily counted using factorials per
            Proposition \ref{prop:vanilla}.  That proposition immediately
            follows from the now-proven Theorem \ref{thm:pathint}.
        \end{rmk}

        \begin{prop} \label{prop:vanilla}
            The order $\eta^d$ contribution to the expected test loss of
            one-epoch SGD with singleton batches is:
            \begin{equation*}\label{eq:sgdbasiccoef}
                \frac{(-1)^d}{d!} \sum_{D} 
                |\ords(D)| {N \choose P-1} {d \choose d_0,\cdots,d_{P-1}}
                \uvalue(D)
            \end{equation*}
            where $D$ ranges over $d$-edged diagrams.  Here, $D$'s parts have
            sizes $d_p: 0\leq p\leq P$, and $|\ords(D)|$ counts the total
            orderings of $D$ s.t.\ children precede parents and parts are
            contiguous.
        \end{prop}

    \subsection{Interlude: a review of M\"obius inversion}          \label{appendix:mobius}

    \subsection{Theorems \ref{thm:resum} and \ref{thm:converge}}    \label{appendix:resum}

        The diagrams summed in Theorem \ref{thm:resum} and \ref{thm:converge}
        may be grouped by their geometric realizations.  Each nonempty class of
        diagrams with a given geometric realization has a unique element with
        minimally many edges, and in this way all and only irreducible diagrams
        arise. 

        We encounter two complications: on one hand, that the sizes of
        automorphism groups might not be uniform among the class of diagrams
        with a given geometric realization.  On the other hand, that the
        embeddings of a specific member of that class might be hard to count.
        The first we handle using Orbit-Stabilizer.  The second we address as
        described by \label{subsubsect:mobius} via M\"obius sums.
           
        \begin{proof}[Proof of Theorem \ref{thm:resum}]
            We apply M\"obius inversion (\S\ref{appendix:mobius}) to Theorem
            \ref{thm:pathint} (\S\ref{appendix:toward-diagrams}).  The result
            is that chains of embeddings  
            {\color{red} FILL IN}

            The difference in loss from the noiseless case is given by all the
            diagram embeddings with at least one fuzzy tie, where the fuzzy tie
            pattern is actually replaced by a difference between noisy and
            noiseless cases as prescribed by the preceding discussion on
            M\"obius Sums.  Beware that even relatively noiseless embeddings
            may have illegal collisions of non-fuzzily-tied nodes within a
            single spacetime (data) row.  Throughout the rest of this proof, we
            permit such illegal embeddings of the fuzz-less diagrams that arise
            from the aforementioned decomposition.  

            Because the Taylor series for analytic functions converge
            absolutely in the interior of the disk of convergence, the
            rearrangement of terms corresponding to a grouping by geometric
            realizations preserves the convergence result of Theorem
            \ref{thm:pathint}.  

            Let us then focus on those diagrams $\sigma$ with a given geometric
            realization represented by an irreducible diagram $\rho$.  By
            Theorem \ref{thm:pathint}, it suffices to show that
            \begin{equation} \label{eq:hard}
                \sum_{f:\rho\to S}
                \sum_{\substack{
                    \tilde f:\sigma\to S \\
                    \exists i_\star: f=\tilde f \circ i_\star
                }}
                \frac{1}{\wabs{\Aut_{\tilde f}(\sigma)}}
                =
                \sum_{f:\rho\to S}
                \sum_{\substack{
                    \tilde f:\sigma\to S \\
                    \exists i_\star: f=\tilde f \circ i_\star
                }}
                \sum_{\substack{
                    i:\rho\to\sigma \\
                    f = \tilde f \circ i
                }}
                \frac{1}{\wabs{\Aut_{f}(\rho)}}
            \end{equation}
            Here, $f$ is considered up to an equivalence defined by
            precomposition with an automorphism of $\rho$.  We likewise
            consider $\tilde f$ up to automorphisms of $\sigma$.  And above,
            $i$ ranges through maps that induce isomorphisms of geometric
            realizations, where $i$ is considered equivalent to $\hat i$ when
            for some automorphism $\phi \in \Aut_{\tilde f}(\sigma)$, we have
            $\hat i = i \circ \phi$.  Name as $X$ the set of all such $i$s
            under this equivalence relation.

            In equation \ref{eq:hard}, we have introduced
            redundant sums to structurally align the two expressions on the
            page; besides this rewriting, we see that equation \ref{eq:hard}'s
            left hand side matches Theorem \ref{thm:pathint} resulting formula
            and tgat its right hand side is the desired formula of Theorem
            \ref{thm:resum}. 

            To prove equation \ref{eq:hard}, it suffices to show (for any
            $f, \tilde f, i$ as above) that
            $$
                \wabs{\Aut_f(\rho)}
                =
                \wabs{\Aut_{\tilde f}(\sigma)}
                \cdot
                \wabs{X}
            $$
            We will prove this using the Orbit Stabilizer Theorem by presenting
            an action of $\Aut_f(\rho)$ on $X$.  We simply use precomposition
            so that $\psi\in \Aut_f(\rho)$ sends $i\in X$ to $i\circ \psi$.
            Since $f\circ\psi = f$, $i\circ \psi \in X$.  Moreover, the action
            is well-defined, because if $i\sim \hat i$ by $\phi$, then $i \circ
            \psi \sim \hat i \circ \psi$ also by $\phi$.
            
            The stabilizer of $i$ has size $\wabs{\Aut_{\tilde f}(\rho)}$.
            For, when $i \sim i \circ \psi$ via $\phi \in \Aut_{\tilde
            f}(\rho)$, we have $i\circ \psi = \phi \circ i$.  This relation in
            fact induces a bijective correspondence: \emph{every} $\phi$
            induces a $\psi$ via $\psi = i^{-1} \circ \phi \circ i$, so we have
            a map $\text{stabilizer}(i) \hookleftarrow \Aut_{\tilde f}(\rho)$
            seen to be well-defined and injective because structure set
            morphisms are by definition strictly increasing and because $i$s
            must induce isomorphisms of geometric realizations.  Conversely,
            every $\psi$ that stabilizes enjoys \emph{only} one $\phi$ via
            which $i \sim i \circ \phi$, again by the same (isomorphism and
            strict increase) properties.  So the stabilizer has the claimed
            size.

            Meanwhile, the orbit is all of $\wabs{X}$.  Indeed, suppose $i_A,
            i_B \in X$.  We will present $\psi \in \Aut_f(\rho)$ such that $i_B
            \sim i_A \circ \psi$ by $\phi=\text{identity}$.  We simply define
            $\psi = i_A^{-1} \circ i_B$, well-defined by the aforementioned
            (isomorphisms and strict increase) properties.  It is then routine
            to verify that
            $
                f \circ \psi
                =
                \tilde f \circ i_A \circ i_A^{-1} \circ i_B
                =
                \tilde f \circ i_B
                = f.
            $
            So the orbit has the claimed size, and by the Orbit Stabilizer
            Theorem, the coefficients in the expansions of Theorems 
            \ref{thm:resum} and \ref{thm:pathint} match.
        \end{proof}

        \begin{proof}[Proof of Theorem \ref{thm:converge}]
            Since we assumed hessians are positive: for any $m$, the propagator
            $K^t = \wrap{(I-\eta H)^{\otimes m}}^t$ exponentially decays to $0$
            (at a rate dependent on $m$).  Since up to degree $d$ only a finite
            number of diagrams exist and hence only a finite number of possible
            $m$s, the exponential rates are bounded away from $0$.  Moreover,
            for any fixed $t_{\text{big}}$, the number of diagrams ---
            involving no exponent $t$ exceeding $t_{\text{big}}$ --- is
            eventually constant as $T$ grows.  Meanwhile, the number involving
            at least one exponent $t$ exceeding that threshold grows
            polynomially in $T$ (with degree $d$).  The exponential decay of
            each term overwhelms the polynomial growth in the number of terms,
            and the convergence statement follows.
        \end{proof}

    \subsection{How to modify proofs to handle variants}            \label{appendix:prove-variants}

    \subsection{Proofs of corollaries}                              \label{appendix:corollaries}

        \subsubsection{Corollary \ref{cor:entropic}}

            \begin{proof}
                The relevant irreducible diagram is $\sdia{c(01-2-3)(02-12-23)}$
                {color{red} (amputated as in the previous subsubsection)}.   
                An embedding of this diagram into $E=B=1$ SGD's spacetime
                is determined by two durations --- 
                $t$ from {\color{moor}red} to {\color{moog}green} and
                $\tilde t$ from {\color{moog}green} to {\color{moob}blue} ---
                obeying $t+\tilde t \leq T$.
                The automorphism group of each embedding has size $2$: identity
                or switch the {\color{moor}red} nodes.  So the answer is: 
                $$
                    C_{\mu \nu}
                    J^{\rho\lambda}_{\sigma}
                    \wrap{\int_{t+\tilde t\leq T}
                        \wrap{\exp(-t \eta H) \eta}^{\mu\rho}
                        \wrap{\exp(-t \eta H) \eta}^{\nu\lambda}
                        \wrap{\exp(-\tilde t \eta H) \eta}^{\sigma\pi}
                    }
                $$
                Standard calculus then gives the desired result.
            \end{proof}

        \subsubsection{Corollary \ref{cor:overfit}'s first part}

            \begin{proof}[Proof.]
                The relevant irreducible diagram is $\sdia{(01-2)(02-12)}$
                (which equals $\sdia{c(01-2)(02-12)}$ because we are at a test
                minimum).  This diagram has one embedding for each pair of
                same-row shaded cells, potentially identical, in spacetime; for
                GD, the spacetime has every cell shaded, so each
                \emph{non-decreasing} pair of durations in $[0,T]^2$ is
                represented; the symmetry factor for the case where the cells
                is identical is $1/2$, so we lose no precision by interpreting
                a automorphism-weighted sum over the \emph{non-decreasing}
                pairs as half of a sum over all pairs.  Each of these may embed
                into $N$ many rows, hence the factor below of $N$.  The two
                integration variables (say, $t, \tilde t$) separate, and we
                have:
                $$
                    \frac{N}{B^{\text{degree}}}
                    \frac{C_{\mu\nu}}{2}
                    \int_t \wrap{\exp(-t \eta H)}^\mu_\lambda
                    \int_{\tilde t} \wrap{\exp(-\tilde t \eta H)}^\nu_\rho
                    \eta^{\lambda\sigma}
                    \eta^{\rho\pi}
                    H_{\sigma\pi}
                $$
                Since for GD we have $N=B$ and we are working to degree $2$,
                the prefactor is $1/N$.  Since $\int_t \exp(a t) = (I-\exp(-a
                T))/a$, the desired result follows. 
            \end{proof}

        \subsubsection{Corollary \ref{cor:overfit}'s second part}

            We apply the generalization gap modification (described in
            \S\ref{prove-variants}) to Theorem \ref{thm:resum}'s result about
            test losses.

            \begin{proof}[Proof]
                The relevant irreducible diagram is $\sdia{c(01)(01)}$.  This
                diagram has one embedding for each shaded cell of spacetime;
                for GD, the spacetime has every cell shaded, so each duration
                from $0$ to $T$ is represented.  So the generalization gap is,
                to leading order,
                $$
                    + \frac{C_{\mu\nu}}{N}
                    \int_t \wrap{\exp(-t \eta H)}^\mu_\lambda
                    \eta^{\lambda\nu}
                $$
                Here, the minus sign from the gen-gap modification canceled
                with the minus sign from the odd power of $-\eta$.  Integration
                finishes the proof.
            \end{proof}
 
        \subsubsection{Corollaries \ref{cor:epochs} and \ref{cor:batch}}

            Corollary \ref{cor:epochs} and the first part of Corollary
            \ref{cor:batch} follow from plugging appropriate values of $M,
            N, B$ into the following proposition.

            \begin{prop}\label{prop:ordtwo}
                To order $\eta^2$, the test loss of SGD --- on $N$
                samples for $M$ epochs with batch size $B$ dividing $N$ and with any
                shuffling scheme --- has expectation
                {\small
                \begin{align*}
                                                            l              
                    &- MN                                   G_\mu G^\mu       
                     + MN\wrap{MN - \frac{1}{2}}            G_\mu H^{\mu}_{\nu} G^\nu \\
                    &+ MN\wrap{\frac{M}{2}}                 C_{\mu \nu} H^{\mu \nu}
                     + MN\wrap{\frac{M-\frac{1}{B}}{2}}     \wrap{\nabla_\mu C^{\nu}_{\nu}} G^\mu / 2
                \end{align*}
                }
            \end{prop}

            \begin{proof}[of Proposition \ref{prop:ordtwo}]
                To prove Proposition \ref{prop:ordtwo}, we simply count
                the embeddings of the diagrams, noting that the automorphism groups
                are all of size $1$ or $2$.  Since we use fuzzy outlines instead of
                fuzzy ties, we allow untied nodes to occupy the same row, since the
                excess will be canceled out by the term subtract in the definition of
                fuzzy outlines.  See Table \ref{tbl:ordtwo}.
                \begin{table}[H]
                    \centering
                    \begin{tabular}{cll}
                        diagram                 & embed.s w/ $\wabs{\Aut_f}=1$  & embed.s w/ $\wabs{\Aut_f}=2$   \\ \hline
                        $\sdia{(0)()}$          & $1$                           & $0$                            \\  
                        $\sdia{(0-1)(01)}$      & $MNB$                         & $0$                            \\                  
                        $\sdia{(0-1-2)(01-12)}$ & ${MNB\choose 2}$              & $0$                            \\
                        $\sdia{c(01-2)(01-12)}$ & $N{MB\choose 2}$              & $0$                            \\
                        $\sdia{(0-1-2)(02-12)}$ & ${MNB\choose 2}$              & $0$                            \\
                        $\sdia{c(01-2)(02-12)}$ & $N{MB\choose 2}$              & $MNB$                             
                    \end{tabular}
                    \label{tbl:ordtwo}
                \end{table}
            \end{proof}

            \begin{proof}[Proof of Corollary \ref{cor:batch}'s second part]
                {\color{red} FILL IN}
            \end{proof}

        \subsubsection{Corollary \ref{cor:vsode}}

            The corollary's first part follows immediately from {\color{red}
            Remark
            \ref{rmk:vsode} in the case that $d=2$, $P=2$, and $(\eta N)^d$ is
            considered fixed while $N^{P-d-1}$ is considered changing.}

            \begin{proof}[Proof of second part]
                Because $\expct{\nabla l}$ vanishes at initialization, all
                diagrams with a degree-one vertex that is a singleton vanish.
                Because we work at order $\eta^3$, we consider $3$-edged
                diagrams.  Finally, because all first and second moments match
                between the two landscapes, we consider only diagrams with at
                least one partition of size at least $3$.  The only such test
                diagram is $\sdia{c(012-3)(03-13-23)}$.  This embeds in $T$
                ways (one for each spacetime cell) and has
                symmetry factor $1/3!$ for a total of
                $$
                    \frac{T \eta^3 }{6}
                    \expct{\nabla^3 l}
                    \expct{\nabla l_{n_{t_a}} \nabla l_{n_{t_b}} \nabla l_{n_{t_c}}}
                $$
            \end{proof}

    \subsection{Future topics}                                      \label{appendix:future}

        Our diagrams invite exploration of Lagrangian formalisms and curved
        backgrounds:\footnote{
            \cite{la60, la51} review these concepts.
        }
        \begin{quest}
            Does some least-action principle govern SGD; if not, what is an
            essential obstacle to this characterization?
        \end{quest}
        Lagrange's least-action formalism intimately intertwines with the
        diagrams of physics.  Together, they afford a modular framework for
        introducing new interactions as new terms or diagram nodes.  In fact,
        we find that some \emph{higher-order} methods --- such as the
        Hessian-based update
        $
            \theta \leftsquigarrow
            \theta -
            (\eta^{-1} + \lambda \nabla \nabla l_t(\theta))^{-1}
            \nabla l_t(\theta)
        $
        parameterized by small $\eta, \lambda$ --- admit diagrammatic analysis
        when we represent the $\lambda$ term as a second type of diagram node.
        Though diagrams suffice for computation, it is Lagrangians that most
        deeply illuminate scaling and conservation laws.

        Our work assumes a flat metric $\eta^{\mu\nu}$, but it might
        generalize to weight spaces curved in the sense of Riemann.\footnote{
            One may represent the affine connection as a node, thus giving
            rise to non-tensorial and hence gauge-dependent diagrams.
        }  Such curvature finds concrete application in the \emph{learning on
        manifolds} paradigm of \cite{ab07, zh16}, notably specialized to
        \cite{am98}'s \emph{natural gradient descent} and \cite{ni17}'s
        \emph{hyperbolic embeddings}.  While that work focuses on
        \emph{optimization} on curved weight spaces, in machine learning we
        also wish to analyze \emph{generalization}.
        %
        Starting with the intuition that ``smaller'' hypothesis classes
        generalize better and that curvature controls the volume of small
        neighborhoods, we conjecture that sectional curvature regularizes
        learning:
        \begin{conj}[Sectional curvature regularizes]
            If $\eta(\tau)$ is a Riemann metric on weight space, smoothly
            parameterized by $\tau$, and if the sectional curvature through
            every $2$-form at $\theta_0$ increases as $\tau$ grows, then
            the gen.\ gap attained by fixed-$T$ SGD with learning rate $c
            \eta(\tau)$ (when initialized from $\theta_0$) decreases as $\tau$
            grows, for all sufficiently small $c>0$.
        \end{conj}
        We are optimistic our formalism may resolve conjectures such as above.

%~~~~~~~~~~~~~~~~~~~~~~~~~~~~~~~~~~~~~~~~~~~~~~~~~~~~~~~~~~~~~~~~~~~~~~~~~~~~~~
%~~~~~~~~~~~~~  Experiments  ~~~~~~~~~~~~~~~~~~~~~~~~~~~~~~~~~~~~~~~~~~~~~~~~~~

\newpage
\section{Experimental methods}\label{appendix:experiments}

    \subsection{What artificial landscapes did we use?}             \label{appendix:artificial}

        We define three artificial landscapes, called
        \Gauss, \Archimedes, and \MeanEstimation.

        \subsubsection*{\Gauss}
            Consider fitting a centered normal $\Nn(0, \sigma^2)$ to some
            centered standard normal data.  We parameterize the landscape by
            $h=\log(\sigma^2)$ so that the Fisher information matches the
            standard dot product \citep{am98}.   
            %
            More explicitly, the \Gauss\, landscape is a probability
            distribution $\Dd$ over functions $l_x:\RR^1\to \RR$ on
            $1$-dimensional weight space, indexed by standard-normally
            distributed $1$-dimensional datapoints $x$ and defined by the
            expression:
            $$
                l_x(h)
                \triangleq
                \frac{1}{2} \wrap{h + x^2 \exp(-h)}
            $$
            The gradient at sample $x$ and weight $\sigma$ is then $g_x(h) =
            (1-x^2\exp(-h))/2$.  Since $x\sim \Nn(0, 1)$, the gradient $g_x(h)$
            will be affinely related to a chi-squared, and in particular
            non-Gaussian.
            
            To measure overfitting, we initialize at the true test minimum
            $h=0$, then train and see how much the test loss increases.  At
            $h=0$, the expected gradient vanishes, and the test loss of SGD
            involves only diagrams that have no leaves of size one.
            
        \subsubsection*{\Archimedes}
            The \Archimedes\, landscape has chirality, much like its namesake's
            screw \cite{vi00}.  Specifically, the \Archimedes\ landscape has
            %
            weights     $\theta = (u,v,z) \in \RR^3$,
            %
            data points $x \sim \Nn(0, 1)$,
            %
            and loss:
            %
            $$
                l_x(\theta)
                \triangleq
                \frac{1}{2} H(\theta) + x \cdot S(\theta)
            $$
            %
            Here,
            $$
                H(\theta) = u^2 + v^2 + (\cos(z) u + \sin(z) v)^2
            $$
            is quadratic in $u, v$, and
            $$
                S(\theta) = \cos(z-\pi/4) u + \sin(z-\pi/4) v
            $$
            is linear in $u, v$.
            Also, since $x \sim \Nn(0,1)$, the $x \cdot S(\theta)$ term has
            expectation $0$.
            %
            In fact, the landscape has a three-dimensional continuous screw
            symmetry consisting of translation along $z$ and simulateous
            rotation in the $u-v$ plane.  Our experiments are initialized at
            $u=v=z=0$, which lies within a valley of global minima defined by
            $u=v=0$.  

            The paper body showed that SGD travels in \Archimedes' $+z$
            direction.  By topologically quotienting the weight space, say by
            identifying points related by a translation by $\Delta z = 200\pi$,
            we may turn the line-shaped valley into a circle-shaped valley.
            Then SGD eternally travels, say, counterclockwise.
           
        \subsubsection*{\MeanEstimation}
            The \MeanEstimation\, family of landscapes has $1$ dimensional
            weights $\theta$ and $1$-dimensional datapoints $x$.  It is defined
            by the expression:
            $$
                l_x(\theta)
                \triangleq
                \frac{1}{2} H \theta^2 + x S \theta
            $$
            Here, $H, S$ are positive reals parameterizing the family; they
            give the hessian and (square root of) gradient covariance,
            respectively.

            For our hyperparameter-selection experiment (Figure
            \ref{fig:takreg}\ofthree{2}) we introduce an $l_2$
            regularization term as follows:
            $$
                l_x(\theta, \lambda)
                \triangleq
                \frac{1}{2} (H + \lambda) \theta^2 + x S \theta
            $$
            Here, we constrain $\lambda\geq 0$ during optimization using
            projections; we found similar results when parameterizing $\lambda
            = \exp(h)$, which obviates the need for projection but necessitates
            a non-canonical choice of initialization.  We initialize
            $\lambda=0$.

    \subsection{What image-classification landscapes did we use?}   \label{appendix:natural}

        \subsubsection*{Architectures}
            In addition to the artificial loss landscapes
            \Gauss, \Archimedes, and \MeanEstimation, 
            we tested our predictions on logistic linear regression
            and simple convolutional networks (2 convolutional weight layers
            each with kernel $5$, stride $2$, and $10$ channels, followed by
            two dense weight layers with hidden dimension $10$) for the
            CIFAR-10 \cite{kr09} and Fashion-MNIST datasets \cite{xi17}.  The
            convolutional architectures used $\tanh$ activations and Gaussian
            Xavier initialization.  To set a standard distance scale on weight
            space, we parameterized the model so that the
            Gaussian-Xavier initialization of the linear maps in each layer
            differentially pulls back to standard normal initializations of the
            parameters.
            
        \subsubsection*{Datasets}
            For image classification landscapes, we regard the finite amount of
            available data as the true (sum of diracs) distribution $\Dd$ from
            which we sample test and training sets in i.i.d.\ manner (and hence
            ``with replacement'').  We do this to gain practical access to a
            ground truth against which we may compare our predictions.  One
            might object that this sampling procedure would cause test and
            training sets to overlap, hence biasing test loss measurements.  In
            fact, test and training sets overlap only in reference, not in
            sense: the situation is analogous to a text prediction task in
            which two training points culled from different corpora happen to
            record the same sequence of words, say, ``Thank you!''.  In any
            case, all of our experiments focus on the limited-data regime, e.g.
            $10^1$ datapoints out of $\sim 10^{4.5}$ dirac masses, so overlaps
            are rare.

    \subsection{Measurement process}                                \label{appendix:measure}

        \subsubsection*{Diagram evaluation on real landscapes}
            We implemented the formulae of \S\ref{sect:bessel} in order
            to estimate diagram values from real data measured at
            initialization from batch averages of products of derivatives.

        \subsubsection*{Descent simulations}
            We recorded test and train losses for each of the trials below.  To
            improve our estimation of average differences, when we compared two
            optimizers, we gave them the same random seed (and hence the same
            training sets).

            We ran $2 \cdot 10^5$ trials of \Gauss\, with SDE and SGD,
            initialized at the test minimum with $T=1$ and $\eta$ ranging from
            $5\cdot 10^{-2}$ to $2.5\cdot 10^{-1}$.
            We ran $5 \cdot 10^1$ trials of \Archimedes with SGD with $T=10^4$
            and $\eta$ ranging from $10^{-2}$ to $10^{-1}$.
            We ran $10^3$ trials of \MeanEstimation with GD and STIC
            with $T=10^2$, $H$ ranging from $10^{-4}$ to $4 \cdot 10^0$,
            a covariance of gradients of $10^2$, and the true mean $0$ or
            $10$ units away from initialization.

            We ran $5 \cdot 10^4$ trials of the CIFAR-10 convnet on each of $6$
            Glorot-Xavier initializations we fixed once and for all through
            these experiments for the optimizers SGD, GD, and GDC, with $T=10$
            and $\eta$ between $10^{-3}$ and $2.5 \cdot 10^{-2}$.  We did
            likewise for the linear logistic model on the one initialization of
            $0$.

            We ran $4 \cdot 10^4$ trials of the Fashion-MNIST convnet on each
            of $6$ Glorot-Xavier initializations we fixed once and for all
            through these experiments for the optimizers SGD, GD, and GDC with
            $T=10$ and $\eta$ between $10^{-3}$ and $2.5 \cdot 10^{-2}$.  We
            did likewise for the linear logistic model on the one
            initialization of $0$. 

    \subsection{Implementing optimizers}                            \label{appendix:optimizers}

        We approximated SDE by refining time discretization by a factor of
        $16$, scaling learning rate down by a factor of $16$, and introducing
        additional noise in the shape of the covariance in proportion as
        prescribed by the Wiener process scaling.

        Our GDC regularizer was implemented using the unbiased estimator
        $$
            \hat{C} \triangleq (l_x - l_y)_\mu {l_x}_\nu / 2
        $$
        
        For our tests of regularization based on Corollary \ref{cor:overfit},
        we exploited the low-dimensional special structure of the artificial
        landscape in order to avoid diagonalizing to perform the matrix
        exponentiation: precisely, we used that, even on training landscapes,
        the covariance of gradients would be degenerate in all but one
        direction, and so we need only exponentiate a scalar.

    \subsection{Software frameworks and hardware}                   \label{appendix:frameworks}

        All code and data-wrangling scripts can be found on
        {\color{mooteal}github.com/???????/perturb}.  This link will be made
        available after the period of double-blind review.

        Our code uses PyTorch 0.4.0 \cite{pa19} on Python 3.6.7; there are no
        other substantive dependencies.  The code's randomness is parameterized
        by random seeds and hence reproducible.

        We ran experiments on a Lenovo laptop and on our institution's
        clusters; we consumed about $100$ GPU-hours.

    \subsection{Unbiased estimators of landscape statistics}        \label{appendix:bessel}
        %
        We use the following method --- familiar to some of our colleagues but
        hard to find writings on --- for obtaining unbiased estimates for
        various statistics of the loss landscape.  The method is merely an
        elaboration of Bessel's factor \citep{ga23}.  For completeness, we
        explain it here. 
        
        Given samples from a joint probability space $\prod_{0\leq d<D} X_d$,
        we seek unbiased estimates of \emph{multipoint correlators} (i.e.\ products of
        expectations of products) such as $\wang{x_0 x_1 x_2}\wang{x_3}$.  Here,
        angle brackets denote expectations over the population. 
        For
        example, say $D=2$ and from $2S$ samples we'd like to estimate
        $\wang{x_0 x_1}$.  Most simply, we could use $\Avg_{0\leq s<2S}
        x_0^{(s)} x_1^{(s)}$, where $\Avg$ denotes averaging over the sample.  In fact, the
        following also works:
        %
        \begin{equation} \label{eq:bessel}
            S
            \wrap{\Avg_{0\leq s< S} x_0^{(s)}}
            \wrap{\Avg_{0\leq s< S} x_1^{(s)}}
            +
            (1-S)
            \wrap{\Avg_{0\leq s< S} x_0^{(s)}}
            \wrap{\Avg_{S\leq s<2S} x_1^{(s)}}
        \end{equation}
        %
        When multiplication is expensive (e.g. when each $x_d^{(s)}$ is a
        tensor and multiplication is tensor contraction), we prefer the latter,
        since it uses $O(1)$ rather than $O(S)$ multiplications.  This in turn
        allows more efficient use of batch computations on GPUs.  We now
        generalize this estimator to higher-point correlators (and $D\cdot S$
        samples).

        For uniform notation, we assume without loss that each of the $D$
        factors appears exactly once in the multipoint expression of interest;
        such expressions then correspond to partitions on $D$ elements, which
        we represent as maps $\mu:\wasq{D}\to \wasq{D}$ with $\mu(d)\leq d$ and
        $\mu\circ \mu=\mu$.  Note that $\wabs{\mu} \coloneqq \wabs{im(\mu)}$
        counts $\mu$'s parts.  We then define the statistic
        %
        $$
            \wurl{x}_\mu
            \triangleq
            \prod_{0\leq d<D} \Avg_{0\leq s<S} x_d^{(\mu(d)\cdot S + s)}
        $$
        %
        and the correlator $\wang{x}_\mu$ we define to be the expectation of 
        $\wurl{x}_\mu$ when $S=1$.  In this notation, \ref{eq:bessel} says: 
        $$
            \wang{x}_{\partitionbox{0}\partitionbox{1}}
            =
            \expct{
                S       \cdot \wurl{x}_{\partitionbox{0 1}} +
                (1-S)   \cdot \wurl{x}_{\partitionbox{0}\partitionbox{1}}
            }
        $$
        %
        Here, the boxes indicate partitions of $\wasq{D}=\wasq{2}=\{0,1\}$.
        Now, for general $\mu$, we have:
        %
        \begin{equation} \label{eq:newbessel}
            \expct{S^D \wurl{x}_\mu}
            =
            \sum_{\tau\leq \mu} \wrap{
                \prod_{0\leq d<D}
                    \frac{S!}{\wrap{S-\wabs{\tau(\mu^{-1}(d))}}!}
            }
            \wang{x}_\tau
        \end{equation}
        %
        where `$\tau \leq \mu$' ranges through partitions \emph{finer} than 
        $\mu$, i.e. maps $\tau$ through which $\mu$ factors.   
        In smaller steps, \ref{eq:newbessel} holds because
        %
        \begin{align*}
            \expct{S^D \wurl{x}_\mu}
            &=
            \expct{
                \sum_{(0\leq s_d<S) \in \wasq{S}^D}
                \prod_{0\leq d<D}
                x_d^{\wrap{\mu(d)\cdot S + s_d}}
            }\\
            &=
            \sum_{\substack{(0\leq s_d<S) \\ \in \wasq{S}^D}}
            \expct{
                \prod_{0\leq d<D}
                x_d^{\wrap{\min \wurl{
                    \tilde{d}~:~\mu(\tilde{d})\cdot S+s_{\tilde{d}} = \mu(d)\cdot S+s_d
                }}}
            }\\
            &=
            \sum_{\tau} \wabs{\wurl{\substack{
                (0\leq s_d<S)~\in~[S]^D~: \\
                \wrap{\substack{
                    \mu(d)=\mu(\tilde{d}) \\
                    \wedge~s_d=s_{\tilde{d}}
                }}
                \Leftrightarrow
                \tau(d)=\tau(\tilde{d})
            }}}
            \wang{x}_\tau \\
            &=
            \sum_{\tau\leq \mu} \wrap{
                \prod_{0\leq d<D}
                    \frac{S!}{\wrap{S-\wabs{\tau(\mu^{-1}(d))}}!}
            }
            \wang{x}_\tau
        \end{align*}

        Solving \ref{eq:newbessel} for $\wang{x}_\mu$, we find:
        %
        \begin{equation*}
            \text{\fbox{$
            \wang{x}_\mu
            =
            \frac{S^D}{S^{\wabs{\mu}}}
            \expct{
                \wurl{x}_\mu
            }
            -
            \sum_{\tau < \mu} \wrap{
                \prod_{d\in im(\mu)}
                \frac{\wrap{S-1}!}{\wrap{S-\wabs{\tau(\mu^{-1}(d))}}!}
            }
            \wang{x}_\tau
            $}}
        \end{equation*}
        %
        This expresses $\wang{x}_\mu$ in terms of the batch-friendly estimator
        $\wurl{x}_\mu$ as well as correlators $\wang{x}_\tau$ for $\tau$
        \emph{strictly} finer than $\mu$.  We may thus (use dynamic programming
        to) obtain unbiased estimators $\wang{x}_\mu$ for all partitions $\mu$.
        Symmetries of the joint distribution and of the multilinear
        multiplication may further streamline estimation by turning a sum over
        $\tau$ into a multiplication by a combinatorial factor.  For example,
        in the case of complete symmetry:
        %
        $$
            \wang{x}_{\partitionbox{012}}
            =
            S^2
            \wurl{x}_{\partitionbox{012}}
            -
            \frac{(S-1)!}{(S-3)!}
            \wurl{x}_{\partitionbox{0}\partitionbox{1}\partitionbox{2}}
            -
            3\frac{(S-1)!}{(S-2)!}
            \wurl{x}_{\partitionbox{0}\partitionbox{12}}
        $$

    \subsection{Additional figures}                                 \label{appendix:figures}

        \begin{figure}[H] 
            \centering
            \centering
            \pmoo{3.5cm}{multi-fashion-logistic-0}
            \pmoo{3.5cm}{vs-sde}
            \pmoo{3.5cm}{tak-reg}
            \caption{
                \textbf{Further experimental results}.
                %
                \textbf{Left}: SGD with $2, 3, 5, 8$ epochs incurs greater test
                loss than one-epoch SGD (difference shown in I bars) by the
                predicted amounts (predictions shaded) for a range of learning
                rates.  Here, all SGD runs have $N=10$; we scale the learning
                rate for $E$-epoch SGD by $1/E$ to isolate the effect of
                inter-epoch correlations away from the effect of larger $\eta
                T$.
                %
                \textbf{Center}: SGD's difference from SDE after $\eta T
                \approx 10^{-1}$ with maximal coarseness on \Gauss.  Two
                effects not modeled by SDE --- time-discretization and
                non-Gaussian noise oppose on this landscape but do not
                completely cancel.  Our theory approximates the above curve
                with a correct sign and order of magnitude; we expect that the
                fourth order corrections would improve it further.
                %
                \textbf{Right}: Blue intervals regularization using Corollary
                \ref{cor:overfit}.  When the blue intervals fall below the
                black bar, this proposed method outperforms plain GD.  For
                \MeanEstimation\ with fixed $C$ and a range of $H$s, initialized
                a fixed distance \emph{away} from the true minimum, descent on
                an $l_2$ penalty coefficient $\lambda$ improves on plain GD for
                most Hessians.  The new method does not always outperform GD,
                because $\lambda$ is not perfectly tuned according to STIC but
                instead descended on for finite $\eta T$.
            }
            \label{fig:takreg}
        \end{figure}

%~~~~~~~~~~~~~~~~~~~~~~~~~~~~~~~~~~~~~~~~~~~~~~~~~~~~~~~~~~~~~~~~~~~~~~~~~~~~~~
%~~~~~~~~~~~~~  History of SGD  ~~~~~~~~~~~~~~~~~~~~~~~~~~~~~~~~~~~~~~~~~~~~~~~

%\section{History of SGD} \label{appendix:history}
%    We were surprised to learn of gradient descent's pre-silicon history:
%
%    It was \cite{ki52} who, in uniting gradient descent \citep{ca47} with
%    stochastic approximation \citep{ro51}, invented SGD.  Since the development
%    of back-propagation for efficient differentiation \citep{we74}, SGD has
%    been used to train connectionist models including neural networks
%    \citep{bo91}, in recent years to remarkable success \citep{le15}.

\end{document}
