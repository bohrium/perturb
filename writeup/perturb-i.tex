
%==============================================================================
%    LATEX PREAMBLE  
%==============================================================================

\documentclass{article}
\usepackage[T1]{fontenc}
\usepackage{microtype}
\usepackage{graphicx, float}
\usepackage{subfigure}
\usepackage{booktabs} % for professional tables
\usepackage{hyperref, xcolor}
\usepackage{amsmath, amssymb, amsthm, hanging, graphicx, txfonts, ifthen}
\usepackage{framed}


\usepackage[perpage]{footmisc}
        
\usepackage[percent]{overpic}

\newcommand{\theHalgorithm}{\arabic{algorithm}}

\theoremstyle{plain}
    \newtheorem*{klem*}{Key Lemma}
    \newtheorem{thm}{Theorem}
    \newtheorem{cor}{Corollary}
    \newtheorem{prop}{Proposition}
    \newtheorem{conj}{Conjecture}
    \newtheorem{quest}{Question}
    \newtheorem*{quest*}{Question}
    \newtheorem*{quests*}{Questions}
%
\theoremstyle{definition}
    \newtheorem{defn}{Definition}
    \newtheorem*{answ*}{Answer}
    \newtheorem{rmk}{Remark}
    \newtheorem*{rmk*}{Remark}

\usepackage{icml2019}
%\usepackage[accepted]{icml2019}

\usepackage{array}   % for \newcolumntype macro
\newcolumntype{L}{>{$}l<{$}}

\definecolor{moolime}{rgb}{0.90,1.00,0.90}
\definecolor{moosky}{rgb}{0.90,0.90,1.00}
\definecolor{moopink}{rgb}{1.00,0.90,0.90}
\definecolor{moor}{rgb}{0.8,0.2,0.2}
\definecolor{moog}{rgb}{0.2,0.8,0.2}
\definecolor{moob}{rgb}{0.2,0.2,0.8}
\definecolor{mooteal}{rgb}{0.1,0.6,0.4}


\renewcommand*{\thefootnote}{\color{red}\fnsymbol{footnote}} 

\newcommand{\nop}[2]{#2}
\newcommand{\Free}{\mathcal{F}}
\newcommand{\Forg}{\mathcal{G}}
\newcommand{\Mod}{\mathcal{M}}
\newcommand{\Hom}{\text{\textnormal{Hom}}}
\newcommand{\Aut}{\text{\textnormal{Aut}}}
\newcommand{\image}{\text{\textnormal{im}}}
\newcommand{\dvalue}{\text{\textnormal{value}}}
\newcommand{\rvalue}{\text{\textnormal{rvalue}}}
\newcommand{\edges}{\text{\textnormal{edges}}}
\newcommand{\ords}{\text{\textnormal{ords}}}
\newcommand{\parts}{\text{\textnormal{parts}}}
\newcommand{\Aa}{\mathcal{A}}
\newcommand{\Bb}{\mathcal{B}}
\newcommand{\Cc}{\mathcal{C}}
\newcommand{\Dd}{\mathcal{D}}
\newcommand{\Ee}{\mathcal{E}}
\newcommand{\Ff}{\mathcal{F}}
\newcommand{\Gg}{\mathcal{G}}
\newcommand{\Hh}{\mathcal{H}}
\newcommand{\Ll}{\mathcal{L}}
\newcommand{\Mm}{\mathcal{M}}
\newcommand{\Nn}{\mathcal{N}}
\newcommand{\Oo}{\mathcal{O}}
\newcommand{\Pp}{\mathcal{P}}
\newcommand{\Qq}{\mathcal{Q}}
\newcommand{\Rr}{\mathcal{R}}
\newcommand{\Ss}{\mathcal{S}}
\newcommand{\Tt}{\mathcal{T}}
\newcommand{\SGD}{\text{\textnormal{SGD}}}
\newcommand{\RR}{\mathbb{R}}
\newcommand{\expc}{\mathbb{E}}
\newcommand{\expct}[1]{\mathbb{E}\left[#1\right]}
\newcommand{\wrap}[1]{\left(#1\right)}
\newcommand{\wasq}[1]{\left[#1\right]}
\newcommand{\wang}[1]{\left\langle#1\right\rangle}
\newcommand{\wive}[1]{\left\llbracket#1\right\rrbracket}
\newcommand{\worm}[1]{\left\|#1\right\|}
\newcommand{\wabs}[1]{\left|#1\right|}
\newcommand{\wurl}[1]{\left\{#1\right\}}
\newcommand{\partbox}[1]{
    \text{
        \fboxsep=0.5pt
        \tiny
        \fbox{#1}
    }
}
\DeclareMathOperator*{\Avg}{\text{\sffamily A}}

\newcommand{\plotplace}[3]{
    \begin{overpic}[width=#2, height=#3]{../plots/blank.png}
        \put( 5, 85){
            \begin{tabular}{p{#2-1.0cm}}
                #1
            \end{tabular}
        }
    \end{overpic}
}

\newcommand{\plotmoo}[3]{
    \includegraphics[width=#2           ]{../#1}
}
\newcommand{\plotmooh}[3]{
    \includegraphics[          height=#3]{../#1}
}

\newcommand{\bdia}[1]{\begin{gathered}\includegraphics[scale=0.22]{../diagrams/#1.png}\end{gathered}}
\newcommand{\dia} [1]{\begin{gathered}\includegraphics[scale=0.18]{../diagrams/#1.png}\end{gathered}}
\newcommand{\mdia}[1]{\begin{gathered}\includegraphics[scale=0.14]{../diagrams/#1.png}\end{gathered}}
\newcommand{\sdia}[1]{\begin{gathered}\includegraphics[scale=0.10]{../diagrams/#1.png}\end{gathered}}

\newcommand{\half}{\frac{1}{2}}
\newcommand{\sixth}{\frac{1}{6}}

\newcommand{\ofsix}[1]{
    {\tiny $\substack{
        \ifthenelse{\equal{#1}{0}}{\blacksquare}{\square}
        \ifthenelse{\equal{#1}{1}}{\blacksquare}{\square} \\
        \ifthenelse{\equal{#1}{2}}{\blacksquare}{\square} 
        \ifthenelse{\equal{#1}{3}}{\blacksquare}{\square} \\
        \ifthenelse{\equal{#1}{4}}{\blacksquare}{\square}
        \ifthenelse{\equal{#1}{5}}{\blacksquare}{\square}
    }$}
}


\newcommand{\lorem}[1]{
    Lorem ipsum dolor sit amet, consectetur adipiscing elit...\\
    \nopagebreak\vspace{#1cm} \ \\
    ...sunt in culpa qui officia deserunt mollit anim id est laborum.
}


\begin{document}

%==============================================================================
%       TITLE AND AUTHOR
%==============================================================================

\icmltitlerunning{Descent as Scattering}

\twocolumn[
    \icmltitle{A Space-Time Approach to Analyzing Stochastic Gradient Descent}
    
    \begin{icmlauthorlist}
        \icmlauthor{Samuel C.~Tenka}{mit}
    \end{icmlauthorlist}
    \icmlaffiliation{mit}{
        Computer Science and Artificial Intelligence Lab,
        Massachusetts Institute of Technology,
        Cambridge, Massachusetts, USA
    }
    \icmlcorrespondingauthor{Samuel C.~Tenka}{coli@mit.edu}
    
    \icmlkeywords{Machine Learning, SGD, ICML}
    
    \vskip 0.3in
]
\printAffiliationsAndNotice{}

%==============================================================================
%       ABSTRACT        
%==============================================================================

\begin{abstract}
    We harness Feynman Diagrams to reason about the behavior of Stochastic
    Gradient Descent (SGD) at small learning rates $\eta$.
    %
    Illustrating this technique:
        We construct a regularizer that causes large-batch GD to emulate
        small-batch SGD.
        %
        We exhibit a non-conservative entropic force driving SGD.
        %
        We generalize the Akaike Information Criterion (AIC) to a smooth
        quantity liable to descent.
        %
        We quantify the differences
        %--- due to time-discretization,
        %inter-epoch correlations, and non-gaussian noise ---
        between SGD and
        the popular approximation SDE. 
    We empirically verify our theoretical predictions on artificial
    datasets and convnets for CIFAR-10 and Fashion-MNIST.
\end{abstract}

%==============================================================================
%       INTRODUCTION    
%==============================================================================

\section{Introduction}
    %\subsection{Overview}

        %----------------------------------------------------------------------
        %       Orienting Invitation 
        %----------------------------------------------------------------------

        Stochastic gradient descent (SGD) decreases an unknown objective $l$ by
        discrete-time $\eta$-steepest\footnote{
            The ``steepness'' concept depends on the choice of metric for
            $l$'s domain.  We thus consider all (flat) metrics at once.
            Specifically, we will Taylor expand an (inverse) metric
            $\eta^{\mu\nu}$ around $0$.
        } descent on noisy estimates of $l$.
        Natural questions concern this noise's effects:
        \begin{quests*}
            Intuitively, SGD and GD differ in their sensitivity to noise;
                when and why does one outperform the other?
            How does SGD select from among a valley of minima?
            By how much do sharp or flat minima overfit to noise?
        \end{quests*}
        To answer the above, we interpret SGD as a superposition of
        various weight-data interactions, each represented by a Feynman
        diagram\footnote{
            Section \ref{sect:calculus} explains these words in detail.
        }.
        For small $\eta$, few diagrams substantively contribute and
        our formalism permits concrete computation.

        %----------------------------------------------------------------------
        %       Killer Applications  
        %----------------------------------------------------------------------

        Unlike prior work, our work models discrete time and hence non-Gaussian
        noise.  Indeed, we give the finite-$N$, finite-$\eta^{-1}$ corrections
        to continuous-time approximations such as ordinary or stochastic
        differential equations (ODE, SDE).  We thus obtain new results
        quantifying how epoch number and batch size affect test loss.  Our
        theory of noise recommends two novel regularizers that respectively
        induce GD to mimic SGD and help to tune hyperparameters such as $l_2$
        coefficients.  We verify our predictions via precision order-$\eta^3$
        tests on convnets for CIFAR-10 and Fashion-MNIST and 
        by examining toy landscapes (details in Appendix \ref{sect:landscape}).

        %----------------------------------------------------------------------
        %       Connection to Physics
        %----------------------------------------------------------------------

        Path integrals offer not only quantitative predictions but also an
        exciting new viewpoint of optimization as physical scattering.  Indeed,
        we import physical tools such as \emph{crossing symmetries}
        \citep{dy49b} and \emph{renormalization} \citep{ge54} to simplify and
        refine our calculations.  The diagrams' combinatorics yield qualitative
        insight, e.g. that to order $\eta^3$, inter-epoch shuffling does not
        affect expected test loss.  Overall, we present the diagram method as a
        flexible and elegant tool for optimization theorists.  In the
        conclusion, we discuss Hessian methods and natural GD as
        low-hanging fruit for future analyses.

    \subsection{Overview of Formalism}

            %------------------------------------------------------------------
            %       The Idea of Diagrams Embedded in Spacetime
            %------------------------------------------------------------------

            Deferring details, we survey our formalism.  Say the $n$th
            datapoint participates in the $t$th update; we represent all such
            $(n, t)$ pairs of an SGD run as the shaded cells of a grid, where
            datapoints index rows and times index columns.  The shaded cells
            comprise the SGD run's \emph{spacetime}.
            \begin{figure}[H] 
                \centering  
                \plotmooh{diagrams/spacetime-e}{}{0.265\columnwidth}
                \plotmooh{diagrams/spacetime-f}{}{0.265\columnwidth}
                \caption{
                    Two spacetimes (grids of shaded cells) for $16$ steps
                    on $8$ datapoints along with some embedded
                    diagrams.  We arbitrarily choose colors to aid reference. 
                    Nodes (here black) outside the grid represent a
                    post-training measurement of test loss; their coordinates
                    are arbitrary.
                    {\bf Left}: SGD run with size-$1$ batches and with
                        shuffling.  
                        Reading across, we see:
                        an illegal embedding of
                        $\sdia{(01-2)(01-12)}$, a legal embedding of
                        $\sdia{(01-2)(01-12)}$, and a legal embedding of
                        $\sdia{(0-1-2)(01-12)}$.
                        Since they have $4$ edges, they decay as $\eta^2$.   
                    {\bf Right}: SGD run with size-$2$ batches and no
                        shuffling.  Depicted is an embedding of a $4$-edged
                        diagram, annotated with the data-weight interactions it
                        represents.
                        Since it has $4$ edges, it decays as $\eta^4$.   
                        Section \ref{sect:calculus} gives numerical meaning
                        to such diagrams.  The
                        test loss is a sum of infinitely many terms, each given
                        by a different diagram embedding.  
                }
                \label{fig:spacetimes}
            \end{figure}

            We interpret an SGD run as a superposition of many concurrent
            data-weight interactions.  Diagrams represent such
            processes (Figure \ref{fig:spacetimes}, right).  Diagrams evaluate
            to numbers, and an SGD run's expected test loss is a \emph{sum over
            diagrams} of those numbers, weighted by the number of ways a
            diagram can \emph{embed} or ``fit'' into the SGD run's spacetime.

            Specifically, diagrams are graphs such as $\sdia{(0-1)(01)}$,
            $\sdia{(01-2)(02-12)}$, $\sdia{(01-2)(01-12)}$, and
            $\sdia{(0-123)(02-12-23)}$: they are composed of thin edges and
            fuzzy ties, where the thin edges form a rooted tree --- by
            convention drawn so that children are left of their parents --- and
            where the fuzzy ties represent a partition of nodes.  {\color{moor}
            Colors} are arbitrarily chosen: they are labels --- we might refer
            to a diagram's ``green nodes'' --- and lack mathematical meaning.

            We may \emph{embed} a diagram into spacetime by placing each node
            in a shaded cell.  We allow only embeddings that both preserve
            left-right relationships along thin edges and map two
            nodes into the same row exactly when they are fuzzily tied.
            Figure \ref{fig:spacetimes}, left, illustrates these rules.

            Each $d$-edged diagram contributes to order $\eta^d$, so, for small
            $\eta$, the few-edged diagrams suffice to predict test losses.  Our
            recipe for predicting test losses to within error $o(\eta^d)$ is
            thus: \textsc{First}, draw all the diagrams with at most $d$ edges.
            \textsc{Then}, for each diagram, compute how many ways it can
            embed into the spacetime.  \textsc{Finally}, sum the diagrams,
            weighed by its number of embeddings.  We will formally state and
            prove that this correctly models SGD. 

        %----------------------------------------------------------------------
        %       Epoch Number Example     
        %----------------------------------------------------------------------

    \subsection{Appetizing Example: Role of Epoch Number}
            As an illustration, let us compare the test losses of one- and
            multi-epoch SGD.  We scale $M$-epoch SGD's $\eta$ by
            $1/M$\footnote{
                This scaling ensures that one- and $M$-epoch SGD
                agree on noiseless linear landscapes and thus a 
                more interesting comparison.
            }.
            One- and $M$-epoch SGD differ in their spacetimes and hence in how
            many times each diagram can embed.  Figure \ref{fig:epoch} shows
            that most diagrams with at most $2$ edges embed in corresponding
            ways into the two spacetimes.  Since those embeddings'
            contributions agree, we may neglect them as we compute the SGDs'
            difference.  Only the diagram
            $\sdia{c(01-2)(01-12)}$ has embeddings in $M$-epoch SGD that do not
            correspond to embeddings in one-epoch SGD (Figure \ref{fig:epoch}). 

            \begin{figure}[h!] 
                \centering  
                \plotmooh{diagrams/spacetime-c}{0.32\columnwidth}{0.32\columnwidth}
                \plotmooh{diagrams/spacetime-d}{0.64\columnwidth}{0.32\columnwidth}
                \caption{
                    Examples of diagrams embedded in spacetime for one-epoch
                    ({\bf left}) and two-epoch ({\bf right}) SGD.  The diagrams
                    for one and two epochs correspond by horizontal shifts
                    \emph{except} when same-row nodes are connected --- such as
                    in the top right diagram $\sdia{c(01-2)(01-12)}$. 
                }
                \label{fig:epoch}
            \end{figure}

            Clearly, $\sdia{c(01-2)(01-12)}$ embeds in $M$-epoch spacetime in
            $N\cdot{M\choose 2}$ many ways.  The order $\eta^2$ discrepancy
            between one- and multi- epoch SGD is therefore\footnote{
                The factor $1/M^2$ comes from our choice to scale $M$-epoch's
                learning rate by $1/M$ and that $2$-edged diagrams 
                are in $\Theta(\eta^2)$.
            }
            $$
                \text{test loss difference} =  
                \frac{N\cdot {M\choose 2}}{M^2}~
                \dvalue\wrap{\sdia{c(01-2)(01-12)}}
                + o(\eta^2)
            $$
            We complete the calculation by reading from the diagram's graph
            structure that it evaluates to\footnote{
                Hidden in the RHS's notation is a factor $\eta^2$: see  
                Section \ref{sect:background}.
            }
            \begin{align*}
                \dvalue\wrap{\sdia{c(01-2)(01-12)}}
                = 
                \frac{1}{2} G^\mu \nabla_\mu C^{\nu}_{\nu} 
            \end{align*}
            We have thus computed the leading order test loss difference
            between one- and multi- epoch SGD.

            In short: by using diagrams, we matched terms so that all but the
            $\sdia{c(01-2)(01-12)}$ terms canceled, thus avoiding an algebraic
            mess.  Moreover, when we later renormalize toward the large-$\eta
            T$ regime (Sections \ref{subsect:effective},
            \ref{subsect:entropic}, and \ref{subsect:overfit}), diagrams will
            become essential.  While diagrams may by direct use of our Key
            Lemma be sometimes avoided, Appendix \ref{sect:compare} shows that
            to avoid them convolutes analysis.  The more complicated the
            computation, the greater the savings of using diagrams.

%==============================================================================
%       BACKGROUND AND NOTATION
%==============================================================================

\section{Background and Notation} \label{sect:background}

    %--------------------------------------------------------------------------
    %           The Loss Landscape 
    %--------------------------------------------------------------------------

    \subsection{The Loss Landscape}
        We henceforth fix a loss landscape on a weight space $\Mm$, i.e. a
        mean-$l$ distribution over smooth functions $l_n:\Mm\to\RR$.
        We refer both to $n$ and to $l_n$ as \emph{datapoints}.

        More precisely, we fix a real-valued stochastic process indexed by the
        points of the \emph{weight space} $\Mm$, an affine manifold.  The
        process furnishes a \emph{train sequence} $(l_n: 0\leq n<N)$ of i.i.d.
        samples, each an unbiased estimate of $l$. We henceforth assume the
        regularity conditions listed in Appendix \ref{sect:proofs}, for
        instance that $l, l_n$ are analytic and that all moments exist.

        For example, these conditions admit $\tanh$ networks with cross entropy
        loss on bounded data --- and with arbitrary weight sharing, skip
        connections, soft attention, dropout, batch-normalization with disjoint
        batches, and weight decay.
        
    %--------------------------------------------------------------------------
    %           Tensor Conventions
    %--------------------------------------------------------------------------

    \subsection{Tensor Conventions}
        We use $G_\mu, H_{\mu\nu}, J_{\mu\nu\lambda}$ for the first, second,
        and third derivatives of $l$ and $C_{\mu \nu}$ for the covariance of
        gradients.  By convention, our repeated Greek indices are implicitly
        summed: if $A_\mu, B^\mu$ are the coefficients of a covector $A$ and a
        vector
        $B$\footnote{
            Vectors/covectors are also called column/row vectors.
        }, indexed by basis elements $\mu$, then
        $
            A_\mu B^\mu
            \triangleq
            \sum_\mu A_\mu \cdot B^\mu
        $.
        To expedite dimensional analysis, we regard the learning rate as an
        inverse metric $\eta^{\mu\nu}$ that converts a gradient covector into a
        vector displacement \citep{bo13}.  We use $\eta$ to \emph{raise}
        indices.  In
        $
            H^{\mu}_{\lambda}
            \triangleq
            \eta^{\mu\nu} H_{\nu\lambda}
        $, for instance,
        $\eta$ raises one of $H_{\mu\nu}$'s indices.  Another example is
        $
            C^{\mu}_{\mu}
            \triangleq
            \sum_{\mu \nu} \eta^{\mu\nu} \cdot C_{\nu\mu}
        $.
        Standard syntactic constraints make manifest which expressions
        transform naturally with respect to optimization dynamics.  

        We say two expressions \emph{agree to order $\eta^d$} when
        their difference, divided by some homogeneous degree-$d$
        polynomial of $\eta$, tends to $0$ as $\eta$ shrinks.  Their
        difference is then $\in o(\eta^d)$.

    %--------------------------------------------------------------------------
    %           Names of SGD Parameters
    %--------------------------------------------------------------------------

    \subsection{SGD Terminology}
        SGD updates on the train sequence $(l_n: 0\leq n<N)$.
        For each of $M\cdot B$ \emph{epochs}, SGD partitions the $N$ datapoints
        into a length-$N/B$ sequence of size-$B$ \emph{batches}.  For each
        batch $\Bb$, SGD updates
        $
            \theta^\mu
            \leftsquigarrow
            \theta^\mu -
            \eta^{\mu\nu} \nabla_\nu
                \wrap{\frac{1}{B} \sum_{n\in \Bb} l_n(\theta)}
        $.
        SGD performs $N\cdot M$ many updates in total.  We write $l_t$ for the
        loss $\frac{1}{B}\sum_\Bb\cdots$ over the $t$th batch. 
        Together with an
        inter-epoch shuffling pattern, $N, B, M$ determine the SGD algorithm.
        The cases $B=1$ and $B=N$ we call \emph{pure SGD} and \emph{pure GD}.
        The $M=1$ case of pure SGD we call \emph{vanilla SGD}.

%==============================================================================
%    DIAGRAM CALCULUS FOR SGD
%==============================================================================

\section{Diagram Calculus for SGD} \label{sect:calculus}

    %--------------------------------------------------------------------------
    %           Role of Diagrams      
    %--------------------------------------------------------------------------

    \subsection{How Diagrams Arise}
        Suppose $s$ is analytic on weight space, for example $s=l$.
        We track $s(\theta)$ as SGD updates $\theta$: %(c.f. \citet{dy49a}):
        \begin{klem*} %\label{lem:dyson}
            For all $T$: for $\eta$ sufficiently small, $s(\theta_T)$ is
            \begin{equation}\label{eq:dyson}
                \sum_{(d_t: 0\leq t<T)}
                (-\eta)^{\sum_t d_t}
                \left(
                    \prod_{0 \leq t < T}
                        \left.
                            \frac{(g \nabla)^{d_t}}{d_t!}
                        \right|_{g={\color{moor}\nabla l_t(\theta)}}
                \right)
                (s) (\theta_0)
            \end{equation}
            Moreover, the expectation symbol (over train sets) commutes with
            the sum over $d$s.
        \end{klem*}
        In averaging over train sets we may factor the expectation of the above
        product according to independence relations between the $l_t$.  We view
        various training procedures (e.g. pure GD, pure SGD) as
        \emph{prescribing different independence relations} that lead to
        different factorizations and hence to potentially different
        generalization behavior at each order.
    
        An instance of the above product (for $s=l_a$ drawn from a test set and
        $0\leq c\leq b<T$) is $-\eta^3 (\nabla l_c \nabla)^2 (\nabla l_b
        \nabla) l_a$, which is
        {\small
        \begin{align*}
            - (\nabla^\lambda l_c) (\nabla^\mu l_c) (\nabla_\lambda \nabla_\mu \nabla^\nu l_b) (\nabla_\nu l_a)   
            - (\nabla^\lambda l_c) (\nabla^\mu l_c) (\nabla_\lambda \nabla^\nu l_b) (\nabla_\mu \nabla_\nu l_a) \\
            - (\nabla^\lambda l_c) (\nabla^\mu l_c) (\nabla_\mu \nabla^\nu l_b) (\nabla_\lambda \nabla_\nu l_a)   
            - (\nabla^\lambda l_c) (\nabla^\mu l_c) (\nabla^\nu l_b) (\nabla_\lambda \nabla_\mu \nabla_\nu l_a)
        \end{align*}
        }
        To reduce clutter, we adapt the string notation of \citet{fe49, pe71}.
        Then, in expectation over $(l_c, l_b, l_a)$ drawn i.i.d.:
        \begin{align}
            \cdots
            &= 
                 \sdia{(01-2-3)(02-12-23)}
                +\sdia{(01-2-3)(02-13-23)}
                +\sdia{(01-2-3)(03-12-23)}
                +\sdia{(01-2-3)(03-13-23)} \\
                \label{eq:simpl}
            &=
                \underbrace{2\sdia{(01-2-3)(02-12-23)}}_{
                   -2~\expct{{\color{moor}(\nabla l)(\nabla l)}}~\expct{{\color{moog}\nabla\nabla\nabla l}}~\expct{{\color{moob} \nabla l}}
                }
                +
                \underbrace{2\sdia{(01-2-3)(02-13-23)}}_{
                   -2~\expct{{\color{moor}(\nabla l)(\nabla l)}}~\expct{{\color{moog}\nabla \nabla l}}~\expct{{\color{moob}\nabla \nabla l}}
                }
        \end{align}
        Each degree-$g$ node corresponds to an $l_n$ (here,
            {\color{moor} $l_c$},
            {\color{moog} $l_b$},
            {\color{moob} $l_a$}) differentiated $g$ times
        (for instance, {\color{moog} $l_b$} is differentiated thrice in the
        first diagram and twice in the second).  Thin \emph{edges} mark
        contractions by $-\eta$.  Fuzzy \emph{ties} denote correlations by
        connecting identical loss functions (here, {\color{moor} $l_c$} with
        {\color{moor} $l_c$}), and are visually emphasized by otherwise
        meaningless colors.
        \begin{defn}
            A diagram $D$ evaluates to the expected value
            $\dvalue(D)$ of the corresponding tensor expression.
        \end{defn}
        % 
        Usefully, % for a fixed, i.i.d.  distribution over $(l_c, l_b, l_a)$,
        \emph{the topology of a diagram determines its value}.  For instance,
        simplification \ref{eq:simpl} is licensed because
        $
            \dvalue\wrap{\sdia{(01-2-3)(02-12-23)}}
            =
            \dvalue\wrap{\sdia{(01-2-3)(03-13-23)}}
        $.
        We will sometimes write a diagram $D$ and mean $\dvalue(D)$. 
    
        \begin{defn}[Fuzzy Outlines Denote the Effect of Noise]
            As we study the effect of noise, we'd like to concisely notate the
            fuzzy ties that represent it.
            So, if $D$ is a diagram and $p, \tilde p$ are two parts of $D$'s
            parition, let $D_{p\tilde p}$ be the diagram with $p, \tilde p$
            joined into a single part.  So:
            $
                (\sdia{(0-1)(01)})_{\text{\color{moor}red}~\text{\color{moog}green}}
                \triangleq
                \sdia{(01)(01)}
            $
            and
            $
                (\sdia{(01-2-3)(02-12-23)})_{\text{\color{moor}red}~\text{\color{moob}blue}}
                \triangleq
                \sdia{(013-2)(02-12-23)}
            $.
            Differences such as $D_{p\tilde p}-D$ will often concern us, so we
            define a diagram with fuzzy \emph{outlines} to represent the
            difference between the fuzzy tied and untied versions so that
            $
                \sdia{c(0-12)(01-12)}
                \triangleq
                (\sdia{(0-1-2)(01-12)})_{\text{\color{moog}green}~\text{\color{moob}blue}}
                -
                \sdia{(0-1-2)(01-12)}
                =
                \sdia{(0-12)(01-12)}
                -
                \sdia{(0-1-2)(01-12)}
            $.
        \end{defn}
            
    %--------------------------------------------------------------------------
    %           Recipe for Test Loss and Generalization Gap
    %--------------------------------------------------------------------------

    \subsection{Recipe for SGD's Test Loss and Generalization}
        Our main tool, proved in Appendix \ref{sect:proofs}, follows.
        
        \begin{thm}[Test Loss as a Path Integral] \label{thm:sgdcoef}
            For all $T$: for $\eta$ sufficiently small, SGD's expected test
            loss is
            \begin{equation*}\label{eq:sgdcoef}
                \sum_{D \in \image(\Free)} \wrap{
                    \sum_{f: D\to\Free(S)}
                    \frac{1}{\wabs{\Aut_f(D)}}
                }
                \frac{\dvalue(D)}{B^{|\edges(D)|}}
            \end{equation*}
            Here, $D$ is a diagram whose root $r$ participates in no fuzzy
            edge,    $f$ represents an embedding of $D$ into spacetime, and
            $\wabs{\Aut_f(D)}$ counts the graph-automorphisms of $D$ that
            preserve $f$'s assignment of nodes to cells.
            %
            Replacing each $\dvalue(D)$ by
            $
                \wrap{-\sum_{p \in \parts(D)} \dvalue(D_{rp} - D)/N}
            $,
            we obtain the expected generalization gap (test minus train
            loss).
        \end{thm}
    
        \begin{prop}[Specialization to Vanilla SGD] \label{prop:vanilla}
            The order $\eta^d$ contribution to the expected test loss of
            one-epoch SGD with singleton batches is:
            \begin{equation*}\label{eq:sgdbasiccoef}
                \frac{(-1)^d}{d!} \sum_{D\in \image(\Free)} 
                |\ords(D)| {N \choose P-1} {d \choose d_0,\cdots,d_{P-1}}
                \dvalue(D)
            \end{equation*}
            where $D$ ranges over $d$-edged diagrams whose equivalence classes
            have sizes $d_p: 0\leq p\leq P$, with $d_P=1$
            and, without loss, are each antichains.  The modification to
            compute the generalization gap is the same as in Theorem
            \ref{thm:sgdcoef}.
        \end{prop}
        A $P$-part, $d$-edged diagram then contributes $\Theta\left((\eta N)^d
        N^{P-d-1}\right)$ to the loss.  For example, there are six diagrams to
        third order, and they have $(4+2)+(2+2+3)+(1)$ many orderings --- see
        Table \ref{tab:scatthree}.  Intuitively, $\eta N$ measures the physical
        time of descent and $1/N$ measures the coarseness of time
        discretization.  So we have a double-series in $(\eta N)^d N^{P-d-1}$,
        where $d$ counts thin edges and $d+1-P$ counts fuzzy ties; the $P=d+1$
        terms correspond to a discretization-agnostic (hence continuous-time,
        noiseless) ODE approximation to SGD, while $P\leq d$ gives correction
        terms modeling time-discretization and hence noise.  

    %--------------------------------------------------------------------------
    %           Theoretical Corollaries                     
    %--------------------------------------------------------------------------

    \subsection{Consequences of the Recipe}
        For accessibility, we write our results with Greek indices instead of
        diagrams.  However, as Appendices \ref{sect:tutorial},
        \ref{sect:compare}, and \ref{sect:calculations} show, diagrams are
        crucial for proving --- and for identifying the data-weight processes
        that govern --- these results.

        \begin{cor}[SGD Differs from ODE and SDE] \label{cor:vsode}
            The test loss of vanilla SGD deviates at order $N^{-1}$ from
            ODE by
            $
                %({{T^2 N^{-1}}/{2}}) \sdia{c(01-2)(02-12)}
                \frac{T^2 N^{-1}}{2} ~ C_{\mu\nu} H^{\mu\nu}
            $.
            Its order $N^{-2}$ deviation due to non-gaussian noise is
            $
                \frac{T^3 N^{-2}}{6} \wrap{
                    \sdia{c(012-3)(03-13-23)}
                    -
                    3 \sdia{c(01-2-3)(03-13-23)}
                }
                =
                -
                \frac{T^3 N^{-2}}{6}  
                \wrap{
                    \wrap{
                        \expct{\nabla_\mu l_x \nabla_\nu l_x \nabla_\lambda l_x}-
                        G_\mu G_\nu G_\lambda
                    }
                    J^{\mu\nu\lambda}
                    - 
                    3
                    C_{\mu \nu} G_\lambda J^{\mu\nu\lambda}
                }
            $.
            These effects contribute to SGD's difference from SDE.
        \end{cor}
        For finite $N$, these effects separate SDE from SGD.  SDE also fails to
        model multi-epoch SGD's inter-update correlations.  Conversely, as
        $N\to\infty$ so that SDE matches SGD, optimization and generalization
        respectively become computationally intractable and trivial and hence
        less interesting.
    
        \begin{table}[h]
            \centering 
            \resizebox{\columnwidth}{!}{%
            \begin{tabular}{c|c|c}
                {\LARGE $\Theta\left((\eta N)^3 N^{-0}\right)$} &
                {\LARGE $\Theta\left((\eta N)^3 N^{-1}\right)$} &
                {\LARGE $\Theta\left((\eta N)^3 N^{-2}\right)$} \\ \hline
                \begin{tabular}{c}
                    \begin{tabular}{LL}
                        \bdia{(0-1-2-3)(01-12-23)} & \bdia{(0-1-2-3)(01-13-23)}
                    \end{tabular} \\
                    \begin{tabular}{LL}
                        \bdia{(0-1-2-3)(02-13-23)} & \bdia{(0-1-2-3)(03-12-23)}
                    \end{tabular} \\ \hline
                    \begin{tabular}{LL}
                        \bdia{(0-1-2-3)(03-13-23)} & \bdia{(0-1-2-3)(02-12-23)}
                    \end{tabular}
                \end{tabular}
                &
                \begin{tabular}{c}
                    \begin{tabular}{LL}
                        \bdia{(01-2-3)(02-13-23)} & \bdia{(01-2-3)(03-12-23)}
                    \end{tabular} \\ \hline
                    \begin{tabular}{LL}
                        \bdia{(0-12-3)(01-13-23)} & \bdia{(0-12-3)(02-13-23)}
                    \end{tabular} \\ \hline
                    \begin{tabular}{LLL}
                        \bdia{(01-2-3)(03-13-23)} & \bdia{(0-12-3)(03-13-23)} & \bdia{(01-2-3)(02-12-23)} 
                    \end{tabular}
                \end{tabular}
                &
                \begin{tabular}{c}
                    \begin{tabular}{L}
                        \bdia{(012-3)(03-13-23)}
                    \end{tabular}
                \end{tabular}
            \end{tabular}
            }
            \caption{
                Degree-$3$ scattering diagrams for $B=M=1$ SGD's test loss.
                {\bf Left:} $(d, P) = (3, 3)$.  Diagrams for ODE behavior.
                {\bf Center:} $(d, P) = (3, 2)$.  $1$st order deviation of SGD
                away from ODE.
                {\bf Right:} $(d, P) = (3, 1)$.  $2$nd order deviation of SGD
                from ODE with appearance of non-Gaussian statistics.
            }
            \label{tab:scatthree}
        \end{table}
    
        %A quick combinatorial argument shows:
        \begin{cor}[Shuffling Barely Matters] \label{cor:shuffle}
            To order $\eta^3$, inter-epoch shuffling doesn't affect SGD's
            expected test loss.
        \end{cor}
        Indeed, for any inter-epoch shuffling scheme: 
        \begin{prop}\label{prop:ordtwo}
            To order $\eta^2$, the test loss of SGD --- on $N$
            samples for $M$ epochs with batch size $B$ dividing $N$ and with any
            shuffling scheme --- has expectation
            %{\small
            %\begin{align*}
            %                                            \mdia{(0)()}
            %    &+ MN                                   \mdia{(0-1)(01)}
            %     + MN\wrap{MN - \frac{1}{2}}            \mdia{(0-1-2)(01-12)} \\
            %    &+ MN\wrap{\frac{M}{2}}                 \mdia{c(01-2)(02-12)}  
            %     + MN\wrap{\frac{M-\frac{1}{B}}{2}}     \mdia{c(01-2)(01-12)}
            %\end{align*}
            %}
            %This is:
            {\small
            \begin{align*}
                                                        l              
                &- MN                                   G_\mu G^\mu       
                 + MN\wrap{MN - \frac{1}{2}}            G_\mu H^{\mu}_{\nu} G^\nu \\
                &+ MN\wrap{\frac{M}{2}}                 C_{\mu \nu} H^{\mu \nu}
                 + MN\wrap{\frac{M-\frac{1}{B}}{2}}     \wrap{\nabla_\mu C^{\nu}_{\nu}} G^\mu / 2
            \end{align*}
            }
        \end{prop}
    
        \begin{cor}[The Effect of Epoch Number] \label{cor:epochs}
            To order $\eta^2$, one-epoch SGD has 
            $
                 %\alpha
                 %\sdia{c(01-2)(01-12)}
                 %=
                 \wrap{\frac{M-1}{M}}\wrap{\frac{B+1}{B}}\wrap{\frac{N}{2}}
                 \wrap{\nabla_\mu C^{\nu}_{\nu}} G^\mu / 2
            $
            less test loss than $M$-epoch SGD with learning rate $\eta/M$.
        \end{cor}
    
        Given a smooth unbiased estimator $\hat{C}$ of gradient covariance, we
        may cause GD to mimic SGD:
        \begin{cor}[The Effect of Batch Size] \label{cor:batch}
            The expected test loss of pure SGD is, to order $\eta^2$,
            less than that of pure GD by
            $
                  %\alpha \sdia{c(01-2)(01-12)}
                  %=
                  \frac{M(N-1)}{2} ~
                  \wrap{\nabla_\mu C^{\nu}_{\nu}} G^\mu / 2
            $.
            Moreover, GD on a modified loss 
            $
                \tilde l_n = l_n +
                    \frac{N-1}{4N} ~
                    \hat{C}_\nu^\nu(\theta)
            $
            has an expected test loss that agrees with SGD's to second order.
        \end{cor}
    
        \begin{cor}[An Entropic Force, Unrenormalized] \label{cor:noncons}
            When vanilla SGD is initialized at a test minimum, the weight is
            displaced to order $\eta^3$ by 
            $
                %{T \choose 2} \sdia{c(01-2-3)(02-12-23)} / 2
                %= 
                -\frac{T(T-1)}{4} C_{\mu_\nu} J^{\mu \nu}_{\lambda} G^{\lambda}
            $.
        \end{cor}
        Corollary \ref{cor:noncons} was first proven by \citet{ya19b}; our
        methods reproduce it elegantly.  We will renormalize it to Corollary
        \ref{cor:entropic}, which applies for large times, hence demonstrating
        a persistent and non-conservative entropic force.
    
    %--------------------------------------------------------------------------
    %           Toward Effective Theories                   
    %--------------------------------------------------------------------------

    \subsection{Effective Theories} \label{subsect:effective}
        An important idea is that of \emph{renormalization}, i.e. the
        summarization of myriad small-scale interactions into an effective
        large-scale theory.  We will renormalize --- specifically, we will
        pair main diagrams with topologically related modifications in order
        to refine our $d$th order estimates for any fixed $d$.  For
        some large-$\eta T$ limits in a positive-Hessian setting, the
        unrenormalized theory does not converge while the renormalized theory
        does.  Thus, renormalization will help us reason about large-time
        behavior. 

        As a motivating example, consider that, to order $\eta^1$, SGD reduces
        test loss by $T G_\mu G^\mu$.  Though useful for small $\eta T$, this
        approximation diverges as $T$ grows.  Intuitively, the  
        consider the diagrams that are uncorrelated
        chains:
        $
            \sdia{(0-1)(01)}, \sdia{(0-1-2)(01-12)},
            \sdia{(0-1-2-3)(01-12-23)}, \cdots
        $.  
        When embedded with initial and
        final nodes separated by duration $t$, this series of diagrams sums to
        contributes
        $
            G (I-\eta H)^{t-1} \eta G
            \approx
            G \exp(-\eta T H) \eta G
            \cdot (1+o(\eta))
        $.
        The $\exp(-\eta T H) \eta$ is thus an ``effective learning rate'' to
        be used in place of $\eta$ in renormalized calculations.
        Summed over all embeddings, the above chains contribute 
        $G \sum_{0\leq t<T} (I-\eta H)^{T-t-1} \eta G$
        to vanilla SGD's test loss, or:
        \begin{align*}
            \approx
            G \wrap{\int_t \exp(-\eta (T-t) H)} \eta G
            =
            G \wrap{\frac{I - \exp(-\eta T H)}{H}} G
        \end{align*}
        In contrast to the unrenormalized prediction $T G_\mu G^\mu$,
        the above converges in the large-$T$ limit (when $H$ is positive).

        So the chains form a class of diagrams usefully analyzed
        together.  We may likewise organize other diagrams.  We will
        represent each class by its minimal element:
        \begin{defn}
            A diagram is \emph{irreducible} when it lacks non-root
            un-fuzzily-tied degree-two nodes (such as $\sdia{(0-1-2)(01-12)}$'s
            {\color{moog}green} node).
            So
            $\sdia{(0-1-2)(02-12)}$ and
            $\sdia{(01-2)(01-12)}$
            but neither
            $\sdia{(0-1-2)(01-12)}$ nor
            $\sdia{(02-1-3)(01-12-23)}$
            are irreducible.
        \end{defn}

        \begin{defn}[Embedding-Sensitive Values]
            \label{defn:rvalue}
            Let $\rvalue_f(D)$ be the expected value of $D$'s corresponding
            tensor expression, where instead of using $-\eta$ to contract
            two tensors embedded to times $t, t+\Delta t$, we use
            $
                -(I-\eta H)^{\Delta t - 1} \eta
            $.
            Actually, as we study the effect of noise, it will be most
            convenient to let $\rvalue$s represent a \emph{difference} from the
            noiseless case.  For the small diagrams we consider, we obtain
            $\rvalue$s by replacing fuzzy ties by fuzzy outlines; larger
            diagrams present complications addressed in
            Appendix \ref{subsubsect:mobius}.
        \end{defn}

        \textsc{Renormalized Recipe:}~
            In general, one sums over embeddings of irreducible diagrams, using
            $\rvalue_f(D)$ instead of $\dvalue(D)$.  In practice, we prefer to
            approximate sums over embeddings by integrals over times and
            $(I-\eta H)^t$ by $\exp(- \eta H t)$.  This incurs a term-by-term
            multiplicative error of $1 + o(\eta)$ that preserves leading order
            results.  Diagrams thus induce easily evaluated integrals of
            exponentials.
       
        \begin{thm}[Renormalization Gives Large-$T$ Limits] \label{thm:renorm}
            For any $T$: for $\eta$ sufficiently small, SGD's expected test
            loss exceeds the noiseless case by 
            \begin{equation*} \label{eq:renorm}
                \sum_{\substack{D \in \image(\Free)\\ \text{\color{moor} irreducible}}}
                \wrap{
                    \sum_{f: D\to\Free(S)}
                    \frac{1}{\wabs{\Aut_f(D)}}
                }
                \frac{{\color{moor} \rvalue_f}(D)}{B^{|\edges(D)|}}
            \end{equation*}
            In contrast to Theorem \ref{thm:sgdcoef}: when $H$ is positive, the
            $d$th order truncation converges as $T$ diverges.
        \end{thm}
        That convergence property gives us formal license to take long-term
        limits of renormalized predictions.  In fact, the convergence is
        uniform in $T$ for sufficiently regular landscapes.

        \begin{figure}[h!]
            \centering  
            \plotmooh{diagrams/spacetime-g}{}{0.210\columnwidth}
            \plotmooh{diagrams/spacetime-h}{}{0.210\columnwidth}
            \caption{
                Illustrations of renormalization.
                {\bf Left}:
                    Some embeddings contributing to the single
                    renormalized term $\rvalue_f(\sdia{(0-1)(01)})$ for an
                    example embedding $f$.  Theorem \ref{thm:renorm} shows how
                    to evaluate this total $\rvalue_f$ simply, and it shows
                    that by considering these related diagrams together, the
                    complicated diagrams temper the simpler diagrams to yield
                    better convergence.
                {\bf Right}:
                    A contribution to
                    $\rvalue_f(\sdia{(01-2)(02-12)})$. 
                    The leftmost {\color{mooteal}teal} nodes depict a disturbance by noise
                    (multi-node cells or rows correspond higher moments and
                    hence to noise).  The noise travels through two
                    channels (top, bottom) before interacting with the test
                    Hessian (degree-two black node).  Such embeddings give the
                    small-$\eta$, large-$T$ interaction of noise and curvature. 
                {\bf Both}:
                    In violation of our rule for valid embeddings, the nodes
                    inserted into the irreducible diagrams may occupy the same
                    row despite lacking fuzzy ties.  This
                    overcounting accompanies our $(I-\eta H)^{\Delta t-1}$
                    prescription; it is countered by inclusion-exclusion as
                    described Appendix \ref{subsubsect:mobius}.
            }
            \label{fig:renormspacetime}
        \end{figure}

        \begin{cor}[A Nonconservative Entropic Force]\label{cor:entropic}
            Initialized at a test minimum, vanilla SGD's weight moves to
            order $\eta^2$ with a long-time-averaged\footnote{
                That is, $T$ so large that $C \exp(-\eta K T)$ is negligible.
                Appendix \ref{sect:calculations} gives a similar expression for general $T$.
            }
            expected velocity of
            $$
                v^\pi = C_{\mu \nu}
                \wrap{K^{-1}}^{\mu\nu}_{\rho\lambda}
                J^{\rho\lambda}_{\sigma}
                \wrap{\frac{I - \exp(-T \eta H)}{T \eta H} \eta}^{\sigma \pi}
            $$
            per timestep.
            Here, $K = \eta H \otimes I + I \otimes \eta H$, a
            $4$-valent tensor. 
        \end{cor}
        Unlike prior work \cite{we19b}, we make no assumptions of
        thermal equilibrium, fast-slow mode separation, or constant covariance.
        This added generality allows us to predict a qualitatively new
        dynamical phenomenon, namely that the velocity field above need not be
        conservative.  We verify this in experiments.
        \begin{cor}[Flat and Sharp Minima Overfit Less]\label{cor:overfit}
            Initialized at a test minimum, pure GD's test loss is to
            order $\eta$
            $$
                \frac{1}{2N} ~
                    C_{\mu\nu}
                    \wrap{(I - \exp(-\eta T H))^{\otimes 2}}^{\mu\nu}_{\rho\lambda}
                    \wrap{H^{-1}}^{\rho\lambda}
            $$
            above the minimum.  This vanishes when $H$ does. 
            Likewise, pure GD's generalization gap is to order $\eta$:  
            $$
                \frac{1}{N} ~
                    C_{\mu\nu}
                    \wrap{(I - \exp(-\eta T H))}^{\nu}_{\lambda}
                    \wrap{H^{-1}}^{\lambda\mu}
            $$
            In contrast to the later-mentioned Takeuchi estimate, this does not
            diverge as $H$ shrinks.
        \end{cor}
        Corollary \ref{cor:overfit}'s generalization gap converges after
        large $T$ to $C_{\mu\nu}(H^{-1})^{\mu\nu}/N$, also known as Takeuchi's
        Information Criterion (TIC).  In turn, in the classical setting of
        maximum likelihood (ML) estimation (with no model mis-specification)
        near the ``true'' test minimum, $C=H$ is the Fisher metric, so we
        recover the Akaike Information Criterion (AIC) $(\textnormal{number of
        parameters})/N$.  Unlike AIC, our more general expression is
        descendably smooth, may be used with MAP or ELBO tasks instead of just
        ML, and makes no model well-specification assumptions.

        \begin{figure}[h!]
            \centering
            \plotmooh{diagrams/entropic-force-diagram}{}{0.35\columnwidth} 
            \plotmooh{diagrams/springs}{}{0.35\columnwidth}
            \caption{
                {\bf Left}:
                    The entropic force mechanism: gradient noise induces a flow
                    toward minima flat \emph{with respect to to the
                    covariance}.  Although our analysis assumes neither thermal
                    equilibrium nor fast-slow mode separation, we label ``fast
                    and slow directions'' in this cartoon to ease comparison
                    with \citet{we19b}.  In this cartoon, red densities denote
                    the spread predicted by a renormalized $C^{\mu\nu}$, and
                    the spatial variation of curvature corresponds to
                    $J_{\mu\nu\lambda}$. 
                {\bf Right}:
                    Noise structure determines how curvature affects
                    overfitting.  To use a physical metaphor, for a fixed
                    displacement scale, stiffer springs incur greater energy
                    gains.  But for a fixed force scale, limper springs incur
                    greater energy gains.  Geometrically, for (empirical risk
                    minimization on) a vector-perturbed landscape, small
                    Hessians are favored (top row), while for
                    covector-perturbed, landscape large Hessians are favored
                    (bottom row).  Corollary \ref{cor:overfit} shows how the
                    implicit regularization of fixed-$\eta T$ descent mediates
                    between the two rows.
            }
            \label{fig:entropic}
        \end{figure}

%==============================================================================
%    EXPERIMENTS AND APPLICATIONS
%==============================================================================

\section{Experiments and Applications}

    %--------------------------------------------------------------------------
    %           Vanilla SGD                                 
    %--------------------------------------------------------------------------

    \subsection{Vanilla SGD}
        {\color{moor} FILL IN}
        We test Proposition \ref{prop:vanilla} for smooth convolutional
        architectures for CIFAR-10 and Fashion-MNIST by comparing measured 
        test losses and generalization gaps with un-renormalized predictions
        (see Appendix \ref{sect:landscape}).  We find good 
        agreement between our order $\eta^3$ perturbative predictions up to
        $\eta T \approx 10^0$.  This test verifies that our application of the
        diagram method hides no mistakes of proportionality of sign.
        See Appendix \ref{sect:figures} for figures.

    %--------------------------------------------------------------------------
    %           Epochs and Overfitting                      
    %--------------------------------------------------------------------------
 
    %\subsection{Epochs and Overfitting}
        %{\color{moor} FILL IN}
        Likewise, Figure \ref{fig:bmmulti} shows that our predictions of how 
        epochs affect overfitting are in good agreement with experiment. 
        %\lorem{3}
    
    %--------------------------------------------------------------------------
    %           Emulating Small Batches with Large Ones     
    %--------------------------------------------------------------------------

    \subsection{Emulating Small Batches with Large Ones}
        Figure \ref{fig:bmmulti} shows that the regularizer proposed in
        Proposition \ref{cor:batch} indeed enables GD to emulate SGD
        on a range of image-classification landscapes.  For these experiments,
        we used a covariance
        estimator $\hat C \propto \nabla l_x (\nabla l_x - \nabla l_x)$ evaluated on
        two batches $x, y$ that partition the train set.
        \begin{figure}[h!] 
            \centering
            \plotmoo{plots/big-bm-new}{0.48\columnwidth}{4.0cm}
            \plotmoo{plots/multi-fashion-logistic-0}{0.48\columnwidth}{4.0cm}
            \caption{
                {\bf Left}: with equal-scaled axes, this plot shows that GDC
                matches SGD (small vertical variation) better than GD matches
                SGD (large horizontal variation) in test loss, for a variety of
                learning rates (from 0.0025 to 0.1) and initializations (zero
                and multiple independent Xavier-Glorot trials) on several of
                image classification landscapes (logistic and convolutional
                CIFAR-10 and Fashion-MNIST).  $T=10$ throughout.
                {\bf Right}: SGD with $2, 3, 5, 8$ epochs incurs greater test
                loss than one-epoch SGD (difference shown in I bars) by the
                predicted amounts (predictions shaded) for a range of learning
                rates.
            }
            \label{fig:bmmulti}
        \end{figure}
  
    %--------------------------------------------------------------------------
    %           Comparison to Continuous Time               
    %--------------------------------------------------------------------------

    \subsection{Comparison to Continuous Time}
        Consider fitting a centered normal $\Nn(0, \sigma^2)$ to some centered
        standard normal data.  We parameterize the landscape by
        $h=\log(\sigma^2)$ so that the Fisher information matches the standard
        dot product \citet{am98}.  The gradient at sample $x$ and weight
        $\sigma$ is then $g_x(h) = (1-x^2\exp(-h))/2$.  Since $x\sim \Nn(0,
        1)$, $g_x(h)$ will be affinely related to a chi-squared, and in
        particular non-gaussian.  At $h=0$, the expected gradient vanishes, and
        the test loss of vanilla SGD only involves diagrams with no singleton
        leaves; to third order, it is
        $
            \sdia{(0)()}
            +\frac{T}{2} \sdia{c(01-2)(02-12)}
            +{T\choose 2} \sdia{c(03-1-2)(01-12-23)}
            +\frac{T}{6} \sdia{c(012-3)(03-13-23)}
        $
        In particular, the ${T\choose 2}$ differs from $T^2/2$ and hence
        contributes to the time-discretization error of SDE as an approximation
        for SDE.  Moreover, non-gaussian noise contributes via
        $\sdia{c(012-3)(03-13-23)}$ to that error.  Figure \ref{fig:thermo}
        shows that, indeed, SDE and one-epoch SGD differ.  For multi-epoch
        SGD, the effect of overfitting to finite training data further
        separates SDE and SGD.

    %--------------------------------------------------------------------------
    %           Thermodynamic Engine                        
    %--------------------------------------------------------------------------

    \subsection{A Nonconservative Entropic Force} \label{subsect:entropic}
        To test Corollary \ref{cor:entropic}'s predicted force, 
        we construct a counter-intuitive loss landscape wherein, for
        arbitrarily small learning rates, SGD steadily increases the weight's
        z component despite 0 test gradient in that direction.
        Our mechanism differs from that discovered by \citet{ch18}.
        Specifically, because in this landscape the force is
        $\eta$-perpendicular to the image of $\eta C$, that work predicts an
        entropic force of $0$.  This disagreement in predictions is possible
        because our analysis does not make any assumptions of equilibrium,
        conservatism, or continuous time.
        
        Intuitively, the presence of the term
        $
            \sdia{c(01-2-3)(02-12-23)}
        $
        in our test loss expansion indicates that 
        \emph{SGD descends on a covariance-smoothed landscape}.
        So, even in a valley of global minima, SGD will move away from minima
        whose Hessian aligns with the current covariance.  However, by the time
        it moves, the new covariance might differ from the old one, and SGD will
        be repelled by different Hessians than before.  By setting the
        covariance to lag the Hessian by a rotational phase, we construct
        a landscape in which this entropic force dominates.
        This ``\emph{linear-screw}'' landscape is defined for
        three-dimensional $w\in \RR^3$ (initialized at the origin) and
        one-dimensional $x \sim \Nn(0, 1)$:
        $$
            l_x(w)
            \triangleq
            \frac{1}{2} H(z)(w, w) + x \cdot S(z)(w)  
        $$
        Here, $H(z)(w, w) = w_x^2 + w_y^2 + (\cos(z) w_x + \sin(z) w_y)^2$
        and   $S(z)(w)    = \cos(z-\pi/4) w_x + \sin(z-\pi/4) w_y$.
        We see that there is a valley of global minima defined by $x=y=0$. 
        If SGD is initialized there, then to leading order in $\eta$ and for
        large $T$, the renormalized theory predicts a $z$-speed of $\eta^2/6$ 
        per timestep.  This prediction, unlike the
        un-renormalized prediction, agrees for large $\eta T$
        with experiment (Figure \ref{fig:thermo}).

        \begin{figure}[h!]
            \centering
            \plotmoo{plots/vs-sde}{0.48\columnwidth}{4.0cm}
            \plotmoo{plots/thermo-linear-screw}{0.48\columnwidth}{4.0cm}
            \caption{
                {\bf Left}: SGD's difference from SDE after $\eta T \approx
                10^{-1}$ with maximal coarseness on the gaussian-fit problem.  
                Two effects not modeled by SDE --- time-discretization and
                non-gaussian noise oppose on this landscape but do not
                completely cancel. 
                {\bf Right}: On the linear screw landscape, the persistent
                entropic force pushes the weight through a valley of global
                minima not at a $T^{1/2}$ diffusive rate but at a directional
                $T^1$ rate.  Hessians and covariances  are uniformly bounded
                throughout the valley, and this effect appears at all
                sufficiently small $\eta$s with strength $\eta^2$.  Thus, the
                effect is not a pathological artifact of well-chosen learning
                rate or divergent covariance noise.  Observe that the net
                displacement of $\approx 10^{1.5}$ well exceeds the $z$-period
                of $2\pi$. 
            }
            \label{fig:thermo}
        \end{figure}

        By stitching together copies of this example, we may cause SGD to
        travel along paths that are closed loops or unbounded curves, and we
        may even add a small linear component so that SGD steadily climbs
        uphill.  


    %--------------------------------------------------------------------------
    %           Sharp vs Flat Minima                        
    %--------------------------------------------------------------------------

    \subsection{Sharp and Flat Minima Both Overfit Less} \label{subsect:overfit}
        Prior work has varyingly found that \emph{sharp} minima overfit less
        (after all, $l^2$ regularization increases curvature) or that
        \emph{flat} minima overfit less (after all, flat minima are more
        robust to small displacements in weight space).  Corollary
        \ref{cor:overfit} reconciles these competing intuitions by showing
        how the relationship of generalization and curvature depends on the
        learning task's noise structure and how the metric $\eta^{-1}$ mediates
        this distinction.
        
        \begin{figure}[h!] 
            \centering
            \plotmoo{plots/tak}{0.48\columnwidth}{4.0cm}
            \plotmoo{plots/tak-reg}{0.48\columnwidth}{4.0cm}
            \caption{
                {\bf Left}: For artificial quadratic landscapes with fixed
                covariance and a range of Hessians, initialized at the true
                minimum, the test losses after fixed-$\eta T$ optimization are
                smallest for very small and very large curvatures.  This
                evidences our renormalized theory's prediction that both sharp
                and flat minima overfit less!  In particular, the Takeuchi
                estimate's singularity is, as predicted, suppressed.
                {\bf Right}: For artificial quadratic landscapes with fixed
                covariance and a range of Hessians, initialized a fixed
                distance \emph{away} from the true minimum, joint descent on 
                an $l_2$ penalty coefficient $\lambda$ by means of STIC
                improves on plain gradient descent for most Hessians.  That
                there is at all a discrepancy from theory is possible because
                $\lambda$ is not perfectly tuned according to STIC but instead
                descended on for finite $\eta T$.
            }
            \label{fig:tak}
        \end{figure}

        Because the TIC estimates a smooth hypothesis class's generalization
        gap, it is tempting to use it as an additive regularization term.
        However, the TIC is singular where the Hessian is (see Figure
        \ref{fig:entropic}), and as such gives insensible results for
        over-parameterized models.  Indeed, a prior attempt ran into numerical
        difficulties requiring an arbitrary cutoff \citep{di18}. 

        Fortunately, by Corollary \ref{cor:overfit}, the implicit
        regularization of gradient descent both demands and enables a
        singularity-removing correction to the TIC --- see Figure
        \ref{fig:tak}.  The resulting \emph{Stabilized TIC} (STIC) uses the
        metric implicit in gradient descent to threshold flat from sharp
        minima.  It thus offers a principled method for optimizer-aware model
        selection easily compatible with automatic differentiation systems.  By
        descending on STIC, we may tune smooth hyperparameters such as $l_2$
        coefficients.  Experiments on the mean-estimation problem validate STIC
        for such model selection, especially when $C/N$ dwarves $H$ as in the
        noisy, small-$N$ regime (see Figure \ref{fig:tak}).  Because matrix
        exponentiation is expensive, STIC regularization without further
        approximations is most useful for models of small dimension
        on very noisy or scanty data.

%==============================================================================
%    RELATED WORK    
%==============================================================================

\section{Related Work} \label{sect:related}

    %--------------------------------------------------------------------------
    %           History of SGD
    %--------------------------------------------------------------------------

    It was \citet{ki52} who, in uniting gradient descent \citep{ca47} with
    stochastic approximation \citep{ro51}, invented SGD.  Since the development
    of back-propagation for efficient differentiation \citep{we74}, SGD
    has been used to train connectionist models including neural
    networks \citep{bo91}, in recent years to remarkable success \citep{le15}.

    %--------------------------------------------------------------------------
    %           Analyzing Overfitting; Relevance of Optimization; SDE Errs  
    %--------------------------------------------------------------------------

    Several lines of work quantify the overfitting of SGD-trained networks
    \citep{ne17a}.  For instance, \citet{ba17} controls the Rademacher
    complexity of deep hypothesis classes, leading to generalization bounds
    that are optimizer-agnostic.  However, since SGD-trained networks
    generalize despite their seeming ability to shatter large sets
    \citep{zh17}, one infers that generalization arises from the aptness to
    data of not only architecture but also optimization \citep{ne17b}.  Others
    have focused on the implicit regularization of SGD itself, for instance by
    modeling descent via stochastic differential equations (SDEs) (e.g.
    \citet{ch18}).  However, per \citet{ya19a}, such continuous-time analyses
    cannot treat covariance correctly, and so they err when interpreting
    results about SDEs as results about SGD for finite trainsets.

    %--------------------------------------------------------------------------
    %           We Extend Dan's Approach                     
    %--------------------------------------------------------------------------

    Following
    %\citet{li17} and
    \citet{ro18}, we avoid making a continuous-time
    approximation by instead Taylor-expanding around the learning rate
    $\eta=0$.  In fact, we develop a diagrammatic method for evaluating each
    Taylor term that is inspired by the field theory methods popularized by
    \citet{dy49a}.  We then quantify the overfitting effects
    of batch size and epoch number, and based on this analysis, propose a
    regularizing term that causes large-batch GD to emulate small-batch SGD,
    thus establishing a precise version of the relationship --- between
    covariance, batch size, and generalization --- conjectured by \citet{ja18}.  
    
    %--------------------------------------------------------------------------
    %           Phenomenology of Rademacher Correlates such as Hessians
    %--------------------------------------------------------------------------

    While we make rigorous, architecture-agnostic predictions of learning
    curves, these predictions become vacuous for large $\eta$. 
    Other discrete-time dynamical analyses allow large $\eta$ by treating deep
    generalization phenomenologically, whether by fitting to an
    empirically-determined correlate of Rademacher bounds \citep{li18}, by
    exhibiting generalization of local minima \emph{flat} with respect to the
    standard metric (see \citet{ho17}, \citet{ke17}, \citet{wa18}), or by
    exhibiting generalization of local minima \emph{sharp} with respect to the
    standard metric (see \citet{st56}, \citet{di17}, \citet{wu18}).  Our work,
    which shows how generalization depends on the underlying
    metric and on the form of gradient noise, reconciles those
    seemingly clashing claims.
    
    %--------------------------------------------------------------------------
    %           Our Work vs Other Perturbative Approaches            
    %--------------------------------------------------------------------------

    Others have imported the perturbative methods of physics to analyze descent
    dynamics:  \citet{dy19} perturb in inverse network width, employing 't
    Hooft diagrams to compute deviations of a specific class of deep
    architectures from Gaussian processes.  Meanwhile, \cite{ch18} and
    \citet{li17} perturb in learning rate to second order by approximating
    noise between updates as gaussian and uncorrelated.  This approach does not
    generalize to higher orders, and, because correlations and heavy tails are
    essential obstacles to concentration of measure and hence to
    generalization, it does not model the generalization behavior of SGD.  By
    contrast, we use Penrose diagrams to compute test and train losses to
    arbitrary order in learning rate, quantifying the effect of non-gaussian
    and correlated noise.  Our method accounts for optimization and applies to
    \emph{any} sufficiently smooth architecture.  For example, since our work
    does not rely on information-geometric relationships between $C$ and $H$
    \citep{am98}, it applies to inexact-likelihood landscapes such as VAEs'.
    We hence extend \citet{ro18} beyond leading order and beyond $2$ time
    steps, allowing us to compare, for instance, the expected test losses of
    multi-epoch and one-epoch SGD.

%==============================================================================
%    CONCLUSION      
%==============================================================================

\section{Conclusion} \label{sect:concl}

    %--------------------------------------------------------------------------
    %           Summarize Contributions                     
    %--------------------------------------------------------------------------

    We present an elegant diagram-based framework for understanding small-$\eta
    T$ SGD.
        In fact, our Renormalization Theorem licences
    predictions for large $\eta T$, beyond the reach of direct perturbation.
    The theory makes novel predictions:

        \textsc{Story of $\sdia{c(01-2)(02-12)}$:~}
        Flat and sharp minima both overfit less than minima of curvature
        comparable to $(\eta T)^{-1}$.  Flat minima are robust to
        vector-valued noise, sharp minima are robust to
        covector-valued noise, and medium minima attain the worst of both
        worlds.  We thus reconcile prior intuitions that sharp \citep{ ke17,
        wa18} or flat \citep{di17, wu18} minima overfit worse.  These
        considerations lead us to a smooth generalization of AIC and to tune
        hyperparameters by gradient descent.

        \textsc{Story of $\sdia{c(01-2-3)(02-12-23)}$:~}
        Refining \citet{we19b} to nonconstant, nonisotropic covariance, we find
        that SGD descends on a loss landscape smoothed by the \emph{current}
        covariance.  As the covariance evolves, the smoothing mask and thus the
        effective landscape evolves.  This dynamics is generically
        nonconservative.  In contrast to \citet{ch18}'s SDE approximation,
        SGD does not generically converge to a limit cycle. 

        \textsc{Story of $\sdia{c(01-2)(01-12)}$:~}
        As conjectured by \citet{ro18}, large-batch GD can be made to emulate
        small-batch SGD.  We show how to do this by adding a multiple of an
        unbiased covariance estimator to the descent objective.  This emulation
        is significant because, while small batch sizes can lead to better
        generalization \citep{bo91}, modern infrastructure increasingly rewards
        large batch sizes \citep{go18}.  

    %--------------------------------------------------------------------------
    %           Ask Questions                               
    %--------------------------------------------------------------------------

    The diagram method invites further exploration of Lagrangian
    characterizations and curved backgrounds\footnote{
        Landau and Lifshitz introduce these concepts with grace and brevity
        \yrcite{la60, la51}.
    }:
    \begin{quest}
        Does some least-action principle govern SGD; if not,
        what is an essential obstacle to this characterization?
    \end{quest}
        Lagrange's least-action formalism intimately intertwines with the
        diagrams of physics.  Together, they afford a modular framework for
        introducing new interactions as new terms or diagram nodes.  In fact,
        we find that some \emph{higher-order} methods --- such as the
        Hessian-based update
        $
            \theta \leftsquigarrow
            \theta -
            (\eta^{-1} + \lambda \nabla \nabla l_t(\theta))^{-1}
            \nabla l_t(\theta)
        $
        parameterized by small $\eta, \lambda$ --- admit diagrammatic analysis
        when we represent the $\lambda$ term as a second type of
        diagram node.  Though diagrams suffice for computations, it is
        Lagrangians that most deeply illuminate scaling and conservation laws.
    \begin{conj}
        To leading order in $\eta$, the generalization gap of SGD  
        increases with sectional curvature.
    \end{conj}
        Though our work so far assumes a flat metric $\eta^{\mu\nu}$, it 
        generalizes to curved weight spaces\footnote{
            One may represent the affine connection as a node, thus giving
            rise to non-tensorial and hence gauge-dependent diagrams.
        }.
        Curvature finds concrete application in the \emph{learning on
        manifolds} paradigm of \citet{bo13}, notably specialized to
        \citet{am98}'s \emph{natural gradient descent} and \citet{ni17}'s
        \emph{hyperbolic embeddings}.  We are optimistic our formalism may
        resolve conjectures such as above.

%==============================================================================
%    ACKNOWLEDGEMENTS
%==============================================================================

{\colorbox{moopink}{acknowledgements commented}}
%\subsection{Acknowledgements}
%    We feel deeply grateful to
%        Sho Yaida,
%        Dan A. Roberts, and
%        Josh Tenenbaum
%    for posing several of the questions we here resolved and for their
%    compassionate and patient guidance.  We appreciate the generosity of
%        Andrzej Banburski,
%        Ben R. Bray,
%        Sasha Rakhlin,
%        Greg Wornell, and
%        Wenli Zhao
%    in offering feedback on earlier stages of writing.  Without the
%    encouragement of
%        Jason Corso,
%        Chloe Kleinfeldt,
%        Alex Lew, 
%        Ari Morcos, and
%        David Schwab,
%    this paper would not be.

%==============================================================================
%    REFERENCES      
%==============================================================================

%\section*{References}
    \bibliography{perturb}
    \bibliographystyle{icml2019}

%==============================================================================
%    APPENDICES      
%==============================================================================

\renewcommand{\thesection}{\Alph{section}}
\setcounter{section}{0}

\section*{Invitation to Appendices}
    The first three appendices deal with \textsc{Calculation} and contain the
    proofs --- some heavily annotated for pedagogical purposes --- of all the
    corollaries in the text.
    %
    \textsc{Appendix} \ref{sect:tutorial} guides the reader through a hands-on
        tutorial of the diagram method, including renormalization.  It
        concludes with a simple project for the reader, the solution of which
        is a novel result easily obtained by the diagram method but not present
        in the literature. 
    \textsc{Appendix} \ref{sect:compare} demonstrates how and why the diagram
        method excels over direct application of the Key Lemma, even if we
        desire only unrenormalized results.
    \textsc{Appendix} \ref{sect:calculations} provides the remaining
        computations promised.

    The next two appendices offer the foundational \textsc{Proofs} that
    license the computations and corollaries treated in the preceding
    appendices.
    %
    \textsc{Appendix} \ref{sect:morebackground} sets up the objects of study
        with full mathematical precision.
    \textsc{Appendix} \ref{sect:proofs}, after cataloging the regularity
        conditions required of loss landscapes (and implicitly assumed as a
        hypothesis of all results herein), proves Theorems \ref{thm:sgdcoef}
        and \ref{thm:renorm}, along the way proving the Key Lemma.

    The next three appendices further describe \textsc{Experiments}, methods,
    and results.
    %
    \textsc{Appendix} \ref{sect:bessel} explains how we obtained loss landscape
        statistics such as expected entropic force for real datasets.
    \textsc{Appendix} \ref{sect:landscape} details our training, testing, and
        architecture for the datasets (e.g. CIFAR-10) that we used in
        experiments.
    \textsc{Appendix} \ref{sect:figures} exhibits additional plots that could
        (but fail to) falsify our theory.

    The final appendix is for convenient \textsc{Reference}.
    %
    \textsc{Appendix} \ref{sect:glossary} glosses physical and mathematical
        terminology and tabulates the values and interpretations for some
        common diagrams.

\section{Tutorial on Diagram Rules} \label{sect:tutorial}

    After reviewing the diagram method's recipe step by step, we will work
    through three sample problems.  As in the main text, we
    relegate precise combinatorial definitions to Section
    \ref{sect:morebackground}, prefering in this section to appeal to intuition
    and to hands-on examples.

    Recall the computational recipe.  To compute a quantity of interest, list
    all the relevant diagrams, then evaluate each diagram over all of its
    embeddings into spacetime.  After reviewing the intuitive interpretation
    and combinatorial character of diagrams, we illustrate each step of this
    process.  We finish with example computations and an easy project.

    \subsection{Anatomy of a Diagram}
        \subsubsection*{The Data of a Diagram}
            A diagram has thin edges and fuzzy ties.  The thin edges must form
            a rooted tree.  When drawing diagrams on a pager, our convention is
            to record the root by drawing it to the right of all other nodes.
            The fuzzy edges are just a mechanism to represent a partition of
            the diagram's nodes.  For a diagram with $d+1$ nodes and $p$ parts,
            one can specify the parts using $d+1-p$ many fuzzy ties and no
            fewer.  Intuitively: thin edges represent differentiation and fuzzy
            ties represent noise.

            The following two diagrams are equivalent, not only in that they
            evaluate to the same numbers but in that their combinatorial data
            is the same: $\sdia{(0-12-3)(03-12-23)}$,
            $\sdia{(02-1-3)(02-13-23)}$.  Only a redundancy of our ink
            description makes them look different.  The following two diagrams
            are inequivalent but evaluate to the same numbers in an
            un-renormalized setting: $\sdia{(01-2-3)(03-13-23)}$,
            $\sdia{(01-2-3)(02-12-23)}$.  We recognize them as inequivalent
            when we observe that their roots (here {\color{moob}blue}) play
            different roles.  The diagrams thus depict different data-weight
            processes.  In a renormalized setting, they will evaluate to
            different numbers.

        \subsubsection*{The Interpretation of a Diagram}
            {\color{moor} FILL IN}

        \subsubsection*{Example: The Diagram $\sdia{(01-2-3)(02-12-23)}$}
            {\color{moor} FILL IN}

        \subsubsection*{Example: The Diagram $\sdia{c(01)(01)}$}
            {\color{moor} FILL IN}


    \subsection{Listing Relevant Diagrams}

        \subsubsection*{Review of Rules}
            A diagram with $d$ thin edges evaluates to an order-$\eta^d$
            expression.  Thus, if we wish the order $\eta^d$ contribution to a
            quantity, we know immediately that we will only need to consider
            diagrams with $d$ thin edges.

            Depending on what we want to compute, we can rule out more
            diagrams.  The type of quantity (test loss / test loss minus train
            loss), the optimization schedule (one-epoch / effect of batch
            size comparison between two optimizers), and special posited
            structure (initialize at minimum / no noise / gaussian noise,
            quadratic loss) can all rule out diagrams.

            Indeed, Theorem \ref{thm:sgdcoef} says that the \emph{test loss}
            involves only diagrams whose root (a.k.a. rightmost node)
            participates in no fuzzy ties.  And it says that the
            \emph{generalization gap} (ie. test minus train loss) involves only
            diagrams whose root participates in a fuzzy outline with one other
            node.
            \begin{rmk*}[Role of Fuzzy Outlines]
                As originally defined, diagrams have thin edges and fuzzy ties
                but not fuzzy outlines.  But because taking train losses minus
                test losses is such a common operation, we have introduced the
                fuzzy outline notation to mean a difference of diagrams, one
                diagram with the fuzzy outline replaced by a fuzzy tie, and the
                other diagram with the fuzzy outline deleted.  For example,
                $\sdia{c(0-12)(01-12)}$ is shorthand for $\sdia{(0-12)(01-12)}
                - \sdia{(0-1-2)(01-12)}$, which is a (summand in the) train
                loss minus (a corresponding summand in the) test loss.    
            \end{rmk*}

            Moreover, because diagrams must ultimately embed into spacetime in
            order to contribute, we might rule out ill-fitting diagrams if
            we know about our specific spacetime.  As a simplest example, 
            one-epoch SGD has only one spacetime cell per row and thus does
            not permit diagrams such as $\sdia{(01-2)(01-12)}$ to be embedded
            because such embeddings would have intra-cell edges.  Thus, for
            one-epoch SGD, we may ignore diagrams whose thin-edge trees have
            a fuzzily tied ancestor-descendant pair.
            As another example of this type of simplification: the spacetimes
            of any two $(N,B,M)$-style SGD runs, with same $N, M$ for fair
            comparison but different $B$, correspond in many-to-one fashion so
            that no diagram embedded with each node in a different epoch
            contributes to the difference between GD and SGD.

            Finally, knowledge of loss landscape structure can further rule
            out diagrams.  As a simplest example, it is often interesting to
            initialize at a test minimum to then measure subsequent
            overfitting.  At a test minimum, the expected gradient $G$
            vanishes, and hence every diagram with a factor of $G$ vanishes.
            These are the diagrams with a leaf node that is not fuzzily tied. 
            So, at a test minimum, we may ignore diagrams such as
            $\sdia{(0-12)(02-12)}$
            and
            $\sdia{(01-2-3)(02-12-23)}$.
            As a further example of this type of simplification: in the
            noiseless case where covariances are $0$, we may ignore every
            diagram with size-$2$ fuzzy-outline-cliques, e.g.
            $\sdia{(0-12-3)(01-12-23)}$.

        \subsubsection*{Examples of Listing Relevant Diagrams}
            \begin{quest*}
                Which diagrams contribute to general SGD's generalization gap
                (i.e. test minus train loss) at order $\eta^1$?
            \end{quest*}
            \begin{answ*}
                The $d$-edged diagrams contribute to order $eta^d$, so we wish
                to enumerate the diagrams with one edge.  Per Theorem
                \ref{thm:sgdcoef}, the generalization loss is a sum over
                diagrams whose root (a.k.a.  rightmost node) participates in
                one fuzzy outline.  So we seek $1$-edged diagrams whose
                rightmost node is fuzzily outlinedly tied to one other node.
                It turns out there is just one such diagram:
                $\sdia{c(01)(01)}$.  Indeed, there is only one possible rooted
                tree of thin edges.  And since there are only two nodes, the
                rightmost node's fuzzy-outline partner is determined.
            \end{answ*}
            \begin{rmk*}[Fuzzy Outlines as Shorthand]
                $\sdia{c(01)(01)}$ is shorthand for $\sdia{(01)(01)} -
                \sdia{(0-1)(01)}$, which is a (summand in the) train loss minus
                (a corresponding summand in the) test loss.    
            \end{rmk*}

            \begin{quest*}
                Which diagrams contribute to general SGD's test loss at order
                $\eta^2$? 
            \end{quest*}
            \begin{answ*}
                The $d$-edged diagrams contribute to order $eta^d$, so we wish
                to enumerate the diagrams with two edges.  Per Theorem
                \ref{thm:sgdcoef}, the test loss is a sum over diagrams whose
                root (a.k.a. rightmost node) participates in no fuzzy ties.  So
                we seek $2$-edged diagrams whose rightmost node isn't fuzzily
                tied.  It turns out there are four such diagrams:
                $\sdia{(0-1-2)(01-12)}$,
                $\sdia{(01-2)(01-12)}$,
                $\sdia{(0-1-2)(02-12)}$, and
                $\sdia{(01-2)(02-12)}$.
                Indeed, two permissible rooted trees multiply with two
                permissible fuzzy tie patterns to yield four diagrams.
            \end{answ*}

            \begin{quest*}
                Which diagrams contribute to one-epoch SGD's test
                loss at order $\eta^2$? 
            \end{quest*}
            \begin{answ*}
                We seek $2$-edged diagrams whose rightmost node isn't fuzzily
                tied.  As we saw before, there are four such diagrams.
                However, consideration of one-epoch, singleton-batch SGD's
                spacetime shows us that no node in the thin-edge tree may be
                fuzzily tied to any of its descendents.  Thus, of the four
                possible diagrams (namely, $\sdia{(01-2)(01-12)}$), one has no
                embeddings, hence has coefficient zero, hence may be ignored.
                Explictly, the remaining diagrams are:
                $\sdia{(0-1-2)(01-12)}$,
                $\sdia{(01-2)(01-12)}$, and
                $\sdia{(01-2)(02-12)}$.
            \end{answ*}

            \begin{quest*}
                Which diagrams contribute to one-epoch SGD's
                generalization gap at order $\eta^2$? 
            \end{quest*}
            \begin{answ*}
                We seek $2$-edged diagrams (order $\eta^2$) whose rightmost
                node has a fuzzy outline connecting to one other node
                (generalization gap).  We ignore trees whose
                fuzzy-outline-erased versions involve fuzzy ties between any
                pair of ancestor and descendant (one-epoch).  To be explicit,
                we write out the fuzzy outlines as the differences they
                represent: 
                $\sdia{(0-12)(01-12)}-\sdia{(0-1-2)(01-12)}$,
                $\sdia{(0-12)(02-12)}-\sdia{(0-1-2)(02-12)}$, and
                $\sdia{(012)(02-12)} -\sdia{(01-2)(02-12)} $.
            \end{answ*}

    \subsection{Summing a Diagram's Spacetime Embeddings}
        \subsubsection*{Unrenormalized Technique} 
            The unrenormalized technique of Theorem \ref{thm:sgdcoef} is
            simpler than the renormalized technique of Theorem
            \ref{thm:renorm}, but it only has guarantees for small $\eta T$.
        \subsubsection*{Renormalized Technique} 
            Though the unrenormalized technique of Theorem \ref{thm:sgdcoef} is
            simple, the renormalized technique of Theorem \ref{thm:renorm}
            gives convergent results even for large $\eta T$.
        \subsubsection*{Examples of Summing a Diagram's Spacetime Embeddings}

    \subsection{
        The Generalization Gap of SGD.  Proof of Corollary \ref{cor:overfit}. 
    }
        The relevant irreducible diagram is $\sdia{c(01)(01)}$.   
        {\color{moor} FILL IN}
 
    \subsection{
        The 3rd Order Curl: Which Minima Does SGD Prefer?
        Proof of Corollaries \ref{cor:noncons} and \ref{cor:entropic}. 
    }
        The relevant irreducible diagram is $\sdia{c(01-2-3)(02-12-23)}$.   
        {\color{moor} FILL IN}

    \subsection{
        SGD vs ODE.
        Proof of Part of Corollary \ref{cor:vsode}
    }
        The rest of the proof is in Appendix \ref{sect:compare}.
        {\color{moor} FILL IN}

    \subsection{
        The Effect of Batch Size and Epochs.
        When Does SGD Outperform GD?
        Proof of Proposition \ref{prop:ordtwo} and Corollary \ref{cor:epochs}. 
    }
        {\color{moor} FILL IN}

    \subsection{
        Project: Nongaussian Noise for Large Times.
    }
        %We saved a result just for you, the reader!  This result is absent from
        %this paper and the literature.  It might be \emph{you} who first writes
        %it out fully.
        One may obtain a novel and unpublished result by answering the
        following questions.

        \begin{rmk*}[Gaussian Third Moments]
            Recall that gaussian processes fit arbitrary first and second
            moments, but that their third moments are determined by 
            $$
                \wang{x^3} = 3\wang{x^2}\wang{x} - 2\wang{x}^3
            $$
        \end{rmk*}
        \begin{quest*}
            Which diagrams that contribute to one-epoch SGD's test loss at
            order $\eta^3$ detect non-gaussian noise?
        \end{quest*}

        \begin{quest*}
            What is the renormalized contribution of $\sdia{(012-3)(03-13-23)}$
            as embedded in the spacetime of one-epoch, singleton-batch SGD?
        \end{quest*}
       
        \begin{quest*}
            What is the leading order contribution of nongaussian noise to 
            vanilla SGD's test loss, for large $\eta T$?
        \end{quest*}
        %\begin{rmk*}[upon finishing]
        %    Congratulations!
        %\end{rmk*}

      
\section{
    {\colorbox{moolime}{Diagram Rules}} vs {\colorbox{moosky}{Direct Perturbation}}
} \label{sect:compare}
    Diagram methods from Stueckelberg to Peierls have flourished in physics
    because they enable swift computations and offer immediate intuition that
    would otherwise require laborious algebraic manipulation.  We demonstrate
    how our diagram formalism likewise streamlines analysis of descent by
    comparing direct perturbation\footnote{
        By ``direct perturbation'', we mean direct application of our Key
        Lemma.
    }
    to the new formalism on three sample problems.

    Aiming for a conservative comparison of derivation ergonomics, we lean
    toward explicit routine when using diagrams and allow ourselves to use
    clever and lucky simplifications when doing direct perturbation.  For
    example, while solving the first sample problem by direct perturbation,
    we structure the SGD and GD computations so that the coefficients (that in
    both the SGD and GD cases are) called $a(T)$ manifestly agree in their
    first and second moments.  This allows us to save some lines of argument.

    Despite these efforts, the diagram method yields arguments about \emph{four
    times shorter} --- and strikingly more conceptual --- than direct
    perturbation yields.  These examples specifically suggest that: diagrams
    obviate the need for meticulous index-tracking, from the start focus one's
    attention on non-cancelling terms by making visually obvious which terms
    will eventually cancel, and allow immediate exploitation of a setting's
    special posited structure, for instance that we are initialized at a test
    minimum or that the batch size is $1$.  We regard these examples as
    evidence that diagrams offer a practical tool for the theorist.

    We make no attempt to compare the renormalized version of our formalism
    to direct perturbation because the algebraic manipulations involved for
    the latter are too complicated to carry out.  

    \subsection{
        Effect of Batch Size.
        Proof of Corollary \ref{cor:batch}.
    }
        We compare the test losses of pure SGD and pure GD.  Because pure
        SGD and pure GD differ in how samples are correlated, their test loss
        difference involves a covariance and hence occurs at order $\eta^2$.  

        \subsubsection*{Diagram Method}
        \colorlet{shadecolor}{moolime}
        \begin{shaded}
            Since SGD and GD agree on noiseless landscapes, we consider only
            diagrams with fuzzy ties.  Since we are working to second order, we
            consider only two-edged diagrams.  There are only two such
            diagrams, $\sdia{(01-2)(02-12)}$ and $\sdia{(01-2)(01-12)}$.  The
            first diagram, $\sdia{(01-2)(02-12)}$, embeds in GD's space time in
            $N^2$ as many ways as it embeds in SGD's spacetime, due to
            horizontal shifts.  Likewise, there are $N^2$ times as many
            embeddings of $\sdia{(01-2)(02-12)}$ in distinct epochs of GD's
            spacetime as there are in distinct epochs of SGD's spacetime.
            However, each same-epoch embedding of $\sdia{(01-2)(01-12)}$ within
            any one epoch of GD's spacetime corresponds by vertical shifts to
            an embedding of $\sdia{(0-1-2)(01-12)}$ in SGD.  There are
            $MN{N\choose 2}$ many such embeddings in GD's spacetime, so GD's
            test loss exceeds SGD's by 
            $
                \frac{MN{N\choose 2}}{N^2}~
                \sdia{c(01-2)(01-12)}
            $.
            Reading the diagram's value from its graph structure, we
            unpack that expression as:
            $$
                \eta^2 \frac{M(N-1)}{4} G \nabla C 
            $$
        \end{shaded}

        \subsubsection*{Direct Perturbation} 
        \colorlet{shadecolor}{moosky}
        \begin{shaded}
            We compute the displacement $\theta_T-\theta_0$ to order $\eta^2$ 
            for pure SGD and separately for pure GD.  Expanding
            $
                \theta_t \in \theta_0 + \eta a(t) + \eta^2 b(t) + o(\eta^2)
            $, we find:
            \begin{align*}
                \theta_{t+1} &=     \theta_t - \eta \nabla l_{n_t} (\theta_t) \\
                             &\in       \theta_0
                                    +   \eta a(t) + \eta^2 b(t)
                                    -   \eta (
                                                \nabla l_{n_t}
                                            +   \eta \nabla^2 l_{n_t} a(t) 
                                        )
                                    +   o(\eta^2) \\
                             &=     \theta_0
                                +   \eta (a(t) - \nabla l_{n_t})
                                +   \eta^2 (b(t) - \nabla^2 l_{n_t} a(t)) 
                                +   o(\eta^2)
            \end{align*}
            To save space, we write $l_{n_t}$ for $l_{n_t}(\theta_0)$.  It's
            enough to solve the recurrence $a(t+1) = a(t) - \nabla l_{n_t}$ and
            $b(t+1) = b(t) - \nabla^2 l_{n_t} a(t)$.  Since $a(0), b(0)$
            vanish, we have $a(t) =-\sum_{0\leq t<T} \nabla l_{n_t}$ and $b(t)
            = \sum_{0\leq t_0 < t_1 < T} \nabla^2 l_{n_{t_1}} \nabla
            l_{n_{t_0}}$.  We now expand $l$:
            \begin{align*}
                l(\theta_T) \in    l   &+   (\nabla l) (\eta a(T) + \eta^2 b(T)) \\
                                       &+   \frac{1}{2} (\nabla^2 l) (\eta a(T) + \eta^2 b(T))^2
                                        +   o(\eta^2) \\
                            =      l   &+   \eta ((\nabla l) a(T))
                                        +   \eta^2 ((\nabla l) b(T) + \frac{1}{2} (\nabla^2 l) a(T)^2 )
                                        +   o(\eta^2)
            \end{align*}
            Then $\expct{a(T)} = -MN(\nabla l)$ and, since the $N$ many
            singleton batches in each of $M$ many epochs are pairwise
            independent,
            \begin{align*}
                \expct{(a(T))^2}
                ~&=
                \sum_{0\leq t<T} \sum_{0\leq s<T} \nabla l_{n_t} \nabla l_{n_s} \\
                ~&= 
                M^2N(N-1)   \expct{\nabla l}^2 +
                M^2N        \expct{(\nabla l)^2}
            \end{align*}
            Likewise, 
            \begin{align*}
                \expct{b(T)}
                = 
                ~&\sum_{0\leq t_0 < t_1 < T} \nabla^2 l_{n_{t_1}} \nabla l_{n_{t_0}} \\
                =
                ~&\frac{M^2N(N-1)}{2} \expct{\nabla^2 l} \expct{\nabla l} + \\
                ~&\frac{M(M-1)N}{2}  \expct{(\nabla^2 l) (\nabla l)} 
            \end{align*}

            Similarly, for pure GD, we may demand that $a, b$ obey recurrence
            relations $a(t+1) = a(t) - \sum_n \nabla l_n/N$ and
            $b(t+1) = b(t) - \sum_n \nabla^2 l_n a(t)/N$, meaning that
            $a(t) = -t \sum_n \nabla l_n/N$ and
            $b(t) = {t \choose 2} \sum_{n_0} \sum_{n_1} \nabla^2 l_{n_0} \nabla l_{n_1}/N^2$.
            So $\expct{a(T)} = -MN(\nabla l)$ and
            \begin{align*}
                \expct{(a(T))^2}
                ~&=
                M^2 
                \sum_{n_0} \sum_{n_1} \nabla l_{n_0} \nabla l_{n_1} \\
                ~&= 
                M^2 N(N-1)  \expct{\nabla l}^2 + 
                M^2 N       \expct{(\nabla l)^2}
            \end{align*}
            and
            \begin{align*}
                \expct{b(T)}
                = 
                ~&{MN \choose 2}\frac{1}{N^2}
                \sum_{n_0} \sum_{n_1} \nabla^2 l_{n_0} \nabla l_{n_1} \\
                =
                ~&\frac{M(MN-1)(N-1)}{2} \expct{\nabla^2 l} \expct{\nabla l} + \\
                ~&\frac{M(MN-1)}{2}      \expct{(\nabla^2 l) (\nabla l)} 
            \end{align*}
            We see that the expectations for $a$ and $a^2$ agree between pure
            SGD and pure GD.  So only $b$ contributes.  We conclude that pure
            GD's test loss exceeds pure SGD's by
            \begin{align*}
                   ~&\eta^2
                    \wrap{\frac{M(MN-1)(N-1)}{2}  - \frac{M^2N(N-1)}{2}}
                    \expct{\nabla^2 l} \expct{\nabla l}^2 \\
                +   ~&\eta^2 
                    \wrap{\frac{M(MN-1)N}{2} - \frac{M(M-1)N}{2}}
                    \expct{(\nabla^2 l) (\nabla l)} \expct{\nabla l} \\
                = 
                    ~&\eta^2     \frac{M(N-1)}{2}
                \expct{\nabla l} \wrap{
                      \expct{(\nabla^2 l) (\nabla l)}
                    - \expct{\nabla^2 l} \expct{\nabla l}
                }
            \end{align*}
            Since $(\nabla^2 l) (\nabla l) = \nabla((\nabla l)^2)/2$, we can 
            summarize this difference as
            $$
                \eta^2 \frac{M(N-1)}{4}
                G \nabla C 
            $$
        \end{shaded}

    \subsection{
        Effect of Nongaussian Noise at a Minimum.
        Proof of Part of Corollary \ref{cor:vsode}.
    }
        The rest of the proof is in Appendix \ref{sect:tutorial}.

        We consider vanilla SGD initialized at a local minimum of the test loss.
        One expects $\theta$ to diffuse around that minimum according to
        gradient noise.  We compute the effect on test loss of nongaussian
        diffusion.  Specifically, we compare SGD test loss on the loss
        landscape to SGD test loss on a different loss landscape defined as a
        Gaussian process whose every covariance agrees with the original
        landscape's.  We work to order $\eta^3$ because at lower orders,
        the gaussian landscapes will by construction match their nongaussian
        counterparts.

        \subsubsection*{Diagram Method}
        \colorlet{shadecolor}{moolime}
        \begin{shaded}
            Because $\expct{\nabla l}$ vanishes at initialization, all diagrams
            with a degree-one vertex that is a singleton vanish.  Because we
            work at order $\eta^3$, we consider $3$-edged diagrams.  Finally,
            because all first and second moments match between the two
            landscapes, we consider only diagrams with at least one partition
            of size at least $3$.  The only such test diagram is
            $\sdia{c(012-3)(03-13-23)}$.  This embeds in $T$ ways (one for each
            spacetime cell of vanilla SGD) and has symmetry factor $1/3!$ for a
            total of
            $$
                \frac{T \eta^3 }{6}
                \expct{\nabla^3 l}
                \expct{\nabla l_{n_{t_a}} \nabla l_{n_{t_b}} \nabla l_{n_{t_c}}}
            $$
        \end{shaded}

        \subsubsection*{Direct Perturbation}
        \colorlet{shadecolor}{moosky}
        \begin{shaded}
            We compute the displacement $\theta_T-\theta_0$ to order $\eta^3$ 
            for vanilla SGD.  Expanding
            $
                \theta_t \in \theta_0 + \eta a_t + \eta^2 b_t + \eta^3 c_t 
                + o(\eta^3)
            $, we find:
            \begin{align*}
                \theta_{t+1}
                =
                \theta_t    &-  \eta \nabla l_{n_t} (\theta_t) \\
                \in\theta_0 &+  \eta a_t + \eta^2 b_t + \eta^3 c_t \\
                            &-  \eta \wrap{
                                     \nabla l_{n_t}
                                    +\nabla^2 l_{n_t} (\eta a_t + \eta^2 b_t)
                                    +\frac{1}{2} \nabla^3 l_{n_t} (\eta a_t)^2
                                }
                             +  o(\eta^3) \\
                =
                \theta_0    &+   \eta   \wrap{a_t - \nabla l_{n_t}} \\
                            &+   \eta^2 \wrap{b_t - \nabla^2 l_{n_t} a_t} \\ 
                            &+   \eta^3 \wrap{
                                     c_t
                                    -\nabla^2 l_{n_t} b_t
                                    -\frac{1}{2} \nabla^3 l_{n_t} a_t^2
                                 }
                             +   o(\eta^3)
            \end{align*}
            We thus have the recurrences
            $
                a_{t+1} = a_t - \nabla l_{n_t}
            $,
            $
                b_{t+1} = b_t - \nabla^2 l_{n_t} a_t
            $, and
            $
                c_{t+1} = c_t -\nabla^2 l_{n_t} b_t 
                              -\frac{1}{2} \nabla^3 l_{n_t} a_t^2
            $
            with solutions:
            $a_t = -\sum_{t} \nabla l_{n_t}$ and
            $\eta^2 b_t = +\eta^2 \sum_{t_0 < t_1} \nabla^2 l_{n_{t_1}} \nabla l_{n_{t_0}}$.
            %\begin{align*}
            %    \eta a_t = &-\eta \sum_{t} \nabla l_{n_t}
            %    \\ 
            %    \eta^2 b_t = &+\eta^2 \sum_{t_0 < t_1} \nabla^2 l_{n_{t_1}} \nabla l_{n_{t_0}}
                %\\
                %\eta^3 c_t^\mu =
                %    &-\sum_{t_0 < t_1 < t_2} 
                %        \nabla^\mu \nabla_\nu l_{n_{t_2}}
                %        \nabla^\nu \nabla_\sigma l_{n_{t_1}} \nabla^\sigma l_{n_{t_0}} \\
                %    &-\frac{1}{2}
                %        \sum_{t_a, t_b < t}
                %        \nabla^\mu \nabla^\nu \nabla^\sigma l_{n_t}
                %        \nabla_\nu l_{n_{t_a}}
                %        \nabla_\sigma l_{n_{t_b}}
            %\end{align*}
            %We use tensor indices above because the contraction pattern would
            %otherwise be ambiguous.
            We do not compute $c_t$ because we will soon see that it will be
            multiplied by $0$.

            To third order, the test loss of SGD is
            \begin{align*}
                l(\theta_T)
                \in
                        l(\theta_0)
                &+     (\nabla   l)   (\eta a_T + \eta^2 b_T + \eta^3 c_T)                              \\
                &+\frac{\nabla^2 l}{2}(\eta a_T + \eta^2 b_T             )^2                            \\
                &+\frac{\nabla^3 l}{6}(\eta a_T                          )^3 
                 +o(\eta)^3                                                                             \\
                =
                    l(\theta_0)
                &+  \eta       \wrap{(\nabla l) a_T                               }                     \\
                &+  \eta^2     \wrap{(\nabla l) b_T + \frac{\nabla^2 l}{2} a_T^2  }                     \\
                &+  \eta^3     \wrap{(\nabla l) c_T + (\nabla^2 l) a_T b_T + \frac{\nabla^3 l}{6} a_T^3}
                 +o(\eta)^3                                                                             
            \end{align*}
            Because $\expct{\nabla l}$ vanishes at initialization, we neglect
            the $(\nabla l)$ terms.  The remaining $\eta^3$ terms involve
            $a_T b_T$, and $a_T^3$.  So let us
            compute their expectations:
            \begin{align*}
                \expct{a_T b_T}
                    =&- \sum_{t} \sum_{t_0 < t_1}
                        \expct{\nabla l_{n_t} \nabla^2 l_{n_{t_1}} \nabla l_{n_{t_0}}}
                    \\
                    =&- \sum_{t_0 < t_1}  
                        \sum_{t \notin \{t_0, t_1\}} 
                            \expct{\nabla l_{n_t}} \expct{\nabla^2 l_{n_{t_1}}} \expct{\nabla l_{n_{t_0}}}
                    \\&- \sum_{t_0 < t_1}  
                        \sum_{t = t_0}
                            \expct{\nabla l_{n_t} \nabla l_{n_{t_0}}} \expct{\nabla^2 l_{n_{t_1}}}
                    \\&- \sum_{t_0 < t_1}  
                        \sum_{t = t_1}
                            \expct{\nabla l_{n_t} \nabla^2 l_{n_{t_1}}} \expct{\nabla l_{n_{t_0}}}
            \end{align*}
            Since $\expct{\nabla l}$ divides $\expct{a_T b_T}$, the latter
            vanishes.
            \begin{align*}
                \expct{a_T^3}
                    =&- \sum_{t_a, t_b, t_c}
                            \expct{\nabla l_{n_{t_a}} \nabla l_{n_{t_b}} \nabla l_{n_{t_c}}}
                    \\
                    =&- \sum_{\substack{t_a, t_b, t_c\\ \text{disjoint}}}  
                            \expct{\nabla l_{n_{t_a}}} \expct{\nabla l_{n_{t_b}}} \expct{\nabla l_{n_{t_c}}}
                    \\&-3 \sum_{t_a=t_b\neq t_c}  
                            \expct{\nabla l_{n_{t_a}} \nabla l_{n_{t_b}}} \expct{\nabla l_{n_{t_c}}}
                    \\&-\sum_{t_a=t_b=t_c}  
                            \expct{\nabla l_{n_{t_a}} \nabla l_{n_{t_b}} \nabla l_{n_{t_c}}}
            \end{align*}
            As we initialize at a test minimum, only the last line remains, at
            it has $T$ identical summands.
            When we plug into the expression for SGD test loss, we get
            $$
                \frac{T \eta^3 }{6}
                \expct{\nabla^3 l}
                \expct{\nabla l_{n_{t_a}} \nabla l_{n_{t_b}} \nabla l_{n_{t_c}}}
            $$
        \end{shaded}

            %\begin{align*}
            %    \expct{a_T^3}
            %       &-\eta^3 \sum_{t} \sum_{t_0 < t_1}
            %            \nabla l_{n_t} \expct{\nabla^2 l_{n_{t_1}} \nabla l_{n_{t_0}}}
            %       \\
            %       &-\eta^3 \sum_{t} \sum_{t_0 < t_1}
            %            \nabla l_{n_t} \nabla^2 l_{n_{t_1}} \nabla l_{n_{t_0}}
            %\end{align*}

    %\subsection{A Nonconservative Force (Unrenormalized)}
    %    We identify the leading order nonconservative component in vanilla
    %    SGD's evolution of a weight initialized at a test minimum.
    %    \subsubsection*{Diagram Method}
    %    \subsubsection*{Direct Perturbation}

    %\subsection{
    %    The Effect of Inter-Epoch Shuffling.
    %    Proof of Corollary \ref{cor:shuffle}.
    %}
    %    We identify the leading order effect of shuffling on test loss for
    %    pure, multi-epoch SGD.  It is much harder to see by direct perturbation
    %    than by diagrams that this is an order $\eta^4$ effect, so for fairness
    %    we will assume this order it already known for both methods.

    %    \subsubsection*{Diagram Method}
    %    \colorlet{shadecolor}{moolime}
    %    \begin{shaded}
    %        {\color{moor} FILL IN}
    %    \end{shaded}
    %        
    %    \subsubsection*{Direct Perturbation}
    %    \colorlet{shadecolor}{moosky}
    %    \begin{shaded}
    %        We compute the displacement $\theta_T-\theta_0$ to order $\eta^4$ 
    %        for vanilla SGD.  Expanding
    %        $
    %            \theta_t \in \theta_0
    %                + \eta a_t + \eta^2 b_t + \eta^3 c_t + \eta^4 d_t 
    %                + o(\eta^4)
    %        $, we find:
    %        \begin{align*}
    %            \theta_{t+1}
    %            =
    %            \theta_t    &- \eta \nabla l_{n_t} (\theta_t) \\
    %            \in\theta_0 &+ \eta a_t + \eta^2 b_t + \eta^3 c_t + \eta^4 d_t \\
    %                        &- \eta \wrap{
    %                                \nabla l_{n_t}
    %                               +\nabla^2 l_{n_t}             (\eta a_t + \eta^2 b_t + \eta^3 c_t)
    %                           } \\
    %                        &- \eta \wrap{
    %                                \frac{1}{2} \nabla^3 l_{n_t} (\eta a_t + \eta^2 b_t)^2
    %                               +\frac{1}{6} \nabla^4 l_{n_t} (\eta a_t)^3
    %                           }
    %                         + o(\eta^4) \\
    %            =
    %            \theta_0    &+ \eta   \wrap{a_t - \nabla l_{n_t}} \\
    %                        &+ \eta^2 \wrap{b_t - \nabla^2 l_{n_t} a_t} \\ 
    %                        &+ \eta^3 \wrap{
    %                               c_t
    %                              -\nabla^2 l_{n_t} b_t
    %                              -\frac{1}{2} \nabla^3 l_{n_t} a_t^2
    %                           } \\
    %                        &+ \eta^4 \wrap{
    %                               d_t
    %                              -            \nabla^2 l_{n_t} c_T
    %                              -\frac{1}{2} \nabla^3 l_{n_t} b_T^2 
    %                              -\frac{1}{6} \nabla^4 l_{n_t} a_T^3 
    %                           }
    %                         + o(\eta^4)
    %        \end{align*}
    %        We thus have the recurrences
    %        $
    %            a_{t+1} = a_t - \nabla l_{n_t}
    %        $,
    %        $
    %            b_{t+1} = b_t - \nabla^2 l_{n_t} a_t
    %        $,
    %        $
    %            c_{t+1} = c_t -\nabla^2 l_{n_t} b_t 
    %                          -\frac{1}{2} \nabla^3 l_{n_t} a_t^2
    %        $, and
    %        $
    %            d_{t+1} = d_t -             \nabla^2 l_{n_t} c_T
    %                          - \frac{1}{2} \nabla^3 l_{n_t} b_T^2 
    %                          - \frac{1}{6} \nabla^4 l_{n_t} a_T^3 
    %        $
    %        with solutions:
    %        \begin{align*}
    %            \eta a_t = &-\eta \sum_{t} \nabla l_{n_t}
    %            \\ 
    %            \eta^2 b_t = &+\eta^2 \sum_{t_0 < t_1} \nabla^2 l_{n_{t_1}} \nabla l_{n_{t_0}}
    %            \\
    %            \eta^3 c_t^\mu =
    %                &-\sum_{t_0 < t_1 < t_2} 
    %                    \nabla^\mu \nabla_\nu l_{n_{t_2}}
    %                    \nabla^\nu \nabla_\sigma l_{n_{t_1}} \nabla^\sigma l_{n_{t_0}} \\
    %                &-\frac{1}{2}
    %                    \sum_{t_a, t_b < t}
    %                    \nabla^\mu \nabla^\nu \nabla^\sigma l_{n_t}
    %                    \nabla_\nu l_{n_{t_a}}
    %                    \nabla_\sigma l_{n_{t_b}}
    %            \\
    %            \eta^4 d_t^\mu =
    %                &-\sum_{t_0 < t_1 < t_2} 
    %                    \nabla^\mu \nabla_\nu l_{n_{t_2}}
    %                    \nabla^\nu \nabla_\sigma l_{n_{t_1}} \nabla^\sigma l_{n_{t_0}} \\
    %                &-\frac{1}{2}
    %                    \sum_{t_a, t_b < t}
    %                    \nabla^\mu \nabla^\nu \nabla^\sigma l_{n_t}
    %                    \nabla_\nu l_{n_{t_a}}
    %                    \nabla_\sigma l_{n_{t_b}}
    %        \end{align*}
    %        We use tensor indices above because the contraction pattern would
    %        otherwise be ambiguous.

    %        {\color{moor} FILL IN}
    %    \end{shaded}



\section{Other Perturbative Calculations} \label{sect:calculations}
    %\subsection{SGD vs ODE and SDE}
    %    {\color{moor} FILL IN}
    %\subsection{Interepoch Shuffling}
    %    {\color{moor} FILL IN}
    %\subsection{Effect of Epochs}
    %    {\color{moor} FILL IN}
    %\subsection{Renormalized Nonconservative Entropic Force}
    %    {\color{moor} FILL IN}


\section{Mathematical Background} \label{sect:morebackground}
    This Appendix provides the mathematical context for our proofs.  For the
    practitioner of the diagram method, such background is likely unnecessary.
    We list the regularity conditions assumed of the loss landscape not here
    but in Appendix \ref{sect:proofs}.

    \subsection{The Combinatorial Costumes: Structure Sets}
        The main text and Appendix \ref{sect:tutorial} give practical
        perspective on spacetimes and diagrams.  To define those concepts with
        mathematical precision, however, we employ the language of category
        theory.  What linear algebra does to clarify and systematize an
        otherwise unwieldy world of coordinate transformations, category theory
        does for combinatorial structures and notions of sameness.  We
        recommend \citet{fo19} for a practical introduction to applied category
        theory.

        We define both diagrams and spacetimes in terms of \emph{structure
        sets}, i.e. sets $S$ equipped with a preorder $\leq$ and an equivalence
        relation $\sim$.  There need not be any relationship between $\leq$ and
        $\sim$.  Morphisms between structure sets are strictly increasing maps
        that preserve $\sim$ and its negation.  A structure set is
        \emph{pointed} when it has a unique $\leq$-maximum element and this
        element forms a singleton $\sim$-class.  The categories $\Ss$ of
        structure sets and $\Pp$ of pointed structure sets enjoy a
        free-forgetful adjunction $\Free, \Forg$.
        
        When $\leq$ is a total preorder, we say that $S$ is a \emph{spacetime}.
        When $\leq$ has arbitrary joins and its geometric realization is a
        tree, we say that $S$ is a \emph{diagram}. 
        
        Let $\parts(D)$ give the $\sim$-parts of $D$.  An $\Ss$-map from $D$ to
        $
            (\Forg\circ\Free)^{|\parts(D)|}(\text{empty set})
        $
        is an \emph{ordering} of $D$.  Let $|\edges(D)|$ and $|\ords(D)|$ count
        edges and orderings of $D$.  In any category, and with any morphism
        $f: x\to y$, let $\wabs{\Aut_f(x)}$ count the automorphisms of $x$
        that commute with $x$.
    
    \subsection{The Parameterized \emph{Personae}: Forms of SGD}
        SGD decreases an objective $l$ by updating on smooth, unbiased i.i.d.
        estimates $(l_n: 0\leq n<N)$ of $l$.  The pattern of updates is
        determined by a spacetime $S$: for a map
        $\pi:S\to [N]$ that induces $\sim$, we define SGD inductively as
        $\text{SGD}_{S}(\theta) = \theta$ when $S$ is empty and otherwise
        $$
            \SGD_S(\theta)
            =
            \SGD_{S\setminus M}(
                \theta^\mu - \eta^{\mu\nu} \nabla_\nu l_{M}(\theta)
            )
        $$
        where $M = \min S \subseteq S$ specifies a batch and $l_M = \sum_{m\in
        M} l_{\pi(m)} / \wabs{M}$ is a batch average.  Since the distribution
        of $l_n$ is permutation invariant, the non-canonical choice of $\pi$
        does not affect the distribution of output $\theta$s.
    
        Of special interest are spacetimes that divide sequentially into
        $M\times B$ many \emph{epochs} each with $N/B$ many disjoint
        \emph{batches} of size $B$.  An SGD instance is then determined by $N,
        B, M$, and an \emph{inter-epoch shuffling scheme}.  The cases $B=1$ and
        $B=N$ we call \emph{pure SGD} and \emph{pure GD}.  The $M=1$ case of
        pure SGD we call \emph{vanilla SGD}.

        We follow convention in calling using the word ``set'' for
        \emph{ordered} sequences of training points. 

\section{Proofs of Theorems} \label{sect:proofs}
   
    \subsection{Regularity Hypotheses}
        We assume throughout this work several regularity properties of the
        loss landscape.  \emph{Existence of Taylor Moments} --- we assume that
        each finite collection of polynomials of the $0$th and higher
        derivatives of the $l_x$, all evaluated at any point $\theta$, may be
        considered together as a random variable insofar
        as they are equipped with a probability measure upon of the standard
        Borel algebra.  \emph{Analyticity Uniform in Randomness} --- we
        moreover assume that the functions $\theta \mapsto l_x(\theta)$, as
        well as the expectations of polynomials of their $0$th and higher
        derivatives, exist and are analytic with shared (but
        $\theta$-dependent) radii of convergence.  \emph{Boundedness of
        Gradients} --- we also assume that the gradients $\nabla l_x(\theta)$,
        considered as random covectors, are bounded by some continuous function
        of $\theta$.\footnote{
            A metric-independent way of expressing this boundedness constraint
            is that the gradients all lie in some subset $\Ss \subseteq TM$ of
            the tangent bundle of weight space, where, for any compact $\Cc
            \subseteq M$, we have that the topological pullback --- of
            $\Ss \hookrightarrow TM \twoheadrightarrow M$
            and
            $\Cc \hookrightarrow M$ ---
            is compact.  We hope that the results of this paper expose how
            important the choice of metric can be and hence underscore the
            value of determining  whether a concept is metric-independent.
        }\footnote{
            Some of our experiments involve Gaussian noise, which is not
            bounded and hence violates one of our hypotheses.  For experimental
            purposes, however, Gaussians are effectively bounded, on the one
            hand in the sense that with high probability no standard normal
            sample encountered on Gigahertz hardware within the age of the
            universe will much exceed $\sqrt{2 \log(10^{30})} \approx 12$, and
            on the other hand in the sense that our predictions vary smoothly
            with the first few moments of this distribution, so that a $\pm
            12$-clipped gaussian will yield almost the same predictions.
        }

        Kol\'{a}\u{r} gives a careful introduction to these differential
        geometric ideas \yrcite{ko93}.

        \subsubsection*{Pathologies Illustrating the Hypotheses' Utility}
            {\color{moor} FILL IN}


    \subsection{Dyson Series for Iterative Optimizers}
        We first give intuition, then worry about $\epsilon$s and $\delta$s.
        \subsubsection*{The Key Lemma: Proof Idea}
            Intuitively, since $\nabla$ Lie-generates translation, the operator
            $
                \exp\wrap{
                    -\eta^{\mu\nu} g_\mu \nabla_\nu
                }
            $
            performs translatation by $-\eta g$.  In particular, the case
            $g=\nabla l_t(\theta)$ effects a gradient step on the $t$th batch.
            A product of such exponential operators will give the loss after a
            sequence of updates $\theta \mapsto \theta - \eta^{\mu\nu}
            \nabla_\mu l(\theta)$ on losses $(l_t: 0\leq t < T)$.  Because the
            operators might not commute, we may not compose the product of
            exponentials into an exponential of a sum.  We instead compute an
            expansion in powers of $\eta$, collecting terms of like degree
            while maintaining the order of operators:
            \begin{align*}
                s(\theta_T)
                &=
                \wrap{\prod_{0\leq t<T} \wrap{
                    \sum_{0\leq d_t}
                        \left.
                            \frac{(-\eta^{\mu\nu} g_\mu \nabla_\nu)^{d_t}}{d_t!}
                        \right|_{g=\nabla l_t(\theta)}
                }
                s} (\theta_0) \\
                &= 
                \sum_{0\leq d < \infty} (-\eta)^d
                \sum_{\substack{(d_t: 0\leq t<T) \\ \sum_t d_t = d}}
                \wrap{
                    \prod_{0 \leq t < T} \left.
                        \frac{(g \nabla)^{d_t}}{d_t!}
                    \right|_{g=\nabla l_t(\theta)}
                } s (\theta_0)
            \end{align*}
            We finish by taking expectations.

        \subsubsection*{The Key Lemma: Proof}
            We work in a neighborhood of the initialization so that the tangent
            space of weight space is a trivial bundle.  For convenience, we fix
            a flat coordinate system, and with it the induced flat,
            non-degenerate inverse metric $\tilde\eta$; the benefit is that we
            may compare our varying $\eta$ against one fixed $\tilde\eta$.
            Henceforth, a ``ball'' unless otherwise specified will mean a ball
            with respect to $\tilde\eta$ around the initialization $\theta_0$.
            Since $s$ is analytic, its Taylor series converges to $s$ within
            some positive radius $\rho$ ball.  By assumption, every $l_t$ is
            also analytic with shared radius of convergence around $\theta_0$,
            without loss also $\rho$.  Since gradients are $x$-uniformly
            bounded by a continuous function of $\theta$, and since in finite
            dimensions the closed $\rho$-ball is compact, we have a strict
            gradient bound $b$ uniform in both $x$ and $\theta$ on gradient
            norms within that closed ball.  When
            \begin{equation} \label{eq:smalleta}
                2 \eta T b < \rho \tilde\eta
            \end{equation}
            as norms, stochastic gradient after $T$ steps on any train set
            will necessarily stay within the $\rho$-ball.  In fact, we will see
            that the factor of $2$ ensures that SGD initialized at any point
            within a $\rho/2$ ball will necessarily stay within the
            $\rho$-ball.  We note that the above condition on $\eta$ is weak
            enough to permit all $\eta$ within some open neighborhood of
            $\eta=0$.  

            Condition \ref{eq:smalleta} together with analyticity of $s$ then
            implies that
            $
                \wrap{\exp(-\eta g \nabla) s}(\theta) = s(\theta - \eta g)
            $
            when $\theta$ lies in the $\tilde\eta$ ball (of radius $\rho$) and
            its $\eta$-distance from that $\tilde\eta$ ball's boundary exceeds
            $b$, and that both sides are analytic in $\eta, \theta$ on the 
            same domain --- and \emph{a fortiori} when $\theta$ lies in the
            ball of radius $\rho (1 - 1/(2T))$.
            Likewise, a routine induction through $T$ gives the value of $s$
            (after doing $T$ gradient steps from an initialization $\theta$) as
            $$
                \wrap{
                    \prod_{0\leq t<T}
                        \left.
                            \exp(-\eta g \nabla)
                        \right|_{g=\nabla l_t(\theta)}
                }
                (s)(\theta)
            $$
            for any $\theta$ in the $\rho (1-T/(2T)$-ball (that is, the
            $\rho/2$-ball), and that both sides are analytic in $\eta, \theta$
            on that same domain.  Note that in each exponential, the
            $\nabla_\nu$ does not act on the $\nabla_\mu l(\theta)$ with which
            it pairs.  

            Now we use the standard expansion of $\exp$.  Because (by
            analyticity) the order $d$ coeffients of $l_t, s$ are bounded by some
            exponential decay in $d$, by assumption at an $x$-uniform rate, we
            have absolute convergence and may rearrange sums.  We choose to
            group by total degree:
            \begin{equation} \label{eq:expansion}
                \cdots 
                =
                \sum_{0\leq d < \infty} (-\eta)^d
                \sum_{\substack{(d_t: 0\leq t<T) \\ \sum_t d_t = d}}
                \wrap{
                    \prod_{0 \leq t < T} \left.
                        \frac{(g \nabla)^{d_t}}{d_t!}
                    \right|_{g=\nabla l_t(\theta)}
                } s (\theta)
            \end{equation}
            The first part of the Key Lemma is proved.  It remains to show that
            expectations over train sets commute with the above summation.

            We will apply Fubini's Theorem.  To do so, it suffices to show that   
            $$
                \wabs{
                    \sum_{\substack{(d_t: 0\leq t<T) \\ \sum_t d_t = d}}
                    \wrap{
                        \prod_{0 \leq t < T} \left.
                            \frac{(g \nabla)^{d_t}}{d_t!}
                        \right|_{g=\nabla l_t(\theta)}
                    } s (\theta)
                }
                = \wabs{c_d((l_t: 0\leq t<T))} 
            $$
            has an expectation that decays exponentially with $d$.  The symbol
            $c_d$ we introduce purely for convenience; that its value depends
            on the train set we emphasize using function application
            notation.  Crucially, no matter the train set, we have shown
            that the expansion \ref{eq:expansion} (that features $c_d$ appear
            as coefficients) converges to an analytic function for all $\eta$
            bounded as in condition \ref{eq:smalleta}.  The uniformity of this
            demanded bound on $\eta$ implies by the standard relation between
            radii of convergence and decay of coefficients that $\wabs{c_d}$
            decays exponentially in $d$ at a rate uniform over train sets.
            If the expectation of $\wabs{c_d}$ exists at all, then, it will
            likewise decay at that same shared rate.
            
            But $\wabs{c_d}$ indeed has an expectation, for it is a bounded
            continuous function of a (finite-dimensional) space of $T$-tuples
            (each of whose entries can specify the first $d$ derivatives of an
            $l_t$) and because the latter space enjoys a joint distribution (of
            course over the standard Borel algebra!).  So Fubini's Theorem
            applies. The Key Lemma
            %(Lemma \ref{lem:dyson})
            follows.   

    \subsection{Terms and Diagram Embeddings Correspond}
        \subsubsection*{Path Integral Theorem: Proof Idea}
            We now seek to describe the terms that appear in the Key Lemma. 
            Theorem \ref{thm:sgdcoef} does so by matching each term to an
            embedding of a diagram in spacetime, so that the infinite sum
            becomes a sum over all diagram spacetime configurations.
            The main idea is that the combinatorics of diagrams parallels the
            combinatorics of repeated applications of the product rule for
            derivatives applied to the expression in the Key Lemma. 
            Balancing against this combinatorial explosion are factorial-style 
            denominators, again from the Key Lemma, that we summarize in terms
            of the sizes of automorphism groups.

        \subsubsection*{Consolation}
            The following proof is messy.  It compresses into a reusable
            package the messy intricacies of direct perturbation (see
            Appendix \ref{sect:compare} for samples of uncompressed computations), and as such 
            equates two conceptually clean sides via a jungle of canceling
            sums and factorials.  

            How can we increase our confidence in the correctness of a theorem
            so unappetizingly proved?  We regard three pieces of evidence as
            supplementing this proof: \emph{Aesthetic} evidence --- the Theorem
            assumes a form familiar to mathematicians and physicists: it is a
            sum over combinatorial objects weighted inversely by the order of
            their respective automorphism groups.  \emph{Comparative}
            evidence --- the Theorem's predictions agree with direct
            perturbation in the cases we report in Appendix \ref{sect:compare}.
            \emph{Empirical} evidence --- the Theorem, though compactly stated,
            precisely predicts the existence and intensity of the phenomena we
            report in the main body up to third order.
 
        \subsubsection*{Path Integral Theorem: Proof}
            We first prove the statement about test losses.
            Due to the analyticity property established in our proof of the
            Key Lemma, it suffices to show agreement at each degree $d$ and
            train set individually.  That is, it suffices to show --- for
            each train set $(l_n: 0\leq n<N)$, spacetime $S$, function $\pi:
            S\to [N]$ that induces $\sim$, and natural $d$ --- that
            \begin{align} \label{eq:toprove}
                (-\eta)^d
                \sum_{\substack{
                    (d_t: 0\leq t<T) \\
                    \sum_t d_t = d
                }}
                \wrap{
                    \prod_{0 \leq t < T} \left.
                        \frac{(g \nabla)^{d_t}}{d_t!}
                    \right|_{g=\nabla l_t(\theta)}
                } l (\theta)
                = \nonumber \\
                \sum_{\substack{
                    D \in \image(\Free) \\
                    \textnormal{with $d$ edges}
                }}
                \wrap{
                    \sum_{f: D\to\Free(S)}
                    \frac{1}{\wabs{\Aut_f(D)}}
                }
                \frac{\dvalue_\pi(D, f)}{B^{d}}
            \end{align}
            Here, $\dvalue_\pi$ is the value of a diagram embedding before
            taking expectations over train sets.  We have for all $f$ that
            $\expct{\dvalue_\pi(D, f)} = \dvalue(D)$.
            Observe that both sides are finitary sums.

            \begin{rmk}[Differentiating Products]
                The product rule of Leibniz easily generalizes to higher
                derivatives of finitary products:
                $$
                    \nabla^{\wabs{M}} \prod_{k \in K} p_k
                    = 
                    \sum_{\nu:M\to K} \prod_{k\in K} \wrap{
                        \nabla^{\wabs{\nu^{-1}(k)}} p_k
                    }
                $$
                The above has $\wabs{K}^m$ many term indexed by functions to
                $K$ from $M$.
            \end{rmk}

            We proceed by joint induction on $d$ and $S$.  The base cases
            wherein $S$ is empty or $d=0$ both follow immediately from the Key
            Lemma, for then the only embedding is the unique embedding of
            $\sdia{(0)()}$.  For the induction step, suppose $S$ is a sequence
            of $\Mm = \min S \subseteq S$ followed by a strictly smaller $S$
            and that the result is proven for $(\tilde d, \tilde S)$ for every
            $\tilde d \leq d$.  Let us group the terms in the left hand side of
            desideratum \ref{eq:toprove} by $d_0$; by applying the induction
            hypothesis with $\tilde d = d - d_0$, we find that that left hand
            side is:
            \begin{align*}
                \sum_{\substack{
                    0 \leq d_0 \leq d
                }}
                \sum_{\substack{
                    \tilde D \in \image(\Free) \\
                    \textnormal{with $d-d_0$ edges}
                }}
                \frac{1}{d_0!}
                \sum_{\tilde f: \tilde D\to\Free(\tilde S)} \wrap{
                    \frac{1}{\wabs{\Aut_{\tilde f}(\tilde D)}}
                }
                ~\cdot~
                \\ %---------------------------------------------
                (-\eta)^{d_0}
                \left.
                    (g \nabla)^{d_0}
                \right|_{g=\nabla l_0(\theta)}
                \frac{\dvalue_\pi(\tilde D, \tilde f)}{B^{d-d_0}}
            \end{align*}
            Since $\dvalue_\pi(\tilde D, \tilde f)$ is a multilinear product of
            $d-d_0+1$ many tensors, the product rule for derivatives tells us
            that $(g \nabla)^{d_0}$ acts on $\dvalue_\pi(\tilde D, \tilde f)$
            to produce $(d-d_0+1)^d_0$ terms.  In fact, if we expand out
            $
                g = \sum_{m\in \Mm} \nabla l_m(\theta) / B
            $ 
            then there are $B^{d_0}(d-d_0+1)^{d_0}$ terms conveniently indexed
            by a pair of functions $\beta:[d_0]\to \Mm$ and $\nu:[d_0]\to
            \tilde D$.  The $(\beta, \nu)$-term corresponds to an embedding
            $f$ of a larger diagram $D$ in the sense that it contributes
            $\dvalue_\pi(D, f)/B^{d_0}$ to the sum.  Here, $(f, D)$ is $(\tilde
            f, \tilde D)$ with $\wabs{\wrap{\beta \times \nu}^{-1}(n, v)}$ many
            additional edges from the cell of datapoint $n$ at time $0$ to the
            $v$th node of $\tilde D$ as embedded by $\tilde f$.

            By the general Leibniz rule remarked on above, the sum over
            terms indexed by $(\beta, \nu)$ corresponds to a sum over
            embeddings $f$ that restrict to $\tilde f$, whose terms are multiples
            of the corresponding and embedded $D$.  Together with the sum
            over $\tilde f$, this gives a sum over all embeddings $f$.  So we
            now only need to check that the coefficients for each $f:D\to S$  
            are as claimed.

            We note that the $(\beta, \nu)$ diagram and term agrees with the
            $(\beta \circ \sigma, \nu \circ \sigma)$ diagram and term for any
            permutation $\sigma$ of $[d_0]$.
            The corresponding orbit has size
            \begin{align*}
                \frac{d_0!}{
                    \prod_{(m, i) \in \Mm \times \tilde D}
                        \wabs{(\beta \times \nu)^{-1}(m, i)}!
                }
            \end{align*}
            by the Orbit Stabilizer Theorem of elementary group theory.   

            It is thus enough to show that
            $$
                \wabs{\Aut_f(D)} = 
                \wabs{\Aut_{\tilde f}(D)}
                \prod_{(m, i) \in \Mm \times \tilde D}
                    \wabs{(\beta \times \nu)^{-1}(m, i)}!
            $$
            We will show this by a direct bijection.  First, observe that
            $
                f = \beta \sqcup \tilde f:
                    [d_0] \sqcup \tilde D \to \Mm \sqcup \tilde S
            $. 
            So each automorphism $\phi: D\to D$ that commutes with $f$ induces
            both an automorphism
            $
                \Aa = \phi|_{\tilde D}: \tilde D\to \tilde D
            $
            that commutes with $\tilde f$ together with the data of a map
            $
                \Bb = \phi_{[d_0]}: [d_0] \to [d_0] 
            $
            that both commutes with $\beta$.  However, not every such pair of
            maps arises from a $\phi$.  For, in order for $\Aa \sqcup \Bb: D
            \to D$ to be an automorphism, it must respect the order structure
            of $D$.  In particular, if $x\leq_D y$ with $x \in [d_0]$ and $y
            \in \tilde D$, then we need
            $$
                \Bb(x) \leq_D \Aa(y)
            $$
            as well.  The
            pairs $(\Aa, \Bb)$ that thusly preserve order are in bijection with
            the $\phi \in \Aut_f(D)$.  There are $\wabs{\Aut_{\tilde f}(\tilde
            D)}$ many $\Aa$.  For each $\Aa$, there are as many $\Bb$ as there
            are sequences $(\sigma_i: i \in \tilde D)$ of permutations on
            $
                \{j\in [d_0]: j\leq_D i\} \subseteq [d_0]
            $ 
            that commute with $\Bb$.  These permutations may be chosen
            independently; there are 
            $
                \prod_{m\in \Mm}
                    \wabs{(\beta \times \nu)^{-1}(m, i)}!
            $
            many choices for $\sigma_i$.  The counting claim follows
            and with it the correctness of coefficients.
 
            The analogous statement about generalization gaps follows similarly
            when we use $\sum_n l_n/N$ instead of $l$ as the value for $s$. 
            The Path Integral Theorem (Theorem \ref{thm:sgdcoef}) follows.

            \begin{rmk}[The Case of Vanilla SGD]
                Because the spacetime of vanilla SGD permits only embeddings
                with each fuzzy-connected-component in one distinct cell, the
                embeddings of a diagram are easily counted using factorials
                per Proposition \ref{prop:ordtwo}.  That proposition immediately
                follows from the now-proven Theorem \ref{thm:sgdcoef}.
            \end{rmk}

    \subsection{Coefficient Convergence upon Renormalization}
        \subsubsection*{Renormalization: $\rvalue_f(D)$'s General Recipe} \label{subsubsect:mobius}
            As mentioned in Definition \ref{defn:rvalue}, because we wish to
            study the difference between noisy and non-noisy optimization, we
            define our $\rvalue$s as differences of diagrams, some
            more fuzzily tied than others.  In the simplest cases 
            encountered in the text, these differences are simply those captured
            by our fuzzy tie notation, e.g.
            $
                \sdia{c(01-2-3)(02-12-23)}
                \triangleq
                \sdia{(01-2-3)(02-12-23)}
                -
                \sdia{(0-1-2-3)(02-12-23)}
            $.
            However, because in the renormalized picture, each irreducible
            diagram implicitly represents a whole family of topologically
            related diagrams, to use this difference as our definition for
            larger diagrams leads to combinatorial overcounting.

            Indeed, let $D$ be an irreducible diagram
            (e.g., $\sdia{(012-3)(01-12-23)}$).
            Removing all fuzzy ties, we obtain another diagram $D_\flat$
            (in the continuing example, $\sdia{(0-1-2-3)(01-12-23)}$),
            in turn is in the same class as some potentially smaller
            irreducible diagram $D_\star$
            ($\sdia{(0-1)(01)}$).
            The overcounting problem arises because some terms in $D_\star$'s
            renormalized value overlap with some terms in $D$'s renormalized
            value, even though they are distinct irreducible diagrams in
            correspondingly distinct classes.

            Let us counter this overcounting using standard technique of
            M\"obius inversion \citep{ro64}. The relevant partial ordered set
            is the set of diagrams in the interval $[D_\flat, D]$, where in
            this discussion we consider diagrams with isomorphic thin edge
            structures as ordered by refinement of partitions.  We proceed to
            define the relevant M\"obius function.  If $\rvalue^\star_f$ gives
            the renormalized values as in \ref{defn:rvalue} before replacing
            fuzzy ties by fuzzy outlines, we let for any $D_a \leq D_b$:
            $$
                \rvalue^\sharp_f(D_a, D_b)
                \triangleq
                \rvalue^\star(D_b) 
                - 
                \sum_{D_a \leq D_c < D_b}
                    \rvalue^\sharp_f(D_b, D_c) 
            $$
            We then set $\rvalue_f(D) \triangleq \rvalue^\sharp_f(D_\flat, D)$.

            For example, 
            \begin{align*}
                \rvalue_f(\sdia{(012-3)(01-12-23)})
                \triangleq
                    &\rvalue^\star_f(\sdia{(012-3)(01-12-23)}) \\
                -   &\wrap{\rvalue^\star_f(\sdia{(01-2-3)(01-12-23)}) - \sdia{(0-1-2-3)(01-12-23)}} \\
                -   &\wrap{\rvalue^\star_f(\sdia{(02-1-3)(01-12-23)}) - \sdia{(0-1-2-3)(01-12-23)}} \\
                -   &\wrap{\rvalue^\star_f(\sdia{(0-12-3)(01-12-23)}) - \sdia{(0-1-2-3)(01-12-23)}} \\
                -   &      \rvalue^\star_f(\sdia{(0-1-2-3)(01-12-23)}
            \end{align*}

        \subsubsection*{Renormalization Theorem: Proof Idea}
            The diagrams summed in Theorem \ref{thm:renorm} may be grouped by
            their geometric realizations.  Each nonempty class of diagrams with
            a given geometric realization has a minimal element, and in this
            way all and only irreducible diagrams arise. 

            We encounter two complications: on one hand, that the sizes of
            automorphism groups might not be uniform among the class of
            diagrams with a given geometric realization.  On the other hand,
            that the embeddings of a specific member of that class might be
            hard to count.  The first we handle by Orbit-Stabilizer.  The
            second we handle as described by \label{subsubsect:mobius} via
            M\"obius sums.
           
        \subsubsection*{Renormalization Theorem: Proof}
            We focus on test loss instead of generalization gap; the proofs are
            similar. The
            difference from the noiseless case is given by all the diagram
            embeddings with at least one fuzzy tie, where the fuzzy tie pattern
            is actually replaced by a difference between noisy and noiseless
            cases as prescribed by the discussion on Mobius Sums.
            Beware that the relatively noiseless embeddings may have illegal
            collisions of non-tied nodes within a single spacetime (data) row.
            Throughout the rest of this proof, we permit such illegal
            embeddings of the fuzz-less diagrams that arise from the
            aforementioned decomposition.  

            Because the Taylor series for analytic functions converge
            absolutely in the interior of the disk of convergence, the
            rearrangement of terms corresponding to a grouping by geometric
            realizations preserves the convergence result of \ref{thm:sgdcoef}.  

            Let us then focus on those diagrams $\sigma$ with a given geometric
            realization represented by $\rho$ irreducible.  By Theorem
            \ref{thm:sgdcoef}, it suffices to show that
            \begin{equation} \label{eq:hard}
                \sum_{f:\rho\to S}
                \sum_{\substack{
                    \tilde f:\sigma\to S \\
                    \exists i_\star: f=\tilde f \circ i_\star
                }}
                \frac{1}{\wabs{\Aut_{\tilde f}(\sigma)}}
                =
                \sum_{f:\rho\to S}
                \sum_{\substack{
                    \tilde f:\sigma\to S \\
                    \exists i_\star: f=\tilde f \circ i_\star
                }}
                \sum_{\substack{
                    i:\rho\to\sigma \\
                    f = \tilde f \circ i
                }}
                \frac{1}{\wabs{\Aut_{f}(\rho)}}
            \end{equation}
            Here, $f$ is considered up to equivalence by precomposition by an
            automorphism of $\rho$; likewise for $\tilde f$ and automorphisms
            of $\sigma$; $i$ ranges through maps that induce isomorphisms of
            geometric realizations; and $i$ is considered equivalent to $\hat
            i$ when for some automorphism $\phi \in \Aut_{\tilde f}(\sigma)$,
            we have $\hat i = i \circ \phi$.  Let the set of all such $i$s
            under this equivalence relation be called $X$.  The left hand side
            is the expression of Theorem \ref{thm:sgdcoef} and the right hand
            side is the expression of Theorem \ref{thm:renorm}; we have
            introduced redundant sums to structurally align the two expressions
            on the page.

            To prove equation \ref{eq:hard}, it suffices to show (for any
            $f, \tilde f, i$ as above) that
            $$
                \wabs{\Aut_f(\rho)}
                =
                \wabs{\Aut_{\tilde f}(\sigma)}
                \cdot
                \wabs{X}
            $$
            We will prove this using Orbit-Stabilizer by presenting an
            action of $\Aut_f(\rho)$ on $X$.  We simply use precomposition so
            that $\psi\in \Aut_f(\rho)$ sends $i\in X$ to $i\circ \psi$.  Since
            $f\circ\psi = f$, $i\circ \psi \in X$.  Moreover, the action is
            well-defined, because if $i\sim \hat i$ by $\phi$, then 
            $i \circ \psi \sim \hat i \circ \psi$ also by $\phi$.
            
            The size of $i$'s the stabilizer is $\wabs{\Aut_{\tilde f}(\rho)}$.
            For, when $i \sim i \circ \psi$ via $\phi \in \Aut_{\tilde
            f}(\rho)$, we have $i\circ \psi = \phi \circ i$.  This relation in
            fact induces a bijective correspondence: \emph{every} $\phi$
            induces a $\psi$ via $\psi = i^{-1} \circ \phi \circ i$, so we have
            a map $\text{stabilizer}(i) \hookleftarrow \Aut_{\tilde f}(\rho)
            $seen to be well-defined and injective by the strictly increasing
            nature of structure set morphisms together with the fact that $i$s
            must induce isomorphisms of geometric realizations.  Conversely,
            every $\psi$ that stabilizes enjoys \emph{only} one $\phi$ via
            which $i \sim i \circ \phi$, again by the same (isomorphism and
            strict increase) properties.  So the stabilizer has the claimed
            size.

            Meanwhile, the orbit is all of $\wabs{X}$.  Indeed, suppose
            $i_A, i_B \in X$.  We will present $\psi \in \Aut_f(\rho)$ such
            that $i_B \sim i_A \circ \psi$ by $\phi=\text{identity}$.  We
            simply define $\psi = i_A^{-1} \circ i_B$, well-defined by the
            aforementioned (isomorphisms and strict increase) properties.
            It is then routine to verify that
            $
                f \circ \psi
                =
                \tilde f \circ i_A \circ i_A^{-1} \circ i_B
                =
                \tilde f \circ i_B
                = f.
            $
            So the orbit has the claimed size, and by the Orbit-Stabilizer
            Theorem, the coefficients in the expansions of Theorems 
            \ref{thm:renorm} and \ref{thm:sgdcoef} match.

            To prove the rest of Theorem \ref{thm:renorm}, we assume that $H$
            is positive.  Then, for any $m$, the propagator ${(I-\eta
            H)^{\otimes m}}^t$ converges via an exponential decay with $t$ to
            $0$ (with a rate dependent on $m$).  The Renormalization Theorem.
            Since up to degree $d$ only a finite number of diagrams exist and
            hence only a finite number of possible $m$s, the exponential rates
            are bounded away from $0$.  Moreover, for any fixed
            $t_{\text{big}}$, the number of diagrams --- involving no exponent
            $t$ exceeding  $t_{\text{big}}$ --- is eventually constant as $T$
            grows.  Meanwhile, the number involving at least one exponent $t$
            exceeding that threshold grows polynomially in $T$ (with degree
            $d$).  The exponential decay of each term overwhelms the polynomial
            growth in the number of terms, and the convergence statement of
            Theorem (\ref{thm:renorm}) follows.


\section{Bessel Factors for Estimating Multipoint Correlators from Data}\label{sect:bessel}

    Given samples from a joint probability space $\prod_{0\leq d<D} X_d$, we
    seek unbiased estimates of multipoint correlators (i.e. products of
    expectations of products) such as $\wang{x_0 x_1 x_2}\wang{x_3}$.  For
    example, say $D=2$ and from $2S$ samples we'd like to estimate $\wang{x_0
    x_1}$.  Most simply, we could use $\Avg_{0\leq s<2S} x_0^{(s)} x_1^{(s)}$,
    where $\Avg$ denotes averaging.  In fact, the following also works:
    \begin{equation} \label{eq:bessel}
        S
        \wrap{\Avg_{0\leq s< S} x_0^{(s)}}
        \wrap{\Avg_{0\leq s< S} x_1^{(s)}}
        +
        (1-S)
        \wrap{\Avg_{0\leq s< S} x_0^{(s)}}
        \wrap{\Avg_{S\leq s<2S} x_1^{(s)}}
    \end{equation}
    When multiplication is expensive (e.g. when each $x_d^{(s)}$ is a tensor
    and multiplication is tensor contraction), we prefer the latter, since it
    uses $O(1)$ rather than $O(S)$ multiplications.  This in turn allows more
    efficient use of large-batch computations on GPUs.  We now generalize this
    estimator to higher-point correlators (and $D\cdot S$ samples).

    For uniform notation, we assume without loss that each of the $D$ factors
    appears exactly once in the multipoint expression of interest; such
    expressions then correspond to partitions on $D$ elements, which we
    represent as maps $\mu:\wasq{D}\to \wasq{D}$ with $\mu(d)\leq d$ and
    $\mu\circ \mu=\mu$.  Note that $\wabs{\mu} \coloneqq \wabs{im(\mu)}$ counts
    $\mu$'s parts.  We then define the statistic
    $$
        \wurl{x}_\mu
        \coloneqq
        \prod_{0\leq d<D} \Avg_{0\leq s<S} x_d^{(\mu(d)\cdot S + s)}
    $$
    and the correlator $\wang{x}_\mu$ we define to be the expectation of 
    $\wurl{x}_\mu$ when $S=1$.  In this notation, \ref{eq:bessel} says: 
    $$
        \wang{x}_{\partbox{0}\partbox{1}}
        =
        \expct{
            S       \cdot \wurl{x}_{\partbox{0 1}} +
            (1-S)   \cdot \wurl{x}_{\partbox{0}\partbox{1}}
        }
    $$
    Here, the boxes indicate partitions of $\wasq{D}=\wasq{2}=\{0,1\}$.
    Now, for general $\mu$, we have:
    \begin{equation} \label{eq:newbessel}
        \expct{S^D \wurl{x}_\mu}
        =
        \sum_{\tau\leq \mu} \wrap{
            \prod_{0\leq d<D}
                \frac{S!}{\wrap{S-\wabs{\tau(\mu^{-1}(d))}}!}
        }
        \wang{x}_\tau
    \end{equation}
    where `$\tau \leq \mu$' ranges through partitions \emph{finer} than 
    $\mu$, i.e. maps $\tau$ through which $\mu$ factors.   
    In smaller steps, \ref{eq:newbessel} holds because
    \begin{align*}
        \expct{S^D \wurl{x}_\mu}
        &=
        \expct{
            \sum_{(0\leq s_d<S) \in \wasq{S}^D}
            \prod_{0\leq d<D}
            x_d^{\wrap{\mu(d)\cdot S + s_d}}
        }\\
        &=
        \sum_{\substack{(0\leq s_d<S) \\ \in \wasq{S}^D}}
        \expct{
            \prod_{0\leq d<D}
            x_d^{\wrap{\min \wurl{
                \tilde{d}~:~\mu(\tilde{d})\cdot S+s_{\tilde{d}} = \mu(d)\cdot S+s_d
            }}}
        }\\
        &=
        \sum_{\tau} \wabs{\wurl{\substack{
            (0\leq s_d<S)~\in~[S]^D~: \\
            \wrap{\substack{
                \mu(d)=\mu(\tilde{d}) \\
                \wedge~s_d=s_{\tilde{d}}
            }}
            \Leftrightarrow
            \tau(d)=\tau(\tilde{d})
        }}}
        \wang{x}_\tau \\
        &=
        \sum_{\tau\leq \mu} \wrap{
            \prod_{0\leq d<D}
                \frac{S!}{\wrap{S-\wabs{\tau(\mu^{-1}(d))}}!}
        }
        \wang{x}_\tau
    \end{align*}

    Solving \ref{eq:newbessel} for $\wang{x}_\mu$, we find:
    \begin{equation*}
        \text{\fbox{$
        \wang{x}_\mu
        =
        \frac{S^D}{S^{\wabs{\mu}}}
        \expct{
            \wurl{x}_\mu
        }
        -
        \sum_{\tau < \mu} \wrap{
            \prod_{d\in im(\mu)}
            \frac{\wrap{S-1}!}{\wrap{S-\wabs{\tau(\mu^{-1}(d))}}!}
        }
        \wang{x}_\tau
        $}}
    \end{equation*}
    This expresses $\wang{x}_\mu$ in terms of the batch-friendly estimator
    $\wurl{x}_\mu$ as well as correlators $\wang{x}_\tau$ for $\tau$ 
    \emph{strictly} finer than $\mu$.  We may thus (use dynamic programming to)
    obtain unbiased estimators $\wang{x}_\mu$ for all partitions $\mu$. 
    Symmetries of the joint distribution and of the multilinear multiplication
    may further streamline estimation by turning a sum over $\tau$ into a
    multiplication by a combinatorial factor.  For example, with complete
    symmetry:
    $$
        \wang{x}_{\partbox{012}}
        =
        S^2
        \wurl{x}_{\partbox{012}}
        -
        \frac{(S-1)!}{(S-3)!}
        \wurl{x}_{\partbox{0}\partbox{1}\partbox{2}}
        -
        3\frac{(S-1)!}{(S-2)!}
        \wurl{x}_{\partbox{0}\partbox{12}}
    $$
    We use such expressions throughout our experiments to estimate the
    (expected) values of diagrams.




\section{Loss Landscapes Used for Experiments}\label{sect:landscape}

    In addition to the clarifyingly artificial loss landscapes (Gaussian Fit,
    Linear Screw, and Mean Estimation) described in the main text, we tested
    our predictions on logistic linear regression and simple convolutional
    networks (2 convolutional weight layers each with kernel $5$, stride $2$,
    and $10$ channels, followed by two dense weight layers with hidden
    dimension $10$) for the CIFAR-10 and Fashion-MNIST datasets.  The
    convolutional architectures used $\tanh$ activations and Gaussian Xavier
    initialization.  We parameterized the model so that the Gaussian-Xavier
    initialization of the linear maps in each layer differentially pulls back
    to standard normal initializations of the parameters.
    
    For these non-artificial landscapes, we regard the finite amount of
    available data as the true (sum of diracs) distribution from which we
    sample test and train sets in i.i.d.  manner (and hence ``with
    replacement'').  We do this to gain practical access to a ground truth
    against which we may compare our predictions.  One might object that this
    sampling procedure would cause test and train sets to overlap, hence
    biasing test loss measurements.  In fact, test and train sets overlap only
    in reference, not in sense: the situation is analogous to a text prediction
    task in which two training points culled from different corpora happen to
    record the same sequence of words, say, ``Thank you!''.  In any case, all
    of our experiments focus on the scanty-data regime, e.g. $10^1$ datapoints
    out of $\sim 10^{4.5}$ dirac masses, so overlaps are diluted. 

\section{Additional Figures}\label{sect:figures}

    \begin{figure}[H]
        \centering
        \plotmoo{plots/test-vanilla-fashion}{0.48\columnwidth}{3.0cm} 
        \plotmoo{plots/gen-cifar}{0.48\columnwidth}{3.0cm}
        \caption{
            {\bf Left}: Test loss vs learning rate on an image
            classification task.  For the instance shown and all $11$ other
            initializations unshown, the degree-$3$ prediction agrees with
            experiment through $\eta T \approx 10^0$.
            {\bf Right}:
            Generalization gap (test minus train) vs learning rate on an
            image classification task.  For the instance shown and all $11$
            other initializations unshown, the degree-$2$ prediction agrees
            with experiment through $\eta T \approx 10^0$.  Throughout,
            measurements are in blue and theory is in other colors.
            Vertical spans indicate 95\% confidence intervals for the mean.
        }
    \end{figure}


\section{Glossary}\label{sect:glossary}

    As our work uses physical methods to solve problems of computer science, it
    may contain unfamiliar terminology.  We strive in the main text to explain
    vital terms before their point of use; we hope this glossary complements
    that attempt.  The entries are intuitive and, while mathematically
    imprecise, sufficient for daily use.

    \subsection{Terminology}

    \begin{description}
        \item[affine manifold] a shape closed under a well-defined notion of displacement.  For example, a plane or a circle but not a disc.  The problem with a disc is that gradients might point toward the boundary and an SGD update might overshoot and fall off.  By contrast, a circle is okay because the gradients will all be clockwise and counterclockwise, directions under which the circle is closed.  We do not deal with projection onto feasible sets in this work.  We treat weight spaces as affine manifolds in this work not for the sake of generality but for the sake of conceptual encapsulation: the results of this work are coordinate-invariant! 
        \item[akaike information criterion] an estimate of generalization gap: $(\text{number of parameters})/N$.  Discrete-valued for fixed $N$, hence not liable to gradient descent. 
        \item[analytic] (of a function) locally equal to a convergent Taylor series.  Most smooth functions one encounters in daily life are analytic.  The ReLU function is not (smooth or) analytic. 
        \item[automorphism] a structure preserving map from an object to itself that has a structure preserving inverse.  The automorphisms of sets are permutations; the automorphisms of graphs are concepually analogous.  For example, $\sdia{(01-2)(02-12)}$ has two automorphisms: the identity map and the map that switches the two {\color{moor}red} nodes.   
        \item[conservative] (of a covector field) with zero curl.  The ODE approximation to gradient descent always yields a conservative covector field of gradients.  See also nonconservative. 
        \item[contraction] (of two tensors) the numpy operation of multiplying along a pair of axes, then summing.  The product of a row and column vector gives the simplest example of contraction.  For more complicated tensors, one performs that simplest product along the specified pair of axes while maintaining all other axes in an SQL-style join operation.  
        \item[covector] a numpy array of shape $1\times p$.  For example, a gradient.  Compare to vector.
        \item[crossing symmetry] a numerical relationship between diagrams with related shapes. 
        \item[datapoint] for us, a heavily overloaded term: an image of a cat or of a dog {\bf OR} an index into the train set of the corresponding image {\bf OR} the induced loss function of a given neural network on that corresponding image.  
        \item[diagram] a representation of an interaction between weights and data as a rooted tree equipped with a partition of nodes.  Drawing conventions: thin edges represent the tree, and fuzzy ties indicate the partition.  The root is specified from among the nodes by placing it rightmost on the page.
        \item[embedding] (of a diagram into a spacetime) an assignment of diagram nodes into spacetime cells such that two nodes occupy the same row precisely when they are fuzzily tied and such that leafward-rootward relations along thin edges are reflected as past-future relations.
        \item[entropic force] a macroscopic force arising from a tendency toward disorder and marked by strong temperature (noise) dependence.  For instance, rubber bands are stretchy because of an entropic force.  SGD tends toward minima flat with respect to the covariance due to a non-conservative entropic force.    
        \item[epoch] a temporal interval (within an SGD run) within which each datapoint participates in exactly one update.  Our notation represents the number of epochs as $B\cdot M$.
        \item[generalization gap] for a model trained on some train set, the test loss minus train loss.  Compare to overfitting.
        \item[geometric realization] (of a diagram) the data of a diagram that remains when we consider chains of thin edges as equivalent to single edges.  See also irreducible.
        \item[index] (of a train set) a name for a datapoint that singles it out from among a train set.
        \item[index] (of a stochastic process) a name for a random variable among the collection of random variables.
        \item[index] (of a tensor) a name for a numpy axis.  For example, a shape $a\times b\times c$ tensor will have three index positions. 
        \item[inverse metric] a notion of size for row vectors, in the sense of kernel methods.  For instance, a covariance of weight displacements.  Compare to metric.
        \item[irreducible] (of a diagram) the property of being minimal among diagrams with the same geometric realization.  Concretely, the property that every thin-edge-degree two node participates in some fuzzy edges.  See also geometric realization. 
        \item[landscape] a neural network, considered as a function from weights $\theta$ and data $x$ to losses $l_x(\theta)$. 
        \item[metric] a notion of size for column vectors, in the sense of kernel methods.  For example, a covariance of gradients.  Compare to inverse metric.
        \item[non-conservative] (of a covector field) with nonzero curl.  In striking contrast to the ODE approximation to gradient descent, SGD experiences a non-conservative force.  See also conservative.
        \item[overfitting] the process of responding to noise as if it is signal.  In this work, we quantify overfitting by initializing a weight at a test minimum, then training.  The greater the net gain in test loss, the more we regard the optimization process as having overfitted.  Compare to generalization gap.
        \item[perturbation] the technique of analyzing a complicated system by decomposing it into a simple system plus a small complication, then applying Taylor's theorem to extract the effects of that complication.  
        \item[raising indices] the algebraic step of contracting an inverse metric with a covector to produce a vector.  To use an example from smooth convex optimization in the quadratic case, the step of identifying a point's position from the objective's slope at that point.
        %\item[recursive] in the spirit of Knuth and Hofstadter.  A tempting glossary entry ripe for meta-humor and anti-humor. 
        \item[renormalization] the theory-building process of summarizing myriad small-scale interactions into a more easily manipulated large-scale interaction, often by appeal to scaling and symmetry.  Here and historically, intertwined with resolving issues of convergence.  
        \item[scattering process] the brief interaction of otherwise self-contained and easily-understood objects.  For example, billiard balls scatter off of each other.  And weights scatter off of data.  Often treated perturbatively. 
        \item[spacetime] the stage on which weights and data interact.  A grid-like summary of which training points participate in which gradient updates through time. 
        \item[spring] the familiar simple machine with energy quadratic in its degrees of freedom.  The sequential composition of two stiff springs yields a limper spring.  We thus see that the potential energy of a static spring bearing a weight scales \emph{inversely} with the spring's stiffness.  This story formally parallels the Takeuchi prediction that flat minima generalize divergently badly.
        \item[stabilized takeuchi information criterion] an estimate of generalization gap: $C_{\mu\nu}\wrap{(I - \exp(-\eta T H))}^{\nu}_{\lambda}\wrap{H^{-1}}^{\lambda\mu}$. 
        \item[stochastic process] a collection of related random variables.  For example, the validation losses of two neural networks with different and frozen weights forms a collection of $2$ random variables, where the randomness is over a shared validation set.  See also index (of a stochastic process).
        \item[takeuchi information criterion] an estimate of generalization gap: $C_{\mu\nu}(H^{-1})^{\mu\nu}/N$.  Diverges for small $H$, hence not suited to gradient descent.
        \item[tensor] a numpy array of potentially long shape, e.g. shape $a\times b\times c\times d$.  For example, the collection of $3$rd derivatives of a multivariate function comprise a shape-$p\times p\times p$ tensor. 
        \item[vector] a numpy array of shape $p\times 1$.  For example, a weight displacement.  Compare to covector.
    \end{description}

    \clearpage
    \newpage

    \subsection{
        Diagrams for Computing Test Losses
    }

    \begin{tabular}{p{1.0\textwidth}}
        We present all $3$rd order diagrams relevant to test loss computations.
        Actually, the rows are indexed by topological families of diagrams.  For
        example, the diagrams $\sdia{(0-1-2)(01-12)}$ and
        $\sdia{(0-1-2)(02-12)}$, through distinct as diagrams, are
        topologically equivalent.  They thus have the same unrenormalized value
        (an example of crossing symmetry!), and for brevity we treat them
        in the same row, labeled arbitrarily with one of them (here
        $\sdia{(0-1-2)(02-12)}$).  The interpretation of a diagram as a
        weight-data interaction process depends on the exact diagram, not just
        its topological family.  So the interpretation row should be regarded
        as providing examples instead of being a complete enumeration.
    \end{tabular}    

    \begin{tabular}{p{0.1\textwidth}p{0.45\textwidth}p{0.4\textwidth}}
           \textsc{Diagram}             &  \textsc{Unrenormalized Value}                                                & \textsc{Interpretation} \\ \hline
           $\sdia{(0)()}$               &  $+l$                                                                         & Trivial process: no data-weight interaction 
        \\ $\sdia{(0-1)(01)}$           &  $-\eta^{\mu\nu} G_\mu G_\nu$                                                 & A datapoint directly affects the test loss   
        \\ $\sdia{(01-2)(01-12)}$      &  $+\eta^{\mu\nu}\eta^{\lambda\rho} \wrap{\nabla_\lambda C_{\mu\nu}} G_\rho/2$ & A datapoint affects the test loss through a later encountered instance of itself
        \\ $\sdia{(0-1-2)(02-12)}$      &  $+\eta^{\mu\nu}\eta^{\lambda\rho} G_\mu G_\lambda H_{\nu\rho}$               & Two different datapoints both affect the test loss
        \\ $\sdia{(01-2)(02-12)}$      &  $+\eta^{\mu\nu}\eta^{\lambda\rho} C_{\mu\lambda} H_{\nu\rho}$                & A datapoint twice affects the test loss
        \\ $\sdia{(0-1-2-3)(01-12-23)}$ &  $-\eta^{\mu\nu}\eta^{\lambda\rho}\eta^{\sigma\pi} G_{\mu} H_{\nu\lambda} H_{\rho\sigma} G_{\pi}$& A datapoint affects a different datapoint that affects yet another datapoint that affects the test loss
        \\ $\sdia{(0-1-2-3)(03-13-23)}$ &  $-\eta^{\mu\nu}\eta^{\lambda\rho}\eta^{\sigma\pi} G_{\mu} G_{\lambda} G_{\sigma} J_{\nu\rho\pi}$& Three different datapoints affect the test loss
        \\ $\sdia{(01-2-3)(02-13-23)}$ &  $-\eta^{\mu\nu}\eta^{\lambda\rho}\eta^{\sigma\pi} G_{\mu} H_{\nu\lambda} G_{\sigma} H_{\rho\pi} $& A datapoint twice affects the loss, once directly and once through a later-encountered and different datapoint 
        \\ $\sdia{(0-12-3)(01-13-23)}$ &  $-\eta^{\mu\nu}\eta^{\lambda\rho}\eta^{\sigma\pi} \wrap{G_\sigma \nabla_\pi \wrap{C_{\mu\lambda} H_{\nu\rho}}/2 - C_{\mu\lambda} G_{\sigma} J_{\nu\rho\pi}}$& A datapoint affects a different point that itself twice affects the test loss 
        \\ $\sdia{(01-2-3)(03-13-23)}$ &  $-\eta^{\mu\nu}\eta^{\lambda\rho}\eta^{\sigma\pi} C_{\mu\lambda} G_{\sigma} J_{\nu\rho\pi}$& A datapoint twice affects the test loss while a different datapoint affects the test loss
        \\ $\sdia{(012-3)(03-13-23)}$  &  $-\eta^{\mu\nu}\eta^{\lambda\rho}\eta^{\sigma\pi} \wrap{\expct{\nabla_{\mu} l_x\nabla_{\lambda} l_x\nabla_{\sigma} l_x} - G_{\mu} G_{\lambda} G_{\sigma}}  J_{\nu\rho\pi}$& A datapoint thrice affects the test loss

        \\ $\sdia{(01-2-3)(01-12-23)}$  &  $-\eta^{\mu\nu}\eta^{\lambda\rho}\eta^{\sigma\pi} \wrap{\nabla_\lambda C_{\mu\nu}} H_{\rho\sigma} G_\pi/2$      &   A datapoint affects a later instance of itself, which affects a different point, which affects the test loss
        \\ $\sdia{(0-12-3)(01-12-23)}$  &  $-\eta^{\mu\nu}\eta^{\lambda\rho}\eta^{\sigma\pi} \wrap{\expct{\nabla_\nu \nabla_\lambda l_x \nabla_\rho\nabla_\sigma l_x} - H_{\nu\lambda} H_{\rho\sigma}} G_\mu G_\pi$      &   A datapoint affects another datapoint, which in turn affects a later instance of itself, which affects the test loss
        \\ $\sdia{(012-3)(01-12-23)}$  &  $-\eta^{\mu\nu}\eta^{\lambda\rho}\eta^{\sigma\pi} \wrap{\expct{\nabla_{\mu} l_x \nabla_\nu \nabla_{\lambda} l_x \nabla_\rho \nabla_{\sigma} l_x} - G_{\mu} H_{\nu \lambda} H_{\rho\sigma}} G_\pi$      &    A datapoint affects a later instance of itself, which affects a yet later instance of itself, which affects the test loss
        \\ $\sdia{(012-3)(03-12-23)}$  &  $-\eta^{\mu\nu}\eta^{\lambda\rho}\eta^{\sigma\pi} \wrap{\expct{\nabla_{\mu} l_x \nabla_\lambda l_x \nabla_\rho \nabla_{\sigma} l_x} - G_{\mu} G_{\lambda} H_{\rho\sigma}} H_{\nu\pi}$      &    A datapoint affects the test loss twice, once directly, and once through a later instance of itself 
        \\ $\sdia{(0-12-3)(02-12-23)}$  &  $-\eta^{\mu\nu}\eta^{\lambda\rho}\eta^{\sigma\pi} \wrap{\expct{\nabla_\lambda l_x \nabla_\nu\nabla_\rho\nabla_\sigma l_x} - G_{\lambda} J_{\nu\rho\sigma}} G_\mu G_\pi$     &    A datapoint affects the test loss through a later encountered instance of itself, which was earlier affected by a different datapoint
        \\ $\sdia{(012-3)(02-12-23)}$  &  $-\eta^{\mu\nu}\eta^{\lambda\rho}\eta^{\sigma\pi} \wrap{\expct{\nabla_{\mu} l_x \nabla_\lambda l_x \nabla_\nu \nabla_\rho \nabla_{\sigma} l_x} - G_{\mu} G_{\lambda} J_{\nu\rho\sigma}} G_\pi$      &    A datapoint twice affects a later instance of itself that in turn affects the test loss
    \end{tabular}

\end{document}
