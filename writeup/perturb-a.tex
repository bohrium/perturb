
%==============================================================================
%    LATEX PREAMBLE  
%==============================================================================

\documentclass{article}
\usepackage[T1]{fontenc}
\usepackage{microtype}
\usepackage{graphicx, float}
\usepackage{subfigure}
\usepackage{booktabs} % for professional tables
\usepackage{hyperref, xcolor}
\usepackage{amsmath, amssymb, amsthm, hanging, graphicx, txfonts, ifthen}
        
\usepackage[percent]{overpic}

\newcommand{\theHalgorithm}{\arabic{algorithm}}
\newtheorem{klem}{Key Lemma}
\newtheorem{thm}{Theorem}
\newtheorem{cor}{Corollary}
\newtheorem{prop}{Proposition}
\newtheorem{quest}{Question}
\newtheorem{rmk}{Remark}

\usepackage{icml2019}
%\usepackage[accepted]{icml2019}

\usepackage{array}   % for \newcolumntype macro
\newcolumntype{L}{>{$}l<{$}}

\definecolor{moor}{rgb}{0.8,0.2,0.2}
\definecolor{moog}{rgb}{0.2,0.8,0.2}
\definecolor{moob}{rgb}{0.2,0.2,0.8}

\renewcommand*{\thefootnote}{\color{red}\fnsymbol{footnote}} 

\newcommand{\Free}{\mathcal{F}}
\newcommand{\Forg}{\mathcal{G}}
\newcommand{\Mod}{\mathcal{M}}
\newcommand{\Hom}{\text{\textnormal{Hom}}}
\newcommand{\image}{\text{\textnormal{im}}}
\newcommand{\dvalue}{\text{\textnormal{value}}}
\newcommand{\edges}{\text{\textnormal{edges}}}
\newcommand{\ords}{\text{\textnormal{ords}}}
\newcommand{\parts}{\text{\textnormal{parts}}}
\newcommand{\Ss}{\mathcal{S}}
\newcommand{\Pp}{\mathcal{P}}
\newcommand{\Nn}{\mathcal{N}}
\newcommand{\SGD}{\text{\textnormal{SGD}}}
\newcommand{\RR}{\mathbb{R}}
\newcommand{\expc}{\mathbb{E}}
\newcommand{\expct}[1]{\mathbb{E}\left[#1\right]}
\newcommand{\wrap}[1]{\left(#1\right)}
\newcommand{\wasq}[1]{\left[#1\right]}
\newcommand{\wang}[1]{\left\langle#1\right\rangle}
\newcommand{\wive}[1]{\left\llbracket#1\right\rrbracket}
\newcommand{\worm}[1]{\left\|#1\right\|}
\newcommand{\wabs}[1]{\left|#1\right|}
\newcommand{\wurl}[1]{\left\{#1\right\}}
\newcommand{\partbox}[1]{
    \text{
        \fboxsep=0.5pt
        \tiny
        \fbox{#1}
    }
}
\DeclareMathOperator*{\Avg}{\text{\sffamily A}}

\newcommand{\plotplace}[3]{
    \begin{overpic}[width=#2, height=#3]{../plots/blank.png}
        \put( 5, 85){
            \begin{tabular}{p{#2-1.0cm}}
                #1
            \end{tabular}
        }
    \end{overpic}
}

\newcommand{\plotmoo}[3]{
    \includegraphics[width=#2           ]{../#1}
}
\newcommand{\plotmooh}[3]{
    \includegraphics[          height=#3]{../#1}
}

\newcommand{\bdia}[1]{\begin{gathered}\includegraphics[scale=0.22]{../diagrams/#1.png}\end{gathered}}
\newcommand{\dia} [1]{\begin{gathered}\includegraphics[scale=0.18]{../diagrams/#1.png}\end{gathered}}
\newcommand{\mdia}[1]{\begin{gathered}\includegraphics[scale=0.14]{../diagrams/#1.png}\end{gathered}}
\newcommand{\sdia}[1]{\begin{gathered}\includegraphics[scale=0.10]{../diagrams/#1.png}\end{gathered}}

\newcommand{\half}{\frac{1}{2}}
\newcommand{\sixth}{\frac{1}{6}}

\newcommand{\ofsix}[1]{
    {\tiny $\substack{
        \ifthenelse{\equal{#1}{0}}{\blacksquare}{\square}
        \ifthenelse{\equal{#1}{1}}{\blacksquare}{\square} \\
        \ifthenelse{\equal{#1}{2}}{\blacksquare}{\square} 
        \ifthenelse{\equal{#1}{3}}{\blacksquare}{\square} \\
        \ifthenelse{\equal{#1}{4}}{\blacksquare}{\square}
        \ifthenelse{\equal{#1}{5}}{\blacksquare}{\square}
    }$}
}


\newcommand{\lorem}[1]{
    Lorem ipsum dolor sit amet, consectetur adipiscing elit...\\
    \nopagebreak\vspace{#1cm} \ \\
    ...sunt in culpa qui officia deserunt mollit anim id est laborum.
}


\begin{document}

%==============================================================================
%       TITLE AND AUTHOR
%==============================================================================

\icmltitlerunning{Descent as Scattering}

\twocolumn[
    \icmltitle{A Space-Time Approach to Analyzing Stochastic Gradient Descent}
    
    \begin{icmlauthorlist}
        \icmlauthor{Samuel C.~Tenka}{mit}
    \end{icmlauthorlist}
    \icmlaffiliation{mit}{
        Computer Science and Artificial Intelligence Lab,
        Massachusetts Institute of Technology,
        Cambridge, Massachusetts, USA
    }
    \icmlcorrespondingauthor{Samuel C.~Tenka}{coli@mit.edu}
    
    \icmlkeywords{Machine Learning, SGD, ICML}
    
    \vskip 0.3in
]
\printAffiliationsAndNotice{}

%==============================================================================
%       ABSTRACT        
%==============================================================================

\begin{abstract}
    We adapt Feynman Diagrams to reason about the behavior, at small learning
    rates $\eta$, of SGD and its variants.  We interpret the diagrams as
    histories of scattering events, thus offering a new physical picture of
    descent.
    %
    Illustrating this technique, we
        construct a regularizer that causes large-batch GD to emulate
        small-batch SGD,
        %
        exhibit a landscape that SGD eternally
        \emph{climbs}
        despite well-behaved boundary conditions and zero curvature in
        the direction of motion,
        %
        generalize the
        %Akaike Information Criterion
        AIC to a smooth quantity liable to descent, and
        %
        quantify the differences --- due to time-discretization, inter-epoch
        correlations, and non-gaussian noise --- between SGD and the popular
        approximation SDE. 
\end{abstract}

%==============================================================================
%       INTRODUCTION    
%==============================================================================

\section{Introduction}
    %\subsection{Overview}

        %----------------------------------------------------------------------
        %       Orienting Invitation 
        %----------------------------------------------------------------------

        Stochastic gradient descent (SGD) decreases an unknown objective $l$ by
        performing discrete-time $\eta$-steepest\footnote{
            The ``steepness'' concept depends on the choice of metric for
            $l$'s domain.  We thus consider all (flat) metrics at once.
            Specifically, we will Taylor expand an (inverse) metric
            $\eta^{\mu\nu}$ around $0$.
        } descent on noisy estimates of $l$.  A key question is how such noise
        affects the final objective value.  We connect SGD dynamics to physical
        scattering theory, thus providing a quantitative and qualitative
        toolkit for answering this question.

        %----------------------------------------------------------------------
        %       Killer Applications  
        %----------------------------------------------------------------------

        Specifically, we derive a diagram-based formalism for reasoning about
        SGD via a path integral over possible interactions between weights and
        data.  The formalism permits perturbative analysis, leading to
        predictions of learning curves for small $\eta$.  Unlike the
        continuous-time limits of prior work, this framework models discrete
        time, and with it, the potential \emph{non-Gaussianity} of noise.  We
        thus obtain new results quantifying the effect of \emph{epoch number
        and batch size} on SGD test loss.  We also \emph{contrast SGD against
        popular continuous-time approximations} such as ordinary or stochastic
        differential equations (ODE, SDE): our work gives the finite-$N$,
        finite-$\eta^{-1}$ corrections to these approximations.  From this
        theory, we propose two new regularizers that respectively induce GD to
        mimic SGD and help to tune hyperparameters such as $l_2$ coefficients.
        
        %----------------------------------------------------------------------
        %       Connection to Physics
        %----------------------------------------------------------------------

        Path integrals offer not only quantitative predictions but also an
        exciting new viewpoint --- that of iterative optimization as a
        \emph{scattering process}.  As individual Feynman diagrams 
        \citep{dy49a} depict how local particle interactions compose into
        global outcomes, our diagrams depict how individual SGD updates
        influence each other before affecting a final test loss.  In fact, we
        import physical tools such as \emph{crossing symmetries} 
        \citep{dy49b} and \emph{re-normalization} \citep{ge54} to
        simplify and refine our calculations.  The diagrams' combinatorics
        yield precise qualitative conclusions as well, e.g. that to
        order $\eta^3$, \emph{inter-epoch shuffling} does not affect expected
        test loss.

    \subsection{Appetizing Example: The Role of Epoch Number}
        %----------------------------------------------------------------------
        %       Epoch Number Example     
        %----------------------------------------------------------------------
        Deferring explanation, we use diagrams for a sample problem: to compare
        the expected test losses of one-epoch and multi-epoch SGD (say, with
        learning rates scaled proportionally).  In our formalism, these test
        losses are the \emph{sum over all diagrams embedded in spacetime},
        where the spacetime encodes which samples are seen at which times and
        each diagram evaluates to a specific statistic of the loss landscape.
        One-epoch and multi-epoch SGD differ only in their datapoint-timestep
        schedules, i.e. only in their spacetimes. 

        \begin{figure}[h!] 
            \centering  
            \plotmooh{diagrams/spacetime-c}{0.32\columnwidth}{0.32\columnwidth}
            \plotmooh{diagrams/spacetime-d}{0.64\columnwidth}{0.32\columnwidth}
            \caption{
                Some diagrams embedded in spacetime (gray) for one-epoch ({\bf
                left}) and two-epoch ({\bf right}) SGD.  The diagrams for one
                and two epochs correspond by horizontal shifts \emph{except}
                when same-row nodes are connected --- such as in the top right
                diagram $\sdia{c(01-2)(01-12)}$.  Colors serve no role beyond
                convenient reference.  Nodes (here black) outside the grid
                represent a post-training measurement of test loss; only
                convenience of drawing dictates their coordinates.
            }
            \label{fig:epoch}
        \end{figure}

        The rules are that embedded diagrams \emph{may not contain edges between
        nodes in the same spacetime cell}
        and that $d$-edged diagrams contribute at order $\eta^d$.
        %and that diagram values are invariant to node-by-node horizontal shifts and permutation of rows.
        So to order $\eta^2$, multi epoch SGD has, in addition to the
        diagrams of single epoch SGD, also 
        $\sdia{c(01-2)(01-12)}$
        (see Figure \ref{fig:epoch}).
        Since this diagram embeds in $M$-epoch spacetime in
        $N\cdot{M\choose 2}$ many ways, and since we scaled the learning rate
        by $1/M$, the order $\eta^2$ discrepancy is
        $$
            \text{test loss difference} =  
            \frac{N\cdot {M\choose 2}}{M^2}~
            \dvalue\wrap{\sdia{c(01-2)(01-12)}}
            + o(\eta^2)
        $$
        We complete the calculation by reading from the diagram's graph
        structure that it evaluates to
        \begin{align*}
            \dvalue\wrap{\sdia{c(01-2)(01-12)}}
            =
            &\wrap{
                \expc_{x}\wasq{\nabla_\mu l_x \nabla_\nu \nabla_\lambda l_x} 
                -
                \expc_{x}\wasq{\nabla_\mu l_x}
                \expc_{x}\wasq{\nabla_\nu \nabla_\lambda l_x} 
            } \cdot \\
            &\expc_{x}\wasq{\nabla_\rho l_x}
            \wrap{-\eta^{\mu\nu}} \wrap{-\eta^{\lambda\rho}}
        \end{align*}

        In using diagrams above, we were able to match terms so that all but
        the $\sdia{c(01-2)(01-12)}$ terms canceled, thus delaying and
        diminishing a substantial algebraic mess.  In fact, diagrams  
        concisely represent correlation and tensor-contraction patterns that
        when written explicitly are complicated and when suppressed lead to
        ambiguities (consider Proposition \ref{prop:entropic}, which makes
        use of 4-valent tensors).  Moreover, when we later \emph{renormalize}
        in order to extend our results to the large-$\eta T$ regime (sections
        \ref{subsect:effective}, \ref{subsect:entropic}, and
        \ref{subsect:overfit}), the topology of diagrams will play a key role.  
        While diagrams may in principle be completely avoided,
        Appendix D shows that to avoid them can lead to
        substantively longer and more complicated analyses.   

%==============================================================================
%       BACKGROUND AND NOTATION
%==============================================================================

\section{Background and Notation}

    %--------------------------------------------------------------------------
    %           Tensor Conventions
    %--------------------------------------------------------------------------

    \subsection{A Smooth Stage: Tensor Conventions}
        We adopt summation notation for Greek indices.  To expedite dimensional
        analysis, we regard the learning rate as an inverse metric
        $\eta^{\mu\nu}$ that converts a gradient into a displacement
        (\cite{bo13}).  We use $\eta$ to raise indices; for example, with $C$
        denoting the covariance of gradients, its ``trace'' will be
        $C^{\mu}_{\mu} = \eta^{\mu\nu} C_{\mu\nu}$.  Standard syntactic
        constraints make manifest which expressions transform naturally with
        respect to optimization dynamics.

        We assume some regularity conditions on the loss landscape.  We list
        these conditions in Appendix A.  They hold, for example, for $\tanh$ or
        softplus neural networks with arbitrary weight sharing and skip
        connections trained with with cross entropy loss on bounded data.  We
        use $G, H, J$ for the first, second, and third derivatives of $l$.
        
        Kol\'{a}\u{r} gives a careful introduction to these differential
        geometric ideas \yrcite{ko93}.

    %--------------------------------------------------------------------------
    %           Combinatorial Structures
    %--------------------------------------------------------------------------
    
    \subsection{Combinatorial Costumes: Structure Sets}
        To precisely define diagrams and spacetimes, we define \emph{structure
        sets}, i.e. sets $S$ equipped with a preorder $\leq$ and an equivalence
        relation $\sim$.  The morphisms of structure sets are strictly
        increasing maps that preserve $\sim$ and its negation.  A structure set
        is \emph{pointed} if it has a unique maximum element and this element
        forms a singleton $\sim$-class.  The categories $\Ss$ of structure sets
        and $\Pp$ of pointed structure sets enjoy a free-forgetful adjunction
        $\Free, \Forg$.  When $\leq$ is a total preorder, we say that $S$ is
        a \emph{spacetime}.
    
        A \emph{diagram} is a rooted tree equipped with an equivalence relation
        $\sim$ on nodes.  We draw the tree with thin edges, with the root at
        the far right, and we indicate $\sim$ with fuzzy ties.  Read as a Hasse
        graph (with the root maximal), the diagram $D$ induces a structure set,
        by abuse of notation also named $D$.  Let $\parts(D)$ give the parts of
        $D$.  An $\Ss$-map from $D$ to
        $(\Forg\circ\Free)^{|\parts(D)|}(\text{empty set})$ is an \emph{ordering} of $D$.
        Let $|\edges(D)|$ and
        $|\ords(D)|$ count edges and orderings of $D$.
    
        Fong gives a swift introduction to these category theoretic and
        diagrammatic ideas \yrcite{fo19}.
            
    %--------------------------------------------------------------------------
    %           Forms of SGD          
    %--------------------------------------------------------------------------

    \subsection{The Parameterized \emph{Personae}: Forms of SGD}
        SGD decreases an objective $l$ by updating on smooth, unbiased i.i.d.
        estimates $(l_n: 0\leq n<N)$ of $l$.  The pattern of updates is
        determined by a spacetime $S$: for a map
        $\pi:S\to [N]$ that induces $\sim$, we define SGD inductively as
        $\text{SGD}_{S}(\theta) = \theta$ when $S$ is empty and otherwise
        $$
            \SGD_S(\theta) =
                \SGD_{S\setminus M}(\theta^\mu - \eta^{\mu\nu} \nabla_\nu l_{M}(\theta))
        $$
        where $M = \min S \subseteq S$ specifies a batch and $l_M = \sum_{m\in
        M} l_{\pi(m)} / \wabs{M}$ is a batch average.  Since the distribution
        of $l_n$ is permutation invariant, the non-canonical choice of $\pi$
        does not affect the distribution of output $\theta$s.
    
        Of special interest are spacetimes that divide sequentially into
        $M\times B$ many \emph{epochs} each with $N/B$ many disjoint
        \emph{batches} of size $B$.  An SGD instance is then determined by $N,
        B, M$, and an \emph{inter-epoch shuffling scheme}.  The cases $B=1$ and
        $B=N$ we call \emph{pure SGD} and \emph{pure GD}.  The $M=1$ case of
        pure SGD we call \emph{vanilla SGD}.

%==============================================================================
%    DIAGRAM CALCULUS FOR SGD
%==============================================================================

\section{Diagram Calculus for SGD}

    %--------------------------------------------------------------------------
    %           Role of Diagrams      
    %--------------------------------------------------------------------------

    \subsection{How Diagrams Arise}
        Suppose $s$ is analytic on weight space; e.g. $s$ may be a test loss.
        We track $s(\theta)$ as $\theta$ evolves by SGD:
        \begin{klem}[A Dyson Series] \label{lem:dyson}
            The Maclaurin series of $s(\theta_T)$ with respect to $\eta$ is:
            \begin{equation}\label{eq:dyson}
                \sum_{(d_t: 0\leq t<T)}
                (-\eta)^{\sum_t d_t}
                \left(
                    \prod_{0 \leq t < T}
                        \left.  \frac{(g \nabla)^{d_t}}{d_t!} \right|_{g=\nabla l_t(\theta)}
                \right)
                (s) (\theta_0)
            \end{equation}
        \end{klem}
        In averaging over training sets $(l_t: 0\leq t<T)$ we may factor the
        expectation of the above product according to independence relations
        between the $l_t$.  We view various training procedures (e.g. pure GD,
        pure SGD) as \emph{prescribing different independence relations} that
        lead to different factorizations and hence to potentially different
        generalization behavior at each order of $\eta$.
    
        An instance of the above product (for $s=l_a$ drawn from a test set and
        $0\leq c\leq b<T$) is $-\eta^3 (\nabla l_c \nabla)^2 (\nabla l_b
        \nabla) l_a$, which is
        {\small
        \begin{align*}
            - (\nabla^\lambda l_c) (\nabla^\mu l_c) (\nabla_\lambda \nabla_\mu \nabla^\nu l_b) (\nabla_\nu l_a)   
            - (\nabla^\lambda l_c) (\nabla^\mu l_c) (\nabla_\lambda \nabla^\nu l_b) (\nabla_\mu \nabla_\nu l_a) \\
            - (\nabla^\lambda l_c) (\nabla^\mu l_c) (\nabla_\mu \nabla^\nu l_b) (\nabla_\lambda \nabla_\nu l_a)   
            - (\nabla^\lambda l_c) (\nabla^\mu l_c) (\nabla^\nu l_b) (\nabla_\lambda \nabla_\mu \nabla_\nu l_a)
        \end{align*}
        }
        To reduce clutter, we adapt the string notation of \citet{pe71}.  Then,
        in expectation over $(l_c, l_b, l_a)$ drawn i.i.d.:
        \begin{align}
            \cdots
            &= 
                 \sdia{(01-2-3)(02-12-23)}
                +\sdia{(01-2-3)(02-13-23)}
                +\sdia{(01-2-3)(03-12-23)}
                +\sdia{(01-2-3)(03-13-23)} \\
                \label{eq:simpl}
            &=
                \underbrace{2\sdia{(01-2-3)(02-12-23)}}_{
                   -2~\expct{{\color{moor}(\nabla l)(\nabla l)}}~\expct{{\color{moog}\nabla\nabla\nabla l}}~\expct{{\color{moob} \nabla l}}
                }
                +
                \underbrace{2\sdia{(01-2-3)(02-13-23)}}_{
                   -2~\expct{{\color{moor}(\nabla l)(\nabla l)}}~\expct{{\color{moog}\nabla \nabla l}}~\expct{{\color{moob}\nabla \nabla l}}
                }
        \end{align}
        Above, each degree-$g$ node corresponds to an $l_n$ (here,
            {\color{moor} $l_c$},
            {\color{moog} $l_b$},
            {\color{moob} $l_a$}) differentiated $g$ times
        (for instance, {\color{moog} $l_b$} is differentiated thrice in the
        first diagram and twice in the second).  Thin \emph{edges} mark
        contractions by $-\eta$.  Fuzzy \emph{ties} denote correlations by
        connecting identical loss functions (here, {\color{moor} $l_c$} with
        {\color{moor} $l_c$}).  The colors are redundant with the fuzzy ties.
        A diagram $D$ evaluates to the expected value $\dvalue(D)$ of the
        corresponding tensor expression.
        % 
        Usefully, for a fixed, i.i.d.  distribution over $(l_c, l_b, l_a)$,
        \emph{the topology of a diagram determines its value}.  For instance,
        $
            \dvalue\wrap{\sdia{(01-2-3)(02-12-23)}}
            =
            \dvalue\wrap{\sdia{(01-2-3)(03-13-23)}}
        $.
        Thus follows simplification \ref{eq:simpl}.
        We will sometimes write a diagram $D$ and mean $\dvalue(D)$. 
        %We may convert back to
        %explicit tensor expressions, invoking independence between untied nodes
        %to factor the expression.  However, as we will see, the diagrams offer
        %physical intuition, streamline computations, and determine useful
        %unbiased estimators of the statistics they represent.  
    
        Since many of our results concern the effect of correlations, it will
        be convenient to notate them concisely.  We thus define a diagram with
        fuzzy outlines to be the difference between the fuzzy tied and untied
        versions : $\sdia{c(01-2)(01-12)} =
        \sdia{(01-2)(01-12)}-\sdia{(0-1-2)(01-12)}$.  
        %
        The recipes for writing down test (or train) losses of SGD and its
        variants are now straightforward.  They reduce the problem of
        evaluating the dynamical expressions to the problem of
        counting graph embeddings.  The more complicated the direct
        computation, the greater the savings of using diagrams.  
    
    %--------------------------------------------------------------------------
    %           Recipe for Test Loss and Generalization Gap
    %--------------------------------------------------------------------------

    \subsection{Recipe for SGD's Test Loss and Generalization}

        Our results all follow from this theorem and its analogues.
        Throughout, the $d$-edged diagrams give the order $\eta^d$ terms.
        Though our methods immediately generalize, we will will for notational
        convenience only consider SGD variants with time-invariant batch size
        $B$. 
        \begin{thm}[Test Loss as a Path Integral] \label{thm:test}
            SGD's expected test loss's MacLaurin series
            is a weighted sum of diagrams:
            \begin{equation}\label{eq:sgdcoef}
                \sum_{D \in \image(\Free)}
                \left(
                    \sum_{f: D\to\Free(S)}
                    \prod_{i\in S} \frac{1}{|f^{-1}(i)|!}
                \right)
                \frac{\dvalue(D)}{B^{|\edges(D)|}}
            \end{equation}
            Here, $D$ is (an isomorphism class of) a diagram of form
            $\Free(T)$ and $f$ is a morphism in $\Pp$.
        %\end{thm}
        %\begin{thm}[Generalization Gap as a Path Integral]
            Moreover, if we replace each $\dvalue(D)$ by
            $
                \sum_{p \in \parts(D)} \dvalue(D_p)/N
            $, we obtain the series for
            SGD's expected generalization gap (test loss minus train loss).
            Here, $D_p$ is $D$ with a fuzzy outline connecting $D$'s maximal
            node to $p$, e.g. $(\sdia{(0-1)(01)})_{p=\sdia{(0)()}} =
            \sdia{c(01)(01)}$.
        \end{thm}
    
        In the special case of $B=1, M=1$:
        \begin{prop}[Specialization to Vanilla SGD] \label{prop:vanilla}
            The order $\eta^d$ contribution to the expected test loss of
            one-epoch SGD with singleton batches is:
            \begin{equation}\label{eq:sgdbasiccoef}
                \frac{(-1)^d}{d!} \sum_{D\in \image(\Free)} 
                |\ords(D)| {N \choose P-1} {d \choose d_0,\cdots,d_{P-1}}
                \dvalue(D)
            \end{equation}
            where $D$ ranges over $d$-edged diagrams whose equivalence classes
            have sizes $d_p: 0\leq p\leq P$, with $d_P=1$
            and, without loss, are each antichains.  The modification to
            compute the generalization gap is the same as in Theorem
            \ref{thm:test}.
        \end{prop}
        A $P$-part, $d$-edged diagram then contributes $\Theta\left((\eta N)^d
        N^{P-d-1}\right)$ to the loss.  For example, there are six diagrams to
        third order, and they have $(4+2)+(2+2+3)+(1)$ many orderings --- see
        Table \ref{tab:scatthree}.  Intuitively, $\eta N$ measures the \emph{
        physical time} of descent, and $1/N$ measures the \emph{coarseness} of
    time discretization.  So we have a double-series in $(\eta N)^d N^{P-d-1}$,
    where $d$ counts thin edges and $d+1-P$ counts fuzzy ties; the $P=d+1$
    terms correspond to a discretization-agnostic (hence continuous-time,
    noiseless) ODE approximation to SGD, while $P\leq d$ gives correction terms
    modeling time-discretization and hence noise.  

    %--------------------------------------------------------------------------
    %           Theoretical Corollaries                     
    %--------------------------------------------------------------------------

    \subsection{Consequences of the Recipe}

        \begin{cor}[SGD Differs from ODE and SDE] \label{cor:vsode}
            For one-epoch SGD on singleton batches through fixed physical time
            $T$: the order $N^{-1}$ deviation of SGD's test loss from ODE's is
            $
                ({{T^2 N^{-1}}/{2}}) \sdia{c(01-2)(02-12)}
            $.
            The order $N^{-2}$ deviation of SGD's test loss due to non-gaussian
            noise is
            $
                -({{T^3 N^{-2}}/{6}}) (\sdia{c(012-3)(03-13-23)} - 3 \sdia{c(01-2-3)(03-13-23)})
            $.
        \end{cor}
        For finite $N$, these effects make SDE different from SGD.  SDE also
        fails to model the correlations between updates in multiepoch SGD.  On
        the other hand, in the $N=\infty$ limit for which SDE matches SGD,
        optimization and generalization become computationally intractable and
        trivial, respectively. 
    
        \begin{table}[h]
            \centering 
            \resizebox{\columnwidth}{!}{%
            \begin{tabular}{c|c|c}
                {\LARGE $\Theta\left((\eta N)^3 N^{-0}\right)$} &
                {\LARGE $\Theta\left((\eta N)^3 N^{-1}\right)$} &
                {\LARGE $\Theta\left((\eta N)^3 N^{-2}\right)$} \\ \hline
                \begin{tabular}{c}
                    \begin{tabular}{LL}
                        \bdia{(0-1-2-3)(01-12-23)} & \bdia{(0-1-2-3)(01-13-23)}
                    \end{tabular} \\
                    \begin{tabular}{LL}
                        \bdia{(0-1-2-3)(02-13-23)} & \bdia{(0-1-2-3)(03-12-23)}
                    \end{tabular} \\ \hline
                    \begin{tabular}{LL}
                        \bdia{(0-1-2-3)(03-13-23)} & \bdia{(0-1-2-3)(02-12-23)}
                    \end{tabular}
                \end{tabular}
                &
                \begin{tabular}{c}
                    \begin{tabular}{LL}
                        \bdia{(01-2-3)(02-13-23)} & \bdia{(01-2-3)(03-12-23)}
                    \end{tabular} \\ \hline
                    \begin{tabular}{LL}
                        \bdia{(0-12-3)(01-13-23)} & \bdia{(0-12-3)(02-13-23)}
                    \end{tabular} \\ \hline
                    \begin{tabular}{LLL}
                        \bdia{(01-2-3)(03-13-23)} & \bdia{(0-12-3)(03-13-23)} & \bdia{(01-2-3)(02-12-23)} 
                    \end{tabular}
                \end{tabular}
                &
                \begin{tabular}{c}
                    \begin{tabular}{L}
                        \bdia{(012-3)(03-13-23)}
                    \end{tabular}
                \end{tabular}
            \end{tabular}
            }
            \caption{
                Degree-$3$ scattering diagrams for $B=M=1$ SGD's test loss.
                {\bf Left:} $(d, P) = (3, 3)$.  Diagrams for ODE behavior.
                {\bf Center:} $(d, P) = (3, 2)$.  $1$st order deviation of SGD
                away from ODE.
                {\bf Right:} $(d, P) = (3, 1)$.  $2$nd order deviation of SGD
                from ODE with appearance of non-Gaussian statistics.
            }
            \label{tab:scatthree}
        \end{table}
    
        A quick combinatorial argument shows:
        \begin{cor}[Shuffling Barely Matters] \label{cor:shuffle}
            To order $\eta^3$, inter-epoch shuffling doesn't affect SGD's
            expected test loss.
        \end{cor}
        Indeed, for any inter-epoch shuffling scheme: 
        \begin{prop}\label{prop:ordtwo}
            To order $\eta^2$, the test loss of SGD --- on $N$
            samples for $M$ epochs with batch size $B$ dividing $N$ and with any
            shuffling scheme --- has expectation
            {\small
            \begin{align*}
                                                        \mdia{(0)()}
                &+ MN                                   \mdia{(0-1)(01)}
                 + MN\wrap{MN - \frac{1}{2}}            \mdia{(0-1-2)(01-12)} \\
                &+ MN\wrap{\frac{M}{2}}                 \mdia{c(01-2)(02-12)}  
                 + MN\wrap{\frac{M-\frac{1}{B}}{2}}     \mdia{c(01-2)(01-12)}
            \end{align*}
            }
        \end{prop}
    
        \begin{cor}[The Effect of Epoch Number] \label{cor:epochs}
            To order $\eta^2$, one-epoch SGD has 
            $
                 \wrap{\frac{M-1}{M}}\wrap{\frac{B+1}{B}}\wrap{\frac{N}{2}} \sdia{c(01-2)(01-12)}
            $
            less test loss than $M$-epoch SGD with learning rate $\eta/M$. 
        \end{cor}
    
        Given a smooth unbiased estimator $\hat{C}$ of gradient covariance, we
        may cause GD to mimic SGD:
        \begin{cor}[The Effect of Batch Size] \label{cor:batch}
            The expected test loss of pure SGD is, to order $\eta^2$,
            less than that of pure GD by
            $
                  \wrap{\frac{M(N-1)}{2}} \sdia{c(01-2)(01-12)}
            $.
            Moreover, GD on a modified loss 
            $
                \tilde l_n = l_n + \wrap{\frac{N-1}{4N}} \hat{C}_\nu^\nu(\theta)
            $
            has an expected test loss that agrees with SGD's to second order.
        \end{cor}
    
        \begin{prop}[An Entropic Force, Unrenormalized] \label{prop:noncons}
            When vanilla SGD is initialized at a test minimum, the weight is
            displaced to order $\eta^3$ by 
            $
                {T \choose 2} \sdia{c(01-2-3)(02-12-23)} / 2
            $.
        \end{prop}
        Proposition \ref{prop:noncons} was first proven by \citet{ya19b}, and
        our methods reproduce it elegantly.  We will renormalize it to
        Proposition \ref{prop:entropic}, which will apply for large times, 
        hence demonstrating a persistent and non-conservative entropic force.
    
    %--------------------------------------------------------------------------
    %           Descent as Scattering                       
    %--------------------------------------------------------------------------

    \subsection{Descent as Scattering}
        In short, SGD's test loss is a weighted sum of diagrams, each $d$-edged
        diagram contributing to order $\eta^d$.  We depict the $\Ss$-maps 
        $f:D\to S$ of Theorem \ref{thm:test} as embeddings of the graph $D$
        into spacetime $S$.  Thus, SGD's test loss is a sum over all embeddings
        in spacetime; see Figure (\ref{fig:spacetime}).  This loss depends on
        the shape of spacetime, which encodes correlations between updates.  To
        compute a test loss, we simply count embeddings.  For instance, the
        order $\eta^2$ diagrams in (\ref{fig:spacetime}) all contribute the
        same amount to test loss, so we just need to find out how many such
        embedded diagrams there are.   
        \begin{figure}[h!] 
            \centering  
            \plotmoo{diagrams/spacetime}{\columnwidth}{3.0cm}  
            \caption{
                Some diagrams embedded in spacetime.  The left four diagrams
                give order $\eta^1$, $\eta^1$, $\eta^3$, and $\eta^5$
                contributions to pure GD's test loss.  The right four each
                contribute $\eta^2 \expct{\nabla^2} \expct{\nabla}^2$; their
                equivalence demonstrates crossing symmetry.  Only diagrams
                whose nodes fall within shaded diagonals contribute to pure SGD's
                test loss (with an extra factor $N$ per edge). 
            }
            \label{fig:spacetime}
        \end{figure}
        \begin{figure}[h!] 
            \centering  
            \plotmoo{diagrams/spacetime-b}{\columnwidth}{3.0cm}
            \caption{
                Comparison of pure GD's vs pure SGD's test loss.  We may
                normalize almost every order $\eta^2$ GD diagram to an
                equivalent SGD diagram by horizontal or vertical shifts (see
                left ten diagrams).  The exception is that $MN{N\choose 2}$ many
                $\sdia{(01-2)(01-12)}$s turn into $\sdia{(0-1-2)(01-12)}$s (see
                right two diagrams).  So pure GD's test loss exceeds pure SGD's
                test loss by $M ((N-1)/2) \sdia{c(01-2)(01-12)}$.
            }
            \label{fig:vsmulti}
        \end{figure}
        Likewise, as shown in Figure (\ref{fig:vsmulti}), the order $\eta^2$
        diagrams of pure GD and pure SGD are nearly in correspondence, except
        for a discrepancy that shows the two test losses differ by $M ((N-1)/2)
        \sdia{c(01-2)(01-12)}$.
        %This argument generalizes to yield proposition \ref{prop:ordtwo}.
    
    %--------------------------------------------------------------------------
    %           Toward Effective Theories                   
    %--------------------------------------------------------------------------

    \subsection{Effective Theories} \label{subsect:effective}
        An important idea is that of \emph{renormalization}, i.e. the
        summarization of myriad small-scale interactions into an effective
        large-scale theory.  We will renormalize --- specifically, we will
        pair main diagrams with topologically related modifications in order
        to refine our $d$th order estimates for any fixed $d$.  For
        some large-$\eta T$ limits in a positive-Hessian setting, the
        unrenormalized theory does not converge while the renormalized theory
        does.  Thus, renormalization will help us reason about large-time
        behavior. 

        For example, consider the diagrams that are uncorrelated chains ---
        $\sdia{(0-1)(01)}, \sdia{(0-1-2)(01-12)},
        \sdia{(0-1-2-3)(01-12-23)}, \cdots$.  When embedded with initial and
        final nodes separated by duration $t$, this series of diagrams in sum
        contributes
        $
            G (I-\eta H)^{t-1} \eta G
            \approx
            G \exp(-\eta T H) \eta G
            +
            o(\eta)
        $.
        We may thus organize diagrams together by the homeomorphism classes of
        their \emph{geometric realizations}; each class yields a sum over
        durations.  For example, the above chains contribute 
        $G \sum_{0\leq t<T} (I-\eta H)^{T-t-1} \eta G$
        to vanilla SGD's test loss, or approximately:
        \begin{align*}
            \approx
            G \wrap{\int_t \exp(-\eta (T-t) H)} \eta G
            =
            G \wrap{\frac{I - \exp(-\eta T H)}{H}} G
        \end{align*}
        To use physical language: after integrating the
        ``effective propagator'' $\exp(-\eta t H) \eta$ over possible
        spacetime configurations, we arrive at a revised inverse metric
        $
            \wrap{I - \exp(-\eta T H)}/H
        $
        to be used instead of $\eta$.  When $H$ is positive, the above
        converges in the large-$T$ limit.

        \textsc{Renormalized Recipe:}~
        In general, one sums over all embeddings of diagrams that are
        \emph{irreducible} in the sense that they lack non-root
        $\sim$-singleton degree-two nodes (such as the middle node of
        $\sdia{(0-1-2)(01-12)}$).  The sum is like the unrenormalized case
        except that the embedding-agnostic value $\dvalue(D)$ is replaced by an
        embedding-sensitive value $\dvalue_f(D)$, which has the same expression
        as $\dvalue(D)$ except that each contraction by $-\eta$ between nodes
        embedded to times $t, \tilde t$ is replaced by a factor $(I-\eta
        H)^{\wabs{t-\tilde t}} \eta$.  In practice, we prefer to approximate
        sums over embeddings by integrals over times and $(I-\eta H)^t$ by
        $\exp(- \eta H t)$, thus incurring a term-by-term multiplicative error
        of $1 + o(\eta)$ that does not affect leading order results; in this
        way, a diagram induces an easily evaluated integral of exponentials.
       
        \begin{thm}[Renormalization Licenses Large-$T$ Limits] \label{thm:renorm}
            SGD's test loss's MacLaurin series may be written as follows
            (Theorem \ref{thm:test}'s modification for the generalization gap
            stands).
            \begin{equation} \label{eq:renorm}
                \sum_{\substack{D \in \image(\Free)\\ \text{irreducible}}}
                \sum_{f: D\to\Free(S)}
                    \left(
                        \prod_{i\in S} \frac{1}{|f^{-1}(i)|!}
                    \right)
                \frac{\dvalue_f(D)}{B^{|\edges(D)|}}
            \end{equation}
            In contrast to Theorem \ref{sgd:coef}, when $H$ is positive,
            the $d$th order truncation converges as $T$ diverges.
        \end{thm}
        Due to the latter convergence property, we have formal license to 
        take long-term limits of renormalized predictions.  In fact, the
        convergence is uniform in $T$ for sufficiently regular landscapes.
        \begin{prop}[A Nonconservative Entropic Force]\label{prop:entropic}
            When initialized at a test minimum, vanilla SGD's weight moves to
            order $\eta^2$ with a long-time-averaged\footnote{
                That is, $T$ so large that $C \exp(-\eta K T)$ is negligible.
                An appendix gives a similar expression for general $T$.
            }
            expected velocity of
            $$
                v^\pi = C_{\mu \nu}
                \wrap{K^{-1}}^{\mu\nu}_{\rho\lambda}
                J^{\rho\lambda}_{\sigma}
                \wrap{\frac{I - \exp(-T \eta H)}{T \eta H} \eta}^{\sigma \pi}
            $$
            per timestep.
            Here, $K = \eta H \otimes I + I \otimes \eta H$, a
            $4$-valent tensor. 
        \end{prop}
        Unlike prior work \cite{we19b}, we make no assumptions of
        thermal equilibrium, fast-slow mode separation, or constant covariance.
        This added generality allows us to predict a qualitatively new
        dynamical phenomenon, namely that the velocity field above need not be
        conservative.  We verify this in experiments.
        \begin{prop}[Flat and Sharp Minima Overfit Less]\label{prop:overfit}
            When initialized at a test minimum, pure GD's test loss is, to
            order $\eta$, 
            $$
                \frac{1}{2N} ~
                    C_{\mu\nu}
                    \wrap{(I - \exp(-\eta T H))^{\otimes 2}}^{\mu\nu}_{\rho\lambda}
                    \wrap{H^{-1}}^{\rho\lambda}
            $$
            above the minimum.  This vanishes when $H$ does. 
            Likewise, pure GD's generalization gap is to order $\eta$:  
            $$
                \frac{1}{N} ~
                    C_{\mu\nu}
                    \wrap{(I - \exp(-\eta T H))}^{\nu}_{\lambda}
                    \wrap{H^{-1}}^{\lambda\mu}
            $$
            Contrary to the later-mentioned Takeuchi estimate, this does not
            diverge as $H$ shrinks.
        \end{prop}
        Proposition \ref{prop:overfit}'s generalization gap converges after
        large $T$ to $C_{\mu\nu}(H^{-1})^{\mu\nu}/N$, also known as Takeuchi's
        Information Criterion (TIC).  In turn, in the classical setting of
        maximum likelihood (ML) estimation (with no model mis-specification)
        near the ``true'' test minimum, $C=H$ is the Fisher metric, so we
        recover the Akaike Information Criterion (AIC) $(\textnormal{number of
        parameters})/N$.  Unlike AIC, our more general expression is
        descendably smooth, may be used with MAP or ELBO tasks instead of just
        ML, and makes no model well-specification assumptions.

        \begin{figure}[h!]
            \centering
            \plotmooh{diagrams/entropic-force-diagram}{}{0.35\columnwidth} 
            \plotmooh{diagrams/springs}{}{0.35\columnwidth}
            \caption{
                {\bf Left}:
                    The entropic force mechanism: gradient noise induces a flow
                    toward minima flat \emph{with respect to to the
                    covariance}.  Although our analysis assumes neither thermal
                    equilibrium nor fast-slow mode separation, we label ``fast
                    and slow directions'' in this cartoon to ease comparison
                    with \citet{we19b}.  In this cartoon, red densities denote
                    the spread predicted by a renormalized $C^{\mu\nu}$, and
                    the spatial variation of curvature corresponds to
                    $J_{\mu\nu\lambda}$. 
                {\bf Right}:
                    Noise structure determines how curvature affects
                    overfitting.  To use a physical metaphor, for a fixed
                    displacement scale, stiffer springs incur greater energy
                    gains.  But for a fixed force scale, limper springs incur
                    greater energy gains.  Geometrically, for (empirical risk
                    minimization on) a vector-perturbed landscape, small
                    hessians are favored (top row), while for
                    covector-perturbed, landscape large hessians are favored
                    (bottom row).  Proposition \ref{prop:overfit} shows how the
                    implicit regularization of fixed-$\eta T$ descent mediates
                    between the two rows.
            }
            \label{fig:entropic}
        \end{figure}

%==============================================================================
%    EXPERIMENTS AND APPLICATIONS
%==============================================================================

\section{Experiments and Applications}

    %--------------------------------------------------------------------------
    %           Vanilla SGD                                 
    %--------------------------------------------------------------------------

    \subsection{Vanilla SGD}
        We test Proposition \ref{prop:vanilla} for smooth convolutional
        architectures for CIFAR-10 and Fashion-MNIST by comparing measured 
        test losses and generalization gaps with un-renormalized predictions
        (see Appendices E and F for experiment details).  We find good
        agreement between our order $\eta^3$ perturbative predictions up to
        $\eta T \approx 10^0$.  Though of little direct import, this test
        verifies that our application of the diagram method hides no factor
        mistakes or sign errors.
        \begin{figure}[h!]
            \centering
            \plotmoo{plots/test-vanilla-fashion}{0.48\columnwidth}{3.0cm} 
            \plotmoo{plots/gen-cifar}{0.48\columnwidth}{3.0cm}
            \caption{
                {\bf Left}: Test loss vs learning rate on an image
                classification task.  For the instance shown and all $11$ other
                initializations unshown, the degree-$3$ prediction agrees with
                experiment through $\eta T \approx 10^0$.
                {\bf Right}:
                Generalization gap (test minus train) vs learning rate on an
                image classification task.  For the instance shown and all $11$
                other initializations unshown, the degree-$2$ prediction agrees
                with experiment through $\eta T \approx 10^0$.  Throughout,
                measurements are in blue and theory is in other colors.
                Vertical spans indicate 95\% confidence intervals for the mean.
            }
        \end{figure}

    %--------------------------------------------------------------------------
    %           Epochs and Overfitting                      
    %--------------------------------------------------------------------------
 
    %\subsection{Epochs and Overfitting}
        %{\color{moor} FILL IN}
        Likewise, Figure \ref{fig:bmmulti} shows that our predictions of how 
        epochs affect overfitting are in good agreement with experiment. 
        %\lorem{3}
    
    %--------------------------------------------------------------------------
    %           Emulating Small Batches with Large Ones     
    %--------------------------------------------------------------------------

    \subsection{Emulating Small Batches with Large Ones}
        Figure \ref{fig:bmmulti} shows that the regularizer proposed in
        Proposition \ref{cor:batch} indeed enables GD to emulate SGD
        on a range of image-classification landscapes.  For these experiments,
        we used a covariance
        estimator $\hat C \propto \nabla l_x (\nabla l_x - \nabla l_x)$ evaluated on
        two batches $x, y$ that partition the training set.
        \begin{figure}[h!] 
            \centering
            \plotmoo{plots/big-bm-new}{0.48\columnwidth}{4.0cm}
            \plotmoo{plots/multi-fashion-logistic-0}{0.48\columnwidth}{4.0cm}
            \caption{
                {\bf Left}: with equal-scaled axes, this plot shows that GDC
                matches SGD (small vertical variation) better than GD matches
                SGD (large horizontal variation) in test loss, for a variety of
                learning rates (from 0.0025 to 0.1) and initializations (zero
                and multiple independent Xavier-Glorot trials) on several of
                image classification landscapes (logistic and convolutional
                CIFAR-10 and Fashion-MNIST).  $T=10$ throughout.
                {\bf Right}: SGD with $2, 3, 5, 8$ epochs incurs greater test
                loss than single-epoch SGD (difference shown in I bars) by the
                predicted amounts (predictions shaded) for a range of learning
                rates.
            }
            \label{fig:bmmulti}
        \end{figure}
  
    %--------------------------------------------------------------------------
    %           Comparison to Continuous Time               
    %--------------------------------------------------------------------------

    \subsection{Comparison to Continuous Time}
        Consider fitting a centered normal $\Nn(0, \sigma^2)$ to some centered
        standard normal data.  We parameterize the landscape by
        $h=\log(\sigma^2)$ so that the Fisher information matches the standard
        dot product \citet{am98}.  The gradient at sample $x$ and weight
        $\sigma$ is then $g_x(h) = (1-x^2\exp(-h))/2$.  Since $x\sim \Nn(0,
        1)$, $g_x(h)$ will be affinely related to a chi-squared, and in
        particular non-gaussian.  At $h=0$, the expected gradient vanishes, and
        the test loss of vanilla SGD only involves diagrams with no singleton
        leaves; to third order, it is
        $
            \sdia{(0)()}
            +\frac{T}{2} \sdia{c(01-2)(02-12)}
            +{T\choose 2} \sdia{c(03-1-2)(01-12-23)}
            +\frac{T}{6} \sdia{c(012-3)(03-13-23)}
        $
        In particular, the ${T\choose 2}$ differs from $T^2/2$ and hence
        contributes to the time-discretization error of SDE as an approximation
        for SDE.  Moreover, the $\sdia{c(012-3)(03-13-23)}$ contributes to
        non-gaussian noise to that error.  Figure \ref{fig:thermo} shows that,
        indeed, SDE and single-epoch SGD differ.  For multi-epoch SGD, the
        effect of overfitting to finite training data further separates SDE and
        SGD.

    %--------------------------------------------------------------------------
    %           Thermodynamic Engine                        
    %--------------------------------------------------------------------------

    \subsection{A Nonconservative Entropic Force} \label{subsect:entropic}
        To test Proposition \ref{prop:entropic}'s predicted force, 
        we construct a counter-intuitive loss landscape wherein, for
        arbitrarily small learning rates, SGD steadily increases the weight's
        z component despite 0 test gradient in that direction.
        Our mechanism differs from that discovered by \citet{ch18}.
        Specifically, because in this landscape the force is
        $\eta$-perpendicular to the image of $\eta C$, that work predicts an
        entropic force of $0$.  This disagreement in predictions is possible
        because our analysis does not make any assumptions of equilibrium,
        conservatism, or assumptions.
        
        Intuitively, the presence of the term
        $
            \sdia{c(01-2-3)(02-12-23)}
        $
        in our test loss expansion indicates that 
        \emph{SGD descends on a covariance-smoothed landscape}.
        So, even in a valley of global minima, SGD will move away from minima
        whose Hessian aligns with the current covariance.  However, by the time
        it moves, the new covariance might differ from the old one, and SGD will
        be repelled by different Hessians than before.  By setting the
        covariance to lag the Hessian by a rotational phase, we construct
        a landscape in which this entropic force occurs forever. 
        This ``\emph{linear-screw}'' landscape is defined for
        three-dimensional $w\in \RR^3$ (initialized at the origin) and
        one-dimensional $x \sim \Nn(0, 1))$:
        $$
            l_x(w) = \frac{1}{2} H(z)(w, w) + x \cdot S(z)(w)  
        $$
        Here, $H(z)(w, w) = w_x^2 + w_y^2 + (\cos(z) w_x + \sin(z) w_y)^2$
        and   $S(z)(w)    = \cos(z-\pi/4) w_x + \sin(z-\pi/4) w_y$.
        We see that there is a valley of global minima defined by $x=y=0$. 
        If SGD is initialized there, then to leading order in $\eta$ and for
        large $T$, the renormalized theory predicts a $z$-speed of $\eta^2/6$ 
        per timestep.  We see that this prediction, unlike the
        un-renormalized prediction, agrees with experiment.

        \begin{figure}[h!]
            \centering
            \plotmoo{plots/vs-sde}{0.48\columnwidth}{4.0cm}
            \plotmoo{plots/thermo-linear-screw}{0.48\columnwidth}{4.0cm}
            \caption{
                {\bf Left}: SGD's difference from SDE after $\eta T \approx
                10^{-1}$ with maximal coarseness on the gaussian-fit problem.  
                Two effects not modeled by SDE --- time-discretization and
                non-gaussian noise oppose on this landscape but do not
                completely cancel. 
                {\bf Right}: On the linear screw landscape, the persistent
                entropic force pushes the weight through a valley of global
                minima not at a $T^{1/2}$ diffusive rate but at a directional
                $T^1$ rate.  Hessians and covariances  are uniformly bounded
                throughout the valley, and this effect appears at all
                sufficiently small $\eta$s with strength $\eta^2$.  Thus, the
                effect is not a pathological artifact of well-chosen learning
                rate or divergent covariance noise.  Observe that the net
                displacement of $\approx 10^{1.5}$ well exceeds the $z$-period
                of $2\pi$. 
            }
            \label{fig:thermo}
        \end{figure}

        Moreover, by stitching together copies of this example, we may cause
        SGD to travel along paths that are closed loops or unbounded curves of
        global minima, or we may add a small linear component to the valleys so
        that SGD steadily climbs uphill.  


    %--------------------------------------------------------------------------
    %           Sharp vs Flat Minima                        
    %--------------------------------------------------------------------------

    \subsection{Sharp and Flat Minima Both Overfit Less} \label{subsect:overfit}
        Prior work has varyingly found that \emph{sharp} minima overfit less
        (after all, $l^2$ regularization increases curvature) or that
        \emph{flat} minima overfit less (after all, flat minima are more
        robust to small displacements in weight space).  Proposition
        \label{prop:overfit} reconciles these competing intuitions by showing
        how the relationship of generalization and curvature depends on the
        learning task's noise structure and how the metric $\eta^{-1}$ mediates
        this distinction.
        
        \begin{figure}[h!] 
            \centering
            \plotmoo{plots/tak}{0.48\columnwidth}{4.0cm}
            \plotmoo{plots/tak-reg}{0.48\columnwidth}{4.0cm}
            \caption{
                {\bf Left}: For artificial quadratic landscapes with fixed
                covariance and a range of hessians, initialized at the true
                minimum, the test losses after fixed-$\eta T$ optimization are
                smallest for very small and very large curvatures.  This
                evidences our renormalized theory's prediction that both sharp
                and flat minima overfit less!  In particular, the Takeuchi
                estimate's singularity is, as predicted, suppressed.
                {\bf Right}: For artificial quadratic landscapes with fixed
                covariance and a range of hessians, initialized a fixed
                distance \emph{away} from the true minimum, joint descent on 
                an $l_2$ penalty coefficient $\lambda$ by means of STIC
                improves on plain gradient descent for most hessians.  That
                there is at all a discrepancy from theory is possible because
                $\lambda$ is not perfectly tuned according to STIC but instead
                descended on for finite $\eta T$.
            }
            \label{fig:tak}
        \end{figure}

        Because the TIC estimates a smooth hypothesis class's generalization
        gap, it is tempting to use it as an additive regularization term.
        However, the TIC is singular where the Hessian is (see Figure
        \ref{fig:entropic}), and as such gives insensible results for
        over-parameterized models.  Indeed, a prior attempt ran into numerical
        difficulties requiring an arbitrary cutoff \citep{di18}. 

        Fortunately, by Proposition \ref{prop:overfit}, the implicit
        regularization of gradient descent both demands and enables a
        singularity-removing correction to the TIC --- see Figure
        \ref{fig:tak}.  The resulting \emph{Stabilized TIC} (STIC) uses the
        metric implicit in gradient descent to threshold flat from sharp
        minima.  It thus offers a principled method for optimizer-aware model
        selection easily compatible with automatic differentiation systems.  By
        descending on STIC, we may tune smooth hyperparameters such as $l_2$
        coefficients.  Experiments on the mean-estimation problem validate STIC
        for such model selection, especially when $C/N$ dwarves $H$ as in the
        noisy, small-$N$ regime (see Figure \ref{fig:tak}).  Because matrix
        exponentiation is expensive, STIC regularization without further
        approximations is most useful for models of small dimension
        on very noisy or scanty data.

%==============================================================================
%    RELATED WORK    
%==============================================================================

\section{Related Work}

    %--------------------------------------------------------------------------
    %           History of SGD
    %--------------------------------------------------------------------------

    It was \citet{ki52} who, in uniting gradient descent \citep{ca47} with
    stochastic approximation \citep{ro51}, invented SGD.  Since the development
    of back-propagation for efficient differentiation \citep{we74}, SGD
    has been used to train connectionist models including neural
    networks \citep{bo91}, in recent years to remarkable success \citep{le15}.

    %--------------------------------------------------------------------------
    %           Analyzing Overfitting; Relevance of Optimization; SDE Errs  
    %--------------------------------------------------------------------------

    Several lines of work quantify the overfitting of SGD-trained networks
    \citep{ne17a}.  For instance, \citet{ba17} controls the Rademacher
    complexity of deep hypothesis classes, leading to generalization bounds
    that are optimizer-agnostic.  However, since networks trained via SGD
    generalize despite their seeming ability to shatter large sets
    \citep{zh17}, one infers that generalization arises from the aptness to
    data of not only architecture but also optimization \citep{ne17b}.  Others
    have focused on the implicit regularization of SGD itself, for instance by
    modeling descent via stochastic differential equations (SDEs) (e.g.
    \citet{ch18}).  However, per \citet{ya19a}, such continuous-time analyses
    cannot treat covariance correctly, and so they err when interpreting
    results about SDEs as results about SGD for finite trainsets.

    %--------------------------------------------------------------------------
    %           We Extend Dan's Approach                     
    %--------------------------------------------------------------------------

    Following
    %\citet{li17} and
    \citet{ro18}, we avoid making a continuous-time
    approximation by instead Taylor-expanding around the learning rate
    $\eta=0$.  In fact, we develop a diagrammatic method for evaluating each
    Taylor term that is inspired by the field theory methods popularized by
    \citet{dy49a}.  Using this technique, we quantify the overfitting effects
    of batch size and epoch number, and based on this analysis, propose a
    regularizing term that causes large-batch GD to emulate small-batch SGD,
    thus establishing a precise version of the relationship --- between
    covariance, batch size, and generalization --- conjectured by \citet{ja18}.  
    
    %--------------------------------------------------------------------------
    %           Phenomenology of Rademacher Correlates such as Hessians
    %--------------------------------------------------------------------------

    While we make rigorous, architecture-agnostic predictions of learning
    curves, these predictions become vacuous for large $\eta$.  In particular,
    while our work does not assume convexity of the loss landscape, before
    renormalization it also is blind to large-$\eta T$ convergence of SGD.
    Other discrete-time dynamical analyses allow large $\eta$ by treating deep
    generalization phenomenologically, whether by fitting to an
    empirically-determined correlate of Rademacher bounds \citep{li18}, by
    exhibiting generalization of local minima \emph{flat} with respect to the
    standard metric (see \citet{ho17}, \citet{ke17}, citet{wa18}), or by
    exhibiting generalization of local minima \emph{sharp} with respect to the
    standard metric (see \citet{st56}, \citet{di17}, \citet{wu18}).  Our work,
    which makes explicit the dependence of generalization on the underlying
    metric and on the form of gradient noise, reconciles those
    seemingly clashing claims.
    
    %--------------------------------------------------------------------------
    %           Our Work vs Other Perturbative Approaches            
    %--------------------------------------------------------------------------

    Others have imported the perturbative methods of physics to analyze descent
    dynamics:  \citet{dy19} perturb in inverse network width, employing 't
    Hooft diagrams to compute deviations of a specific class of deep
    architectures from Gaussian processes.  Meanwhile, \cite{ch18} and
    \citet{li17} perturb in learning rate to second order by approximating
    noise between updates as gaussian and uncorrelated.  This approach does not
    generalize to higher orders, and, because correlations and heavy tails are
    essential obstacles to concentration of measure and hence of
    generalization, it does not model the generalization behavior of SGD.  By
    contrast, we use Penrose diagrams to compute test and train losses to
    arbitrary order in learning rate, quantifying the effect of non-gaussian
    and correlated noise.  Our method accounts for optimization and applies to
    any smooth architecture.  We hence extend \citet{ro18} beyond leading order
    and beyond $2$ time steps, allowing us to compare, for instance, the
    expected test losses of multi-epoch and single-epoch SGD.

%==============================================================================
%    CONCLUSION      
%==============================================================================

\section{Conclusion}

    %--------------------------------------------------------------------------
    %           Summarize Contributions                     
    %--------------------------------------------------------------------------

    Feynman diagrams make novel and precise predictions about SGD: 

    %\par\noindent\framebox{\parbox{\dimexpr\linewidth-2\fboxsep-2\fboxrule}{
        \textsc{Story of $\sdia{c(01-2)(02-12)}$:~}
        Flat and sharp minima both overfit less than minima of curvature
        comparable to $(\eta T)^{-1}$.  While flat minima withstand
        vector-valued weight displacements and sharp minima withstand
        covector-valued loss offsets, medium minima attain the worst of both
        worlds.  Such considerations lead us to a smooth generalization of AIC
        and to tune hyperparameters by gradient descent.
    %}}

    %\par\noindent\framebox{\parbox{\dimexpr\linewidth-2\fboxsep-2\fboxrule}{
        \textsc{Story of $\sdia{c(01-2-3)(02-12-23)}$:~}
        Refining \citet{we19b} to nonconstant, nonisotropic covariance, we find
        SGD descends on a loss landscape smoothed by the current covariance.
        As the weight and thus the covariance evolve, the smoothing mask
        evolves and thus the effective landscape evolves.  This dynamics
        is generically nonconservative.  In contrast to \citet{ch18}, this SGD
        dynamics does not need to converge to a limit cycle. 
    %}}

    %\par\noindent\framebox{\parbox{\dimexpr\linewidth-2\fboxsep-2\fboxrule}{
        \textsc{Story of $\sdia{c(01-2)(01-12)}$:~}
        As conjectured by \citet{ro18}, large-batch GD can be made to emulate
        small-batch SGD.  We show how to do this by adding a multiple of an
        unbiased covariance estimator to the descent objective.  This emulation
        is significant because, while small batch sizes can lead to better
        generalization \citep{bo91}, modern infrastructure increasingly rewards
        large batch sizes \citep{go18}.  
    %}}

    %\par\noindent\framebox{\parbox{\dimexpr\linewidth-2\fboxsep-2\fboxrule}{
    %    \textsc{Story of $\sdia{c(012-3)(03-13-23)}$:~}
    %    In multi-epoch SGD, inter-epoch shuffling induces no $3$rd order effect
    %    on test loss.  In other words, we proved that \emph{nongaussian effects
    %    matter more than shuffling order} for fixed finite $N$ and small $\eta$.
    %}}

    SGD's test loss and generalization gap are sums of infinitely many
    diagrams, and for small learning rates, the behavior is controlled by the
    finitely many diagrams with few edges.  The base theory gives predictions
    for any fixed $T$ as long as $\eta$ is much smaller than $1/T$.  On
    strictly convex landscapes, renormalization improves the $\eta$-expansion's
    convergence to be \emph{uniform} in $T$.  Most of the calculations here
    presented can by means of diagrams be done fully in one's head.  We see
    this as evidence that our method is helpful.

    %--------------------------------------------------------------------------
    %           Ask Questions                               
    %--------------------------------------------------------------------------

    In enriching our intuitions with physical analogies, the diagram method
    invites further exploration.  Especially interesting are Lagrangian methods
    and curved backgrounds: 
    \begin{quest}
        Do Lagrangians govern effective theories of SGD and its variants, or
        is there a fundamental obstacle to such a representation?
    \end{quest}
        More than Hamiltonian formalisms for SGD \cite{ch14}, a Langrangian
        formalism could illuminate hidden conservation and scaling laws in
        SGD's variants.  Problematically, SGD lacks the symplectic structure
        typically used to translate between Hamiltonians and Lagrangians.
        Still, one may hope for a formal analogy between diagram parts and
        terms of a Lagrangian.  Indeed, we have found that some
        \emph{higher-order} methods --- such as the Hessian-based update
        $
            \theta \leftsquigarrow
            \theta -
            (\eta^{-1} + \lambda \nabla \nabla l_t(\theta))^{-1}
            \nabla l_t(\theta)
        $
        parameterized by small $\eta, \lambda$ --- are amenable to diagrammatic
        analysis, with the Hessian term yielding a new type of vertex.
        Though diagrams suffice for computations, it is Lagrangians that offer
        the deepest non-perturbative insight.
    \begin{quest}
        How does curvature of weight space affect the generalization behavior
        of gradient descent?
    \end{quest}
        Throughout this work, we have assumed a flat metric $\eta^{\mu\nu}$.
        However, the general case of curved spaces have concrete applications
        in the \emph{learning on manifolds} paradigm of \citet{bo13}, notably
        specialized to \citet{am98}'s \emph{natural gradient descent} and
        \citet{ni17}'s \emph{hyperbolic embeddings}.  The diagram formalism may
        be adapted to curved weight spaces: one may represent the affine
        connection as a node, thus giving rise to non-tensorial and hence
        gauge-dependent diagrams.  A promising line of work is thus to adapt
        our current formalism to the curved case, especially to extract
        qualitative results about the role of Riemannian curvature in
        generalization.

%==============================================================================
%    ACKNOWLEDGEMENTS
%==============================================================================

\subsection{Acknowledgements}
    We thank Dan A. Roberts and Sho Yaida for introductions to their work and
    for posing several of the questions we answer here.  We feel deeply
    grateful to Sho Yaida and Josh B. Tenenbaum for their compassionate and
    patient guidance.  We appreciate the generosity of
        Andrzej Banburski
        and
        Wenli Zhao
    in offering feedback on writing.

%==============================================================================
%    REFERENCES      
%==============================================================================

%\section*{References}
    \bibliography{perturb}
    \bibliographystyle{icml2019}

%==============================================================================
%    APPENDICES      
%==============================================================================

\section*{A. Proofs of Theorems}

    \subsection*{Regularity Hypotheses}
        We make several assumptions throughout this work.  We assume that the
        functions $\theta \mapsto l_x(\theta)$, as well as the expectations of
        polynomials of their $0$th and higher derivatives, are analytic with
        shared ($\theta$-dependent) radii of convergence.  We also assume that
        the gradients $\nabla l_x(\theta)$ are bounded, uniformly in $x$, by a
        continuous function of $\theta$.  And finally, we assume that
        expectations with respect to $x$ and derivatives with respect to
        $\theta$ commute in all expressions mentioned in the text.

    \subsection*{Dyson Series for Iterative Optimizers}
        If a density $\rho$ governs a point $\theta$ in weight space, then
        after a sequence of updates $\theta \mapsto \theta - \eta^{\mu\nu}
        \nabla_\mu l(\theta)$ on losses $(l_t: 0\leq t < T)$, the following
        density will govern the new point:
        \begin{equation}\label{eq:descexp}
            \exp\left(+\eta^{\mu\nu} \nabla_\mu l_{T-1}(\theta) \nabla_\nu\right)
            \cdots
            \exp\left(+\eta^{\mu\nu} \nabla_\mu l_0(\theta) \nabla_\nu\right)
            \rho
        \end{equation}
        We write 
        $
            \prod \exp\left(+\eta \nabla l \nabla\right) \rho
        $
        for short.  The exponent above is a linear operator that acts on a
        space of sufficiently smooth maps; in particular, the $\nabla_\nu$ does
        not act on the $\nabla_\mu l(\theta)$ with which it pairs.  Integrating
        by parts, we write the expectation over initial values after $T$ steps
        of an analytic function $s$ of weight space (e.g. $s$ may be test loss)
        as:
        \begin{align}\label{eq:contraexp}
            \int_\theta \rho(\theta) \left(
                \prod_{0 \leq t \leq T} \exp\left(
                    -\eta^{\mu\nu} \nabla_\mu l_t(\theta) \nabla_\nu
                \right) s
            \right)(\theta)
        \end{align}
        Since the exponentials above might not commute, we may not compose the
        product of exponentials into an exponential of a sum.  We instead
        compute an expansion in powers of $\eta$.  Setting the initialization
        $\rho(\theta) = \delta(\theta-\theta_0)$ to be deterministic, and
        labeling as $\theta_t$ the weight after $t$ steps, we find:
        \begin{equation}\label{eq:dyson}
            s(\theta_T) =
            \sum_{0\leq d < \infty} (-\eta)^d
            \sum_{\substack{(d_t: 0\leq t<T) \\ \sum_t d_t = d}}
            \left(
                \prod_{0 \leq t < T} \left.
                    \frac{(g \nabla)^{d_t}}{d_t!}
                \right|_{g=\nabla l_t(\theta)}
            \right) s (\theta_0)
        \end{equation}
        Though above we use standard physical shorthand, our assumptions that
        $s$ is analytic and that gradients are $x$-uniformly bounded by a
        continous function permit the argument to be made rigorous for any
        $\eta$ smaller than a radius of convergence that may depend on $T$ but
        not on the sampling of losses $(l_t: 0\leq t<T)$ from the fixed
        distribution of loss functions.  In fact, by those assumptions, both
        sides of the above have an expectation analytic --- and hence smooth
        --- in $\eta$.  The Key Lemma (\ref{lem:dyson}) follows.

    \subsection*{Terms and Diagram Embeddings Correspond}
        {\color{moor} FILL IN}
        The first Path Integral Theorem (\ref{thm:sgdcoef}) follows.

    \subsection*{Coefficient Convergence upon Renormalization}
        Because the Taylor series for analytic functions converge absolutely
        in the interior of the disk of convergence, the rearrangement of terms
        corresponding to a grouping by geometric realizations preserves the
        proof and results of \ref{thm:test}.  

        To prove the rest of Theorem \ref{thm:renorm}, we assume that $H$ is
        positive.  Then, for any $m$, the propagator ${(I-\eta H)^{\otimes
        m}}^t$ converges exponentially with $t$ to $0$.
        {\color{moor} FILL IN}
        The Renormalization Theorem (\ref{thm:renorm}) follows.

\section*{B. Tutorial on Diagram Rules}
    After reviewing how diagrams correspond to specific landscape statistics,
    we will work through four examples in using diagrams to analyze descent.

    \subsection*{Evaluating a Diagram}
        {\color{moor} FILL IN}
    \subsection*{Integrating a Diagram over Spacetime}
        {\color{moor} FILL IN}
    \subsection*{The 3rd Order Curl: Which Minima Does SGD Prefer?}
        {\color{moor} FILL IN}
    \subsection*{The Roles of Covariance: the Generalization Gap of SGD}
        {\color{moor} FILL IN}
        
\section*{C. Perturbative Calculations}

    \subsection*{SGD vs ODE and SDE}
        {\color{moor} FILL IN}
    \subsection*{Interepoch Shuffling}
        {\color{moor} FILL IN}
    \subsection*{Effect of Epochs}
        {\color{moor} FILL IN}
    \subsection*{Renormalized Nonconservative Entropic Force}
        {\color{moor} FILL IN}

\section*{D. Diagram Rules vs Direct Perturbation} \label{sect:compare}
    Diagram methods from Stueckelberg to Peierls have flourished in physics
    because they enable swift computations and offer immediate intuition that
    would otherwise require laborious algebraic manipulation.  We demonstrate
    how our diagram formalism likewise streamlines analysis of descent by
    comparing direct perturbation to the new formalism on three sample
    problems.

    Aiming for a conservative comparison of derivation ergonomics, we lean
    toward explicit routine when using diagrams and allow ourselves to use
    clever and lucky simplifications when doing direct perturbation.  For
    example, while solving the first sample problem by direct perturbation,
    we structure the SGD and GD computations so that the coefficients (that in
    both the SGD and GD cases are) called $a(T)$ manifestly agree in their
    first and second moments.  This allows us to save some lines of argument.

    Despite these efforts, the diagram method yields arguments about four times
    shorter --- and strikingly more conceptual --- than direct perturbation
    yields.  These examples specifically suggest that: diagrams obviate the
    need for meticulous index-tracking, from the start focus one's attention on
    non-cancelling terms by making visually obvious which terms will eventually
    cancel, and allow immediate exploitation of a setting's special posited
    structure, for instance that we are initialized at a test minimum or that
    the batch size is $1$.  We regard these examples as evidence that diagrams
    offer a practical tool for the theorist.

    We make no attempt to compare the renormalized version of our formalism
    to direct perturbation because the algebraic manipulations involved for
    the latter are too complicated to carry out.  

    \subsection*{Effect of Batch Size}
        We compare the test losses of pure SGD and pure GD.  Because pure
        SGD and pure GD differ in how samples are correlated, their test loss
        difference involves a covariance and hence occurs at order $\eta^2$.  

        \subsubsection*{Diagram Method}
            Since SGD and GD agree on noiseless landscapes, we consider only
            diagrams with fuzzy ties.  Since we are working to second order, we
            consider only two-edged diagrams.  There are only two such
            diagrams, $\sdia{(01-2)(02-12)}$ and $\sdia{(01-2)(01-12)}$.  The
            first diagram, $\sdia{(01-2)(02-12)}$, embeds in GD's space time in
            $N^2$ as many ways as it embeds in SGD's spacetime, due to
            horizontal shifts.  Likewise, there are $N^2$ times as many
            embeddings of $\sdia{(01-2)(02-12)}$ in distinct epochs of GD's
            spacetime as there are in distinct epochs of SGD's spacetime.
            However, each same-epoch embedding of $\sdia{(01-2)(01-12)}$ within
            any one epoch of GD's spacetime corresponds by vertical shifts to
            an embedding of $\sdia{(0-1-2)(01-12)}$ in SGD.  There are
            $MN{N\choose 2}$ many such embeddings in GD's spacetime, so GD's
            test loss exceeds SGD's by 
            $$
                %\eta^2 \frac{MN{N\choose 2}}{N^2}
                %(\sdia{(01-2)(01-12)} - \sdia{(0-1-2)(01-12)}) 
                %=
                \wrap{\frac{\eta}{N}}^2 MN{N\choose 2}~
                \sdia{c(01-2)(01-12)}
            $$
            Since $(\nabla^2 l) (\nabla l) = \nabla((\nabla l)^2)/2$, we can 
            summarize this difference as
            $$
                \eta^2 \frac{M(N-1)}{4} G \nabla C 
            $$

        \subsubsection*{Direct Perturbation} 
            We compute the displacement $\theta_T-\theta_0$ to order $\eta^2$ 
            for pure SGD and separately for pure GD.  Expanding
            $
                \theta_t \in \theta_0 + \eta a(t) + \eta^2 b(t) + o(\eta^2)
            $, we find:
            \begin{align*}
                \theta_{t+1} &=     \theta_t - \eta \nabla l_{n_t} (\theta_t) \\
                             &\in       \theta_0
                                    +   \eta a(t) + \eta^2 b(t)
                                    -   \eta (
                                                \nabla l_{n_t}
                                            +   \eta \nabla^2 l_{n_t} a(t) 
                                        )
                                    +   o(\eta^2) \\
                             &=     \theta_0
                                +   \eta (a(t) - \nabla l_{n_t})
                                +   \eta^2 (b(t) - \nabla^2 l_{n_t} a(t)) 
                                +   o(\eta^2)
            \end{align*}
            To save space, we write $l_{n_t}$ for $l_{n_t}(\theta_0)$.  It's
            enough to solve the recurrence $a(t+1) = a(t) - \nabla l_{n_t}$ and
            $b(t+1) = b(t) - \nabla^2 l_{n_t} a(t)$.  Since $a(0), b(0)$
            vanish, we have $a(t) =-\sum_{0\leq t<T} \nabla l_{n_t}$ and $b(t)
            = \sum_{0\leq t_0 < t_1 < T} \nabla^2 l_{n_{t_1}} \nabla
            l_{n_{t_0}}$.  We now expand $l$:
            \begin{align*}
                l(\theta_T) \in    l   &+   (\nabla l) (\eta a(T) + \eta^2 b(T)) \\
                                       &+   \frac{1}{2} (\nabla^2 l) (\eta a(T) + \eta^2 b(T))^2
                                        +   o(\eta^2) \\
                            =      l   &+   \eta ((\nabla l) a(T))
                                        +   \eta^2 ((\nabla l) b(T) + \frac{1}{2} (\nabla^2 l) a(T)^2 )
                                        +   o(\eta^2)
            \end{align*}
            Then $\expct{a(T)} = -MN(\nabla l)$ and, since the $N$ many
            singleton batches in each of $M$ many epochs are pairwise
            independent,
            \begin{align*}
                \expct{(a(T))^2}
                ~&=
                \sum_{0\leq t<T} \sum_{0\leq s<T} \nabla l_{n_t} \nabla l_{n_s} \\
                ~&= 
                M^2N(N-1)   \expct{\nabla l}^2 +
                M^2N        \expct{(\nabla l)^2}
            \end{align*}
            Likewise, 
            \begin{align*}
                \expct{b(T)}
                = 
                ~&\sum_{0\leq t_0 < t_1 < T} \nabla^2 l_{n_{t_1}} \nabla l_{n_{t_0}} \\
                =
                ~&\frac{M^2N(N-1)}{2} \expct{\nabla^2 l} \expct{\nabla l} + \\
                ~&\frac{M(M-1)N}{2}  \expct{(\nabla^2 l) (\nabla l)} 
            \end{align*}

            Similarly, for pure GD, we may demand that $a, b$ obey recurrence
            relations $a(t+1) = a(t) - \sum_n \nabla l_n/N$ and
            $b(t+1) = b(t) - \sum_n \nabla^2 l_n a(t)/N$, meaning that
            $a(t) = -t \sum_n \nabla l_n/N$ and
            $b(t) = {t \choose 2} \sum_{n_0} \sum_{n_1} \nabla^2 l_{n_0} \nabla l_{n_1}/N^2$.
            So $\expct{a(T)} = -MN(\nabla l)$ and
            \begin{align*}
                \expct{(a(T))^2}
                ~&=
                M^2 
                \sum_{n_0} \sum_{n_1} \nabla l_{n_0} \nabla l_{n_1} \\
                ~&= 
                M^2 N(N-1)  \expct{\nabla l}^2 + 
                M^2 N       \expct{(\nabla l)^2}
            \end{align*}
            and
            \begin{align*}
                \expct{b(T)}
                = 
                ~&{MN \choose 2}\frac{1}{N^2}
                \sum_{n_0} \sum_{n_1} \nabla^2 l_{n_0} \nabla l_{n_1} \\
                =
                ~&\frac{M(MN-1)(N-1)}{2} \expct{\nabla^2 l} \expct{\nabla l} + \\
                ~&\frac{M(MN-1)}{2}      \expct{(\nabla^2 l) (\nabla l)} 
            \end{align*}
            We see that the expectations for $a$ and $a^2$ agree between pure
            SGD and pure GD.  So only $b$ contributes.  We conclude that pure
            GD's test loss exceeds pure SGD's by
            \begin{align*}
                   ~&\eta^2
                    \wrap{\frac{M(MN-1)(N-1)}{2}  - \frac{M^2N(N-1)}{2}}
                    \expct{\nabla^2 l} \expct{\nabla l}^2 \\
                +   ~&\eta^2 
                    \wrap{\frac{M(MN-1)N}{2} - \frac{M(M-1)N}{2}}
                    \expct{(\nabla^2 l) (\nabla l)} \expct{\nabla l} \\
                = 
                    ~&\eta^2     \frac{M(N-1)}{2}
                \expct{\nabla l} \wrap{
                      \expct{(\nabla^2 l) (\nabla l)}
                    - \expct{\nabla^2 l} \expct{\nabla l}
                }
            \end{align*}
            Since $(\nabla^2 l) (\nabla l) = \nabla((\nabla l)^2)/2$, we can 
            summarize this difference as
            $$
                \eta^2 \frac{M(N-1)}{4}
                G \nabla C 
            $$

    \subsection*{Effect of Nongaussian Noise at a Minimum}
        We consider vanilla SGD initialized at a local minimum of the test loss.
        One expects $\theta$ to diffuse around that minimum according to
        gradient noise.  We compute the effect on test loss of nongaussian
        diffusion.  Specifically, we compare SGD test loss on the loss
        landscape to SGD test loss on a different loss landscape defined as a
        Gaussian process whose every covariance agrees with the original
        landscape's.  We work to order $\eta^3$ because at lower orders,
        the gaussian landscapes will by construction match their nongaussian
        counterparts.

        \subsubsection*{Diagram Method}
            Because $\expct{\nabla l}$ vanishes at initialization, all diagrams
            with a degree-one vertex that is a singleton vanish.  Because we
            work at order $\eta^3$, we consider $3$-edged diagrams.  Finally,
            because all first and second moments match between the two
            landscapes, we consider only diagrams with at least one partition
            of size at least $3$.  The only such test diagram is
            $\sdia{c(012-3)(03-13-23)}$.

        \subsubsection*{Direct Perturbation}
            We compute the displacement $\theta_T-\theta_0$ to order $\eta^3$ 
            for vanilla SGD.  Expanding
            $
                \theta_t \in \theta_0 + \eta a_t + \eta^2 b_t + \eta^3 c_t 
                + o(\eta^3)
            $, we find:
            \begin{align*}
                \theta_{t+1}
                =
                \theta_t    &-  \eta \nabla l_{n_t} (\theta_t) \\
                \in\theta_0 &+  \eta a_t + \eta^2 b_t + \eta^3 c_t \\
                            &-  \eta \wrap{
                                     \nabla l_{n_t}
                                    +\nabla^2 l_{n_t} (\eta a_t + \eta^2 b_t)
                                    +\frac{1}{2} \nabla^3 l_{n_t} (\eta a_t)^2
                                }
                             +  o(\eta^3) \\
                =
                \theta_0    &+   \eta   \wrap{a_t - \nabla l_{n_t}} \\
                            &+   \eta^2 \wrap{b_t - \nabla^2 l_{n_t} a_t} \\ 
                            &+   \eta^3 \wrap{
                                     c_t
                                    -\nabla^2 l_{n_t} b_t
                                    -\frac{1}{2} \nabla^3 l_{n_t} a_t^2
                                 }
                             +   o(\eta^3)
            \end{align*}
            We thus have the recurrences
            $
                a_{t+1} = a_t - \nabla l_{n_t}
            $,
            $
                b_{t+1} = b_t - \nabla^2 l_{n_t} a_t
            $, and
            $
                c_{t+1} = c_t -\nabla^2 l_{n_t} b_t 
                              -\frac{1}{2} \nabla^3 l_{n_t} a_t^2
            $
            with solutions:
            $a_t = -\sum_{t} \nabla l_{n_t}$ and
            $\eta^2 b_t = +\eta^2 \sum_{t_0 < t_1} \nabla^2 l_{n_{t_1}} \nabla l_{n_{t_0}}$.
            %\begin{align*}
            %    \eta a_t = &-\eta \sum_{t} \nabla l_{n_t}
            %    \\ 
            %    \eta^2 b_t = &+\eta^2 \sum_{t_0 < t_1} \nabla^2 l_{n_{t_1}} \nabla l_{n_{t_0}}
                %\\
                %\eta^3 c_t^\mu =
                %    &-\sum_{t_0 < t_1 < t_2} 
                %        \nabla^\mu \nabla_\nu l_{n_{t_2}}
                %        \nabla^\nu \nabla_\sigma l_{n_{t_1}} \nabla^\sigma l_{n_{t_0}} \\
                %    &-\frac{1}{2}
                %        \sum_{t_a, t_b < t}
                %        \nabla^\mu \nabla^\nu \nabla^\sigma l_{n_t}
                %        \nabla_\nu l_{n_{t_a}}
                %        \nabla_\sigma l_{n_{t_b}}
            %\end{align*}
            %We use tensor indices above because the contraction pattern would
            %otherwise be ambiguous.
            We do not compute $c_t$ because we will soon see that it will be
            multiplied by $0$.

            To third order, the test loss of SGD is
            \begin{align*}
                l(\theta_T)
                \in
                        l(\theta_0)
                &+     (\nabla   l)   (\eta a_T + \eta^2 b_T + \eta^3 c_T)                              \\
                &+\frac{\nabla^2 l}{2}(\eta a_T + \eta^2 b_T             )^2                            \\
                &+\frac{\nabla^3 l}{6}(\eta a_T                          )^3 
                 +o(\eta)^3                                                                             \\
                =
                    l(\theta_0)
                &+  \eta       \wrap{(\nabla l) a_T                               }                     \\
                &+  \eta^2     \wrap{(\nabla l) b_T + \frac{\nabla^2 l}{2} a_T^2  }                     \\
                &+  \eta^3     \wrap{(\nabla l) c_T + (\nabla^2 l) a_T b_T + \frac{\nabla^3 l}{6} a_T^3}
                 +o(\eta)^3                                                                             
            \end{align*}
            Because $\expct{\nabla l}$ vanishes at initialization, we neglect
            the $(\nabla l)$ terms.  The remaining $\eta^3$ terms involve
            $a_T b_T$, and $a_T^3$.  So let us
            compute their expectations:
            \begin{align*}
                \expct{a_T b_T}
                    =&- \sum_{t} \sum_{t_0 < t_1}
                        \expct{\nabla l_{n_t} \nabla^2 l_{n_{t_1}} \nabla l_{n_{t_0}}}
                    \\
                    =&- \sum_{t_0 < t_1}  
                        \sum_{t \notin \{t_0, t_1\}} 
                            \expct{\nabla l_{n_t}} \expct{\nabla^2 l_{n_{t_1}}} \expct{\nabla l_{n_{t_0}}}
                    \\&- \sum_{t_0 < t_1}  
                        \sum_{t = t_0}
                            \expct{\nabla l_{n_t} \nabla l_{n_{t_0}}} \expct{\nabla^2 l_{n_{t_1}}}
                    \\&- \sum_{t_0 < t_1}  
                        \sum_{t = t_1}
                            \expct{\nabla l_{n_t} \nabla^2 l_{n_{t_1}}} \expct{\nabla l_{n_{t_0}}}
            \end{align*}
            Since $\expct{\nabla l}$ divides $\expct{a_T b_T}$, the latter
            vanishes.
            \begin{align*}
                \expct{a_T^3}
                    =&- \sum_{t_a, t_b, t_c}
                            \expct{\nabla l_{n_{t_a}} \nabla l_{n_{t_b}} \nabla l_{n_{t_c}}}
                    \\
                    =&- \sum_{\substack{t_a, t_b, t_c\\ \text{disjoint}}}  
                            \expct{\nabla l_{n_{t_a}}} \expct{\nabla l_{n_{t_b}}} \expct{\nabla l_{n_{t_c}}}
                    \\&-3 \sum_{t_a=t_b\neq t_c}  
                            \expct{\nabla l_{n_{t_a}} \nabla l_{n_{t_b}}} \expct{\nabla l_{n_{t_c}}}
                    \\&-\sum_{t_a=t_b=t_c}  
                            \expct{\nabla l_{n_{t_a}} \nabla l_{n_{t_b}} \nabla l_{n_{t_c}}}
            \end{align*}
            As we initialize at a test minimum, only the last line remains, at
            it has $T$ identical summands.
            When we plug into the expression for SGD test loss, we get
            $$
                \frac{T \eta^3 }{6}
                \expct{\nabla^3 l}
                \expct{\nabla l_{n_{t_a}} \nabla l_{n_{t_b}} \nabla l_{n_{t_c}}}
            $$

            %\begin{align*}
            %    \expct{a_T^3}
            %       &-\eta^3 \sum_{t} \sum_{t_0 < t_1}
            %            \nabla l_{n_t} \expct{\nabla^2 l_{n_{t_1}} \nabla l_{n_{t_0}}}
            %       \\
            %       &-\eta^3 \sum_{t} \sum_{t_0 < t_1}
            %            \nabla l_{n_t} \nabla^2 l_{n_{t_1}} \nabla l_{n_{t_0}}
            %\end{align*}

    %\subsection*{A Nonconservative Force (Unrenormalized)}
    %    We identify the leading order nonconservative component in vanilla
    %    SGD's evolution of a weight initialized at a test minimum.
    %    \subsubsection*{Diagram Method}
    %    \subsubsection*{Direct Perturbation}

    \subsection*{The Effect of Inter-Epoch Shuffling}
        We identify the leading order effect of shuffling on test loss for
        pure, multi-epoch SGD.  It is much harder to see by direct perturbation
        than by diagrams that this is an order $\eta^4$ effect, so for fairness
        we will assume this order it already known for both methods.

        \subsubsection*{Diagram Method}

            {\color{moor} FILL IN}
            
        \subsubsection*{Direct Perturbation}
            We compute the displacement $\theta_T-\theta_0$ to order $\eta^4$ 
            for vanilla SGD.  Expanding
            $
                \theta_t \in \theta_0
                    + \eta a_t + \eta^2 b_t + \eta^3 c_t + \eta^4 d_t 
                    + o(\eta^4)
            $, we find:
            \begin{align*}
                \theta_{t+1}
                =
                \theta_t    &- \eta \nabla l_{n_t} (\theta_t) \\
                \in\theta_0 &+ \eta a_t + \eta^2 b_t + \eta^3 c_t + \eta^4 d_t \\
                            &- \eta \wrap{
                                    \nabla l_{n_t}
                                   +\nabla^2 l_{n_t}             (\eta a_t + \eta^2 b_t + \eta^3 c_t)
                               } \\
                            &- \eta \wrap{
                                    \frac{1}{2} \nabla^3 l_{n_t} (\eta a_t + \eta^2 b_t)^2
                                   +\frac{1}{6} \nabla^4 l_{n_t} (\eta a_t)^3
                               }
                             + o(\eta^4) \\
                =
                \theta_0    &+ \eta   \wrap{a_t - \nabla l_{n_t}} \\
                            &+ \eta^2 \wrap{b_t - \nabla^2 l_{n_t} a_t} \\ 
                            &+ \eta^3 \wrap{
                                   c_t
                                  -\nabla^2 l_{n_t} b_t
                                  -\frac{1}{2} \nabla^3 l_{n_t} a_t^2
                               } \\
                            &+ \eta^4 \wrap{
                                   d_t
                                  -            \nabla^2 l_{n_t} c_T
                                  -\frac{1}{2} \nabla^3 l_{n_t} b_T^2 
                                  -\frac{1}{6} \nabla^4 l_{n_t} a_T^3 
                               }
                             + o(\eta^4)
            \end{align*}
            We thus have the recurrences
            $
                a_{t+1} = a_t - \nabla l_{n_t}
            $,
            $
                b_{t+1} = b_t - \nabla^2 l_{n_t} a_t
            $,
            $
                c_{t+1} = c_t -\nabla^2 l_{n_t} b_t 
                              -\frac{1}{2} \nabla^3 l_{n_t} a_t^2
            $, and
            $
                d_{t+1} = d_t -             \nabla^2 l_{n_t} c_T
                              - \frac{1}{2} \nabla^3 l_{n_t} b_T^2 
                              - \frac{1}{6} \nabla^4 l_{n_t} a_T^3 
            $
            with solutions:
            \begin{align*}
                \eta a_t = &-\eta \sum_{t} \nabla l_{n_t}
                \\ 
                \eta^2 b_t = &+\eta^2 \sum_{t_0 < t_1} \nabla^2 l_{n_{t_1}} \nabla l_{n_{t_0}}
                \\
                \eta^3 c_t^\mu =
                    &-\sum_{t_0 < t_1 < t_2} 
                        \nabla^\mu \nabla_\nu l_{n_{t_2}}
                        \nabla^\nu \nabla_\sigma l_{n_{t_1}} \nabla^\sigma l_{n_{t_0}} \\
                    &-\frac{1}{2}
                        \sum_{t_a, t_b < t}
                        \nabla^\mu \nabla^\nu \nabla^\sigma l_{n_t}
                        \nabla_\nu l_{n_{t_a}}
                        \nabla_\sigma l_{n_{t_b}}
                \\
                \eta^4 d_t^\mu =
                    &-\sum_{t_0 < t_1 < t_2} 
                        \nabla^\mu \nabla_\nu l_{n_{t_2}}
                        \nabla^\nu \nabla_\sigma l_{n_{t_1}} \nabla^\sigma l_{n_{t_0}} \\
                    &-\frac{1}{2}
                        \sum_{t_a, t_b < t}
                        \nabla^\mu \nabla^\nu \nabla^\sigma l_{n_t}
                        \nabla_\nu l_{n_{t_a}}
                        \nabla_\sigma l_{n_{t_b}}
            \end{align*}
            We use tensor indices above because the contraction pattern would
            otherwise be ambiguous.

            {\color{moor} FILL IN}


\section*{E. Bessel Factors for Estimating Multipoint Correlators from Data}
    Given samples from a joint probability space $\prod_{0\leq d<D} X_d$, we
    seek unbiased estimates of multipoint correlators (i.e. products of
    expectations of products) such as $\wang{x_0 x_1 x_2}\wang{x_3}$.  For
    example, say $D=2$ and from $2S$ samples we'd like to estimate $\wang{x_0
    x_1}$.  Most simply, we could use $\Avg_{0\leq s<2S} x_0^{(s)} x_1^{(s)}$,
    where $\Avg$ denotes averaging.  In fact, the following also works:
    \begin{equation} \label{eq:bessel}
        S
        \wrap{\Avg_{0\leq s< S} x_0^{(s)}}
        \wrap{\Avg_{0\leq s< S} x_1^{(s)}}
        +
        (1-S)
        \wrap{\Avg_{0\leq s< S} x_0^{(s)}}
        \wrap{\Avg_{S\leq s<2S} x_1^{(s)}}
    \end{equation}
    When multiplication is expensive (e.g. when each $x_d^{(s)}$ is a tensor
    and multiplication is tensor contraction), we prefer the latter, since it
    uses $O(1)$ rather than $O(S)$ multiplications.  This in turn allows more
    efficient use of large-batch computations on GPUs.  We now generalize this
    estimator to higher-point correlators (and $D\cdot S$ samples).

    For uniform notation, we assume without loss that each of the $D$ factors
    appears exactly once in the multipoint expression of interest; such
    expressions then correspond to partitions on $D$ elements, which we
    represent as maps $\mu:\wasq{D}\to \wasq{D}$ with $\mu(d)\leq d$ and
    $\mu\circ \mu=\mu$.  Note that $\wabs{\mu} \coloneqq \wabs{im(\mu)}$ counts
    $\mu$'s parts.  We then define the statistic
    $$
        \wurl{x}_\mu
        \coloneqq
        \prod_{0\leq d<D} \Avg_{0\leq s<S} x_d^{(\mu(d)\cdot S + s)}
    $$
    and the correlator $\wang{x}_\mu$ we define to be the expectation of 
    $\wurl{x}_\mu$ when $S=1$.  In this notation, \ref{eq:bessel} says: 
    $$
        \wang{x}_{\partbox{0}\partbox{1}}
        =
        \expct{
            S       \cdot \wurl{x}_{\partbox{0 1}} +
            (1-S)   \cdot \wurl{x}_{\partbox{0}\partbox{1}}
        }
    $$
    Here, the boxes indicate partitions of $\wasq{D}=\wasq{2}=\{0,1\}$.
    Now, for general $\mu$, we have:
    \begin{equation} \label{eq:newbessel}
        \expct{S^D \wurl{x}_\mu}
        =
        \sum_{\tau\leq \mu} \wrap{
            \prod_{0\leq d<D}
                \frac{S!}{\wrap{S-\wabs{\tau(\mu^{-1}(d))}}!}
        }
        \wang{x}_\tau
    \end{equation}
    where `$\tau \leq \mu$' ranges through partitions \emph{finer} than 
    $\mu$, i.e. maps $\tau$ through which $\mu$ factors.   
    In smaller steps, \ref{eq:newbessel} holds because
    \begin{align*}
        \expct{S^D \wurl{x}_\mu}
        &=
        \expct{
            \sum_{(0\leq s_d<S) \in \wasq{S}^D}
            \prod_{0\leq d<D}
            x_d^{\wrap{\mu(d)\cdot S + s_d}}
        }\\
        &=
        \sum_{\substack{(0\leq s_d<S) \\ \in \wasq{S}^D}}
        \expct{
            \prod_{0\leq d<D}
            x_d^{\wrap{\min \wurl{
                \tilde{d}~:~\mu(\tilde{d})\cdot S+s_{\tilde{d}} = \mu(d)\cdot S+s_d
            }}}
        }\\
        &=
        \sum_{\tau} \wabs{\wurl{\substack{
            (0\leq s_d<S)~\in~[S]^D~: \\
            \wrap{\substack{
                \mu(d)=\mu(\tilde{d}) \\
                \wedge~s_d=s_{\tilde{d}}
            }}
            \Leftrightarrow
            \tau(d)=\tau(\tilde{d})
        }}}
        \wang{x}_\tau \\
        &=
        \sum_{\tau\leq \mu} \wrap{
            \prod_{0\leq d<D}
                \frac{S!}{\wrap{S-\wabs{\tau(\mu^{-1}(d))}}!}
        }
        \wang{x}_\tau
    \end{align*}

    Solving \ref{eq:newbessel} for $\wang{x}_\mu$, we find:
    \begin{equation*}
        \text{\fbox{$
        \wang{x}_\mu
        =
        \frac{S^D}{S^{\wabs{\mu}}}
        \expct{
            \wurl{x}_\mu
        }
        -
        \sum_{\tau < \mu} \wrap{
            \prod_{d\in im(\mu)}
            \frac{\wrap{S-1}!}{\wrap{S-\wabs{\tau(\mu^{-1}(d))}}!}
        }
        \wang{x}_\tau
        $}}
    \end{equation*}
    This expresses $\wang{x}_\mu$ in terms of the batch-friendly estimator
    $\wurl{x}_\mu$ as well as correlators $\wang{x}_\tau$ for $\tau$ 
    \emph{strictly} finer than $\mu$.  We may thus (use dynamic programming to)
    obtain unbiased estimators $\wang{x}_\mu$ for all partitions $\mu$. 
    Symmetries of the joint distribution and of the multilinear multiplication
    may further streamline estimation by turning a sum over $\tau$ into a
    multiplication by a combinatorial factor.  For example, with complete
    symmetry:
    $$
        \wang{x}_{\partbox{012}}
        =
        S^2
        \wurl{x}_{\partbox{012}}
        -
        \frac{(S-1)!}{(S-3)!}
        \wurl{x}_{\partbox{0}\partbox{1}\partbox{2}}
        -
        3\frac{(S-1)!}{(S-2)!}
        \wurl{x}_{\partbox{0}\partbox{12}}
    $$
    We use such expressions throughout our experiments to estimate the
    (expected) values of diagrams.

\section*{F. Loss Landscapes Used for Experiments}
    In addition to the clarifyingly artificial loss landscapes (Gaussian Fit,
    Linear Screw, and Mean Estimation) described in the main text, we tested
    our predictions on logistic linear regression and simple convolutional
    networks (2 convolutional weight layers each with kernel $5$, stride $2$,
    and $10$ channels, followed by two dense weight layers with hidden
    dimension $10$) for the CIFAR-10 and Fashion-MNIST datasets.  The
    convolutional architectures used $\tanh$ activations and Gaussian Xavier
    initialization.  We parameterized the model so that the Gaussian-Xavier
    initialization of the linear maps in each layer differentially pulls back
    to standard normal initializations of the parameters.
    
    For these non-artificial landscapes, we regard the finitely much available
    data as the true (sum of diracs) distribution from which we sample test and
    train sets in i.i.d.  manner (and hence ``with replacement'').  We do this
    to gain practical access to a ground truth against which we may compare our
    predictions.  One might object that this sampling procedure would cause
    test and train sets to overlap, hence biasing test loss measurements.  In
    fact, test and train sets overlap only in reference, not in sense: the
    situation is analogous to a text prediction task in which two training
    points culled from different corpora happen to record the same sequence of
    words, say, ``Thank you!''.  In any case, all of our experiments
    focus on the scanty-data regime, e.g. $10^1$ datapoints out of $\sim
    10^{4.5}$ dirac masses, so overlaps are diluted. 

\section*{G. Additional Figures}
    {\color{moor} FILL IN}



\end{document}
