        The following three appendices serve three respective functions:
        \setlist{nolistsep}
        \begin{itemize}[noitemsep]
            \item to explain how to calculate using diagrams;
            \item to prove our results (and pose a conjecture);
            \item to specify our experimental methods and results.
        \end{itemize}

        %.

        %\emph{{\color{moor!90}Warning}:
        %We will revise these appendices substantially between the paper body's
        %due date of Sept 8 and the supplementary materials' due date of Sept
        %11.  For example: the Sept 8 version's appendices have in some places
        %a sign convention (gradient \emph{ascent} to maximize fitness) opposite
        %the paper body's convention and we are still translating everything to
        %\emph{descent}.  We are also working to polish the mathematics
        %for easier digestion and to fix some of \LaTeX cross-references
        %broken in our recent revisions.
        %}

        %.

        In more detail, we organize the appendices as follows.\\
    
        {\bf
        \par\noindent A ~ Tutorial: how to use diagrams}                        \hfill {\bf page \pageref{appendix:tutorial}}
        \par\indent     A.1 ~~ An example calculation: the effect of epochs     \hfill \pageref{appendix:example}
        \par\indent     A.2 ~~ How to identify the relevant grid                \hfill \pageref{appendix:draw-spacetime} 
        \par\indent     A.3 ~~ How to identify the relevant diagram histories  \hfill \pageref{appendix:draw-histories}
        \par\indent     A.4 ~~ How to evaluate each history                   \hfill \pageref{appendix:evaluate-histories}
        \par\indent     A.5 ~~ How to sum the histories' values                \hfill \pageref{appendix:sum-histories}
        %\par\indent     A.6 ~~ Interpreting diagrams intuitively                \hfill \pageref{appendix:interpret-diagrams}
        \par\indent     A.6 ~~ How to solve variant problems                    \hfill \pageref{appendix:solve-variants}
        \par\indent     A.7 ~~ Do diagrams streamline computation?              \hfill \pageref{appendix:diagrams-streamline}
    
        {\bf
        \par\noindent B ~ Mathematics of the theory}                            \hfill {\bf page \pageref{appendix:math}}
        \par\indent     B.1 ~~ Setting and assumptions                          \hfill \pageref{appendix:assumptions}
        \par\indent     B.2 ~~ A key lemma \`a la Dyson                         \hfill \pageref{appendix:key-lemma}
        \par\indent     B.3 ~~ From Dyson to diagrams                           \hfill \pageref{appendix:toward-diagrams}
        %\par\indent     B.4 ~~ Interlude: a review of M\"obius inversion       \hfill \pageref{appendix:mobius}
        \par\indent     B.4 ~~ Proof of Theorem \ref{thm:resum}                 \hfill \pageref{appendix:resum}
        \par\indent     B.5 ~~ Proof of Theorem \ref{thm:converge}              \hfill \pageref{appendix:converge}
        \par\indent     B.6 ~~ Proofs of corollaries                            \hfill \pageref{appendix:corollaries}
        \par\indent     B.7 ~~ Future topics                                    \hfill \pageref{appendix:future}
    
        {\bf
        \par\noindent C ~ Experimental methods}                                 \hfill {\bf page \pageref{appendix:experiments}}
        \par\indent     C.1 ~~ What artificial landscapes did we use?           \hfill \pageref{appendix:artificial}  
        \par\indent     C.2 ~~ What image-classification landscapes did we use? \hfill \pageref{appendix:natural}
        \par\indent     C.3 ~~ Measurement process                              \hfill \pageref{appendix:measure}
        \par\indent     C.4 ~~ Implementing optimizers                          \hfill \pageref{appendix:optimizers}
        \par\indent     C.5 ~~ Software frameworks and hardware                 \hfill \pageref{appendix:frameworks}
        \par\indent     C.6 ~~ Unbiased estimators of landscape statistics      \hfill \pageref{appendix:bessel}
        \par\indent     C.7 ~~ Additional figures                               \hfill \pageref{appendix:figures}

        %{\bf
        %\par\noindent D ~ Review of Tensors}                                    \hfill {\bf page \pageref{appendix:tensors}}
        %\par\indent     D.1 ~~ Vectors versus covectors                         \hfill \pageref{appendix:tensor-variance}
        %\par\indent     D.2 ~~ What is a tensor?                                \hfill \pageref{appendix:what-tensor}  
        %\par\indent     D.3 ~~ Tensors of type $^u_d$                           \hfill \pageref{appendix:what-tensor}  
        %\par\indent     D.4 ~~ Contraction of tensors                           \hfill \pageref{appendix:tensor-contraction}

\onecolumn


