            We sketch a path toward
            Corollary \ref{cor:entropic} while avoiding the generality 
            and notation of 
            \S\ref{sect:calculus}.\squash\squash 

        \subsection{Notation and assumptions, I}\label{sect:setup}
            %--  the landscape  -----------------------------------------------
    
            We formalize the loss --- suffered by a fixed architecture on a
            random datapoint --- as a distribution $\Dd$ over functions from a
            space $\Mm$ of weights.  The \emph{testing loss} $l:\Mm\to\RR$ is
            $\Dd$'s mean.  We write $\theta\in\Mm$, $l_x\sim\Dd$ for generic
            elements.
            %
            We consider training sequences $(l_n: 0\leq n<N) \sim \Dd^N$.  We
            refer to $n$ and to $l_n$ as \emph{training points}.
            %
            Each initialization $\theta_0 \in \Mm$ then induces --- via SGD ---
            a distribution over trajectories $(\theta_t: 0\leq t \leq T)$.
            Specifically, SGD runs $T$ steps of $\eta$-steepest descent:
                \squash
            $$
                \textstyle
                \theta_{t+1}^\mu
                \coloneqq
                \theta_t^\mu -
                \sum_{\nu}
                \eta^{\mu\nu} \nabla_\nu l_{n_t}(\theta_t)
                \squash
            $$
            where each sequence $(n_t: kN\leq t<kN+N)$ is a permutation of $(n:
            0\leq n<N)$.  Here, Greek indices name components of
            $\theta,\eta,\nabla$ with respect to a fixed basis.  We view $\eta$
            as a bilinear form, not for generality but to constrain the natural
            operations to those of geometric significance.
            %

            Throughout this paper we assume:
            %
            \textbf{Derivative Bounds}:
            there compact sets $(K_k: k\geq 0)$ so that
            $\nabla^n l_x(\theta)\in K_k$ for all $\theta,l_x$. 
            Here $\nabla^k
            l_x(\theta)$ is a $k$th derivative, a tensor with $k$ axes.
            %
            \textbf{Analytic Moments}:
            any polynomial $p$ of the $l_x$ and its higher derivatives
            induces a random variable so that
            $\expct{p}:\Mm\to\RR$ (exists and) is analytic in $\theta$;
            moreover, $\expct{p}$'s radii of convergence are strictly bounded from $0$,
            even as $\theta$ varies.
            %
            Consequently, $\nabla \expct{p}=\expct{\nabla p}$.

            We quote a well-known prop
            (\cite{ne04}, \S 2.1) to illustrate our notation:
            \begin{prop}\label{prop:nest}
                $G = \nabla l(\theta_0)$ controls the loss to leading order:
                $
                    \expc[l(\theta_T)-l(\theta_0)]^\mu =
                    - 
                    T \sum_{\mu\nu} G_\mu \eta^{\mu\nu} G_\nu
                    + o(\eta^1)
                $.
            \end{prop}
            A proof inducts on $T$.
            For noiseless linear landscapes, (i.e., when $\nabla l_x(\theta)$
            depends on neither $x$ nor $\theta$), this estimate is exact.
            %
            This paper identifies how noise and curvature correct Prop
            \ref{prop:nest} by developing and using large-$T$ techniques less
            opaque and more convergent than induction.
            %

        \subsection{Taylor series: method and challenges}\label{sect:challenges}
            %\subsubsection{Proving Prop \ref{prop:nest}}
            Let's study $\expct{\theta_T}, \expct{l(\theta_T)}$.  To warm
            up, we'll prove Prop \ref{prop:nest} (c.f.\ \cite{ne04,ro18}). 
            \begin{proof} %(of Prop \ref{prop:nest}).
                By gradient bounds: $\theta_T - \theta_0$ is $O(\eta^1)$.
                We \textbf{claim} that $(\theta_T - \theta_0)^\mu =
                -T\sum_\nu \eta^{\mu\nu}G_\nu + o(\eta^1)$.
                %
                The claim holds when $T=0$.  Say the claim holds for
                ${\tilde T}$-step SGD with
                $T = {\tilde T}+1$.  Then:\squash
                \begin{align*}
                    \wrap{\theta_{T} - \theta_{{\tilde T}}}^\mu
                    &= - \textstyle\sum_{\nu} \eta^{\mu\nu} \nabla_\nu l_{n_{\tilde T}}(\theta_{{\tilde T}}) \\
                    &= - \textstyle\sum_{\nu} \eta^{\mu\nu} \nabla_\nu \wrap{
                             l_{n_{\tilde T}}(\theta_0)
                             + \text{\translucent{moosky}{$\sum_{\xi} \nabla_\xi l_{n_{\tilde T}}(\theta_0) (\theta_{{\tilde T}} - \theta_0)^\xi$}}
                             + o(\theta_{{\tilde T}} - \theta_0)
                         } \\ 
                    &= - \textstyle\sum_{\nu} \eta^{\mu\nu} \nabla_\nu \wrap{
                                l_{n_{\tilde T}}(\theta_0)
                                + \nabla l_{n_{\tilde T}}(\theta_0) \cdot O(\eta^1) + o(O(\eta^1))
                            } \\
                    &= \textstyle\text{\translucent{moolime}{$- \sum_\nu \eta^{\mu\nu} \nabla_\nu l_{{\tilde T}}(\theta_0)$}} + o(\eta^1)
                \end{align*}
                Applying the induction hypothesis proves the claim.
                %
                We plug the claim into $l$'s Taylor series:
                \begin{align*}
                    \expc[l(\theta_T) - l(\theta_0)]
                    &= \textstyle \sum_\mu \nabla_\mu l(\theta_0) \text{\translucent{moopink}{$\expc[\theta_T - \theta_0)]^\mu$}} + \expc[o(\theta_T - \theta_0)] \\
                    &= \textstyle \sum_\mu \nabla_\mu l(\theta_0) (-T\eta G + o(\eta^1)) + o(O(\eta^1)) \\
                    &= \textstyle \text{\translucent{moogold}{$- \sum_{\mu\nu} T G_\mu \eta^{\mu\nu} G_\nu$}}+ o(\eta^1)
                \end{align*}
                Indeed, due to analytic moments, the above expectations of
                $o(\eta^1)$ terms are still $o(\eta^1)$.
            \end{proof}

            \subsubsection{What happens when we keep higher order terms?}

            \textsc{Multiple Moments} ---
            We used above
            that, to order $\eta^1$, $\expct{l(\theta_T)}$ depends on the
            training data only through the first moment
            \translucent{moopink}{$\expc[\theta_T - \theta_0]$}.\squish\squish\ But to compute  
            $\expct{l(\theta_T)}$ to higher order, we'd also need
            the $k$th moments $M_k^{\mu_0\mu_1\cdots} = \expc\wasq{\prod_i (\theta_T - \theta_0)^{\mu_i}}$.\squish\  We may achieve this by inductively
            proving multiple \textbf{claim}s, one for each moment. 

            \textsc{Tuples of Times} --- Complications arise even as we compute $M_1$\squish\ to order $\eta^2$.  We may not neglect the gradient
            correction $\nabla (\text{\translucent{moosky}{$\nabla
            l_{n_{\tilde T}}(\theta_0) \cdot (\theta_{{\tilde T}} -
            \theta_0)$}})$\squash\ at the $\tilde{T}$th induction step. As the
            displacement $\theta_{{\tilde T}} - \theta_0$ contains (to order
            $\eta^1$) $\tilde T$ terms, so will the correction.
            %
            Totalling the correction over time thus yields
            $\sum_{0\leq \tilde T<T}\tilde T = {T \choose 2}$\squish\
            summands, each (e.g.\ $\nabla\nabla l_{5} \nabla l_{2}$)
            involving a \emph{pair} of times.  Order-$d$
            corrections represent the joint influence of $d$-tuples of times.
            %
            Prop \ref{prop:nest}'s result $\sum_{\tilde T}
            \wrap{\text{\translucent{moolime}{$- \eta \nabla l_{{\tilde
            T}}(\theta_0)$}}}$\squish\squish\ is degree $1$ in $T$; but the
            order-$d$ displacement is a degree $d$ polynomial --- very divergent --- in $T$.

            \textsc{Factoring's Failure} --- To obtain \translucent{moogold}{$-TG\eta G$},\squish\ we multiplied
            $l$'s derivatives by the expectations of such summands.
            %
            In contrast to Prop \ref{prop:nest}, these expectations, even those of a fixed
            degree in $\eta$, now vary in form due to noise: some (e.g.\
            $\nabla\nabla l_{5} \nabla l_{2}$) have statistically independent
            factors that permit expectations to factor; others (e.g.\
            $\nabla\nabla l_{5} \nabla l_{5}$) do not.  This is how $\nabla
            l_x$'s higher cumulants (such as the covariance and skew of the gradient distribution) appear in our analysis.

            \textsc{Diverse Derivatives} --- 
            At order $\eta^3$, a hessian correction $\nabla((\theta_{{\tilde T}} -
            \theta_0) \cdot \nabla \nabla l_{n_{\tilde T}}(\theta_0) \cdot
            (\theta_{{\tilde T}} - \theta_0)/2)$ augments the gradient
            correction.
            %
            Then $M_1$'s order-$\eta^3$ summands vary in form, even when all
            expectations factor (as happens on noiseless landscapes).  For
            instance, the hessian and gradient corrections respectively induce
            order-$\eta^3$ summands of $\expct{l(\theta_T)}$ such as
            \squash
            $$
                \textstyle
                \sum_{\substack{\mu\nu\xi \\ \omicron\pi\rho}}
                    \eta^{\mu\omicron} \, \eta^{\nu\pi} \, \eta^{\xi\rho}
                    \,
                    (\nabla_\mu l_x)
                    \,
                    (\nabla_\nu l_y)
                    \,
                    (\nabla_\omicron\nabla_\pi\nabla_\rho l_z)
                    \,
                    (\nabla_\xi l)
                %
                \hspace{0.75cm}
                %
                \sum_{\substack{\mu\nu\xi \\ \omicron\pi\rho}}
                    \eta^{\mu\omicron} \, \eta^{\nu\pi} \, \eta^{\xi\rho}
                    \,
                    (\nabla_\mu l_x)
                    \,
                    (\nabla_\omicron\nabla_\nu l_y)
                    \,
                    (\nabla_\pi \nabla_\xi l_z)
                    \,
                    (\nabla_\rho l)
                \squash
            $$
            And $M_2, M_3$'s terms are yet more diverse.  In short, a Taylor
            expansion even to low degrees yields a combinatorial explosion of
            terms.  Our paper develops tools to organize and interpret these
            terms.

            \subsubsection{Diagrams in brief}\label{sect:diagrams-in-brief}

            That development begins with the observation that each $\eta$
            `connects' two $\nabla$ operators as indices prescribe.  So we
            draw
            $\eta$s as edges, $\nabla^k l$s as nodes, and each
            summand of the form
            \squash
            $$
                \sum_{\text{all Greek indices}} \wrap{\prod_{j\in J} \eta^{\mu_j\nu_j}}
                \wrap{\prod_{i\in I} \wrap{\prod_{k\in K_i} \nabla_{\xi_{i,k}}}
                l_{x_i}} \wrap{\prod_{k\in K_\star} \nabla_{\xi_{\star,k}}}
                l\,\,\,\,\,\,\,\,\text{(evaluated at $\theta=\theta_0$)}
            \squash
            $$
            as an undirected graph with edges indexed by $j\in J$, nodes
            indexed by $i\in I\sqcup \{\star\}$, and an edge $j$ incident
            to a node $i$ when $\{\xi_{i,k}:k\in K_i\}$ meets
            $\{\mu_j,\nu_j\}$.  Per \textsc{factoring}, we also equip these
            graphs with a partition of nodes to account for correlation
            structure.
            %
            Such diagrams have advantages of compactness and of clarity as
            decimal numerals have over unary numerals.  More importantly,
            their topology has dynamical significance; diagrams thus permit a
            physical interpretation of SGD.

        \subsection{An entropic force with curl}\label{sect:entropic-curl}
            The displacement $M_1=\expc[\theta_T - \theta_0]$ contains many
            order-$\eta^3$ summands, including those of the form\squash
            $$
                \textstyle
                \Delta^\xi_{xyz} \propto 
                -
                \sum_{\substack{\mu\nu    \\ \omicron\pi\rho}}
                    \eta^{\mu\omicron} \, \eta^{\nu\pi} \, \eta^{\xi\rho}
                    \,
                \expct{
                    (\nabla_\mu l_x)
                    \,
                    (\nabla_\nu l_y)
                    \,
                    (\nabla_\omicron\nabla_\pi\nabla_\rho l_z)
                }
                \squash
            $$
            where $0\leq x,y,z<N$ label datapoints.
            Let $\Delta_\circ = \expc[\Delta_{xxz}-\Delta_{xyz}]$ for $x,y,z$
            distinct.\squash\  $l_x, l_y, l_z$ are i.i.d., so:
            $ 
                \Delta_\circ^\xi \propto 
                -
                \sum_{\cdots}
                \eta^{\mu\omicron} \, \eta^{\nu\pi} \, \eta^{\xi\rho}
                C_{\mu\nu} J_{\omicron\pi\rho}
            $
            or, schematically, 
                $\boxed{\Delta_\circ^\xi
                \propto -\eta^3 C\nabla H}$.\squish\ 
            Here, $C_{\mu\nu} = \expc_x[\nabla_{\mu} l_x \nabla_{\nu} l_x] - G_\mu G_\nu$ is the covariance of gradients,
            $H_{\pi\rho} = \nabla_\pi\nabla_\rho l$ is $l$'s hessian, and
            $J_{\omicron\pi\rho} = \nabla_\omicron H_{\pi\rho}$
            is $l$'s `jerk'.\footnote{
                `\textbf{J}erk' is a standard term for third derivatives in dynamical systems:
                \href{https://www.iso.org/obp/ui/\#iso:std:iso:2041:ed-3:v1:en}{ISO 2041 (2009)}, \S1.
            }

            \begin{wrapfigure}{r}{0.3\textwidth}
                \centering
                \crunch\squash
                \plotmoow{colt/cubic}{0.3\textwidth}{}
                \caption{%
                    {Gradient noise pushes SGD toward minima flat w.r.t.\ $C$.}%$\mathbf{C}$.}
                        \small
                        A 2D loss near
                        a valley of minima.  Red densities show typical
                        $\theta$s, perturbed by noise ($C$),
                        in two cross sections of the valley.  The hessian
                        changes across the valley: $J \neq 0$.  
                }
                \label{fig:cubic}
                \crunch
            \end{wrapfigure}
            The expression $-\eta^3 C\nabla H$ suggests that SGD moves
            toward flat minima (see  Figure \ref{fig:cubic}).
            %
            The bilinear form
            $F^{\mu\nu}=\textstyle\sum_{\xi\omicron}\eta^{\mu\xi}
            \eta^{\nu\omicron} C_{\xi\omicron}$ determines which $H$s
            are `large' or `small'.  E.g.: if $F$
            degenerates along a covector $v$, then $H$'s 
            $v$-component does not affect $\Delta_\circ$.
            %\footnote{
            %    Explicitly: if
            %        $\sum_{\mu\nu} v_\mu F^{\mu\nu} v_\nu = 0$,
            %    then replacing $H$ by $\tilde H_{\mu\nu} = H_{\mu\nu} + 42 v_\mu v_\nu$ 
            %    will not change $\Delta_\circ$.
            %}
            %
            Diagram techniques establish\squash\
            this `entropic force'\footnote{
                Thermal systems tend toward disorder as if pushed by an
                `entropic force'.
                So arises the tension of rubber
                bands: their polymers can wreathe in
                many ways; can be straight in only one.
                Such `forces' characteristically depend linearly on the
                temperature (i.e., noise scale).  As the Corollary
                scales with $C$, we regard it as describing an entropic force.
            }\footnote{
                Our result ($T\gg 1$) is $\Theta(\eta^2)$; \cite{ya19b}'s
                ($T=2$) is $\Theta(\eta^3)$.  We
                integrate noise over time, amplifying $C$'s
                effect. 
            }
            as dominant when $G=0$, $N=T$.  
            Evaluating a single diagram
            ($
                \sdia{c(01-2-3)(02-12-23)}
            $) yields:\squash\squash
            %
            \begin{cor}\label{cor:entropic}%[Computed from $\sdia{c(01-2-3)(02-12-23)}$]
                Start SGD at a minimum of $l$ with $H>0$, $N=T$ and use
                an eigenbasis of $K=\eta H$.  
                For any $T$, the final displacement is
                \squish\squish
                $$
                    M_1^\xi =
                    -
                    {\textstyle\sum_{\substack{\mu\nu    \\ \omicron\pi\rho}}}
                        C_{\mu\omicron}
                        {\color{gray}{\mathcal P}_T(K_{\mu\mu} + K_{\omicron\omicron})}
                        \eta^{\mu\nu}\eta^{\omicron\pi}
                        J_{\nu\pi\xi}
                        {\color{gray}{\mathcal P}_T(K_{\xi\xi})}\eta^{\xi\rho}/2
                    + o(\eta^3)
                    \squish\squish
                $$
                with ${\mathcal P}_T(s) = (1 - \exp(-Ts))/s$.
            \end{cor}\squash\squash

            Consider a 1D valley of near-minima wherein $\eta H$ has spectrum
            $\lambda_0 \ll 1/T \ll \lambda_1 \leq \cdots$ for eigenvectors
            $v_i$.  Let's ignore diffusion along the valley: $(\eta C) v_0 =
            0$.
            %
            Then ${\mathcal P}_T(\lambda_0)\approx T$ and every
            $C_{ij}{\mathcal P}_T(\lambda_i+\lambda_j)$ is $O(T^0)$.  So $M_1$
            scales linearly with $T$.  We thus expect SGD to move with
            velocity $-f(\eta,H,T)\cdot C\nabla H/2$ toward flat minima.  
            Observe that $\nabla ({\frak G}_C \star l) = \nabla l+C\nabla
            H/2 + o(C)$, where ${\frak G}_C \star$ denotes convolution with a
            \emph{fixed} centered $C$-shaped Gaussian; we conclude with the
            intuition that \emph{SGD descends on a $C$-smoothed landscape that
            changes as $C$ does}.  Since the smoothed $l$ may itself evolve,
            SGD might eternally circulate.
      
        %\subsection{Curl in the entropic force}
            \begin{figure}[h!]
                \centering
                \rotatebox[origin=c]{-90}{\plotmoow{plots/from-above}{0.19\textwidth}{}}
                \plotmooh{colt/screw-trajectory-cropped}{}{0.19\textwidth} 
                \plotmooh{plots/neurips-thermo-linear-screw}{}{0.19\textwidth}
                \squash
                \caption{%
                    \textbf{Leftmost}: \Helix\ is defined on a 3D $\Mm$ that
                    extends into and out of the page.  A helical
                    level surface $S$ (orange-green) of $l$ winds around 
                    a 1D valley of minima orthogonal to the
                    page.  $l$ $\gg$ $1$ outside $S$.  Gradient noise
                    is parallel to the page and to the line between outer tubes.
                    %
                    Thanks to
                    \href{https://www.monroecc.edu/faculty/paulseeburger/calcnsf/CalcPlot3D/}{CalcPlot3D}.
                    %
                    \textbf{Center}: SGD moves
                    into the page.  In green: SGD's trajectory over 
                    cross sections of $\Mm$ that descend progressively
                    into the page.  In blue: $l$'s contours; $l$'s valley
                    intersects each pane's center.  Dotted
                    curves help compare adjacent panes.  Red bi-arrows: $C$'s
                    major axis.
                    %
                    Gradient noise kicks $\theta$ from A; $\theta$ then falls
                    (\hspace{-0.08cm}\protect\offour{1}) to B in \protect\offour{2}.  At C,
                    noise kicks $\theta$ uphill (\hspace{-0.08cm}\protect\offour{3}); $\theta$
                    thus never settles and the phenomena depicted here continue
                    forever.
                    {\bf Right}: Predictions near minima excel for
                    large $\eta T$.
                }
                \squash\squash
                \label{fig:archimedes}
            \end{figure}

            To test Corollary \ref{cor:entropic}'s $C$-dependence,
            \S\ref{appendix:artificial} constructs a landscape, \Helix, on
            whose valley of global minima $C$ varies (Figure
            \ref{fig:archimedes}).  As in Rock-Paper-Scissors, each point
            $\theta$ has a neighbor that is more attractive (flatter) with
            respect to $C(\theta)$.  This permits eternal motion into the page
            despite the landscape's discrete translation symmetry in that
            direction.  Corollary \ref{cor:entropic} predicts a velocity of
            $+\eta^2/6$ per timestep, while \cite{ch18}'s SDE-based analysis
            predicts a constant velocity of $0$.\footnote{
                Indeed, \Helix' velocity is $\eta$-perpendicular to the image
                of $(\eta C)^\mu_\nu$ in tangent space.
            }
            One may add a small linear term to \Helix\ to make SGD eternally
            ascend; one may wrap \Helix\ in a loop to make SGD circulate,
            witnessing a velocity field with curl (\S\ref{appendix:artificial});
            this is
            possible because $C\nabla H$, unlike $\nabla(CH)$, is not a total
            derivative. 
            In avoiding \cite{we19b}'s constant-$C$ assumption, we 
            find that SGD's velocity field can has curl. 
      
