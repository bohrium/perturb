
        %\subsection{SGD}

            %--  object of study  ---------------------------------------------
            %
            Gradient estimates, measured on minibatches and thus noisy, form
            the primary learning signal when training deep neural nets.  While
            users of deep learning benefit from the intuition that such
            \emph{stochastic gradient descent} (SGD) approximates deterministic
            gradient descent (GD) \citep{bo91,le15}, SGD's gradient noise in
            practice alters training dynamics and testing losses
            \citep{go18, %zh19,
            wu20}.  This paper studies SGD on
            short timescales or near minima and shows that \textbf{gradient
            noise biases learning} toward low-curvature, low-noise regions of
            the loss landscape.

            %--  vs ode and sde  ----------------------------------------------
            %
            %
            Generalizing \cite{li18,we19b,zh19,ba21}, we model correlated,
            non-gaussian, non-isotropic, non-constant gradient noise and
            find qualitative differences in dynamics.  For example, we
            construct a non-pathological
            %\footnote{%
            %    All higher derivatives exist and are quadratically bounded; the
            %    gradient noise at each weight vector is $1$-subgaussian.%
            %}
            loss landscape on which SGD's trajectory
            \emph{ascends}.
            %We argue that our theory enhances practical intuitions.
            We verify our theory on convolutional CIFAR-10 and Fashion-MNIST
            loss landscapes.

            %--  soft benefits: retrospective  --------------------------------
            %
            \begin{wrapfigure}{r}{0.40\textwidth}
                \centering  
                \vspace{-0.40cm}
                \plotmoow{diagrams/paradigm}{0.99\linewidth}{}\vspace{-0.10cm}
                \caption{
                    \textbf{A sub-process of SGD}.  Timesteps index
                    columns; training data index rows.  The $5$th datum
                    participates in the $2$nd SGD update.  This
                    {\color{spacetimepurple}$(n=5,t=2)$ event} affects the
                    testing loss both directly and via the
                    {\color{spacetimeteal}$(1,12)$ event}, which is itself
                    modulated by the {\color{spacetimeindigo}$(2,5)$ event}. 
                }\vspace{+0.60cm}
                \label{fig:paradigm}
            \end{wrapfigure}
            Our theory offers a new physics-inspired perspective of SGD as a
            superposition of concurrent information-flow processes.  Indeed, we
            study the post-training testing loss by Taylor expanding it w.r.t.\
            the learning rate $\eta$.  We interpret the resulting terms as
            describing processes by which data influence weights.  E.g.\ an
            instance of the process\footnote{
                Throughout, colors help us refer to parts of diagrams; colors
                lack mathematical meaning.
            }
                \vspace{-0.25cm}
            $$
                \mdia{MOOc(01-2-3-4)(04-13-23-34)}
                \vspace{-0.65cm}
            $$
            is shown on the right.  Notating processes with such diagrams, we
            show in \S\ref{sect:main} how to compute the effect of each process
            and that summing the finitely many processes with $d$ or fewer
            edges suffices to answer dynamical questions to error $o(\eta^d)$.  
            We thus factor the analysis of SGD into the analyses of individual
            processes, a technique that may power future theoretical
            inquiries.  
 
