
        The linear algebra of tensors plays a role in our analysis similar to
        the role of types in C.  One eschew types entirely, but to
        do so would hinder interpretation of the raw bitstrings we manipulate
        and would obscure which operations (e.g.\ dereferencing of a pointer)
        are natural and which are not (e.g.\ dereferencing of a float). 
        %
        The language of tensors supplies our analysis with
        abstractions and operations appropriate to derivatives and
        moments in high dimensional loss landscapes.

        For example, we recognize $C^{-1} + H$ as ill-formed, even
        though both $C^{-1}$ and $H$ are square grids of
        numbers of the same shape.  This is because $C^{-1}$ and $H$ are
        tensors of types $^2_0$ and $^0_2$, respectively.

        So here's a brief refresher for tensors.  
        %
        We recommend these sources for more details:
        \cite{si17} \S1 for motivational background;
        \cite{mi73} \S2.5 for helpful pictures;
        \cite{co14} \S2 for examples of how new tensors arise from old;
        \cite{ko93} \S7 and \S14 for precise formalism; and
        \cite{cu87} \S1.4 for statistics-relevant examples.\footnote{
            Some of these sources contrast the physicists' and statisticians'
            uses of tensors and of linear algebra overall.
            %
            Physicists often use vector spaces whose elements represent changes
            through physical space and time; statisticians often use vector
            spaces whose elements represent mixtures of experimental subjects.
            %
            Our paper doesn't commit to or rely on any domain interpretation
            of our loss landscape and its associated vector spaces, so to us
            these distinctions are irrelevant. 
        }\footnote{
            Some of these sources refer to
            \emph{symmetric tensors} and \emph{antisymmetric tensors} (a.k.a.\
            \emph{alternating forms}).  Our paper does not use these concepts,
            so we invite the reader to black-box adjectives such as
            `symmetric' when consulting those sources.  A symmetric tensor
            (etc) is just a special case of tensors as we define in this
            appendix.
        }
        \subsection{Vectors versus covectors}
            
            Imagine an air-conditioned hallway's temperature gradient
            ($0.1^\circ \text{K}/\text{meter}$) and length
            ($20\,\text{meters}$).  When we switch units from m to cm, the
            temperature gradient numerically \emph{decreases} (to $0.001$) but
            the length numerically \emph{increases} (to $2000$).  So
            temperature gradients and spatial displacements are instances of
            distinct geometric types.  A displacement inhabits a vector
            space $V$ of primary interest; a gradient inhabits $V$'s \emph{dual
            space} $V^\star$, defined as the set of linear maps from $V$ to
            $\RR$.  When $V$ is clear from context (e.g.\ throughout our paper
            $V$ consists of tangent vectors on $\Mm$), we call elements of $V$
            \emph{vectors} and elements of $V^\star$ \emph{covectors}.\footnote{
                This duality is most apparent when we
                recall that elements in $V$ correspond bijectively with 
                linear maps from $\RR$ to $V$.  Then a
                \emph{vector} (a.k.a.\ \emph{column} a.k.a.\
                \emph{displacement} a.k.a.\ \emph{primal} vector) is a linear
                map of type $\RR\to V$ and a \emph{covector} (a.k.a.\
                \emph{row} a.k.a.\ \emph{gradient} a.k.a.\ \emph{dual} vector)
                is a linear map of type $V\to \RR$.
            }\footnote{
                Here's why the word \emph{dual} is apt.
                We may map $V \to (V^\star)^\star$ by evaluation: $v
                \mapsto (f \mapsto f(v))$.  In finite dimensions, each 
                $\omega\in (V^\star)^\star$ arises this way.  So we
                identify $(V^\star)^\star = V$.  So a space's double dual is
                the space itself.
            }

            %Fix a (real) vector space $V$ of (finite) dimension $p$.

            Imagine a smooth real-valued function $f:V\to \RR$.  Then 
            we can specify the first derivative $\nabla f(x)$ (say at $x=0$) 
            by specifying $f$'s $p$ many partials with respect to a basis of
            $V$.
            %
            Imagine a (say, compactly supported) probability measure $\mu$ on
            $V$.  We can specify the first moment $\expc_\mu[v]$ of $\mu$ 
            by specifying $\expc_\mu[v]$'s $p$ many projections with respect
            to a basis of $V$.

            So both $\nabla f(0)$ and $\expc_\mu[v]$ are objects of dimension
            $p$.  But $\expc_\mu[v] \in V$ is a vector while $\nabla f(0) \in
            V^\star$ is a covector: the two differ in their geometric
            properties.
            %
            There is a canonical way to push $\mu$ forward along a
            linear map $\phi:V\to W$ and thereby to send $\expc_\mu[v]$ to
            $\expc_\mu[\phi(v)]$.  And there is a canonical way to pull $f$ back
            along a linear map $\psi:U\to V$ and thereby to send $\nabla f(0)$
            to $\nabla (f\circ \psi)(0)$.  But there is no natural non-zero way
            to pull $\mu, \expc_\mu[v]$ backward or to push $f, \nabla f(0)$
            forward (try it!).

        \subsection{What is a tensor?}
            We've seen that the first derivative $\nabla f$ of a smooth function
            $f:V\to \RR$ (evaluated, say, at $x\in V$) inhabits $V^\star$, not
            $V$.  How about higher derivatives?  To answer this question, we
            introduce \textbf{tensor}s: multi-axis grids of numbers viewed
            as linear algebraic objects.

            Take the third derivative $\nabla\nabla\nabla f$ evaluated at
            $x\in V$.  We can describe $\nabla\nabla\nabla f(x)$ completely by
            specifying the $p^3$ many partial derivatives with respect to a
            basis
            $\mathcal{E} = (e^i: 0\leq i<p)$ of $V$:
            $$ 
                \left[D_{i,j,k}^{\mathcal{E}} f\right](x)
                \triangleq 
                \left.
                \frac{\partial}{\partial t}
                \frac{\partial}{\partial t^{\prime}}
                \frac{\partial}{\partial t^{\prime\prime}}
                f(x + t                \cdot e^i
                    + t^{\prime}       \cdot e^j
                    + t^{\prime\prime} \cdot e^k)
                \right|_{t=t^\prime=t^{\prime\prime}=0}
                \hspace{1.0cm}
                \text{for $i,j,k\in \{i: 0\leq i<p\}^3$}
            $$
            The $p^3$ many components
            $
                \left[D_{i,j,k}^{\mathcal{E}} f\right](x)
            $
            enjoy key algebraic properties such as symmetry:
            $$
                \left[D_{i,j,k}^{\mathcal{E}} f\right](x)
                =
                \left[D_{j,i,k}^{\mathcal{E}} f\right](x)
                =
                \left[D_{i,k,j}^{\mathcal{E}} f\right](x)
            $$
            and linearity (which we see from the chain rule for derivatives):
            $$
                \left[D_{\hat i,\hat j,\hat k}^{\mathcal{F}} f\right](x)
                =
                \sum_{\hat i} 
                \sum_{\hat j}
                \sum_{\hat k}
                C_{\hat i}^i
                C_{\hat j}^j
                C_{\hat k}^k
                \left[D_{i,j,k}^{\mathcal{E}} f\right](x)
            $$
            Here, $\mathcal{F}$ is a basis expressible in terms of
            $\mathcal{E}$ via $f^i = \sum_{\hat i} C^i_{\hat i} e^{\hat i}$.

            %Such linearity rules find application, for instance, when we
            %pre-condition a linear system for better convergence. 
            %
            %More importantly,
            By showing how $\nabla\nabla\nabla f$'s components change
            coherently as we change basis, the linearity rule helps us
            interpret $\nabla\nabla\nabla f$ as a geometric object with
            magnitude and directional information rather than as a mere grid of
            numbers.

            We capture the essence of such linearity rules of by defining what
            it means for a vector space $P$ to be a tensor product of given
            vector spaces $V, W$; a `tensor' is then just an element of $P$.
            %
            The \textsc{defining quality} of $P$ is that \textbf{a linear map
            from $P$ to $X$ is `as good as' a bilinear map from $V, W$ to $X$,
            for any vector space $X$}.\footnote{
                Precisely, a tensor product of $V, W$ is a vector space $P$
                equipped with a natural family of vector-space isomorphisms $$
                \iota_X: \text{Bilin}(V,W;X) \xrightarrow{\sim} \text{Lin}(P;X)
                $$ from the vector space of bilinear maps to the vector space
                of linear maps.
                %
                The word `natural' just means that for any linear map $\phi:
                X\to Y$ and any $b \in \text{Bilin}(V,W;X)$: $$ \phi \circ
                \iota_X(b) = \iota_Y(\phi \circ b) $$ Naturality captures the
                intuition that $\iota$ just `re-packages' data without
                `altering' its content.
                %
                For instance, if $\kappa:X\to X$ is a non-trivial isomorphism,
                then defining $\iota^\prime_X(b) = \kappa \circ \iota_X(b)$ and
                $\iota^\prime_Z = \iota_Z$ for all $Z\neq X$ would turn a
                natural family $\iota$ of isomorphisms into a non-natural
                family $\iota^\prime$ of isomorphisms.
            }
            %
            We write $V\otimes W$ for the two-axis tensor $P$.  More
            generally, we write $U\otimes V\otimes \cdots \otimes W$ for 
            an $m$-axis tensor $P$, the maps from which correspond to
            the $m$-linear maps from $U,V,\cdots,W$.

            We may now interpret the third derivative $\nabla\nabla\nabla f(x)$
            as a linear-algebraic object:
            $$
                \nabla\nabla\nabla f(x) \in  (V\otimes V\otimes V)^\star
            $$
            The defining quality of $V\otimes V\otimes V = P$ says that such an
            element is specified the moment we specify a trilinear map from $V,
            V, V$ to $\RR$.  This trilinear map is simply 
            $$
                (u,v,w) \mapsto 
                \frac{\partial}{\partial t}
                \frac{\partial}{\partial t^{\prime}}
                \frac{\partial}{\partial t^{\prime\prime}}
                f(x + t                \cdot u 
                    + t^{\prime}       \cdot v
                    + t^{\prime\prime} \cdot w)
            $$
            So $\nabla\nabla\nabla f(x)$ is a tensor.  We now explain what we
            mean when we say that it is a tensor of type $^0_3$. 

        \subsection{Tensors of type $^u_d$}
            We've seen two ways to make new finite-dimensional
            vector spaces from old: we can dualize (to get $V^\star$ from $V$)
            and we can tensor (to get $U\otimes V\otimes \cdots \otimes W$ from
            $U,V,\cdots W$).

            No matter how we compose these operations, we'll get a space of the
            standard form 
            $$
                (U \otimes
                V \otimes \cdots
                W) \otimes
                (X^\star \otimes
                Y^\star \otimes \cdots
                Z^\star)
            $$
            In other words, we have a tensor product of $\mathbf{u}$ many non-dualized
            vector spaces and $\mathbf{d}$ many dualized vector spaces.  We won't provide
            a proof.\footnote{
                Here's an special case that contains all the ideas of a general
                proof.
                We'll define a natural isomorphism $\phi:U^\star\otimes
                V^\star\otimes W^\star \to (U\otimes V\otimes W)^\star$.  By
                the defining quality of the domain, to specify $\phi$ is to
                specify a trilinear map from $U^\star, V^\star, W^\star$ into
                the codomain.  By the defining quality of the codomain, to
                specify a map into the codomain is to specify a map into the
                space of trilinear real-valued maps on $U, W, V$.  
                %
                We accomplish all this by sending $f,g,h$ to $((u,v,w) \mapsto
                f(u)\cdot g(v)\cdot h(w))$.  We've thus defined a linear map
                $\phi$.  It is routine to check that $\phi$ sends no element
                except zero to zero, that $\phi$'s domain and codomain have
                equal and finite dimension, and thus that $\phi$ is an
                isomorphism.
            }

            An example is that $((U^\star \otimes V)^\star \otimes
            W^\star)^\star$, though not of the above form, is naturally
            isomorphic to $U^\star \otimes V \otimes W$, which \emph{is} of the
            above form.  That example illustrates a general algorithm: the
            factors in an un-standardized expression correspond with the
            factors in a standardized expression; whether a standardized factor
            is dualized or not depends simply on the \textsc{parity} of the
            number of dual operations that act on the corresponding factor in
            the original expression.  Here, $U,V,W$ appear with $3,2,2$ layers
            of duals, respectively, whence by parity we obtain the $1,0,0$
            layers of duals in the standardization.

            In the common case where all the underlying spaces $U,\cdots, Z$
            agree with a vector space $V$ understood from
            context, we say that elements of the combined space are
            \textbf{tensors of type $\mathbf{^u_d}$}.  We represent such
            tensors as grids with $u+d$ many axes, where each axis has a length
            equal to the dimension of $V$.
            %
            Thus, elements of $((V^\star \otimes V)^\star \otimes
            V^\star)^\star$ are tensors of type $^2_1$.  Numerically, they have
            $2+1=3$ axes: $2$ that transform like vectors and $1$ that
            transforms like a covector.  
            The derivative of $\theta_T$'s covariance matrix with respect to
            the initialization $\theta_0$ is an instance of such a type-$^2_1$
            tensor.

            As special cases, scalars are tensors of type $^0_0$,
            vectors are tensors of type $^1_0$, covectors are tensors of
            type $^0_1$, and linear maps from $V$ to $V$ are tensors of type
            $^1_1$.\footnote{
                Here's an example that our paper doesn't use but that may
                aid geometric intuition.
                %
                If $V$ has dimension $p$, then to specify a linear-algebraic
                notion of volume in $V$ is to specify a tensor of type $^0_p$
                obeying certain antisymmetry properties.
                Intuitively, such a tensor produces for any given sequence
                of $p$ input vectors a real number.  We interpret this real
                number as the volume of a parallelepiped spanned by those $p$
                vectors.
            }

            Whenever we have a tensor of type $^a_a$, we may interpret it as a
            linear map from the $a$-fold tensor product $V\otimes
            V\otimes\cdots\otimes V$ to itself.  In our work, this linear map  
            will sometimes be invertible, in which case we use standard notation
            $(\cdot)^{-1}$ to denote the inverse.
            %
            As a variation on this theme, a tensor (such as $C$) of type
            $^0_2$ we may view as a linear map from $V$ to $V^\star$.
            This linear map  
            is invertible whenever $C$ is positive definite, in which case
            we use the notation $C^{-1}$ to denote its inverse.  Observe that
            this inverse, being a linear map from $V^\star$ to $V$, is a tensor
            of type $^2_0$, unlike $C$.

        \subsection{Contraction}
            Let's define a map $\epsilon_V:V\otimes V^\star \to \RR$ by
            evaluation: $(v,f) \mapsto f(v)$.  We may similarly define
            $$
                {\text{id}_{(T\otimes U)}}\otimes \epsilon_V:(T\otimes
                U)\otimes V\otimes V^\star \to (T\otimes U)
            $$
            by $(x,v,f) \mapsto f(v)\cdot x$ for $x\in T\otimes U$.
            %
            In general, we can map any tensor
            product of the form
            $$
                (U \otimes
                V \otimes \cdots
                W) \otimes
                (U^\star \otimes
                Y^\star \otimes \cdots
                Z^\star)
            $$
            to one of the form
            $$
                (V \otimes \cdots
                W) \otimes
                (Y^\star \otimes \cdots
                Z^\star)
            $$
            by applying the evaluation map $\epsilon_U$.

            Say we have a tensor of type $^{u+1}_{d+1}$.  \textsc{If} we
            specify a specific axis among the ${u+1}$ vector-type axes as well
            as a specific axis among the ${d+1}$ covector-type axes
            \textsc{then} we may apply the above linear map to get a tensor of
            type $^u_d$.  We call this operation \textbf{contraction}.  
            Operationally, contraction just means summing across the two specified
            axes over the $\text{dim}(V)$ many `diagonal' index pairs $(i,i)$.

            For example, a linear map $\phi$ is the same as a tensor of type
            $^1_1$.  We may contract it to get a tensor of type $^0_0$, that is,
            a real number.  The number we get is the trace of $\phi$.

            As another example, consider the learning rate $\eta$ (a tensor
            of type $^2_0$) and the hessian $H=\nabla G$ (a tensor of type
            $^0_2$).  Then $\eta \otimes H$ is a tensor of type $^2_2$ and we
            may contract (in $2\times 2$ many ways, here all equivalent due to
            $\eta,H$'s symmetry) to obtain tensor of type $^1_1$: a linear map.

            What is this linear map, concretely?  It acts on a vector $v$ by
            sending it to the vector $w$ such that a gradient descent step
            $\theta \mapsto \theta - \eta \nabla f$ on the non-standard loss
            function $$ f(x) = G(x)\cdot v \in \RR $$ displaces $\theta$ by
            $-w$.
            %
            We see that the language of tensors concisely expresses otherwise
            cumbersome conceptual gymnastics.  In our experience, this language
            also guides us through otherwise treacherously brittle algebraic
            manipulations.

\end{document}

%%This section prepares for \S\ref{sect:using}, which uses
%%            diagrams to generalize \S\ref{sect:exegesis}'s separation, tame
%%            \S\ref{sect:exegesis}'s combinatorial explosion of terms, and
%%            temper \S\ref{sect:exegesis}'s polynomial divergence.  
%%            %
%%            We start by characterizing the higher order terms.  Suppose
%%            $s$ is an analytic function on $\Mm$.  For example, $s$ might be
%%            the testing loss $l$.  The following Lemma, reminiscent of
%%            \cite{dy49a}'s, tracks $s(\theta)$ as SGD updates $\theta$:
%%            \begin{klem*} \label{lem:dyson}
%%                For all $T$: for $\eta$ sufficiently small, $s(\theta_T)$ is a
%%                sum over tuples of natural numbers:
%%                \begin{equation}\label{eq:dyson}
%%                    \sum_{(d_t: 0\leq t<T) \in \NN^T}
%%                    (-\eta)^{\sum_t d_t}
%%                    \wrap{
%%                        \prod_{0 \leq t < T}
%%                            \wrap{\left.
%%                                \frac{(g \nabla)^{d_t}}{d_t!}
%%                            \right|_{g = \sum_{n\in \Bb_t} \nabla l_n(\theta) / B}}
%%                    }(s) (\theta_0)
%%                \end{equation}
%%                Moreover, an expectation symbol (over training sets) commutes
%%                with the outer sum.
%%            \end{klem*}
%%            Here, we consider each $(g \nabla)^{d_t}$ as a higher order
%%            function that takes in a function $f$ defined on weight space and
%%            outputs a function equal to the $d_t$th derivative of $f$, times
%%            $g^{d_t}$.  The above product then indicates composition of $(g
%%            \nabla)^{d_t}$'s across the different $t$'s.  In total, that
%%            product takes the function $s$ as input and outputs a function
%%            equal to some polynomial of $s$'s derivatives.
%%
%%            For example, the $\eta^3$ terms that appear in the above
%%            (for $s=l$) include:
%%            $$
%%                -\nabla_{\mu} l_{t=2} \nabla_{\nu} l_{t=2}
%%                 \nabla^{\mu}\nabla^{\nu}\nabla_{\lambda} l_{t=5}
%%                 \nabla^{\lambda} l
%%                %
%%                \hspace{2cm}
%%                %
%%                -\nabla_{\mu} l_{t=2} \nabla_{\lambda} l_{t=2}
%%                 \nabla^{\mu}\nabla_{\nu} l_{t=5}
%%                 \nabla^{\nu}\nabla^{\lambda} l
%%            $$
%%            Let us take expectations over training sets.  Suppose $B=1$ and
%%            $N>5$; then batches at $t=2,5$ are independent so the
%%            expectations factor (we use ``$\leftrightsquigarrow$'', not
%%            of ``$=$'', due to page's \pageref{dfn:uvalue-body}'s footnote):  
%%            \begin{align*}
%%               -\expc[\nabla_{\mu} l_{t=2} \nabla_{\nu} l_{t=2}]
%%                \expc[\nabla^{\mu}\nabla^{\nu}\nabla_{\lambda} l_{t=5}]
%%                \expc[\nabla^{\lambda} l]
%%                &&
%%               -\expc[\nabla_{\mu} l_{t=2} \nabla_{\nu} l_{t=2}]
%%                \expc[\nabla^{\mu}\nabla_{\lambda} l_{t=5}]
%%                \expc[\nabla^{\nu}\nabla^{\lambda} l] \\
%%               = -(GG+C)_{\mu\nu}J^{\mu\nu}_{\lambda}G^{\lambda} 
%%                && 
%%               = -(GG+C)_{\mu\nu}H^{\mu}_{\lambda} H^{\nu\lambda} \\
%%                \leftrightsquigarrow -\uvalue(\sdia{c(01-2-3)(02-12-23)})
%%                && 
%%                \leftrightsquigarrow -\uvalue(\sdia{c(01-2-3)(02-13-23)})
%%            \end{align*}
%%            To prepare for \S\ref{sect:using}, we write terms (such as
%%            above) as uvalues (Def.\ \ref{dfn:uvalue-body}) of diagrams.
%%            %(e.g.\ $\sdia{c(01-2-3)(02-12-23)}, \sdia{c(01-2-3)(02-13-23)}$).
%%            See Fig.\ \ref{fig:uvalue-example}.
%%            %We notated the tensors above using diagrams.  Diagrams help us
%%            %organize (\ref{eq:dyson})'s terms.
%%            %Each diagram evaluates to a
%%            %tensor: its \emph{un-resummed value} or
%%            %\translucent{moolime}{\emph{\textbf{uvalue}}, defined as a product containing a $d$th}
%%            %\translucent{moolime}{derivative of $l_x$ for each degree-$d$ node, grouped under
%%            %expectation symbols per the diagram's gray}
%%            %\translucent{moolime}{outlines, and tensor-contracted per the diagram's black edges}
%%            %(\S\ref{appendix:evaluate-histories} provides details).\footnote{
%%            %    We write $\leftrightsquigarrow$ instead of $=$ since diagrams
%%            %    evaluate to products of cumulants $C$ rather than of moments
%%            %    $GG+C$: \S\ref{appendix:evaluate-histories}.
%%            %}
%%            %%
%%            %We express (\ref{eq:dyson}) as a weighted sum of the $\uvalue$s of
%%            %all diagrams formally defined below.
%%            %We supplu The following examples of
%%            %(in)valid diagrams supply an informal sense sufficient to
%%            %read this paper's body.\footnote{
%%            %    Throughout, colors help us refer to parts of diagrams; colors
%%            %    lack mathematical meaning.
%%            %}
%%            \noindent
%%
%            So
%            \begin{align*}
%                \uvalue(\sdia{c(01-2-3)(02-12-23)}) &=
%                \textstyle
%                \sum_{\substack{\mu\nu\xi \\ \omicron\pi\rho}}
%                \expc[\nabla_{\mu} l_{t=2} \nabla_{\nu} l_{t=2}]
%                \expc[\nabla_{\omicron}\nabla_{\pi}\nabla_{\xi} l_{t=5}]
%                \expc[\nabla_{\rho} l]\eta^{\mu\omicron}\eta^{\nu\pi}\eta^{\xi\rho}
%                \\
%                \uvalue(\sdia{c(01-2-3)(02-13-23)}) &= 
%                \textstyle
%                \sum_{\substack{\mu\nu\xi \\ \omicron\pi\rho}}
%                \expc[\nabla_{\mu} l_{t=2} \nabla_{\nu} l_{t=2}]
%                \expc[\nabla_{\omicron}\nabla_{\xi} l_{t=5}]
%                \expc[\nabla_{\pi}\nabla_{\rho} l]\eta^{\mu\omicron}\eta^{\nu\pi}\eta^{\xi\rho} \\
%            \end{align*}
%            %\translucent{moolime}{\noindent\parbox{\textwidth}{
