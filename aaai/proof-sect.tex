    \subsection{Assumptions and Definitions}                        \label{appendix:assumptions}
        We assume throughout this work the following regularity properties of
        the loss landscape.
        
        \textbf{Existence of Taylor Moments} --- we assume
        that each finite collection of polynomials of the $0$th and higher
        derivatives of the $l_x$, all evaluated at any point $\theta$, may be
        considered together as a random variable insofar as they are equipped
        with a probability measure upon the standard Borel algebra.

        \textbf{Analyticity Uniform in Randomness} --- we assume that
        the functions $\theta \mapsto l_x(\theta)$ --- and the expectations
        of polynomials of their $0$th and higher derivatives --- exist and are
        analytic with radii of convergence bounded from $0$ (by a potentially
        $\theta$-dependent function).  So expectations and derivatives commute. 

        \textbf{Boundedness of Gradients} --- we also assume that the gradients
        $\nabla l_x(\theta)$, considered as random covectors, are bounded by
        some continuous function of $\theta$.\footnote{
            Some of our experiments involve Gaussian noise, which is not
            bounded and so violates the hypothesis.  In practice, Gaussians are
            effectively bounded
            %, on the one hand in that with high
            %probability no standard normal sample encountered on Gigahertz
            %hardware within the age of the universe will much exceed $\sqrt{2
            %\log(10^{30})} \approx 12$, and on the other hand
            in that
            our predictions vary smoothly with the first few moments of this
            distribution, so that a $\pm 12$-clipped Gaussian will yield almost
            the same predictions.
        }
        A metric-independent way of expressing this boundedness constraint
        is that the gradients all lie in some subset $\Ss \subseteq TM$ of
        the tangent bundle of weight space, where, for any compact $\Cc
        \subseteq M$, we have that the topological pullback --- of
        $\Ss \hookrightarrow TM \twoheadrightarrow M$
        and
        $\Cc \hookrightarrow M$ ---
        is compact.
        
        Now we turn to definitions.

        \begin{dfn}[Diagrams] \label{dfn:diagrams}
            A diagram is a finite rooted tree equipped with a partition of
            non-root nodes.  We adopt some drawing conventions: we draw the
            tree using thin ``edges''.  We draw each node to the right of its
            children; the root is thus always rightmost.  We draw the partition
            by connecting the nodes within each part via what we'll call
            ``fuzzy outlines'' or ``fuzzy ties''.  For example,
            $\sdia{c(012-3)(03-13-23)}$ has $2$ parts.
            %
            We insist on using as few fuzzy ties as possible so that, if $d$
            counts edges and $c$ counts ties, then $d+1-c$ counts parts of the
            partition.
            %
            There may be multiple ways to draw a single diagram, e.g.\
            $\sdia{MOOc(01-23-4)(03-13-23-34)}=\sdia{MOOc(02-13-4)(03-13-23-34)}$. 
        \end{dfn}
        \begin{dfn}[Histories]
            A history of a diagram %within a grid
            is an assignment of that diagram's non-root nodes to pairs $(n,t)$
            such that each node occurs at a time $t^\prime$ strictly after each
            of its children and such that two nodes occupy the same row $n$ if
            they inhabit the same part of $D$'s partition.
            %
            In situations of interest, the permissible pairs $(n,t)$ are those
            pairs such that the $n$th training point participates in the $t$th
            SGD update.  We prefer to draw all such pairs by shading the
            corresponding cells in a square grid whose rows are indexed by
            training points $n$ and whose columns are indexed by timesteps $t$.
            We thus speak of histories of a diagram with respect to a grid. 
            %
            We always regard the root note as assigned to the time $T$, i.e.,
            the successor time to the actual times $0,\cdots,T-1$ of the $T$
            many training updates.
        \end{dfn}
        We define $\uvalue(D)$ and $\rvalue_f(D)$ as was described precisely in the tutorial
        section ``How to evaluate histories''.

    \subsection{A key lemma \`a la Dyson}                           \label{appendix:key-lemma}

        %\subsection{Diagrams overcome Taylor Series challenges}
        \label{sect:using}\label{sect:diagrams}


    %\subsubsection{Proving Prop \ref{prop:nest}}
    %Let's study $\expct{\theta_T}, \expct{l(\theta_T)}$.  To warm
    %up,
    %we'll prove Prop \ref{prop:nest} (c.f.\ \cite{ne04,ro18}). 

    \subsubsection{Structure of the Taylor expansion} %---
    We discuss how to analyze SGD by expanding in powers of $\eta$.  We warm up
    by proving Prop \ref{prop:nest} (c.f.\ \cite{ne04,ro18}). 
        We'll name four quantities (${\translucent{moosky}{$W$}}$,
        ${\translucent{moolime}{$X$}}$, ${\translucent{moopink}{$Y$}}$,
        ${\translucent{moogold}{$Z$}}$) for later reference in this
        sub-section (their colors merely aid the eye).

    \begin{proof} %(of Prop \ref{prop:nest}).
        By our gradient bound assumption: $\theta_T - \theta_0$ is $O(\eta^1)$.
        We {claim} that $(\theta_T - \theta_0)^\mu =
        -\sum_t \sum_\nu \eta^{\mu\nu} \nabla_\nu l_{n_t}(\theta_0) + o(\eta^1)$.
        %
        The claim holds when $T=0$.  Say the claim holds for
        ${\tilde T}$-step SGD with
        $T = {\tilde T}+1$.  The displacement
        $\wrap{\theta_{T} - \theta_{{\tilde T}}}^\mu$
        is:%evaluates to:
        \begin{align*}
               & - \textstyle\sum_{\nu} \eta^{\mu\nu} \nabla_\nu l_{n_{\tilde T}}(\theta_{{\tilde T}})   
            \\=& - \textstyle\sum_{\nu} \eta^{\mu\nu} \nabla_\nu \wrap{
                       l_{n_{\tilde T}}(\theta_0)
                       + \text{\translucent{moosky}{$\sum_{\xi} \nabla_\xi l_{n_{\tilde T}}(\theta_0) (\theta_{{\tilde T}} - \theta_0)^\xi$}}^{\text{\translucent{moosky}{$_\swarrow W$}}}
                       + o(\theta_{{\tilde T}} - \theta_0)
                   }    
            \\=& - \textstyle\sum_{\nu} \eta^{\mu\nu} \nabla_\nu \wrap{
                          l_{n_{\tilde T}}(\theta_0)
                          + \nabla l_{n_{\tilde T}}(\theta_0) \cdot O(\eta^1) + o(O(\eta^1))
                      }    
            \\=& \textstyle\text{\translucent{moolime}{$- \sum_\nu \eta^{\mu\nu} \nabla_\nu l_{n_{\tilde T}}(\theta_0)$}}^{\text{\translucent{moolime}{$_\swarrow X$}}}
            + o(\eta^1)
        \end{align*}
        Applying the induction hypothesis proves the claim.
        %
        We plug the claim into $l$'s Taylor series:
        \begin{align*}
            \expc[l(\theta_T) - l(\theta_0)]
            &= \textstyle \sum_\mu \nabla_\mu l(\theta_0)
              \text{\translucent{moopink}{$\expc[\theta_T - \theta_0]^\mu$}}^{\text{\translucent{moopink}{$_\swarrow Y$}}}
              + \expc[o(\theta_T - \theta_0)] \\
            &= \textstyle \sum_\mu \nabla_\mu l(\theta_0) (%
              \text{\translucent{moogold}{$-T\eta G + o(\eta^1)$}}^{\text{\translucent{moogold}{$_\swarrow Z$}}})
              + o(O(\eta^1)) \\
            &= \textstyle - \sum_{\mu\nu} T G_\mu \eta^{\mu\nu} G_\nu + o(\eta^1)
        \end{align*}
        Indeed, due our assumption of analytic moments, the above
        expectations of $o(\eta^1)$ terms are still $o(\eta^1)$.
    \end{proof}

    The above proof gives an order-$1$ result.  %Correction terms involve
At higher order, higher derivatives correct
${\translucent{moosky}{$W$}}$
%${\translucent{moosky}{$\nabla l_{n_{\tilde T}}(\theta_0) \cdot (\theta_{{\tilde T}} - \theta_0)$}}$
and higher
moments augment
\translucent{moopink}{$Y$}.
%\translucent{moopink}{$\expc[\theta_T - \theta_0]$}.
%
%%Whereas \translucent{moogold}{$-TG\eta G$}
%%we multiplied
%%$l$'s derivatives by the expectations of such summands.
%
Whereas above the displacement is a sum over $\tilde T$s of
\translucent{moolime}{$X$}s, due to ${\translucent{moosky}{$W$}}$'s
corrections the displacement at higher order is a sum over \emph{tuples} of
times with summands such as $\nabla\nabla l_{n_{T^\prime}} \nabla l_{n_{\tilde
T}}$ instead of $\nabla l_{n_{\tilde T}}$.
%
When we then take expectations of \translucent{moolime}{$X$} to evaluate 
\translucent{moopink}{$Y$} as  
\translucent{moogold}{$Z$},
some summands (e.g.\ $\expc[\nabla\nabla l_{5}
\nabla l_{2}]=\expc[\nabla\nabla l_{5}]
\expc[\nabla l_{2}]$) are uncorrelated and thus factor; others (e.g.\ $\expc[\nabla\nabla
l_{5} \nabla l_{5}]$) do not.  This is how $\nabla l_x$'s higher cumulants such
as $C, S$ appear in our analysis.
%In short, a Taylor
%expansion even to low degrees yields a combinatorial explosion of
%terms.
%

        \subsubsection{Dyson}

        Suppose $s$ is an analytic function defined on the space of weights.
        The following Lemma, reminiscent of \cite{dy49a}, helps us track
        $s(\theta)$ as SGD updates $\theta$:
        \begin{klem*} \label{lem:dyson}
            For all $T$: for $\eta$ sufficiently small, $s(\theta_T)$ is a sum
            over tuples of natural numbers:
            \begin{equation}\label{eq:dyson2}
                s(\theta_T) = 
                \sum_{(d_t: 0\leq t<T) \in \NN^T}
                (-\eta)^{\sum_t d_t}
                \wrap{
                    \prod_{0 \leq t < T}
                        \wrap{\left.
                            \frac{(g \nabla)^{d_t}}{d_t!}
                        \right|_{g = \sum_{n\in \Bb_t} \nabla l_n(\theta) / B}}
                }(s) (\theta_0)
            \end{equation}
            Moreover, the expectation symbol (over training sets) commutes with
            the outer sum.
        \end{klem*}
        Here, we consider each $(g \nabla)^{d_t}$ as a higher order function
        that takes in a function $f$ defined on weight space and outputs a
        function equal to the $d_t$th derivative of $f$, times $g^{d_t}$.
        The above product then indicates composition of $(g \nabla)^{d_t}$'s
        across the different $t$'s.  In total, that product takes the function
        $s$ as input and outputs a function equal to some polynomial of $s$'s
        derivatives.

        \begin{proof}[Proof of the Key Lemma]%
            We work in a neighborhood of the initialization so that the tangent
            space of weight space is a trivial bundle.  For convenience, we fix
            a  coordinate system, and with it the induced flat,
            non-degenerate inverse metric $\tilde\eta$; the benefit is that we
            may compare our varying $\eta$ against one fixed $\tilde\eta$.
            Henceforth, a ``ball'' unless otherwise specified will mean a ball
            with respect to $\tilde\eta$ around the initialization $\theta_0$.
            Since $s$ is analytic, its Taylor series converges to $s$ within
            some positive radius $\rho$ ball.  By assumption, every $l_t$ is
            also analytic with radius of convergence around $\theta_0$ at least
            some $\rho>0$.  Since gradients are $x$-uniformly
            bounded by a continuous function of $\theta$, and since in finite
            dimensions the closed $\rho$-ball is compact, we have a strict
            gradient bound $b$ uniform in both $x$ and $\theta$ on gradient
            norms within that closed ball.  When
            \begin{equation} \label{eq:smalleta}
                2 \eta T b < \rho \tilde\eta
            \end{equation}
            as norms, SGD after $T$ steps on any train set
            will necessarily stay within the $\rho$-ball.\footnote{
                The $2$ ensures that SGD initialized at
                any point within a $\rho/2$ ball will necessarily stay within
                the $\rho$-ball.
            } We note that the above condition on $\eta$ is weak enough to
            permit all $\eta$ within some open neighborhood of $\eta=0$.  

            Condition \ref{eq:smalleta} together with analyticity of $s$ then
            implies that
            $
                \wrap{\exp(-\eta g \nabla) s}(\theta) = s(\theta - \eta g)
            $
            when $\theta$ lies in the $\tilde\eta$ ball (of radius $\rho$) and
            its $\eta$-distance from that $\tilde\eta$ ball's boundary exceeds
            $b$, and that both sides are analytic in $\eta, \theta$ on the same
            domain --- and \emph{a fortiori} when $\theta$ lies in the ball of
            radius $\rho (1 - 1/(2T))$.  Likewise, a routine induction through
            $T$ gives the value of $s$ (after doing $T$ gradient steps from an
            initialization $\theta$) as
            $$
                \wrap{
                    \prod_{0\leq t<T}
                        \left.
                            \exp(-\eta g \nabla)
                        \right|_{g=\nabla l_t(\theta)}
                }
                (s)(\theta)
            $$
            for any $\theta$ in the $\rho (1-T/(2T)$-ball (that is, the
            $\rho/2$-ball), and that both sides are analytic in $\eta, \theta$
            on that same domain.  Note that in each exponential, the
            $\nabla_\nu$ does not act on the $\nabla_\mu l(\theta)$ with which
            it pairs.  

            Now we use the standard expansion of $\exp$.  Because (by
            analyticity) the order $d$ coeffients of $l_t, s$ are bounded by
            some exponential decay in $d$ that has by assumption an $x$-uniform
            rate, we have absolute convergence and may rearrange sums.  We
            choose to group by total degree:
            \begin{equation} \label{eq:expansion}
                \cdots 
                =
                \sum_{0\leq d < \infty} (-\eta)^d
                \sum_{\substack{(d_t: 0\leq t<T) \\ \sum_t d_t = d}}
                \wrap{
                    \prod_{0 \leq t < T} \left.
                        \frac{(g \nabla)^{d_t}}{d_t!}
                    \right|_{g=\nabla l_t(\theta)}
                } s (\theta)
            \end{equation}
            The first part of the Key Lemma is proved.  It remains to show that
            expectations over train sets commute with the above summation.

            We will apply Fubini's Theorem.  To do so, it suffices to show that   
            $$
                \wabs{c_d((l_t: 0\leq t<T))} 
                \triangleq
                \wabs{
                    \sum_{\substack{(d_t: 0\leq t<T) \\ \sum_t d_t = d}}
                    \wrap{
                        \prod_{0 \leq t < T} \left.
                            \frac{(g \nabla)^{d_t}}{d_t!}
                        \right|_{g=\nabla l_t(\theta)}
                    } s (\theta)
                }
            $$
            has an expectation that decays exponentially with $d$.  The symbol
            $c_d$ we introduce purely for convenience; that its value depends
            on the train set we emphasize using function application
            notation.  Crucially, no matter the train set, we have shown
            that the expansion \ref{eq:expansion} (that features $c_d$ appear
            as coefficients) converges to an analytic function for all $\eta$
            bounded as in condition \ref{eq:smalleta}.  The uniformity of this
            demanded bound on $\eta$ implies by the standard relation between
            radii of convergence and decay of coefficients that $\wabs{c_d}$
            decays exponentially in $d$ at a rate uniform over train sets.
            If the expectation of $\wabs{c_d}$ exists at all, then, it will
            likewise decay at that same shared rate.
            
            Finally, $\wabs{c_d}$ indeed has a well-defined expected value, for
            $\wabs{c_d}$ is a bounded continuous function of a
            (finite-dimensional) space of $T$-tuples (each of whose entries can
            specify the first $d$ derivatives of an $l_t$) and because the
            latter space enjoys a joint distribution.  So Fubini's Theorem
            applies.  The Key Lemma follows.   
        \end{proof}

    \subsection{From Dyson to diagrams}                             \label{appendix:toward-diagrams}

        %{\color{red} TODO: define diagrams! FILL IN }

        We now describe the terms that appear in the Key Lemma.  The following
        result looks like Theorem \ref{thm:resum}, except it has $\uvalue(D)$
        instead of $\uvalue_f(D)$, and the sum is over all diagrams, not just
        linkless ones.  In fact, we will use Theorem \ref{thm:pathint} to
        prove Theorem \ref{thm:resum}.

        \begin{thm}[Test Loss as a Path Integral] \label{thm:pathint}
            For all $T$: for $\eta$ sufficiently small, SGD's expected test
            loss is
            \begin{equation*}\label{eq:sgdcoef}
                \sum_{D}
                %\wrap{
                    \sum_{\text{histories}~f}
                    \frac{1}{\wabs{\Aut_f(D)}}
                %}
                \frac{\uvalue(D)}{(-B)^{|\edges(D)|}}
            \end{equation*}
            Here, $D$ is a diagram whose root $r$ does not participate in
            any fuzzy edge, $f$ is a history of $D$ into a grid, and
            $\wabs{\Aut_f(D)}$ counts the graph-automorphisms of $D$ that
            preserve $f$'s assignment of nodes to cells.
            %
            If we replace $D$ by 
            $
                \wrap{-\sum_{p \in \parts(D)} (D_{rp} - D)/N}
            $, where $r$ is $D$'s root,
            we obtain the expected generalization gap (testing minus training loss).
        \end{thm}

        Theorem \ref{thm:pathint} describe the terms that appear in the Key
        Lemma by matching each term to a history of a diagram in a grid,
        so that the infinite sum becomes a sum over all diagram grid 
        configurations.  The main idea is that the combinatorics of diagrams
        parallels the combinatorics of repeated applications of the product
        rule for derivatives applied to the expression in the Key Lemma.
        Balancing against this combinatorial explosion are factorial-style
        denominators, again from the Key Lemma, that we summarize in terms of
        the sizes of automorphism groups.

        \begin{proof}[Proof of Theorem \ref{thm:pathint}]
            We first prove the statement about testing losses.
            Due to the analyticity property established in our proof of the
            Key Lemma, it suffices to show agreement at each degree $d$ and
            train set individually.  That is, it suffices to show --- for
            each train set $(l_n: 0\leq n<N)$, grid $S$, function $\pi:
            S\to [N]$ that induces $\sim$, and natural $d$ --- that
            \begin{align} \label{eq:toprove}
                (-\eta)^d
                \sum_{\substack{
                    (d_t: 0\leq t<T) \\
                    \sum_t d_t = d
                }}
                \wrap{
                    \prod_{0 \leq t < T} \left.
                        \frac{(g \nabla)^{d_t}}{d_t!}
                    \right|_{g=\nabla l_t(\theta)}
                } l (\theta)
                = \nonumber \\
                \sum_{\substack{
                    D \in \image(\Free) \\
                    \textnormal{with $d$ edges}
                }}
                \wrap{
                    \sum_{f: D\to\Free(S)}
                    \frac{1}{\wabs{\Aut_f(D)}}
                }
                \frac{\uvalue_\pi(D, f)}{B^{d}}
            \end{align}
            Here, $\uvalue_\pi$ is the value of a diagram history before
            taking expectations over train sets.  We have for all $f$ that
            $\expct{\uvalue_\pi(D, f)} = \uvalue(D)$.
            Observe that both sides of \ref{eq:toprove} are finitary sums.

            \begin{rmk}[Differentiating Products] \label{rmk:leibniz}
                The product rule of Leibniz easily generalizes to higher
                derivatives of finitary products:
                $$
                    \nabla^{\wabs{M}} \prod_{k \in K} p_k
                    = 
                    \sum_{\nu:M\to K} \prod_{k\in K} \wrap{
                        \nabla^{\wabs{\nu^{-1}(k)}} p_k
                    }
                $$
                The above has $\wabs{K}^{\wabs{M}}$ many term indexed by
                functions to $K$ from $M$.
            \end{rmk}

            We proceed by joint induction on $d$ and $S$.  The base cases
            wherein $S$ is empty or $d=0$ both follow immediately from the Key
            Lemma, for then the only history is the unique history of the
            one-node diagram $\sdia{(0)()}$.  For the induction step, suppose
            $S$ is a sequence of $\Mm = \min S \subseteq S$ followed by a
            strictly smaller $S$ and that the result is proven for $(\tilde d,
            \tilde S)$ for every $\tilde d \leq d$.  Let us group by $d_0$ the
            terms on the left hand side of desideratum \ref{eq:toprove}.
            Applying the induction hypothesis with $\tilde d = d - d_0$, we
            find that that left hand side is:
            \begin{align*}
                \sum_{\substack{
                    0 \leq d_0 \leq d
                }}
                \sum_{\substack{
                    \tilde D \in \image(\Free) \\
                    \textnormal{with $d-d_0$ edges}
                }}
                \frac{1}{d_0!}
                \sum_{\tilde f: \tilde D\to\Free(\tilde S)} \wrap{
                    \frac{1}{\wabs{\Aut_{\tilde f}(\tilde D)}}
                }
                ~\cdot~
                \\ %---------------------------------------------
                (-\eta)^{d_0}
                \left.
                    (g \nabla)^{d_0}
                \right|_{g=\nabla l_0(\theta)}
                \frac{\uvalue_\pi(\tilde D, \tilde f)}{B^{d-d_0}}
            \end{align*}
            Since $\uvalue_\pi(\tilde D, \tilde f)$ is a multilinear product of
            $d-d_0+1$ many tensors, the product rule for derivatives tells us
            that $(g \nabla)^{d_0}$ acts on $\uvalue_\pi(\tilde D, \tilde f)$
            to produce $(d-d_0+1)^{d_0}$ terms.  In fact,
            $
                g = \sum_{m\in \Mm} \nabla l_m(\theta) / B
            $ 
            expands to
            $B^{d_0}(d-d_0+1)^{d_0}$ terms, each conveniently indexed
            by a pair of functions $\beta:[d_0]\to \Mm$ and $\nu:[d_0]\to
            \tilde D$.  The $(\beta, \nu)$-term corresponds to a history
            $f$ of a larger diagram $D$ in the sense that it contributes
            $\uvalue_\pi(D, f)/B^{d_0}$ to the sum.  Here, $(f, D)$ is $(\tilde
            f, \tilde D)$ with $\wabs{\wrap{\beta \times \nu}^{-1}(n, v)}$ many
            additional edges from the cell of datapoint $n$ at time $0$ to the
            $v$th node of $\tilde D$ as embedded by $\tilde f$.

            By the Leibniz rule of Remark \label{rmk:leibniz}, this $(\beta,
            \nu)$-indexed sum by corresponds to a sum over histories $f$ that
            restrict to $\tilde f$, whose terms are multiples of the value of
            the corresponding history of $D$.  Together with the sum over
            $\tilde f$, this gives a sum over all histories $f$.  So we now
            only need to check that the coefficients for each $f:D\to S$ are as
            claimed.

            We note that the $(\beta, \nu)$ diagram (and its value) agrees with
            the $(\beta \circ \sigma, \nu \circ \sigma)$ diagram (and its
            value) for any permutation $\sigma$ of $[d_0]$.  The corresponding
            orbit has size
            \begin{align*}
                \frac{d_0!}{
                    \prod_{(m, i) \in \Mm \times \tilde D}
                        \wabs{(\beta \times \nu)^{-1}(m, i)}!
                }
            \end{align*}
            by the Orbit Stabilizer Theorem of elementary group theory.   

            It is thus enough to show that
            \begin{align*} \label{eqn:countclaim}
                \wabs{\Aut_f(D)} = 
                \wabs{\Aut_{\tilde f}(D)}
                \prod_{(m, i) \in \Mm \times \tilde D}
                    \wabs{(\beta \times \nu)^{-1}(m, i)}!
            \end{align*}
            We will show this by a direct bijection.  First, observe that
            $
                f = \beta \sqcup \tilde f:
                    [d_0] \sqcup \tilde D \to \Mm \sqcup \tilde S
            $. 
            So each automorphism $\phi: D\to D$ that commutes with $f$ induces
            both an automorphism
            $
                \Aa = \phi|_{\tilde D}: \tilde D\to \tilde D
            $
            that commutes with $\tilde f$ together with the data of a map
            $
                \Bb = \phi_{[d_0]}: [d_0] \to [d_0] 
            $
            that both commutes with $\beta$.  However, not every such pair of
            maps arises from a $\phi$.  For, in order for $\Aa \sqcup \Bb: D
            \to D$ to be an automorphism, it must respect the order structure
            of $D$.  In particular, if $x\leq_D y$ with $x \in [d_0]$ and $y
            \in \tilde D$, then we need
            $$
                \Bb(x) \leq_D \Aa(y)
            $$
            as well.  The
            pairs $(\Aa, \Bb)$ that thusly preserve order are in bijection with
            the $\phi \in \Aut_f(D)$.  There are $\wabs{\Aut_{\tilde f}(\tilde
            D)}$ many $\Aa$.  For each $\Aa$, there are as many $\Bb$ as there
            are sequences $(\sigma_i: i \in \tilde D)$ of permutations on
            $
                \{j\in [d_0]: j\leq_D i\} \subseteq [d_0]
            $ 
            that commute with $\Bb$.  These permutations may be chosen
            independently; there are 
            $
                \prod_{m\in \Mm}
                    \wabs{(\beta \times \nu)^{-1}(m, i)}!
            $
            many choices for $\sigma_i$.  Claim \ref{eqn:countclaim} follows,
            and with it the correctness of coefficients.
 
            The argument for generalization gaps parallels the above when we
            use $l-\sum_n l_n/N$ instead of $l$ as the value for $s$. 
            Theorem \ref{thm:pathint} is proved.
        \end{proof}

        \begin{rmk}[The Case of $E=B=1$ SGD]
            The grid of $E=B=1$ SGD permits all and only those
            histories that assign to each part of a diagram's partition  a
            distinct cell.  Such histories factor through a diagram
            ordering and are thus easily counted using factorials per
            Prop \ref{prop:vanilla}.  That prop immediately
            follows from the now-proven Theorem \ref{thm:pathint}.
        \end{rmk}

        \begin{prop} \label{prop:vanilla}
            The order $\eta^d$ contribution to the expected testing loss of
            one-epoch SGD with singleton batches is:
            \begin{equation*}\label{eq:sgdbasiccoef}
                \frac{(-1)^d}{d!} \sum_{D} 
                |\ords(D)| {N \choose P-1} {d \choose d_0,\cdots,d_{P-1}}
                \uvalue(D)
            \end{equation*}
            where $D$ ranges over $d$-edged diagrams.  Here, $D$'s parts have
            sizes $d_p: 0\leq p\leq P$, and $|\ords(D)|$ counts the total
            orderings of $D$ s.t.\ children precede parents and parts are
            contiguous.
        \end{prop}

    %\subsection{Interlude: a review of M\"obius inversion}          \label{appendix:mobius}

    
    \subsection{Proof of Theorem \ref{thm:resum}}    \label{appendix:resum}

        (We say a history is \textbf{strict} if it assigns to each part
        a different datapoint $n$.
        Then, by M\"obius inversion (\cite{ro64}), a sum over strict histories
        of moment values (page \pageref{appendix:evaluate-histories}) matches 
        a sum over all histories of $\uvalue$s.)

        The diagrams summed in Theorem \ref{thm:resum} and \ref{thm:converge}
        may be grouped by their geometric realizations.  Each nonempty class of
        diagrams with a given geometric realization has a unique element with
        minimally many edges, and in this way all and only linkless diagrams
        arise. 

        We encounter two complications: on one hand, that the sizes of
        automorphism groups might not be uniform among the class of diagrams
        with a given geometric realization.  On the other hand, that the
        histories of a specific member of that class might be hard to count.
        The first we handle using Orbit-Stabilizer.  The second we address
        via M\"obius sums.
           
        \begin{proof}[Proof of Theorem \ref{thm:resum}]
            We apply M\"obius inversion %(page \pageref{appendix:mobius})
            to Theorem
            \ref{thm:pathint} (\pageref{appendix:toward-diagrams}).
            %The result
            %is that chains of histories  
            %{\color{red} FILL IN}

            The difference in loss from the noiseless case is given by all the
            diagram histories with at least one fuzzy tie, where the fuzzy tie
            pattern is actually replaced by a difference between noisy and
            noiseless cases as prescribed by the preceding discussion on
            M\"obius Sums.  Beware that even relatively noiseless histories
            may have illegal collisions of non-fuzzily-tied nodes within a
            single grid (data) row.  Throughout the rest of this proof, we
            permit such illegal histories of the fuzz-less diagrams that arise
            from the aforementioned decomposition.  

            Because the Taylor series for analytic functions converge
            absolutely in the interior of the disk of convergence, the
            rearrangement of terms corresponding to a grouping by geometric
            realizations preserves the convergence result of Theorem
            \ref{thm:pathint}. page 
page
            Let us then focus on those diagrams $\sigma$ with a given geometric
            realization represented by an linkless diagram $\rho$.  By
            Theorem \ref{thm:pathint}, it suffices to show that
            \begin{equation} \label{eq:hard}
                \sum_{f:\rho\to S}
                \sum_{\substack{
                    \tilde f:\sigma\to S \\
                    \exists i_\star: f=\tilde f \circ i_\star
                }}
                \frac{1}{\wabs{\Aut_{\tilde f}(\sigma)}}
                =
                \sum_{f:\rho\to S}
                \sum_{\substack{
                    \tilde f:\sigma\to S \\
                    \exists i_\star: f=\tilde f \circ i_\star
                }}
                \sum_{\substack{
                    i:\rho\to\sigma \\
                    f = \tilde f \circ i
                }}
                \frac{1}{\wabs{\Aut_{f}(\rho)}}
            \end{equation}
            Here, $f$ is considered up to an equivalence defined by
            precomposition with an automorphism of $\rho$.  We likewise
            consider $\tilde f$ up to automorphisms of $\sigma$.  And above,
            $i$ ranges through maps that induce isomorphisms of geometric
            realizations, where $i$ is considered equivalent to $\hat i$ when
            for some automorphism $\phi \in \Aut_{\tilde f}(\sigma)$, we have
            $\hat i = i \circ \phi$.  Name as $X$ the set of all such $i$s
            under this equivalence relation.

            In equation \ref{eq:hard}, we have introduced
            redundant sums to structurally align the two expressions on the
            page; besides this rewriting, we see that equation \ref{eq:hard}'s
            left hand side matches Theorem \ref{thm:pathint} resulting formula
            and tgat its right hand side is the desired formula of Theorem
            \ref{thm:resum}. 

            To prove equation \ref{eq:hard}, it suffices to show (for any
            $f, \tilde f, i$ as above) that
            $$
                \wabs{\Aut_f(\rho)}
                =
                \wabs{\Aut_{\tilde f}(\sigma)}
                \cdot
                \wabs{X}
            $$
            We will prove this using the Orbit Stabilizer Theorem by presenting
            an action of $\Aut_f(\rho)$ on $X$.  We simply use precomposition
            so that $\psi\in \Aut_f(\rho)$ sends $i\in X$ to $i\circ \psi$.
            Since $f\circ\psi = f$, $i\circ \psi \in X$.  Moreover, the action
            is well-defined, because if $i\sim \hat i$ by $\phi$, then $i \circ
            \psi \sim \hat i \circ \psi$ also by $\phi$.
            
            The stabilizer of $i$ has size $\wabs{\Aut_{\tilde f}(\rho)}$.
            For, when $i \sim i \circ \psi$ via $\phi \in \Aut_{\tilde
            f}(\rho)$, we have $i\circ \psi = \phi \circ i$.  This relation in
            fact induces a bijective correspondence: \emph{every} $\phi$
            induces a $\psi$ via $\psi = i^{-1} \circ \phi \circ i$, so we have
            a map $\text{stabilizer}(i) \hookleftarrow \Aut_{\tilde f}(\rho)$
            seen to be well-defined and injective because structure set
            morphisms are by definition strictly increasing and because $i$s
            must induce isomorphisms of geometric realizations.  Conversely,
            every $\psi$ that stabilizes enjoys \emph{only} one $\phi$ via
            which $i \sim i \circ \phi$, again by the same (isomorphism and
            strict increase) properties.  So the stabilizer has the claimed
            size.

            Meanwhile, the orbit is all of $\wabs{X}$.  Indeed, suppose $i_A,
            i_B \in X$.  We will present $\psi \in \Aut_f(\rho)$ such that $i_B
            \sim i_A \circ \psi$ by $\phi=\text{identity}$.  We simply define
            $\psi = i_A^{-1} \circ i_B$, well-defined by the aforementioned
            (isomorphisms and strict increase) properties.  It is then routine
            to verify that
            $
                f \circ \psi
                =
                \tilde f \circ i_A \circ i_A^{-1} \circ i_B
                =
                \tilde f \circ i_B
                = f.
            $
            So the orbit has the claimed size, and by the Orbit Stabilizer
            Theorem, the coefficients in the expansions of Theorems 
            \ref{thm:resum} and \ref{thm:pathint} match.
        \end{proof}

    \subsection{Proof of Theorem \ref{thm:converge}}    \label{appendix:converge}

        %Let $L_{d,T}(\eta)$ be the $d$th order truncation of Thm \ref{thm:resum}'s series.
        %The following lemma compares $L_d = \lim_{T\to\infty} L_{d,T}$ to
        %$L_T = \lim_{d\to\infty} L_{d,T}$.
        %%
        %For convenience, we fix some inner product on $\Mm$. 
        %%
        %\begin{lem}\label{lem:strength}
        %    Fix $U\subseteq \Mm$ open and $\theta_\star\in U$ a local minimum
        %    of $l$.  Assume Local Strength.  Then for all $d$ and for all
        %    initializations $\theta_0$ in some neighborhood $V_d$ of
        %    $\theta_\star$: there exist $T_0,A,B>0$ so that $\sup_{T\geq T_0}
        %    {\rm ReLU}(|L_d(\eta)-L_T(\eta)|-\exp(A-BT))$ is well-defined and
        %    is $o(\eta^d)$.
        %    `Well-defined' means the sup is finite on some
        %    neighborhood (in the space of symmetric positive semidefinite
        %    forms) of $\eta=0$.
        %\end{lem}
        %\begin{proof}
        %    In what follows, $\epsilon,\delta,s,\epsilon^\pr$ denote positive
        %    quantities to be regarded as small and under our control.  We
        %    follow proof-writing tradition by \emph{using} each quantity
        %    before \emph{setting} it.

        %    Let $\hat l = \sum_n l_n/N$ denote the training landscape.  Since
        %    $\nabla l_x, \nabla\nabla l_x$ are universally bounded, Chernoff
        %    implies that for any compact $K$ and any $\epsilon,\delta>0$:
        %    there are $T_0, \eta_\star>0$ so that whenever 
        %    $\eta < \eta_\star$ and $T>T_0$, the error $\sup_{\theta\in K}
        %    \|\nabla \hat l-\nabla l\|$ is small ($<\epsilon$) with high
        %    probability ($>1-\delta$).

        %    Since $l_x$ is universally bounded, the low-probability fringe
        %    event contributes $O(\delta)$ to the final testing loss $L_T$.  In
        %    fact, by Chernoff this contribution decays exponentially in $T$.
        %    (Thence arises the lemma's term $\exp(A-BT)$). 

        %    Let's thus focus on the high-probability bulk event.  For any
        %    $s>0$, we may arrange $K, \epsilon$ with reference to Local
        %    Strength's $Q_-$ so that, conditioned on this bulk event, SGD's
        %    trajectory is confined to an open radius-$s$ ball
        %    $B_s(\theta_\star)$ around $\theta_\star$ when it is
        %    initialized at $\theta_\star$.  This confinement being an open
        %    condition and dynamics being continuous, we may in fact arrange $K,
        %    \epsilon$ so that (again conditioned on the bulk event) SGD's
        %    trajectory is confined to $B_s(\theta_\star)$ when 
        %    $\theta_0 \in B_{r(s)}(\theta_\star)$.

        %    Invoking $L_{T_0}$'s analyticity, we re-choose $\eta_\star>0$ as
        %    needed so that for all lesser $\eta$,
        %    $|L_{d,T_0}(\eta)-L_{T_0}(\eta)|<\epsilon^\pr$.  
        %\end{proof}

        \begin{proof}[Proof of Theorem \ref{thm:converge}]
            Since we assumed hessians are positive: for any $m$, the propagator
            $K^t = \wrap{(I-\eta H)^{\otimes m}}^t$ exponentially decays to $0$
            (at a rate dependent on $m$).  Since up to degree $d$ only a finite
            number of diagrams exist and hence only a finite number of possible
            $m$s, the exponential rates are bounded away from $0$.  Moreover,
            for any fixed $t_{\text{big}}$, the number of diagrams ---
            involving no exponent $t$ exceeding $t_{\text{big}}$ --- is
            eventually constant as $T$ grows.  Meanwhile, the number involving
            at least one exponent $t$ exceeding that threshold grows
            polynomially in $T$ (with degree $d$).  The exponential decay of
            each term overwhelms the polynomial growth in the number of terms,
            and Theorem's first part follows.
            %More difficult is the Theorem's statement about Local Strength.

            The Theorem's second part pertains to the setting of quadratic loss
            landscapes, i.e.\ landscapes for which the hessian is positive
            definite and constant over all weights and all data points.  Note
            that in this case the only non-vanishing diagrams have tree
            structures that are chains, that is, whose every node has degree
            one or two (where the root might have degree one or two as well).
            There are only exponentially many histories of such diagrams,
            weighted with exponential decays as in the previous paragraph. 
            Summing for any $T$ over all higher degree histories, 
            one thus obtains tail bounds uniform in $T$, as claimed.
        \end{proof}

    %\subsection{How to modify proofs to handle variants}            \label{appendix:prove-variants}

    \subsection{Proofs of corollaries}                              \label{appendix:corollaries}

        \subsubsection{Corollary \ref{cor:entropic}}

            \begin{proof}
                The relevant linkless diagram is $\sdia{c(01-2-3)(02-12-23)}$
                %{\color{red} (amputated as in the previous subsubsection)}.   
                A history of this diagram into $E=B=1$ SGD's grid 
                is determined by two durations --- 
                $t$ from {\color{moor}red} to {\color{moog}green} and
                $\tilde t$ from {\color{moog}green} to {\color{moob}blue} ---
                obeying $t+\tilde t \leq T$.
                The automorphism group of each history has size $2$: identity
                or switch the {\color{moor}red} nodes.  So the answer is: 
                $$
                    C_{\mu \nu}
                    J^{\rho\lambda}_{\sigma}
                    \wrap{\int_{t+\tilde t\leq T}
                        \wrap{\exp(-t \eta H) \eta}^{\mu\rho}
                        \wrap{\exp(-t \eta H) \eta}^{\nu\lambda}
                        \wrap{\exp(-\tilde t \eta H) \eta}^{\sigma\pi}
                    }
                $$
                Standard calculus then gives the desired result.
            \end{proof}

        \subsubsection{Corollaries \ref{cor:epochs} and \ref{cor:batch}}

            Corollary \ref{cor:epochs} and Corollary \ref{cor:batch} follow
            from plugging appropriate values of $M, N, B$ into the following
            prop.

            \begin{prop}\label{prop:ordtwo}
                To order $\eta^2$, the testing loss of SGD --- on $N$ samples
                for $T=MN$ timesteps with batch size $B$ dividing $N$ and with
                any shuffling scheme --- has expectation
                {\small
                \begin{align*}
                                                            l              
                    &- MN                                   G_\mu G^\mu       
                     + MN\wrap{MN - \frac{1}{2}}            G_\mu H^{\mu}_{\nu} G^\nu \\
                    &+ MN\wrap{\frac{M}{2}}                 C_{\mu \nu} H^{\mu \nu}
                     + MN\wrap{\frac{M-\frac{1}{B}}{2}}     \wrap{\nabla_\mu C^{\nu}_{\nu}} G^\mu / 2
                \end{align*}
                }
            \end{prop}

            \begin{proof}[of Prop \ref{prop:ordtwo}]
                To prove Prop \ref{prop:ordtwo}, we simply count
                the histories of the diagrams, noting that the automorphism groups
                are all of size $1$ or $2$. 
                %Since we use fuzzy outlines instead of
                %fuzzy ties, we allow untied nodes to occupy the same row, since the
                %excess will be canceled out by the term subtract in the definition of
                %fuzzy outlines.
                See Table \ref{tbl:ordtwo}.
                \begin{table}[h]
                    \centering
                    \begin{tabular}{cll}
                        diagram                 & \#histories w/ $\wabs{\Aut_f}=1$  & \#histories w/ $\wabs{\Aut_f}=2$   \\ \hline
                        $\sdia{(0)()}$          & $1$                           & $0$                            \\  
                        $\sdia{(0-1)(01)}$      & $MNB$                         & $0$                            \\                  
                        $\sdia{(0-1-2)(01-12)}$ & ${MN\choose 2}B^2$            & $0$                            \\
                        $\sdia{c(01-2)(01-12)}$ & $N{MB\choose 2}$              & $0$                            \\
                        $\sdia{(0-1-2)(02-12)}$ & ${MNB\choose 2}$              & $MNB$                          \\
                        $\sdia{c(01-2)(02-12)}$ & $N{MB\choose 2}$              & $MNB$                             
                    \end{tabular}
                    \caption{Terms used in proof of Prop \ref{prop:ordtwo}}
                    \label{tbl:ordtwo}
                \end{table}
            \end{proof}

        \subsubsection{Corollary \ref{cor:barret}}

%\moosect{Locating \cit{Ba} in our theory}
    \emph{Here we adopt a more pedagogical tone and develop a several related results}
    relating to the displacement of GD, of ODE, and how those two compare.
    In particular, we prove the Corollary, which is \cite{ba21}'s Theorem 3.1. 

    \cite{ba21}'s Theorem 3.1 computes order-$\eta^2$ weight
    displacements $\theta_T-\theta_0$ in the noiseless case $l_x=l$.  The
    relevant diagrams are thus those with $\leq 2$ edges and that contain no
    fuzzy outlines.
    %
    Indeed, noiseless $\implies$ cumulants vanish $\implies$ any diagram that
    contains one or more fuzzy outlines has a uvalue (and rvalue) equal to
    zero.  So a sum over diagrams is the same as a sum over fuzzless diagrams,
    i.e., over each diagram whose partition is maximally fine. %(\pag{5}\dfn{1}) is maximally fine.

    Per page \pageref{appendix:solve-variants}, we use `rootless' diagrams,
    e.g.\ $\mdia{MOOc(0-1)(01-1)}, \mdia{MOOc(0-1-2)(01-12-2)}$.  These
    diagrams look different from ordinary ones because we are computing weight
    displacements $\Delta_l \triangleq \expc[\theta_T-\theta_0]$, not test
    losses $\expc[l(\theta_T)]$.  Of course, in the noiseless case, those
    expectation symbols are redundant.  Likewise, in the noiseless case
    $\Delta_l$ is a function only of $\eta, T$ (and of the loss landscape $l$
    and the initialization $\theta_0$); in particular, we may set $E,B$ as
    convenient.  Let's set $E=B=1$.

%\moosect{GD's displacement}

    So, we seek rootless fuzzless diagrams width $\leq 2$ edges. 
    $\mdia{MOOc(0)(0)}$ and
    $\mdia{MOOc(0-1)(01-1)}$ are the only such.
    %
    Let's use their uvalues %as in \pag{36}\thm{3}
    to compute $\Delta_l(T,\eta)$.
    We read off:
    $$
    \text{uvalue}(\mdia{MOOc(0)(0)}) = G_\mu \eta^{\mu\nu} = h G 
    \hspace{1cm}
    \text{uvalue}(\mdia{MOOc(0-1)(01-1)}) = G_\mu \eta^{\mu\sigma} H_{\sigma\rho} \eta^{\rho\nu} = h^2 (H G)   
    $$
    The RHSs of the above concretize to the case that $\eta^{\mu\sigma}$ (in our
    directionality-aware theory a symmetric bilinear form that takes two covectors and outputs a
    scalar) is $h$ times the standard dot product and that $G, H$ are represented
    in standard ways as matrices.
    %
    The diagrams embed (into an $E=B=1$ grid)
    %that looks like the rightmost grid on \pag{18})
    in $T$ and in ${T\choose 2}$ many
    ways, respectively.
    The ${T\choose 2}$ arises due to the
    time-ordering condition: $\mdia{MOOc(0-1)(01-1)}$ has one embedding for
    every pair $0\leq t<t^\prime<T$, where $t$ is the red node's column
    and $t^\prime$ is the green node's column.

    These embeddings have trivial Aut groups
    %(as
    %implied by the fact that no two nodes of a diagram inhabit the same cell; see
    %(\pag{28}\exm{5}),
    so any fixed $T$ has a grand total:
    $$
        \Delta_l(T,h)
        =
        -hTG
        hT 
        +
        (h^2(T^2-T)/2)
        HG
        +
        o(\eta^2)
    $$

%\moosect{\cit{Ba}'s regularizer}
    %How does $\Delta_l$ relate to \cit{Ba}'s \thm{3.1}? 
    Since EulerMethod (EM) (simulation time $h$, $k$ steps) 
    is just GD with $\eta=h/k$, $T=k$, we can use $\Delta_{\tilde l}(k,h/k)$ to predict
    EM's behavior ---and hence ODE's behavior---
    on a loss $\tilde l$.
    %
    For $k$ huge and $\eta$ tiny (in a way that depends on $k$),
    $\Delta_{\tilde l}(k,h/k)$ is close to
    $$
        \star(h) = -h\tilde G
        +
        (\tilde H \tilde G)
        h^2/2
    $$
    %(I.e., $\tilde l$ analytic $\implies$
    %$\forall\epsilon$ $\exists k_0\forall k>k_0$ $\forall A>0$ $\exists h_0>h_0\forall h<h_0$:
    %    $\|\Delta_{\tilde l}(k,h/k) - \star(h)\| < A h^2 + \epsilon$.)

    To match $\star$ with ordinary GD's one-step displacement $\Delta_l(1,h)
    = -hG$, we just need $hG = h\tilde{G} - (\tilde H \tilde G) h^2/2 + o(h^2)$;
    it's enough to set $G=\tilde{G}+(\tilde H \tilde G) h/2$.  Recognizing
    the RHS as a total derivative (as $\nabla(\tilde G\cdot \tilde
    G) = 2\tilde{H}\tilde{G}$), we see it's enough that $G = \nabla(\tilde l - (h/4) (\tilde G\cdot \tilde G))$ or:
    \begin{align*}
        l &= \tilde l - (h/4) (\tilde G\cdot \tilde G) \\
          &= \tilde l - (h/4) (G\cdot G) + o(h^2)
    \end{align*}

    This shows how to turn a loss $\tilde l$ (on which we plan to run ODE),
    into a loss $l$ such that running one GD step on $l$ matches ODE on $\tilde
    l$ to leading non-trivial order.  Or how to turn
    $l$ into $\tilde l$.  In either case, the key term is $(h/4) (G\cdot G)$ with
    the appropriate sign.

        \subsubsection{Corollary \ref{cor:vsode}}

            \begin{proof}
            The corollary follows from consideration of $\sdia{c(01-2)(02-12)}$
                (c.f.\ Prop \ref{prop:ordtwo}).  This is the leading order
                effect of $C$ because it is the smallest
                diagram involving $C$ that has more than zero histories
                for $E=B=1$ SGD.
                %
            Now note that
            $\sdia{c(01-2)(02-12)}$ has $N{MB\choose 2}=N$ many embeddings for
            SGD $E=B=1$ and that each embedding has a size-two automorphism
            group.  So the diagram's effect on the testing loss is $N/2$ times
            its uvalue.  We read off the effect $(N/2) \cdot (C\eta^2 H)$
            claimed in the corollary statement.   
            \end{proof}

        \subsubsection{Corollary \ref{cor:vssde}}
            \begin{proof}%[Proof of second part]
                Because $\expct{\nabla l}$ vanishes at initialization, all
                diagrams with a degree-one vertex that is a singleton vanish.
                Because we work at order $\eta^3$, we consider $3$-edged
                diagrams.  Finally, because all first and second moments match
                between the two landscapes, we consider only diagrams with at
                least one part (in their partition) of size at least $3$.  The
                only such test diagram is $\sdia{c(012-3)(03-13-23)}$.  This
                embeds in $T$ ways (one for each grid cell --- recall that
                $E=B=1$) and has symmetry factor $1/3!$ for a total of
                $$
                    \frac{T \eta^3 }{6}
                    \expct{\nabla^3 l}
                    \expct{\nabla l_{n_{t_a}} \nabla l_{n_{t_b}} \nabla l_{n_{t_c}}}
                    = 
                    \frac{T \eta^3 }{6}
                    S_{\mu\nu\sigma}J^{\mu\nu\sigma}
                $$
                This is the un-resummed expression.  To obtain the re-summed
                expression, we replace $\eta^{\mu\nu}$ with $(I-\eta H)^{\Delta t-1}\eta)^{\mu\nu}$. 
                The histories range over $T$ many times uniformly spaced in $[0,T]$. 
                So we may integrate (let's name our variable of integration $\tau$,
                and let's have it represent $\tau=\Delta t-1 = T-t-1$):
                $$
                    \int_{0\leq \tau < T}
                        \wrap{\exp(-\tau \eta H) \eta}^{\mu\nu}
                        \wrap{\exp(-\tau \eta H) \eta}^{\pi\sigma}
                        \wrap{\exp(-\tau \eta H) \eta}^{\lambda\rho}
                        S_{\mu\pi\lambda}J_{\nu\sigma\rho}
                $$
                Observe that 
                $$
                        \wrap{\exp(-\tau \eta H)}_\mu^\nu 
                        \wrap{\exp(-\tau \eta H)}_\pi^\sigma 
                        \wrap{\exp(-\tau \eta H)}_\lambda^\rho
                    =
                        \wrap{\exp(-\tau Y)}_{\mu\nu\pi}^{\sigma\lambda\rho}
                $$                         
                where $Y = \eta H \otimes I \otimes I + I \otimes \eta H \otimes I + I \otimes I \otimes \eta H$
                is a six-index tensor.  We finish by recalling that 
                $$
                    \int_t \exp(A t) = \frac{\exp(A t)}{A} |_t 
                $$
            \end{proof}                     



        \subsubsection{Corollary \ref{cor:overfit}}
        \textsc{Corollary's first part}---

            \begin{proof}[Proof.]
                The relevant linkless diagram is $\sdia{c(01-2)(02-12)}$.
                %(which equals $\sdia{c(01-2)(02-12)}$ because we are at a test
                %minimum).
                This diagram has one history for each pair of
                same-row shaded cells, potentially identical, in a grid; for
                GD, the grid has every cell shaded, so each
                \emph{non-decreasing} pair of durations in $[0,T]^2$ is
                represented; the symmetry factor for the case where the cells
                is identical is $1/2$, so we lose no precision by interpreting
                a automorphism-weighted sum over the \emph{non-decreasing}
                pairs as half of a sum over all pairs.  Each of these may embed
                into $N$ many rows, hence the factor below of $N$.  The two
                integration variables (say, $t, \tilde t$) separate, and we
                have:
                $$
                    \frac{N}{B^{\text{degree}}}
                    \frac{C_{\mu\nu}}{2}
                    \int_t \wrap{\exp(-t \eta H)}^\mu_\lambda
                    \int_{\tilde t} \wrap{\exp(-\tilde t \eta H)}^\nu_\rho
                    \eta^{\lambda\sigma}
                    \eta^{\rho\pi}
                    H_{\sigma\pi}
                $$
                Since for GD we have $N=B$ and we are working to degree $2$,
                the prefactor is $1/N$.  Since $\int_t \exp(a t) = (I-\exp(-a
                T))/a$, the desired result follows. 
            \end{proof}

        %\subsubsection{Corollary \ref{cor:overfit}'s second part}
        \textsc{Corollary's second part}---

            We apply the generalization gap modification (described in
            page \pageref{appendix:solve-variants}) to Theorem \ref{thm:resum}'s
            result about testing losses.

            \begin{proof}[Proof]
                The relevant linkless diagram is $\sdia{c(01)(01)}$.  This
                diagram has one history for each shaded cell of grid;
                for GD, the grid has every cell shaded, so each duration
                from $0$ to $T$ is represented.  So the generalization gap is,
                to leading order,
                $$
                    + \frac{C_{\mu\nu}}{N}
                    \int_t \wrap{\exp(-t \eta H)}^\mu_\lambda
                    \eta^{\lambda\nu}
                $$
                Here, the minus sign from the gen-gap modification canceled
                with the minus sign from the odd power of $-\eta$.  Integration
                finishes the proof.
            \end{proof}
 
            %\newpage
    \subsection{Future topics}                                      \label{appendix:future}

        Our diagrams invite exploration of Lagrangian formalisms and curved
        backgrounds:\footnote{
            \cite{la60, la51} review these concepts.
        }
        \begin{quest}
            Does some least-action principle govern SGD; if not, what is an
            essential obstacle to this characterization?
        \end{quest}
        Lagrange's least-action formalism intimately intertwines with the
        diagrams of physics.  Together, they afford a modular framework for
        introducing new interactions as new terms or diagram nodes.  In fact,
        we find that some \emph{higher-order} methods --- such as the
        Hessian-based update
        $
            \theta \leftsquigarrow
            \theta -
            (\eta^{-1} + \lambda \nabla \nabla l_t(\theta))^{-1}
            \nabla l_t(\theta)
        $
        parameterized by small $\eta, \lambda$ --- admit diagrammatic analysis
        when we represent the $\lambda$ term as a second type of diagram node.
        Though diagrams suffice for computation, it is Lagrangians that most
        deeply illuminate scaling and conservation laws.

        Our work assumes a flat metric $\eta^{\mu\nu}$, but it might
        generalize to weight spaces curved in the sense of Riemann.\footnote{
            One may represent the affine connection as a node, thus giving
            rise to non-tensorial and hence gauge-dependent diagrams.
        }  Such curvature finds concrete application in the \emph{learning on
        manifolds} paradigm of \cite{ab07, zh16}, notably specialized to
        \cite{am98}'s \emph{natural gradient descent} and \cite{ni17}'s
        \emph{hyperbolic histories}.  While that work focuses on
        \emph{optimization} on curved weight spaces, in machine learning we
        also wish to analyze \emph{generalization}.
        %
        Starting with the intuition that ``smaller'' hypothesis classes
        generalize better and that curvature controls the volume of small
        neighborhoods, we conjecture that sectional curvature regularizes
        learning:
        \begin{conj}[Sectional curvature regularizes]
            If $\eta(\tau)$ is a Riemann metric on weight space, smoothly
            parameterized by $\tau$, and if the sectional curvature through
            every $2$-form at $\theta_0$ increases as $\tau$ grows, then
            the gen.\ gap attained by fixed-$T$ SGD with learning rate $c
            \eta(\tau)$ (when initialized from $\theta_0$) decreases as $\tau$
            grows, for all sufficiently small $c>0$.
        \end{conj}
        We are optimistic our formalism may resolve conjectures such as above.


