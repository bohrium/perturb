\documentclass{article}

\usepackage{neurips_2020_author_response}

\usepackage[utf8]{inputenc} % allow utf-8 input
\usepackage[T1]{fontenc}    % use 8-bit T1 fonts
\usepackage{hyperref}       % hyperlinks
\usepackage{url}            % simple URL typesetting
\usepackage{booktabs}       % professional-quality tables
\usepackage{amsfonts}       % blackboard math symbols
\usepackage{nicefrac}       % compact symbols for 1/2, etc.
\usepackage{microtype}      % microtypography

\usepackage{lipsum, xcolor}
\usepackage{amsmath, amssymb, amsthm}

\newcommand{\Ra}{\textmd{\textsf{\color{brown} {R1}}}}
\newcommand{\Rb}{\textmd{\textsf{\color{green} {R2}}}}
\newcommand{\Rc}{\textmd{\textsf{\color{blue} {R3}}}}
\newcommand{\Rd}{\textmd{\textsf{\color{purple} {R4}}}}

\newcommand{\sct}[1]{\textmd{\textsf{Sect #1}}}
\newcommand{\cor}[1]{\textmd{\textsf{Cor #1}}}
\newcommand{\dfn}[1]{\textmd{\textsf{Dfn #1}}}
\newcommand{\pdx}[1]{\textmd{\textsf{Appx #1}}}
\newcommand{\lin}[1]{\textmd{\textsf{Ln #1}}}
\newcommand{\fig}[1]{\textmd{\textsf{Fig #1}}}

\begin{document}

    We thank reviewers \Ra, \Rb, \Rc, \Rd\ for a wealth of constructive
    comments.  The reviewers expressed interest in our results but had concerns
    over our exposition's clarity. In response to these concerns, we will
    thoroughly revise the paper as detailed below, and in addition to the
    NeurIPS submission process we will try to submit to a journal.  As \Ra\
    noted, conferences have the advantage of visibility for ideas we are proud
    of.

%

    \textsc{Planned re-factoring of the narrative}
    ------
    Per \Ra, \Rd, we will

%

    \textsc{Planned economizing of space}
    ------
    \Ra: we will use the full 8 pages and remove distracting content such as
    the Chladni plate image and archaic references from the paper body.  

%

    \textsc{Planned clarification of claims}
    ------
    We will expand \sct{1.2} to less tersely define the test loss $l(\theta)$,
    the generalization gap, and the tensors $M^1_1, M^2_1, \cdots$, where
    we adopt \Ra's suggested $M$ notation.
    ------
    ------
    ------
    Per \Ra, we will render hand-drawn figures cleanly, and rework \fig{4}
    entirely by plotting an SGD trajectory through three cross-sectional slices
    of the landscape, in each slice indicating gradient noise with contour
    lines and expected loss with a colored heatmap.  Only two slices are
    necessary, but to emphasize that the effect happens for all time, we'll
    show the trajectory for three slices.  We will also orient \fig{4} to align
    with \fig{1a}.

%

    \textsc{Planned defense of significance}
    ------

%

    \textsc{Technical clarifications}
    ------
    \textbf{\Ra\ \lin{51}: How is the expression in an eigenbasis of $\eta H$?
    It looks like only $H$.} $H$ is rank $(0,2)$ (no upper idxs, two lower
    idxs), so it maps vectors to covectors and without further structure we
    can't speak of its eigenvalues.  That's why we use $(\eta H)^\mu_\nu$, a
    linear map that maps vectors to vectors, to get an eigenbasis.  Then
    $\eta(H_{\mu\mu} + H_{\nu\nu}+\cdots)$ is short for ``the $\mu$th
    eigval of $\eta H$, plus the $\nu$th eigval ...''.  On \lin{51$\frac{1}{2}$} we
    are explicit instead of using summation convention; accordingly, we 
    disobey the usual syntax for upper/lower idxs.  This is confusing, so we
    will use uniform notation throughout the revision.
    ------
    \textbf{\Ra\ I am skeptical of the authors' claim that this work reconciles
    conflicting results on sharp vs flat minima.}
    {\color{red} FILL IN}
    ------
    \textbf{\Rb\ \sct{3}: What was being plotted, architectures, etc?}
    \fig{3.a} shows the test losses (y axis) attained after fixed-time SGD runs
    with different learning rates (x axis), with one random initialization.
    We'll expand all figures' captions and discussion.  \pdx{C.2.1} lists
    architectures.
    ------
    \textbf{\Rb\ \sct{2.5}, \Rd\ \sct{3.2}: Do corrections proposed satisfy
    these scaling relationships?  That higher-order approximations outperform
    lower-order approximations feels tautological.  Very artificial example
    does not motivate third order dynamics.} Some but not all of our
    corrections obey SDE
    noise-scaling laws in that they are functions of $\eta/B$.  We view our
    experiments as verifying that we forgot no factors of $2$ etc.
    As with many NeurIPS papers, our contribution is theoretical, and we
    suggest but do not demonstrate that this theory may one day improve
    training of modern neural nets.  Our artificial examples are typical in that the
    third order contributions that they isolate are all present in generic loss
    landscapes.  We show how to interpret third order terms, yielding insight
    when they are non-negligible.  These terms may be negligible in practice,
    but experiments on real data (\fig{3} green lines) suggest they are sometimes
    substantial.
    ------
    \textbf{\Rc\ \dfn{1}: What do diagrams stand for?  Are they valid math?}
    Formally, diagrams represents (sets of) terms in a Taylor expansion.
    \pdx{A.6} gives visual intuition.  \pdx{B} gives defns and proofs.
    ------
    \textbf{\Rc\ \lin{119}, \Rd\ \cor{3}: Is this ERM? What does \emph{test}
    refer to? Formally define $l(\theta)$ and generalization gap}
    We study SGD as an approximate method for ERM.  The test loss is the
    expectation $l(\theta) \triangleq
    \mathbb{E}_{x\sim\mathcal{D}}[l_x(\theta)]$ over fresh samples $x$ from the
    underlying distribution $\mathcal{D}$, as suggested by \lin{65}'s word
    ``unbiased''.  The generalization gap on a training set $\mathcal{S}\sim
    \mathcal{D}^N$ is $\mathbb{E}_{x\sim\mathcal{D}}[l_x(\theta)] -
    \mathbb{E}_{x\sim\mathcal{S}}[l_x(\theta)]$.
    Like prior work [e.g. Chaudari], our predictions depend on the underlying
    (and in practice unknown) distribution $\mathcal{D}$; one may obtain
    qualitative insight (e.g. \sct{2.3}) and unbiased estimates (\pdx{C.6})
    with just training data.
    ------
    \textbf{\Rd\ \cor{3}: Can one find a term $l_c$ that works globally? Can it
    be computed at less cost than running SGD?}
    Yes, \pdx{C.6} gives estimates for expressions of arbitrary order with only
    constant factor time overhead.  E.g.
    $2l_x(\theta)\cdot\nabla(l_x(\theta)-l_y(\theta))$ is for any fixed $\theta$
    an unbiased estimate of $\nabla C$, for $x,y\sim \mathcal{D}$.  This local
    estimate may thus be computed at each step as $\theta_t$ evolves.

\end{document}
