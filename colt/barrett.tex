%\documentclass[12pt]{colt2021} %% Anonymized submission
\documentclass[12pt]{article}

\usepackage{times}
\usepackage{soul}
\usepackage[
    top   =1.0in,
    bottom=1.0in,
    left  =1.0in,
    right =1.0in,
]{geometry}


\usepackage[utf8]{inputenc} % allow utf-8 input
\usepackage[T1]{fontenc}    % use 8-bit T1 fonts
\usepackage{hyperref}       % hyperlinks
\usepackage{url}            % simple URL typesetting
\usepackage{booktabs}       % professional-quality tables
\usepackage{amsfonts}       % blackboard math symbols
\usepackage{nicefrac}       % compact symbols for 1/2, etc.
\usepackage{microtype}      % microtypography
\usepackage[clock,weather]{ifsym}

\usepackage[dvipsnames]{xcolor}
\usepackage{amsmath, amssymb, amsthm, bm}

\newcommand{\Ra}{\textmd{\textsf{\color{purple!50} {R1}}}}
\newcommand{\Rb}{\textmd{\textsf{\color{green!60}  {R2}}}}
\newcommand{\Rc}{\textmd{\textsf{\color{blue!50}   {R3}}}}
\newcommand{\PP}{\mathbb{P}}
\newcommand{\Tt}{\mathcal{T}}
\newcommand{\Mm}{\mathcal{M}}
\newcommand{\Hh}{\mathcal{H}}
\newcommand{\NN}{\mathbb{N}}
\newcommand{\EE}{\mathbb{E}}
\newcommand{\RR}{\mathbb{R}}

\newcommand{\cor}[1]{\textmd{{\color{gray}Cor}{#1}}} 
\newcommand{\dfn}[1]{\textmd{\textsf{Def#1}}}
\newcommand{\apx}[1]{\textmd{\textsf{Apdx#1}}}
\newcommand{\pag}[1]{\textmd{{\color{gray}Pg}{#1}}}
\newcommand{\pgph}[1]{\textmd{{\color{gray}Par}{#1}}}
\newcommand{\fig}[1]{\textmd{{\color{gray}Fig}#1}}
\newcommand{\thm}[1]{\textmd{{\color{gray}Thm}{#1}}}
\newcommand{\lem}[1]{\textmd{{\color{gray}Lem}{#1}}}
\newcommand{\prp}[1]{\textmd{{\color{gray}Prp}{#1}}}
\newcommand{\trk}[1]{\textmd{{\color{gray}Trk}{#1}}}
\newcommand{\tab}[1]{\textmd{{\color{gray}Tab}{#1}}}
\newcommand{\exm}[1]{\textmd{{\color{gray}Exm}{#1}}}

\newcommand{\cit}[1]{[\textbf{#1}]}

\newcommand{\moosect}[1]{\par\noindent\hspace{-1cm}\textsc{\textbf{#1}}.}
%\newtheorem*{propA*}{{Prop A}}
%\newtheorem*{thm2*}{{Thm 2}}

\usepackage{amsfonts, makerobust, float}
\usepackage{mathtools, nicefrac, xstring, enumitem, pdflscape, multicol}
\usepackage[export]{adjustbox}
%---------------------  graphics and figures  ---------------------------------
\usepackage{wrapfig, caption}
\usepackage{hanging, txfonts, ifthen}

\definecolor{moor}{rgb}{0.8,0.2,0.2}
\definecolor{moog}{rgb}{0.2,0.8,0.2}
\definecolor{moob}{rgb}{0.2,0.2,0.8}


\newcommand{\offive}[1]{
    {\tiny
        \raisebox{-0.04cm}{\color{gray}\scalebox{2.5}{$\substack{
            \ifthenelse{\equal{#1}{0}}{{\color{moor}\blacksquare}}{\square} 
        }$}}%
        \raisebox{0.04cm}{$\substack{
            \IfSubStr{#1}{1}{{\color{moor}\blacksquare}}{\square}   
            \IfSubStr{#1}{1}{{\color{moor}\blacksquare}}{\square} \\
            \IfSubStr{#1}{2}{{\color{moor}\blacksquare}}{\square}    
            \IfSubStr{#1}{2}{{\color{moor}\blacksquare}}{\square}    
        }$}%
    }%
}


\newcommand{\ofsix}[1]{
    {\tiny \raisebox{0.04cm}{$\substack{
        \IfSubStr{#1}{0}{{\color{moor}\blacksquare}}{\square}    
        \IfSubStr{#1}{1}{{\color{moor}\blacksquare}}{\square}    
        \IfSubStr{#1}{2}{{\color{moor}\blacksquare}}{\square}  \\ 
        \IfSubStr{#1}{3}{{\color{moor}\blacksquare}}{\square}    
        \IfSubStr{#1}{4}{{\color{moor}\blacksquare}}{\square}    
        \IfSubStr{#1}{5}{{\color{moor}\blacksquare}}{\square}    
    }$}}%
}

%   The following reconciles COLT's style with \includegraphics:
%       (see tex.stackexchange.com/questions/520891)
%\makeatletter
%\let\Ginclude@graphics\@org@Ginclude@graphics
%\makeatother



\newcommand{\sizeddia}[2]{%
    \begin{gathered}%
        \includegraphics[scale=#2]{../diagrams/#1.png}%
    \end{gathered}%
}
\newcommand{\bdia}[1]{\protect \sizeddia{#1}{0.22}}
\newcommand{\dia} [1]{\protect \sizeddia{#1}{0.18}}
\newcommand{\mdia}[1]{\protect \sizeddia{#1}{0.14}}
\newcommand{\sdia}[1]{\protect \sizeddia{#1}{0.10}}



\begin{document}
%\title{}

    \newcommand{\LaT}{\Lambda_{\text{\tiny\VarClock}}}
    \newcommand{\Lad}{\Lambda_{\text{\tiny\Thermo{4}}}}

    \noindent
    %This is a brief note on \cit{Ba}, the paper suggested by \Rb.
    \cit{Ba}
    relates strongly with our work; we plan to discuss it in (the revision's
    analogue of) \S{5.1, 3.2}.

\moosect{Locating \cit{Ba} in our theory}
    \cit{Ba}'s \thm{3.1} computes order-$\eta^2$ weight
    displacements $\theta_T-\theta_0$ in the noiseless case $l_x=l$.  The
    relevant diagrams are thus those with $\leq 2$ edges and that contain no
    gray outlines.
    %
    Indeed, noiseless $\implies$ cumulants vanish $\implies$ any diagram that
    contains one or more gray outlines has a uvalue (and rvalue) equal to
    zero.  So a sum over diagrams is the same as a sum over gray-free diagrams,
    i.e., over each diagram whose partition (\pag{5}\dfn{1}) is maximally fine.

    Per \S{A.6}, we use `rootless' diagrams, e.g.\
    $\mdia{MOOc(0-1)(01-1)}, \mdia{MOOc(0-1-2)(01-12-2)}$.  These diagrams look
    different from ordinary ones because we are computing weight displacements
    $\Delta_l \triangleq \EE[\theta_T-\theta_0]$, not test losses
    $\EE[l(\theta_T)]$.  Of course, in the noiseless case, those expectation
    symbols are redundant.  Likewise, in the noiseless case $\Delta_l$ is a
    function only of $\eta, T$ (and of the loss landscape $l$ and the
    initialization $\theta_0$); in particular, we may set $E,B$ as convenient.
    Let's set $E=B=1$.

\moosect{GD's displacement}
    So, we seek rootless gray-free diagrams width $\leq 2$ edges. 
    $\mdia{MOOc(0)(0)}$ and
    $\mdia{MOOc(0-1)(01-1)}$ are the only such.
    %
    Let's use their uvalues as in \pag{36}\thm{3} to compute $\Delta_l(T,\eta)$.
    We read off:
    $$
    \text{uvalue}(\mdia{MOOc(0)(0)}) = G_\mu \eta^{\mu\nu} = h G 
    \hspace{1cm}
    \text{uvalue}(\mdia{MOOc(0-1)(01-1)}) = G_\mu \eta^{\mu\sigma} H_{\sigma\rho} \eta^{\rho\nu} = h^2 (H G)   
    $$
    The RHSs of the above concretize to the case that $\eta^{\mu\sigma}$ (in our
    directionality-aware theory a symmetric bilinear form that takes two covectors and outputs a
    scalar) is $h$ times the standard dot product and that $G, H$ are represented
    in standard ways as matrices.
    %
    The diagrams embed (into an $E=B=1$ grid that looks
    like the rightmost grid on \pag{18}) in $T$ and in ${T\choose 2}$ many
    ways, respectively.\footnote{%
        An embedding of a rootless diagram (e.g.\ $\mdia{MOOc(0-1)(01-1)}$)
        assigns \emph{every} node to a grid cell.  \pag{19}
        decrees that we assign only \emph{non-root} nodes when computing
        $\EE[l(\theta_T)]$; indeed, the root node
        represents the test-time factor $l$ and thus 
        corresponds to no training point or training step.  By contrast,
        every factor of every term in $\EE[\theta_T-\theta_0]$
        corresponds to some training point $n$ and training step $t$.
        So we assign \emph{all} nodes to grid cells.  We'll expand \S{A.6}
        to note as much.
        %(and analogously for \S{A.6}'s other two
        %variants).
    }  
    The ${T\choose 2}$ arises due to \pag{19}'s
    time-ordering condition: $\mdia{MOOc(0-1)(01-1)}$ has one embedding for
    every pair $0\leq t<t^\prime<T$, where $t$ is the red node's column
    and $t^\prime$ is the green node's column.

    These embeddings have trivial Aut groups
    %(as
    %implied by the fact that no two nodes of a diagram inhabit the same cell; see
    (\pag{28}\exm{5}), so any fixed $T$ has a grand total:
    $$
        \Delta_l(T,h)
        =
        -hTG
        hT 
        +
        (h^2(T^2-T)/2)
        HG
        +
        o(\eta^2)
    $$

\moosect{\cit{Ba}'s regularizer}
    %How does $\Delta_l$ relate to \cit{Ba}'s \thm{3.1}? 
    Since EulerMethod (EM) (simulation time $h$, $k$ steps) 
    is just GD with $\eta=h/k$, $T=k$, we can use $\Delta_{\tilde l}(k,h/k)$ to predict
    EM's behavior ---and hence ODE's behavior---
    on a loss $\tilde l$.
    %
    For $k$ huge and $\eta$ tiny (in a way that depends on $k$),
    $\Delta_{\tilde l}(k,h/k)$ is close to
    $$
        \star(h) = -h\tilde G
        +
        (\tilde H \tilde G)
        h^2/2
    $$
    (I.e., $\tilde l$ analytic $\implies$
    $\forall\epsilon$ $\exists k_0\forall k>k_0$ $\forall A>0$ $\exists h_0>h_0\forall h<h_0$:
        $\|\Delta_{\tilde l}(k,h/k) - \star(h)\| < A h^2 + \epsilon$.)

    To match $\star$ with ordinary GD's one-step displacement $\Delta_l(1,h)
    = -hG$, we just need $hG = h\tilde{G} - (\tilde H \tilde G) h^2/2 + o(h^2)$;
    it's enough to set $G=\tilde{G}+(\tilde H \tilde G) h/2$.  Recognizing
    the RHS as a total derivative (as $\nabla(\tilde G\cdot \tilde
    G) = 2\tilde{H}\tilde{G}$), we see it's enough that $G = \nabla(\tilde l - (h/4) (\tilde G\cdot \tilde G))$ or:
    \begin{align*}
        l &= \tilde l - (h/4) (\tilde G\cdot \tilde G) \\
          &= \tilde l - (h/4) (G\cdot G) + o(h^2)
    \end{align*}

    This shows how to turn a loss $\tilde l$ (on which we plan to run ODE),
    into a loss $l$ such that running one GD step on $l$ matches ODE on $\tilde
    l$ to leading non-trivial order.  Or how to turn
    $l$ into $\tilde l$.  In either case, the key term is $(h/4) (G\cdot G)$ with
    the appropriate sign.

\moosect{Mistakes in our work}
    We've found several oversights in our paper.
    %
    E.g.: our \cor{3} fails to state that the computed loss difference
    between SGD and ODE is the leading-order difference \emph{due to noise},
    i.e., that scales with some higher cumulant such as $C$.  Of course, even
    without noise, there is also a difference due to time discretization, given
    by $\mdia{c(0-1-2)(01-12)}$'s embeddings into ODE's $T=kT_0$ grid minus
    its embeddings into SGD's $T=T_0$ grid.  For large $k$, $(h/k)^2
    {kT_0\choose 2}\approx h^2T_0^2/2$, so ODE suffers $(T_0^2/2 - {T_0\choose
    2}) \text{uvalue}(\mdia{c(0-1-2)(01-12)}) = (h^2 T_0/2)(GHG)$ more loss
    than SGD for noiseless loss landscapes.

    And \tab{1}'s caption should clarify the same point. 

    The section introducing resummation should say `spanning 11 timesteps'
    instead of `spanning 12 timesteps' in its example.

%\moosect{References}
    %\small%footnotesize
    %\cit{Am} S-I.Amari, H.Nagaoka. \emph{Information Geometry}, pg 5.  Oxford UP 1993.  
    %\cit{Ba} D.G.Barrett, B.Dherin.  \emph{Implicit Gradient Regularization}.  ICLR 2021.
    %\cit{Cu} P.McCullagh.  \emph{Tensor Methods in Statistics}, \S{1.1-1.4},\S{1.8}.  Dover 2017.
    %\cit{Di} L.Dinh, R.Pascanu, S.Bengio, Y.Bengio.  \emph{Sharp Minima Can Generalize for Deep Nets}, \S{1},\S{5}.  ICML 2017.
    %\cit{Ke} N.S.Keskar et alia.  \emph{Large-Batch Training for Deep Learning}, \S{4}.  ICLR 2017.
    %\cit{Wi} H.Wilf.  \emph{Generatingfunctionology}, \S{2.1-2.3}.  Academic Press 1994.
\end{document}

