\documentclass{article}
\usepackage[
    top   =1in,
    bottom=1in,
    left  =1in,
    right =1in,
]{geometry}

\usepackage[utf8]{inputenc} % allow utf-8 input
\usepackage[T1]{fontenc}    % use 8-bit T1 fonts
\usepackage{hyperref}       % hyperlinks
\usepackage{url}            % simple URL typesetting
\usepackage{booktabs}       % professional-quality tables
\usepackage{amsfonts}       % blackboard math symbols
\usepackage{nicefrac}       % compact symbols for 1/2, etc.
\usepackage{microtype}      % microtypography

\usepackage[dvipsnames]{xcolor}
\usepackage{amsmath, amssymb, amsthm, bm}

\newcommand{\Ra}{\textmd{\textsf{\color{purple!50} {R1}}}}
\newcommand{\Rb}{\textmd{\textsf{\color{green!50}  {R2}}}}
\newcommand{\Rc}{\textmd{\textsf{\color{blue!50}   {R3}}}}
\newcommand{\PP}{\mathbb{P}}

\newcommand{\sct}[1]{\textmd{\textsf{Sct #1}}}
\newcommand{\cor}[1]{\textmd{\textsf{Cor #1}}}
\newcommand{\dfn}[1]{\textmd{\textsf{Dfn #1}}}
\newcommand{\apx}[1]{\textmd{\textsf{Apx #1}}}
\newcommand{\lin}[1]{\textmd{\textsf{Lin #1}}}
\newcommand{\fig}[1]{\textmd{\textsf{Fig #1}}}
\newcommand{\thm}[1]{\textmd{\textsf{Thm #1}}}
\newcommand{\prp}[1]{\textmd{\textsf{Prp #1}}}

\newcommand{\cit}[1]{[\textbf{#1}]}

\newcommand{\moosect}[1]{\par\noindent\textbf{#1} ---}
\newtheorem*{proposition*}{{\textsc{Proposition}}}

\usepackage{amsfonts, makerobust, float}
\usepackage{mathtools, nicefrac, xstring, enumitem, pdflscape, multicol}
\usepackage[export]{adjustbox}
%---------------------  graphics and figures  ---------------------------------
\usepackage{wrapfig, caption}
\usepackage{hanging, txfonts, ifthen}

\definecolor{moor}{rgb}{0.8,0.2,0.2}
\definecolor{moog}{rgb}{0.2,0.8,0.2}
\definecolor{moob}{rgb}{0.2,0.2,0.8}


\newcommand{\offive}[1]{
    {\tiny
        \raisebox{-0.04cm}{\color{gray}\scalebox{2.5}{$\substack{
            \ifthenelse{\equal{#1}{0}}{{\color{moor}\blacksquare}}{\square} 
        }$}}%
        \raisebox{0.04cm}{$\substack{
            \IfSubStr{#1}{1}{{\color{moor}\blacksquare}}{\square}   
            \IfSubStr{#1}{1}{{\color{moor}\blacksquare}}{\square} \\
            \IfSubStr{#1}{2}{{\color{moor}\blacksquare}}{\square}    
            \IfSubStr{#1}{2}{{\color{moor}\blacksquare}}{\square}    
        }$}%
    }%
}



\begin{document}

    We thank reviewers \Ra, \Rb, \Rc\ for their feedback.  Reviewers had
    concerns over our work's correctness (\Ra), counterintuitiveness (\Rc), 
    citations (\Rb), and clarity (\Ra,\Rb,\Rc).  We address these concerns in
    sqeuence. 

\moosect{Limits}
    view the expected testing loss as a function
    $L(\eta, T)$.  For each $d$ is a $d$th order truncation $L_d(\eta, T)$, a
    degree-$d$ polynomial in $\eta$ whose coefficients depend on $T$.
    \thm{2} gives a sufficient condition for $L_{d,\infty}(\eta) \triangleq
    \lim_{T\to\infty} L_{d}(\eta,T)$ to exist as well as 
    a formula for $L_{d,\infty}$. %; \prp{1} specializes \thm{2}.
    %
    \Ra\ points out that, though
    \thm{2} controls $\text{LHS}(\eta)\triangleq \lim_{d\to\infty}
    \lim_{T\to \infty} L_d(\eta,T)$, it is
    $\text{RHS}(\eta)\triangleq \lim_{T\to\infty} \lim_{d\to\infty}
    L_d(\eta,T)$ that more interests us.  So how do LHS and RHS relate?
    %
    \begin{proposition*}\vspace{-0.25cm}
        Assume \apx{B.1}'s setting and suppose
        that $\nabla l(\theta_\star)=0$ and that on some neighborhood $U$ of
        $\theta_\star$ the hessian $\nabla\nabla l_x(\theta)$ is lower-bounded
        by some strictly positive definite form $Q(\theta)$ continuous in
        $\theta$.  \emph{Then} for any initialization $\theta_0\in V$ in some
        neighborhood $V$ of $\theta_\star$ and for any homogeneous polynomial
        $p(\eta)$ (of $\eta$'s $\text{dim}\times\text{dim}$ many components)
        with exactly one root (at $\eta=0$): $\lim_{\eta\to 0}
        (\text{LHS}(\eta)-\text{RHS}(\eta))/p(\eta) = 0$.\vspace{-0.25cm}
    \end{proposition*}
    %
    %Even without this proposition, though, $\text{LHS}(\eta)$ and
    %$\text{RHS}(\eta)$ disagree, we still have that 
    \text{\color{moor} TODO: prove, discuss, and discuss irrelevance!}

\moosect{Sharp Minima}
    \Rb\ notes that \cor{5} statement --- that the amount of overfitting
    (defined as the increase in testing loss $l$ upon initializing at a local
    minimum of $l$ and then training) is, to second order in $\eta$, greatest
    when the hessian has moderate curvature with respect to $\eta$ --- is
    counterintutive.
    %
    Comparing \fig{5 \offive{1}} to \cit{Ke}'s \fig{1}, we note that SGD's
    natural noise structure is \emph{not} that of \emph{displacements} in
    weight space; rather, it is that of additive error terms $\nabla
    l_x(\theta)-\nabla l(x)$ in the \emph{gradient} estimate.
    %
    Consider a one-dimensional $\theta$ and imagine a quadratic testing loss
    $l(\theta)=a \theta^2/2$ and a training loss $\hat
    l(\theta)=l(\theta)+b\theta$.  At the training minimum $\theta=-b/a$, the
    testing loss is $b^2/(2a)$.  Thus, for fixed $b$, sharper minima (larger
    $a$) overfit less.  The gradient covariance $C$ controls $b^2$,
    explaining \cor{5}'s $C/2H$ factor.  This example suggests that if we
    optimize to convergence, sharp minima overfit least.
    %
    But it is near flat minima that
    convergence is slowest, so for fixed $\eta, T$, we expect
    overfitting to vanish as the Hessian shrinks.  By making explicit $\eta$'s
    role in translating gradients into displacements, our theory accounts for
    both effects, thus reconciling \cit{Ke}'s pro-flat intuitions with 
    [Di]'s pro-sharp intuitions.
    %
    We view it as a merit that our formalism makes such counterintuitive
    phenomena visible.
    
\moosect{Citations}
    Thank you for pointing us to Barrett et al.  
    
\moosect{Clarity}
    Our work uses a convention standard in physics and in high-dimensional
    statistics. 

\par\noindent
\vspace{0.1cm}
\hrule

    \noindent
    \cit{Ba} D.G.Barrett, B.Dherin.  Implicit Gradient Regularization.  ICLR 2021.

    \noindent
    \cit{Di} L.Dinh, R.Pascanu, S.Bengio, Y.Bengio.  Sharp Minima Can Generalize for Deep Nets.  ICML 2017.

    \noindent
    \cit{Dy} E.Dyer, G.Gur-Ari.  Asymptotics of Wide Networks from Feynman Diagrams.  ICLR 2020.

    \noindent
    %\cit{Ke} N.S.Keskar, D.Mudigere, J.Nocedal, M.Smelyanskiy, P.T.P.Tang.  On Large-Batch Training for Deep Learning.  ICLR 2017.
    \cit{Ke} N.S.Keskar et alia.  On Large-Batch Training for Deep Learning.  ICLR 2017.

\end{document}
