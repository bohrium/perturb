\documentclass{article}
\usepackage[
    top   =1in,
    bottom=1in,
    left  =1in,
    right =1in,
]{geometry}

\usepackage[utf8]{inputenc} % allow utf-8 input
\usepackage[T1]{fontenc}    % use 8-bit T1 fonts
\usepackage{hyperref}       % hyperlinks
\usepackage{url}            % simple URL typesetting
\usepackage{booktabs}       % professional-quality tables
\usepackage{amsfonts}       % blackboard math symbols
\usepackage{nicefrac}       % compact symbols for 1/2, etc.
\usepackage{microtype}      % microtypography

\usepackage[dvipsnames]{xcolor}
\usepackage{amsmath, amssymb, amsthm, bm}

\newcommand{\Ra}{\textmd{\textsf{\color{purple!50} {R1}}}}
\newcommand{\Rb}{\textmd{\textsf{\color{green!50}  {R2}}}}
\newcommand{\Rc}{\textmd{\textsf{\color{blue!50}   {R3}}}}
\newcommand{\PP}{\mathbb{P}}

\newcommand{\cor}[1]{\textmd{\textsf{Cor #1}}} 
\newcommand{\dfn}[1]{\textmd{\textsf{Defn#1}}}
\newcommand{\apx}[1]{\textmd{\textsf{Apdx#1}}}
\newcommand{\pag}[1]{\textmd{\textsf{Page#1}}}
\newcommand{\fig}[1]{\textmd{\textsf{Fig #1}}}
\newcommand{\thm}[1]{\textmd{\textsf{Thm #1}}}
\newcommand{\prp}[1]{\textmd{\textsf{Prop#1}}}

\newcommand{\cit}[1]{[\textbf{#1}]}

\newcommand{\moosect}[1]{\par\textsc{\textbf{#1}} ---}
\newtheorem*{proposition*}{{Proposition A}}

\usepackage{amsfonts, makerobust, float}
\usepackage{mathtools, nicefrac, xstring, enumitem, pdflscape, multicol}
\usepackage[export]{adjustbox}
%---------------------  graphics and figures  ---------------------------------
\usepackage{wrapfig, caption}
\usepackage{hanging, txfonts, ifthen}

\definecolor{moor}{rgb}{0.8,0.2,0.2}
\definecolor{moog}{rgb}{0.2,0.8,0.2}
\definecolor{moob}{rgb}{0.2,0.2,0.8}


\newcommand{\offive}[1]{
    {\tiny
        \raisebox{-0.04cm}{\color{gray}\scalebox{2.5}{$\substack{
            \ifthenelse{\equal{#1}{0}}{{\color{moor}\blacksquare}}{\square} 
        }$}}%
        \raisebox{0.04cm}{$\substack{
            \IfSubStr{#1}{1}{{\color{moor}\blacksquare}}{\square}   
            \IfSubStr{#1}{1}{{\color{moor}\blacksquare}}{\square} \\
            \IfSubStr{#1}{2}{{\color{moor}\blacksquare}}{\square}    
            \IfSubStr{#1}{2}{{\color{moor}\blacksquare}}{\square}    
        }$}%
    }%
}


\newcommand{\ofsix}[1]{
    {\tiny \raisebox{0.04cm}{$\substack{
        \IfSubStr{#1}{0}{{\color{moor}\blacksquare}}{\square}    
        \IfSubStr{#1}{1}{{\color{moor}\blacksquare}}{\square}    
        \IfSubStr{#1}{2}{{\color{moor}\blacksquare}}{\square}  \\ 
        \IfSubStr{#1}{3}{{\color{moor}\blacksquare}}{\square}    
        \IfSubStr{#1}{4}{{\color{moor}\blacksquare}}{\square}    
        \IfSubStr{#1}{5}{{\color{moor}\blacksquare}}{\square}    
    }$}}%
}


\begin{document}

    We thank reviewers \Ra, \Rb, \Rc\ for their feedback.  Reviewers had
    concerns over our work's correctness (\Ra), counterintuitiveness (\Rc), 
    citations (\Rb), and clarity (\Ra,\Rb,\Rc).  We address these concerns in
    sqeuence. 

\moosect{Limits}
    view the expected testing loss as a function
    $L(\eta, T)$.  For each $d$ is a $d$th order truncation $L_d(\eta, T)$, a
    degree-$d$ polynomial in $\eta$ whose coefficients depend on $T$.
    \thm{2} gives a sufficient condition for $L_{d,\infty}(\eta) \triangleq
    \lim_{T\to\infty} L_{d}(\eta,T)$ to exist as well as 
    a formula for $L_{d,\infty}$. %; \prp{1} specializes \thm{2}.
    %
    \Ra\ observes that, though \thm{2} controls
    $\text{LHS}(\eta)\triangleq \lim_{d\to\infty} \lim_{T\to \infty}
    L_d(\eta,T)$, it is $\text{RHS}(\eta)\triangleq \lim_{T\to\infty}
    \lim_{d\to\infty} L_d(\eta,T)$ that more interests us.  How do LHS and
    RHS relate (and do they exist)?
    %
    \begin{proposition*}\vspace{-0.25cm}
        Assume \S B.1's boundedness and analyticity properties;
        {\textbf{if}} $\nabla l(\theta_\star)=0$ and on some
        open neighborhood $U$ of $\theta_\star$ the hessian $\nabla\nabla
        l_x(\theta)$ is lower-bounded by some strictly positive definite form
        $Q(\theta)$ continuous in $\theta$, {\textbf{then}} for any
        initialization $\theta_0\in V$ in some open neighborhood $V$ of
        $\theta_\star$ and for any homogeneous polynomial $p(\eta)$ (of
        $\eta$'s $\text{dim}\times\text{dim}$ many components; and with no roots
        besides $\eta=0$): $\lim_{\eta\to 0}
        (\text{LHS}(\eta)-\text{RHS}(\eta))/p(\eta) = 0$ is well-defined and vanishes.\vspace{-0.25cm}
    \end{proposition*}
    \begin{proof}[\textbf{Proof idea}]
        Gradient and hessian bounds give for all $\epsilon$ some
        $\delta,\delta^\prime$ so that for all $T$, all $\eta<\delta$, and all
        $\theta_0$ with $|\theta_0-\theta_\star|<\delta^\prime$:
        $|\theta_T-\theta_0|<\epsilon$ with probability $1$.
        %
        In fact, 
        {\color{moor}{FILL IN}}
        \vspace{-0.25cm}
    \end{proof}
    \noindent
    \prp{A} is a straightforward extension of \thm{2}'s proof.  But in
    some sense, \prp{A} is inessential: in practice we have observed none
    of the pathologies that \prp{A} seeks to control (\pag{6, last par}).
    Ultimately, we view our experiments (e.g., with CIFAR conv-nets) as
    verifying that our theory has substance.

\moosect{Sharp Minima}
    \Rc\ finds \cor{5} ({\small that overfitting
    (i.e., the excess testing loss $l$ incurred after initializing at a testing
    minimum then training) is, to second order in $\eta$, greatest
    when the $\eta H$ has moderate eigenvalues}) 
    counterintutive.  We do, too.
    %
    Compare \fig{5\offive{1}} to \cit{Ke}'s \fig{1}; note that SGD's noise
    consists not of \emph{displacements} in weight space but of
    error terms $\nabla l_x(\theta)-\nabla l(x)$ in the \emph{gradient}
    estimate.  
    %
    Say $\text{dim}=1$ with testing loss $l(\theta)=a \theta^2/2$ training loss
    $\hat l(\theta)=l(\theta)+b\theta$.  At the training minimum $\theta=-b/a$,
    the testing loss is $b^2/(2a)$.  So for fixed $b$, sharp minima ($a\gg 1$)
    overfit less.  The covariance $C$ controls $b^2$, explaining
    \cor{5}'s $C/2H$ factor.  (Run this toy example at \href{https://gist.github.com/anonymous-taylor-series/60ee7ca824e44a9e8f25e69ceb60995e}{gist.github.com/anonymous-taylor-series}.)
    In this case, optimization to convergence favors sharp minima
    (\textbf{A}); but convergence is slow near flat minima, so theory and
    measurement agree that flat minima also overfit little (\textbf{B}).  (Our
    small-$\eta$ assumption rules out the possibility that $H$ is so sharp that
    SGD diverges: in the regime $1/T \ll \eta H \ll 1$, sharper minima overfit
    less).
    %
    Prior work (see \pag{12, par 5}) finds that (contrary to \cit{Ke}) sharp
    minima overfit little.  By explicating $\eta$'s role in translating
    gradients into displacements, our theory accounts for both (\textbf{A}) and
    (\textbf{B}) and unifies existing pro-flat and pro-sharp intuitions
    (e.g., \cit{Ke} and \cit{Di}).
    %
    We view it as a merit that our formalism makes such counterintuitive
    phenomena visible.
    
\moosect{Implicit Regularization and ODE}
    We thank \Rb\ for the highly relevant article.  In brief,  
    
\moosect{Verification} \Rc,\Ra\ raise concerns about verifiability.

\moosect{Notation}
    \Rb,\Rc\ note that our use of `Einstein notation' (though found in tensor
    statistics (\cit{Mc}, \cit{Dy})) is alien to CS at large.  Our camera-ready
    will make all `$\sum$'s explicit while also referring the reader to
    \cit{Mc}, as well as a new Einstein-free appendix \S{D}, for examples of tensor manipulations.

\moosect{Organization}
    \Rb,\Rc\ stress the challenge of 
    organizing our paper so as to avoid both the burden of too many
    frontloaded concepts and the confusion of too many forward references to
    concepts delayed.  We'll address this challenge by segmenting the paper
    into three tracks (to be selected by a reader \emph{based on her goals}), each
    with a self-contained subset of concepts: 
    %
    \textbf{Track A} [pgs 1-4], for readers who want a `free sample', will
    eschew diagrams, general theorems, and \S1.1/\S2.2's heavy notations.  It will illustrate Taylor series via \S
    2.1's proof, identify the concrete terms relevant to \S 3.3, state \cor{4}
    (with \S B.1's assumptions explicit and with \prp{A}'s level of precision),
    and conclude with \S 4.2's verification of SGD's sensitivity to curl.  
    %
    \textbf{Track B} [pgs 1-4, 5-12], for she who seeks our results and their
    physical intuitions, will use Track A to motivate (and \S A.4 to
    illustrate) \S1.1/2.2/2.3's definitions, relegating \S2.2/2.1's Lemma and
    discussions of terms to \S B.  \S 2.4 will include \prp{A} per \Ra's
    feedback as well as a cartoon (in \fig{5,7}'s style) of resummation's
    `physics'.  For space, we'll move \S 4 to \S C; but each of \S 3.1/3.2/3.4
    will briefly summarize the relevant empirical confirmation.
    %
    \textbf{Track C} [pgs 5-12, 15-42], for she who wishes to use and extend
    our formalism, will {\color{moor}{FILL IN}}.

\moosect{References}
    \cit{Ba} D.G.Barrett, B.Dherin.  \emph{Implicit Gradient Regularization}.  ICLR 2021.
    ---
    \cit{Di} L.Dinh, R.Pascanu, S.Bengio, Y.Bengio.  \emph{Sharp Minima Can Generalize for Deep Nets}.  ICML 2017.
    ---
    \cit{Dy} E.Dyer, G.Gur-Ari.  \emph{Asymptotics of Wide Networks from Feynman Diagrams}.  ICLR 2020.
    ---
    \cit{Ke} N.S.Keskar et alia.  \emph{On Large-Batch Training for Deep Learning}.  ICLR 2017.
    ---
    \cit{Mc} P.McCullagh.  \emph{Tensor Methods in Statistics}.  Dover Books 2017.
\end{document}

%\moosect{Surprise}
%    We are glad that some of our results surprised \Rb, \Rc.  We believe these
%    results, and more importantly the physical-geometric viewpoint that led to
%    them, are worth sharing.  Recalling that many of our favorite papers
%    are those offering a disorientingly fresh viewpoint, we hope that the
%    reviewers can feel the same about our work.
    %\cit{Ke} N.S.Keskar, D.Mudigere, J.Nocedal, M.Smelyanskiy, P.T.P.Tang.  On Large-Batch Training for Deep Learning.  ICLR 2017.
    %\cit{Mc} J.Ragan-Kelley et alia.  Easy Optimization of Image Processing Pipelines.  ACM Transactions July 2012.


    %We the authors prefer to admit both
    %mathematical analysis and scientific measurement as means of
    %discovering.   

