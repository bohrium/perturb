%\documentclass[anon,12pt]{colt2021} % Anonymized submission
\documentclass[final,12pt]{colt2021} % Include author names

% The following packages will be automatically loaded:
% amsmath, amssymb, natbib, graphicx, url, algorithm2e

\usepackage{amsfonts, makerobust}
\usepackage{mathtools, nicefrac, xstring, enumitem}

%   The following reconciles COLT's style with \includegraphics:
%       (see tex.stackexchange.com/questions/520891)
\makeatletter
\let\Ginclude@graphics\@org@Ginclude@graphics
\makeatother

%---------------------  graphics and figures  ---------------------------------

\usepackage{wrapfig, caption}
\usepackage{hanging, txfonts, ifthen}

\newcommand{\ofsix}[1]{
    {\tiny \raisebox{0.04cm}{$\substack{
        \ifthenelse{\equal{#1}{0}}{{\color{moor}\blacksquare}}{\square}
        \ifthenelse{\equal{#1}{2}}{{\color{moor}\blacksquare}}{\square}    
        \ifthenelse{\equal{#1}{4}}{{\color{moor}\blacksquare}}{\square} \\
        \ifthenelse{\equal{#1}{1}}{{\color{moor}\blacksquare}}{\square}    
        \ifthenelse{\equal{#1}{3}}{{\color{moor}\blacksquare}}{\square}
        \ifthenelse{\equal{#1}{5}}{{\color{moor}\blacksquare}}{\square}
    }$}}%
}

\newcommand{\offive}[1]{
    {\tiny
        \raisebox{-0.04cm}{\color{gray}\scalebox{2.5}{$\substack{
            \ifthenelse{\equal{#1}{0}}{{\color{moor}\blacksquare}}{\square} 
        }$}}%
        \raisebox{0.04cm}{$\substack{
            \IfSubStr{#1}{1}{{\color{moor}\blacksquare}}{\square}   
            \IfSubStr{#1}{1}{{\color{moor}\blacksquare}}{\square} \\
            \IfSubStr{#1}{2}{{\color{moor}\blacksquare}}{\square}    
            \IfSubStr{#1}{2}{{\color{moor}\blacksquare}}{\square}    
        }$}%
    }%
}

\newcommand{\ofthree}[1]{
    {\tiny \raisebox{0.04cm}{$
        \ifthenelse{\equal{#1}{0}}{{\color{moor}\blacksquare}}{\square}
        \ifthenelse{\equal{#1}{1}}{{\color{moor}\blacksquare}}{\square}    
        \ifthenelse{\equal{#1}{2}}{{\color{moor}\blacksquare}}{\square}
    $}}%
}


%---------------------  colors  -----------------------------------------------

\usepackage{xcolor, framed}
\definecolor{moolime}{rgb}{0.90,1.00,0.90}
\definecolor{moosky}{rgb}{0.90,0.90,1.00}
\definecolor{moopink}{rgb}{1.00,0.90,0.90}
\definecolor{moor}{rgb}{0.8,0.2,0.2}
\definecolor{moog}{rgb}{0.2,0.8,0.2}
\definecolor{moob}{rgb}{0.2,0.2,0.8}
\definecolor{mooteal}{rgb}{0.1,0.6,0.4}

%---------------------  intertext: footnotes and hyperlinks  ------------------ 

\usepackage[perpage]{footmisc}
\renewcommand*{\thefootnote}{
    \color{red}
    \arabic{footnote}
    %\fnsymbol{footnote}
} 

\usepackage{hyperref}

%~~~~~~~~~~~~~~~~~~~~~~~~~~~~~~~~~~~~~~~~~~~~~~~~~~~~~~~~~~~~~~~~~~~~~~~~~~~~~~
%~~~~~~~~~~~~~  Theorem Environments  ~~~~~~~~~~~~~~~~~~~~~~~~~~~~~~~~~~~~~~~~~

%---------------------  mathematical results  ---------------------------------

%\theoremstyle{plain}
    \newtheorem*{klem*}{Key Lemma}
    \newtheorem{thm}{Theorem}
    \newtheorem*{thm*}{Theorem}
    \newtheorem{cor}{Corollary}
    \newtheorem{prop}{Proposition}
    \newtheorem*{prop*}{Proposition}

%---------------------  mathematical questions  -------------------------------

    \newtheorem{conj}{Conjecture}
    \newtheorem{quest}{Question}
    \newtheorem*{quest*}{Question}
    \newtheorem*{quests*}{Questions}

%---------------------  definitions, answers, remarks  ------------------------

%\theoremstyle{definition}
    \newtheorem{defn}{Definition}
    \newtheorem*{answ*}{Answer}
    \newtheorem{rmk}{Remark}
    \newtheorem*{midea*}{Main Idea}
    \newtheorem*{rmk*}{Remark}
    \newtheorem{exm}{Example}


%~~~~~~~~~~~~~~~~~~~~~~~~~~~~~~~~~~~~~~~~~~~~~~~~~~~~~~~~~~~~~~~~~~~~~~~~~~~~~~
%~~~~~~~~~~~~~  Custom Math Commands  ~~~~~~~~~~~~~~~~~~~~~~~~~~~~~~~~~~~~~~~~~

%---------------------  expanding containers  ---------------------------------

\newcommand{\wrap}[1]{\left(#1\right)}
\newcommand{\wasq}[1]{\left[#1\right]}
\newcommand{\wang}[1]{\left\langle#1\right\rangle}
\newcommand{\wive}[1]{\left\llbracket#1\right\rrbracket}
\newcommand{\worm}[1]{\left\|#1\right\|}
\newcommand{\wabs}[1]{\left|#1\right|}
\newcommand{\wurl}[1]{\left\{#1\right\}}

\newcommand{\partitionbox}[1]{
    \text{
        \fboxsep=0.5pt
        \tiny
        \fbox{#1}
    }
}

%---------------------  special named objects  --------------------------------


        \newcommand{\nb} { \nabla }
        \newcommand{\lx} { l_x(\theta) }
        \newcommand{\teq} { \triangleq }
        \newcommand{\ex}[1] { \expc_x \wasq{#1} }


\newcommand{\Free}{\mathcal{F}}
\newcommand{\Forg}{\mathcal{G}}
\newcommand{\Mod}{\mathcal{M}}
\newcommand{\Hom}{\text{\textnormal{Hom}}}
\newcommand{\Aut}{\text{\textnormal{Aut}}}
\newcommand{\image}{\text{\textnormal{im}}}
\newcommand{\uvalue}{\text{\textnormal{uvalue}}}
\newcommand{\rvalue}{\text{\textnormal{rvalue}}}
\newcommand{\edges}{\text{\textnormal{edges}}}
\newcommand{\ords}{\text{\textnormal{ords}}}
\newcommand{\parts}{\text{\textnormal{parts}}}
\newcommand{\SGD}{\text{\textnormal{SGD}}}
\DeclareMathOperator*{\Avg}{\text{\sffamily A}}
\newcommand{\expc}{\mathbb{E}}
\newcommand{\expct}[1]{\mathbb{E}\left[#1\right]}

%---------------------  fancy letters  ----------------------------------------

\newcommand{\Aa}{\mathcal{A}}
\newcommand{\Bb}{\mathcal{B}}
\newcommand{\Cc}{\mathcal{C}}   \newcommand{\CC}{\mathbb{C}}
\newcommand{\Dd}{\mathcal{D}}
\newcommand{\Ee}{\mathcal{E}}
\newcommand{\Ff}{\mathcal{F}}
\newcommand{\Gg}{\mathcal{G}}
\newcommand{\Hh}{\mathcal{H}}
\newcommand{\Ll}{\mathcal{L}}
\newcommand{\Mm}{\mathcal{M}}
\newcommand{\Nn}{\mathcal{N}}   \newcommand{\NN}{\mathbb{N}}
\newcommand{\Oo}{\mathcal{O}}
\newcommand{\Pp}{\mathcal{P}}
\newcommand{\Qq}{\mathcal{Q}}   \newcommand{\QQ}{\mathbb{Q}}
\newcommand{\Rr}{\mathcal{R}}   \newcommand{\RR}{\mathbb{R}}
\newcommand{\Ss}{\mathcal{S}}
\newcommand{\Tt}{\mathcal{T}}
\newcommand{\Uu}{\mathcal{U}}
\newcommand{\Vv}{\mathcal{V}}
\newcommand{\Ww}{\mathcal{W}}
\newcommand{\Xx}{\mathcal{X}}
\newcommand{\Yy}{\mathcal{Y}}
\newcommand{\Zz}{\mathcal{Z}}   \newcommand{\ZZ}{\mathbb{Z}}

%~~~~~~~~~~~~~~~~~~~~~~~~~~~~~~~~~~~~~~~~~~~~~~~~~~~~~~~~~~~~~~~~~~~~~~~~~~~~~~
%~~~~~~~~~~~~~  Pictures  ~~~~~~~~~~~~~~~~~~~~~~~~~~~~~~~~~~~~~~~~~~~~~~~~~~~~~

%---------------------  pictures with specified width or height  --------------

\newcommand{\plotmoow}[3]{\includegraphics[width=#2          ]{../#1}}
\newcommand{\plotmooh}[3]{\includegraphics[         height=#3]{../#1}}
\newcommand{\pmoo}[2]{\includegraphics[height=#1]{../plots/#2}}
\newcommand{\dmoo}[2]{\includegraphics[height=#1]{../diagrams/#2}}

%---------------------  inline diagrams of various sizes  ---------------------

\newcommand{\sizeddia}[2]{%
    \begin{gathered}%
        \includegraphics[scale=#2]{../diagrams/#1.png}%
    \end{gathered}%
}
\newcommand{\bdia}[1]{\protect \sizeddia{#1}{0.22}}
\newcommand{\dia} [1]{\protect \sizeddia{#1}{0.18}}
\newcommand{\mdia}[1]{\protect \sizeddia{#1}{0.14}}
\newcommand{\sdia}[1]{\protect \sizeddia{#1}{0.10}}

\newcommand{\mend}{\hfill $\Diamond$}

\newcommand{\Gauss}{\textsc{Gauss}}
\newcommand{\Archimedes}{\textsc{Archimedes}}
\newcommand{\MeanEstimation}{\textsc{Mean Estimation}}




\newcommand{\sS}{\hspace{0.43em}}
\newcommand{\sammail}{%
    C{\sS}%
    O{\sS}%
    L{\sS}%
    I{\tiny{@}}%
    M{\sS}%
    I{\sS}%
    T{\sS}%
    .{\sS}%
    e{\sS}%
    d{\sS}%
    u%
}

\title[SGD at Small Learning Rates]{SGD at Small Learning Rates}
\usepackage{times}
\coltauthor{%
    \Name{Samuel C.\ Tenka} \Email{\sammail}    \\
    \addr MIT, CSAIL
}

\begin{document}

    \maketitle
    
    \begin{abstract}%
        We quantify how gradient noise shapes the dynamics of stochastic
        gradient descent (SGD) by taking Taylor series in the learning rate.
        %
        We present in particular a new diagram-based notation that permits
        resummation to convergent results.
        %
        We employ our theory to contrast SGD against two popular
        approximations: deterministic descent and stochastic differential
        equations.  We find that SGD's trajectory avoids both gradient noise
        and minima that are sharp with respect to gradient noise.  Paired with
        our results that connect overfitting to the width of minima, these
        repulsions suggest a mechanism for the unexpected generalization of
        overparameterized learning scenarios.
    \end{abstract}
    
    \begin{keywords}%
        SGD, learning rates, generalization, gradient noise. 
    \end{keywords}

    \section{Introduction}

        \subsection{Intuitions about SGD}

            %--  object of study  ---------------------------------------------
            %
            Practitioners benefit from the intuition that stochastic gradient
            descent (SGD) approximates deterministic gradient descent (GD)
            \citep{bo91}.  This paper refines that intuition by showing
            how gradient noise biases learning toward certain areas of weight
            space.

            %--  vs ode and sde  ----------------------------------------------
            %
            Departing from prior work, we model discrete time and hence
            non-Gaussian noise.  Indeed, we derive corrections to 
            continuous-time, Gaussian-noise approximations such as ordinary and
            stochastic differential equations (ODE, SDE).
            %--
            Our corrections lead to qualitative differences in dynamics: for
            example, we construct a non-pathological\footnote{
                All higher derivatives exist and quadratically bounded; the
                gradient noise at each weight vector is $1$-subgaussian.
            } loss landscape with a
            discrete translational symmetry \emph{violated} by SGD's
            trajectory.  Adding a linear term to this landscape leads SGD to
            perform perpetual \emph{ascent}.
            %--
            These examples, though artificial, demonstrate that standard
            intuitions about SGD are wrong.  We argue that our theory's
            quantitative results usefully enhance the intuition of practioners,
            and we we verify our theory on convolutional CIFAR-10 and
            Fashion-MNIST loss landscapes.

            %--  soft benefits: retrospective  --------------------------------
            %
            Our analysis offers a novel interpretation of SGD as a
            superposition sum of several concurrent processes, each a basic way
            that data may influence weights.  Diagrams such as
            $\sdia{c(01-2-3)(02-12-23)}$, analogous to those of \cite{fe49} and
            \cite{pe71}, depict these interactions. 
            %
            %--  soft benefits: prospective  ----------------------------------
            %
            \S\ref{appendix:future} discusses this bridge to physics --- and
            its relation to Hessian methods and natural GD --- as topics for
            future research.  
    
        \newpage
        \subsection{Background, notation, assumptions}

            %--  the landscape  -----------------------------------------------
    
            We fix a \emph{testing loss} function $l:\Mm\to\RR$ on a space
            $\Mm$ of weights $\theta$.  We fix a distribution $\Dd$ from which
            unbiased estimates $l_x$ of $l$ are drawn.  We write $(l_n: 0\leq
            n<N)$ for a training sequence drawn i.i.d.\ from $\Dd$.  We refer
            to $n$ and to $l_n$ as \emph{training points}.  We assume
            \S\ref{appendix:assumptions}'s hypotheses, e.g.\ that $l, l_x$ are
            analytic and that all moments exist.
            %
            %--  specialization to a common case  -----------------------------
            %
            For instance, our theory models $\tanh$ networks with cross entropy
            loss on bounded data --- and with weight sharing, skip connections,
            soft attention, dropout, and weight decay.  But it does not model
            $\text{ReLU}$ networks.

            %--  tensor conventions  ------------------------------------------

            \begin{wrapfigure}{r}{0.4\textwidth}
                \vspace{-0.3cm}
                \begin{tabular}{lclcl}
                    $G$ &$=$& $\ex{\nb\lx}        $ &$=$& $\mdia{MOO(0)(0)}       $\\
                    $H$ &$=$& $\ex{\nb\nb\lx}     $ &$=$& $\mdia{MOO(0)(0-0)}     $\\ 
                    $J$ &$=$& $\ex{\nb\nb\nb\lx}  $ &$=$& $\mdia{MOO(0)(0-0-0)}   $\\
                    $C$ &$=$& $\ex{(\nb\lx - G)^2}$ &$=$& $\mdia{MOOc(01)(0-1)}   $\\
                    $S$ &$=$& $\ex{(\nb\lx - G)^3}$ &$=$& $\mdia{MOOc(012)(0-1-2)}$
                \end{tabular}
                \vspace{-0.5cm}
                \caption*{\textbf{Above}: Named tensors.
                \S\ref{ssect:what-are-diagrams} explains how tensors
                correspond to diagrams.}
                \vspace{-0.2cm}
            \end{wrapfigure}
            Our analysis makes heavy use of the tensors defined to the right:
            $G, H, J; C, S$ have $1, 2, 3; 2, 3$ indices, respectively.  We
            shall implicitly sum repeated Greek indices: if a covector $U$ and
            a vector $V$\footnote{
                Vectors/covectors, a.k.a.\ column/row vectors,
                represent distinct geometric concepts \citep{ko93}. 
            } have coefficients $U_\mu, V^\mu$, then 
            $
                U_\mu V^\mu
                \triangleq
                \sum_\mu U_\mu \cdot V^\mu
            $.
            We regard the learning rate as an inverse metric $\eta^{\mu\nu}$
            that converts gradient covectors to displacement vectors
            \citep{bo13}.  We use the learning rate $\eta$ to raise indices;
            thus,
            $
                H^{\mu}_{\lambda}
                \triangleq
                \sum_{\nu} 
                \eta^{\mu\nu} H_{\nu\lambda}
            $ and
            $
                C^{\mu}_{\mu}
                \triangleq
                \sum_{\mu \nu} \eta^{\mu\nu} \cdot C_{\nu\mu}
            $.
            A quantity $q$ \emph{vanishes to order $\eta^d$} when
            $\lim_{\eta\to 0} q/p(\eta) = 0$ for some homogeneous degree-$d$
            polynomial $p$; we then say $q\in o(\eta^d)$.

            %--  names of sgd parameters  -------------------------------------

            SGD performs $\eta$-steepest descent on the estimates $l_n$.  Our
            general theory describes SGD with any number
                 $N$ of training points,
                 $T$ of updates, and 
                 $B$ of points per batch.
            SGD then runs $T$ many updates (i.e. $E=TB/N$ epochs, i.e. $M=T/N$
            updates per point) of the form
            $$
                \textstyle
                \theta^\mu
                \coloneqq
                \theta^\mu -
                \eta^{\mu\nu} \nabla_\nu
                    \sum_{n\in \Bb_t} l_n(\theta) / B
            $$
            where in each epoch, we sample the $t$th batch $\Bb_t$ without
            replacement from the training sequence.  Thus, to each
            initialization $\theta_0 \in \Mm$ there corresponds a joint
            distribution over learning trajectories $(\theta_t: 0\leq t \leq
            T)$.  The expectation of $l(\theta_T)$ especially interests us.
            %For simplicity, our paper body
            %(but not the appendices) will assume unless otherwise stated that
            %\textbf{SGD has $\mathbf{E=B=1}$ and GD has $\mathbf{T=B=N}$}.
 
            %--  example  -----------------------------------------------------

            To illustrate our notation, we quote a well-known proposition
            (\cite{ne04}):
            \begin{prop*}
                $G$ controls the leading order loss decrease:
                $
                    \expc l(\theta_T) - l(\theta_0) =
                    T G_{\mu} \eta^{\mu\nu} G_\nu|_{\theta=\theta_0}
                    + o(\eta^1)
                $.
            \end{prop*}
            One proves this estimate by induction on $T$.  When the loss
            landscape is noiseless and linear (that is, $\nabla l_x(\theta)$
            depends on neither $x$ nor $theta$), this estimate is exact.

        \newpage
        \subsection{Related work}
    
            %-- history of sgd  -----------------------------------------------

            It was \cite{ki52} who, in uniting gradient descent \citep{ca47}
            with stochastic approximation \citep{ro51}, invented SGD.  Since
            the development of back-propagation for efficient differentiation
            \citep{we74}, SGD has been used to train connectionist models,
            e.g.\ neural networks \citep{bo91}, recently to remarkable success
            \citep{le15}.
        
            %-- analyzing overfitting; relevance of optimization; sde errs  ---
        
            Several lines of work treat the overfitting of SGD-trained networks
            \citep{ne17a}.  For example, \cite{ba17} controls the Rademacher
            complexity of deep hypothesis classes, leading to
            optimizer-agnostic generalization bounds.  Yet SGD-trained networks
            generalize despite their ability to shatter large sets
            \citep{zh17}, so generalization must arise from not only
            architecture but also optimization \citep{ne17b}.  Others
            approximate SGD by SDE to analyze implicit regularization (e.g.\
            \cite{ch18}), but, per \cite{ya19a}, such continuous-time analyses
            cannot treat SGD noise correctly.
            %
            %-- we extend dan's approach  -------------------------------------
            %
            We avoid these pitfalls by Taylor expanding around $\eta=0$ as in
            \cite{ro18}.  Unlike that work, we generalize beyond order $\eta^1$
            and $T=2$,.  To support this generalization, we develop new 
            summation techniques with improved large-$T$ convergence.  Our
            interpretion of the resulting terms offers a new qualitative
            picture of SGD as a superposition of several simpler
            information-flow processes.
            
            %-- phenomenology of rademacher correlates such as hessians  ------
        
            Our predictions are vacuous for large $\eta$.  Other analyses treat
            large-$\eta$ learning phenomenologically, whether by finding
            empirical correlates of gen.\ gap \citep{li18}, by showing that
            \emph{flat} minima generalize (\cite{ho17}, \cite{ke17},
            \cite{wa18}), or by showing that \emph{sharp} minima generalize
            (\cite{st56}, \cite{di17}, \cite{wu18}).  Our theory reveals that 
            SGD's implicit regularization mediates between these seemingly
            clashing intuitions.
            
            %-- our work vs other perturbative approaches  --------------------
        
            Prior work analyzes SGD perturbatively: \cite{dy19} perturb in
            inverse network width, using 't Hooft diagrams to correct the
            Gaussian Process approximation for specific deep nets.  Perturbing
            to order $\eta^2$, \cite{ch18} and \cite{li17} are forced to assume
            uncorrelated Gaussian noise.  By contrast, we use Penrose diagrams
            to compute test losses to arbitrary order in $\eta$.  We allow
            correlated, non-Gaussian noise and thus \emph{any} smooth
            architecture.  For instance, we do not assume information-geometric
            relationships between $C$ and $H$,\footnote{
                Disagreement of $C$ and
                $H$ is typical in modern learning \citep{ro12, ku19}
            } so we may model VAEs. 

    \section{Perturbative theory of SGD}
        \subsection{Trivial example}
        \subsection{Perturbation as technique}
        \subsection{The necessity and role of diagrams}\label{ssect:what-are-diagrams}
        \subsection{Insights from diagrams}
        \subsection{Resummation}

            \begin{wraptable}{r}{5cm}
                \begin{tabular}{p{5cm}}
                    \textbf{Examples}:
                    The diagrams
                    $\sdia{c(0-1)(01)}$, $\sdia{c(012-3)(03-13-23)}$ each have $2$
                    parts; $\sdia{c(0-12-3)(03-13-23)}$, $\sdia{c(01-2-3)(02-12-23)}$
                    have $3$.
                    %
                    Corollaries \ref{cor:overfit}, \ref{cor:epochs},
                    \ref{cor:batch} have $E\neq 1 \neq B$, so they feature
                    $\sdia{c(01)(01)}$ and $\sdia{c(01-2)(01-12)}$, generalized
                    diagrams that violate the path condition. 
                    %
                    Diagrams $\sdia{c(0-1)(01)}$, $\sdia{c(0-1-2)(02-12)}$ 
                    are irreducible; due to their green nodes,
                    $\sdia{c(0-1-2)(01-12)}$, $\sdia{c(01-2-3)(03-12-23)}$ are not.
                    %
                    For all $f$,
                    $|\Aut_f(\sdia{c(01-2-3)(03-12-23)})|=1$ and
                    $|\Aut_f(\sdia{c(01-2-3)(02-12-23)})|=2$.
                \end{tabular}
            \end{wraptable}
    
            A \emph{diagram} is a finite rooted tree equipped with a partition
            of its nodes that obeys the \emph{path condition}: no path from leaf to
            root may encounter any part more than once.
            We specify the root by drawing it rightmost.  We draw the parts of 
            the partition by grouping each part's nodes inside fuzzy outlines. 
            %
            A diagram is \emph{irreducible} when each of its degree-$2$ nodes is in
            a part of size one.
            %
            An \emph{embedding} $f$ of a diagram $D$ is an injection from
            $D$'s parts to (integer) times $0 \leq t \leq T$ that sends the
            root to $T$ and s.t., for each path from leaf to root, the
            corresponding sequence of times increases.  So $f$ might
            send $\sdia{c(01-2-3)(03-12-23)}$'s red part to $t=3$ and its green
            part to $t=4$, but --- because the green node has a red child ---
            not vice versa.
            %
            Let $\wabs{\Aut_f(D)}$ count automorphisms of $D$ that preserve $f$.
            %%%%%%%%%
            Up to unbiasing terms,\footnote{
                For example, we actually define $\sdia{MOOc(01)(0-1)}$ to be the
                cumulant $C = \expc\,[(\nb\lx - G)^2]$, not the moment
                $\expc\,[(\nb\lx)^2]$.  This centering is routine (see \S
                \ref{appendix:mobius}), tedious to notate, and un-germane, so we
                ignore it in the paper body.
            }
            we construct the \emph{re-summed value} $\rvalue_f(D)$ as follows:
            %
            \par\textbf{Node rule}: insert a factor a $\nabla^d l_x$for each degree $d$
            node. 
            %
            \par\textbf{Outline rule}: group each part's nodes within brackets $\expc_x []$.
            %
            \par\textbf{Edge rule}: if $f$ sends an edge's endpoints to times $t,
            t^\prime$, insert a factor of $K^{\wabs{t^\prime-t}-1} \eta$, where $K
            \triangleq (I-\eta H)$.
            %
            \par So if $f$ maps $\sdia{c(012-3)(03-13-23)}$'s red part to time $t =
            T-\Delta t$, then (the red part gives $S$; the green part, $J$):
            $$
                \rvalue_f\wrap{\sdia{c(012-3)(03-13-23)}} = 
                S_{\mu\lambda\rho}
                    (K^{\Delta t-1}\eta)^{\mu\nu}
                    (K^{\Delta t-1}\eta)^{\lambda\sigma}
                    (K^{\Delta t-1}\eta)^{\rho\pi}
                J_{\nu\sigma\pi}
            $$
            In fact, we may integrate this expression per Remark
            \ref{rmk:integrate} to recover Example \ref{exm:first}.
    
        \subsection{Main result}
    
        %\subsection{Recipe for SGD's expected test loss}
            %~~~~~~~~~~~~~~~~~~~~~~~~~~~~~~~~~~~~~~~~~~~~~~~~~~~~~~~~~~~~~~~~~~~~~~
            %~~~~~  Recipe for Test Loss  ~~~~~~~~~~~~~~~~~~~~~~~~~~~~~~~~~~~~~~~~~
            Theorem \ref{thm:resum} expresses SGD's test loss as a sum over
            diagrams.  A diagram with $d$ edges scales as $O(\eta^d)$, so the
            following is a series in $\eta$.  We later truncate the series to small
            $d$, thus focusing on few-edged diagrams and simplifying the
            combinatorics of embeddings.
            \begin{thm}[Special case of $E=B=1$] \label{thm:resum}
                For any $T$: for $\eta$ small enough, SGD has expected test loss
                \begin{equation*} \label{eq:resum}
                    \sum_{\substack{D~\text{an irreduc-} \\ \text{-ible diagram}}}
                    ~
                    \sum_{\substack{f~\text{an embed-} \\ \text{-ding of}~D}}
                    ~
                    \frac{(-1)^{|\edges(D)|}}{\wabs{\Aut_f(D)}}
                    \,
                    {\rvalue_f}(D)
                \end{equation*}
            \end{thm}
    
            %~~~~~~~~~~~~~~~~~~~~~~~~~~~~~~~~~~~~~~~~~~~~~~~~~~~~~~~~~~~~~~~~~~~~~~
            %~~~~~  Simplifications  ~~~~~~~~~~~~~~~~~~~~~~~~~~~~~~~~~~~~~~~~~~~~~~
     
            \begin{rmk} \label{rmk:integrate}
                In practice, we approximate sums over embeddings by integrals over
                times and $(I-\eta H)^t$ by $\exp(- \eta H t)$, reducing to a
                routine integration of exponentials at the cost of an error factor
                $1 + o(\eta)$.
            \end{rmk}
    
            %~~~~~~~~~~~~~~~~~~~~~~~~~~~~~~~~~~~~~~~~~~~~~~~~~~~~~~~~~~~~~~~~~~~~~~
            %~~~~~  Convergence  ~~~~~~~~~~~~~~~~~~~~~~~~~~~~~~~~~~~~~~~~~~~~~~~~~~
     
            \begin{thm} \label{thm:converge}
                If $\theta_\star$is a non-degenerate local minimum of $l$ (i.e.\
                $G(\theta_\star)=0$ and $H(\theta_\star) > 0$), then for SGD
                initialized sufficiently close to $\theta_\star$, the $d$th-order
                truncation of Theorem \ref{thm:resum} converges as $T\to \infty$.
            \end{thm}
    
            Caution: the $T\to \infty$ limit in Theorem \ref{thm:converge} might
            not measure any well-defined limit of SGD, since the limit might not
            commute with the infinite sum.  We have not seen such pathologies in
            practice, so we will freely speak of ``SGD in the large-$T$ limit'' as
            informal shorthand when referencing this Theorem.
          
        %\subsection{Insights from the formalism}
    
        \subsection{SGD descends on a $C$-smoothed landscape and prefers
        minima flat w.r.t.\ $C$.}
    
            \begin{cor}[Computed from $\sdia{c(01-2-3)(02-12-23)}$]
                \label{cor:entropic}
                Run SGD for $T \gg 1/\eta H$ from a non-degenerate test
                minimum.  Written in an eigenbasis of $\eta H$, $\theta$ has an
                expected displacement of
                $$
                    - \frac{\eta^3}{2}
                    \sum_{\mu\nu}
                        C_{\mu\nu}
                        \frac{1}{\eta (H_{\mu\mu} + H_{\nu\nu})}
                        J_{\mu\nu\lambda}
                        \frac{1}{H_{\lambda\lambda}}
                    + o(\eta^2)
                $$
            \end{cor}

            Intuitively, $D = \sdia{c(01-2-3)(02-12-23)}$ connects the
            subdiagram $\sdia{c(01-2)(02-12)} \propto CH$, via an extra edge on
            the green node (an extra $\nabla$ on $H$), to $D$'s degree-$1$
            root, $G$.  By l'H\^opital,\footnote{
                %
                Roughly:
                if a displacement $\Delta\theta$ grows loss by $G C\nabla H$
                nats, and by $G$ nats per foot, then $\Delta \theta$ is
                $C\nabla H$ feet.
                %
            } the displacement is $\propto -C\nabla H$.  That is, SGD moves
            toward minima that are flat \emph{with respect to} $C$ (Figure
            \ref{fig:cubicandspring}\offive{0}).
            %
            Taking limits to drop the non-degeneracy hypothesis, we expect
            \emph{sustained} motion toward flat regions in a valley of minima.
            By avoiding \cite{we19b}'s assumptions of constant $C$, we find
            that SGD's velocity field is typically non-conservative, i.e.\ has
            curl (\S\ref{subsect:entropic}).  Indeed, $\nabla(CH)$ is a total
            derivative but $C\nabla H$ is not.  Since, by low-pass
            filter theory, $CH/2+o(C)$ is the loss increase upon convolving $l$
            with a $C$-shaped Gaussian, we say that SGD descends on a
            $C$-smoothed landscape that changes as $C$ does.
            %
            Our $T\gg 1$ result is $\Theta(\eta^2)$, while \cite{ya19b}'s
            similar $T=2$ result is $\Theta(\eta^3)$.  Indeed, our analysis
            integrates the noise over many updates, hence amplifying $C$'s 
            effect.
            Experiments verify our law.
      
        \subsection{Both flat and sharp minima overfit less}
            \label{subsect:curvature-and-overfitting}%

            Intuitively, sharp minima are robust to slight changes in the
            average \emph{gradient} and flat minima are robust to slight
            \emph{displacements} in weight space (Figure
            \ref{fig:cubicandspring}\protect\offive{12}).  However, as SGD by
            definition equates displacements with gradients, it may be unclear
            how to reason about overfitting in the presence of curvature.
            %
            Our theory, by (automatically) accounting for the implicit
            regularization of fixed-$T$ descent, shows that both effects play
            a role.  In fact, by routine calculus on the left hand side of
            Corollary \ref{cor:overfit}, overfitting is maximized for medium
            minima with curvature $H \sim (\eta T)^{-1}$.
            %
            \begin{cor}[from $\sdia{c(01-2)(02-12)}$, $\sdia{c(01)(01)}$]\label{cor:overfit}
                Initialize GD at a non-degenerate test minimum $\theta_\star$.
                The overfitting (test loss minus $l(\theta_\star)$) and gen.\
                gap (test minus train loss) due to training are:
                $$
                    \wrap{\frac{C/N}{2H}}_{\mu\nu}^{\rho\lambda} ~
                        \wrap{(I - \exp(-\eta T H))^{\otimes 2}}^{\mu\nu}_{\rho\lambda}
                        + o(\eta^2)
                    ~~~~~ ; ~~~~~
                    \wrap{\frac{C/N}{H}}_{\mu\nu}^{\mu\lambda} ~
                        \wrap{I - \exp(-\eta T H)}^{\nu}_{\lambda}
                        + o(\eta)
                $$
            \end{cor}
            The gen.\ gap tends  
            to $C_{\mu\nu}(H^{-1})^{\mu\nu}/N$ as $T\to\infty$.  For maximum
            likelihood (ML) estimation in well-specified models near the ``true''
            minimum, $C=H$ is the Fisher metric, so we recover AIC:
            $(\textnormal{model dimension})/N$.  Unlike AIC, our more general
            expression is descendably smooth, may be used with MAP or ELBO tasks
            instead of just ML, and does not assume a well-specified model.
    
            \begin{figure}[h!]
                \centering
                \plotmooh{diagrams/entropic-force-diagram}{}{0.32\columnwidth} 
                \plotmooh{diagrams/sharp}{}{0.31\columnwidth}
                \caption{%
                    \textbf{Geometric intuition for curvature-noise interactions.}
                    \textbf{Left}:
                        Gradient noise pushes SGD toward flat minima
                        (Corollary
                        \ref{cor:entropic}).  The red densities show the 
                        typical $\theta$s, perturbed from the
                        minimum due to noise $C$, in two cross sections of the
                        loss valley.  $J = \nabla H$ measures
                        how curvature changes across the valley.  Our theory
                        does not assume separation between ``fast'' and
                        ``slow'' modes, but we label them in the picture to
                        ease comparison with \cite{we19b}.
                        Compare with Figure \ref{fig:archimedes}.
                    \textbf{\bf Right}:
                        Both curvature and the structure of noise affect
                        overfitting.  In each of the four subplots, the  
                        $\leftrightarrow$ axis represents weight space and the
                        $\updownarrow$ axis represents loss.
                        \protect\offive{1}:
                        \emph{covector}-perturbed landscapes favor large $H$s.
                        \protect\offive{2}:
                        \emph{vector}-perturbed landscapes favor small $H$s.
                        SGD's implicit regularization interpolates between
                        these rows (Corollary \ref{cor:overfit}).
                }
                \label{fig:cubicandspring}
            \end{figure}
    

    \section{Consequences of the theory}

        \subsection{High-$C$ regions repel small-$(E,B)$ SGD more than large-$(E,B)$ SGD}
            \label{subsect:epochs-batch}

            \begin{wrapfigure}[14]{r}{0.25\textwidth} 
                \centering
                \pmoo{3.5cm}{chladni}
                \caption{
                    \textbf{Chladni plate}. 
                    Grains of sand on a vibrating plate tend toward
                    stationary regions.
                }
                \label{fig:chladni}
            \end{wrapfigure}
            Physical intuition (\S\ref{appendix:interpret-diagrams}) suggests
            that noise repels SGD.  
            In particular, if two neighboring regions
            of weight space have high and low levels of gradient noise,
            respectively, then we expect the rate at which $\theta$ jumps from
            the former to the latter to exceed the opposite rate.  There is 
            thus a net movement toward regions of small $C$! 
            This mechanism parallels the Chladni effect \citep{ch87}
             (Figure \ref{fig:chladni}).\footnote{
                From Pierre Dragicevic and Yvonne Jansen's
                \href{http://www.dataphys.org/list/gallery/}{data physicalization project}, Creative
                Commons BY-SA 3.0. 
            }
            %
            Our theory makes this intuition precise; the drift is in the
            direction of $-\nabla C$, and the effect is strongest when gradient
            noise is not averaged out by large batch sizes.
            \begin{cor}[$\sdia{c(01-2)(01-12)}$] \label{cor:batch}
                SGD avoids high-$C$ regions more than GD:
                $
                    l_{C}
                        \triangleq
                    \frac{N-1}{4 N}
                    \nabla^\mu C^{\nu}_{\nu}
                        =
                    \expct{\theta_{GD} - \theta_{SGD}}^\mu - o(\eta^2)
                $.
                If $\hat{l_c}$ is a smooth unbiased estimator of $l_c$, then GD
                on $l + \hat{l_c}$ has an expected test loss that agrees with
                SGD's to order $\eta^2$.  We call this method GDC.
            \end{cor}

            An analogous form of averaging occurs over multiple epochs.  For a
            tight comparison, we scale the learning rates appropriately so
            that, to leading order, few-epoch and many-epoch SGD agree.  Then
            few and many- epoch SGD differ, to leading order, in their
            sensitivity to $\nabla C$:
            \begin{cor}[$\sdia{c(01-2)(01-12)}$] \label{cor:epochs}
                SGD with $M=1$ and $\eta=\eta_0$ avoids high-$C$ regions more
                than SGD with $M=M_0$ and $\eta=\eta_0/M_0$.  Precisely:
                $
                    \expct{\theta_{M=M_0} - \theta_{M=1}}^\mu
                        =
                    \wrap{\frac{M_0-1}{4 M_0}} N
                    \wrap{\nabla^\mu C^{\nu}_{\nu}}
                    + o(\eta^2)
                $.
            \end{cor}
    
        \subsection{Non-Gaussian noise affects SGD but not SDE}
    
            Stochastic differential equations (SDE: see \cite{li18}) are a
            popular theoretical approximation of SGD, but SDE and SGD differ in
            several ways.  For instance, the inter-epoch noise correlations in
            multi-epoch SGD measurably affect SGD's final test loss (Corollary
            \ref{cor:epochs}), but SDE assumes uncorrelated gradient updates.
            Even if we restrict to single-epoch SDE, differences arise due to
            time discretization and non-Gaussian noise.  Intuitively, SGD and
            SDE respond differently to changes in curvature:
            %
            \begin{cor}[$\sdia{c(01-2)(02-12)}$, $\sdia{c(012-3)(03-13-23)}$] \label{cor:vsode}
                SGD's test loss is
                $
                    \frac{T}{2} C_{\mu\nu} H^{\mu\nu} + o(\eta^2)
                $
                more than ODE's and SDE's.
                The deviation from SDE due to skewed noise is
                $
                    - \frac{T}{6} S_{\mu\nu\lambda} J^{\mu\nu\lambda} 
                    + o(\eta^3)
                $.\footnote{
                    This approximation of Example \ref{exm:first}'s more exact
                    expression agrees with the latter to leading order in
                    $\eta$.
                }
            \end{cor}



    \section{Experiments}
        Despite the convergence results in Theorems \ref{thm:resum} and
        \ref{thm:converge}, we have no theoretical bounds for the domain and
        \emph{rate} of convergence.  Instead, we test our predictions by
        experiment.  We perceive support for our theory in drastic rejections of
        the null hypothesis.  For instance, in Figure \ref{fig:vanilla}\ofsix{4},
        \citep{ch18} predicts a velocity of $0$ while we predict a velocity of
        $\eta^2/6$
        %
        Here, \texttt{I} bars, \texttt{+} signs, and shaded regions all mark $95\%$
        confidence intervals based on the standard error of the mean.
        \S\ref{appendix:experiments} describes neural architectures, the definitions
        of artificial landscapes, sample sizes, and further plots.
    
        \subsection{Training time, epochs, and batch size; $C$ repels SGD more than GD}
            %----------------------------------------------------------------------
            %       Vanilla SGD                                 
            %----------------------------------------------------------------------
            We test Theorem \ref{thm:resum}'s order $\eta^3$ truncation on smooth
            convnets for CIFAR-10 and Fashion-MNIST.  Theory agrees with experiment
            through timescales long enough for accuracy to increase by $0.5\%$
            (Figure \ref{fig:vanilla}\ofsix{0},\ofsix{1}).
            %----------------------------------------------------------------------
            %       Epochs and Overfitting                      
            %----------------------------------------------------------------------
            \S\ref{appendix:figures} supports Corollary \ref{cor:epochs}'s
            predictions about epoch number.
            %----------------------------------------------------------------------
            %       Emulating Small Batches with Large Ones     
            %----------------------------------------------------------------------
            Figure \ref{fig:vanilla}\ofsix{2} tests Corollary \ref{cor:batch}'s
            claim that, relative to GD, high-$C$ regions \emph{repel} SGD.  This is
            significant because $C$ controls the rate at which the gen.\ gap (test
            minus train loss) grows (Corollary \ref{cor:overfit}, Figure
            \ref{fig:vanilla}\ofsix{3}).
    
            \begin{figure}[h!] 
                \centering
                \pmoo{3.55cm}{neurips-test-small} \pmoo{3.55cm}{new-big-bm-new}          \pmoo{3.55cm}{neurips-thermo-linear-screw}
                \pmoo{3.55cm}{neurips-test-large} \pmoo{3.55cm}{neurips-gen-cifar-lenet} \pmoo{3.55cm}{neurips-tak}
                \caption{
                    {\bf Experiments on natural and artificial losses.}
                    The label \texttt{rvalue} refers to Theorem \ref{thm:resum}'s
                    predictions, approximated as in Remark \ref{rmk:integrate}.
                    Curves marked \texttt{uvalue} are polynomial approximations to
                    Theorem \ref{thm:resum}'s result (see
                    \S\ref{appendix:sum-embeddings}).  \texttt{uvalue}s are simpler
                    to work with but (see \protect\ofsix{4}) may be less
                    accurate.
                    %%%
                    %%%
                    \newline
                    {\bf Left: Perturbation models SGD for small $\eta T$.}
                    Fashion-MNIST convnet's test loss vs learning rate.  In this
                    small $T$ setting, we choose to use our theory's simpler 
                    un-resummed values (\ref{appendix:evaluate-embeddings})
                    instead of the more precise $\rvalue$s.
                    %
                    \protect\ofsix{0}: For all init.s tested ($1$ shown,
                    $11$ unshown), the order $3$ prediction agrees with experiment
                    through $\eta T \approx 10^0$, corresponding to a decrease
                    in $0\mbox{-}1$ error of $\approx 10^{-3}$.
                    %
                    \protect\ofsix{1}: For large $\eta T$, our predictions
                    break down.  Here, the order-$3$ prediction holds until the
                    $0\mbox{-}1$ error improves by $5\cdot 10^{-3}$.
                    Beyond this, $2$nd order agreement with experiment is
                    coincidental.  
                    %%%
                    %%%
                    \newline
                    {\bf Center: $C$ controls gen.\ gap and distinguishes GD
                    from SGD.}
                    %
                    With equal-scaled axes, \protect\ofsix{2} shows that
                    GDC matches SGD (small vertical varianec) better than GD
                    matches SGD (large horizontal variance) in test loss for a
                    range of $\eta$ ($\approx 10^{-3}-10^{-1}$) and
                    init.s\ (zero and several Xavier-Glorot trials) for
                    logistic regression and convnets.  Here, $T=10$. 
                    %
                    \protect\ofsix{3}: CIFAR-10 generalization gaps.  For all
                    init.s tested ($1$ shown, $11$ unshown), the
                    degree-$2$ prediction agrees with experiment through $\eta T
                    \approx 5\cdot 10^{-1}$.
                    %%%
                    %%%
                    \newline
                    {\bf Right: Predictions near minima excel for large $\eta T$.}%
                    \protect\ofsix{4}: SGD travels \Archimedes' valley of global
                    minima in the positive $z$ direction.  Note: $H$ and $C$ are
                    bounded across the valley, we see drift for all small $\eta$,
                    and we see displacement exceeding the landscape's period of
                    $2\pi$.  So: the drift is not a pathology of well-chosen
                    $\eta$, of divergent noise, or of ephemeral initial conditions.
                    %
                    \protect\ofsix{5}: For \MeanEstimation\, with fixed $C$ and a
                    range of $H$s, initialized at the truth, the test losses after
                    fixed-$T$ GD are smallest for very sharp and very flat $H$.
                    Near $H=0$, our predictions improve on TIC \citep{di18} and
                    thus on AIC.
                    %%%
                    %%%
                }
                \label{fig:vanilla}
            %    \label{fig:batchandgen}
            %    \label{fig:thermoandtak}
            \end{figure}
    
    
        %--------------------------------------------------------------------------
        %           Thermodynamic Engine                        
        %--------------------------------------------------------------------------
    
        \newpage
        \subsection{Minima that are flat \emph{with respect to} $C$ attract SGD}
            %
            \begin{wrapfigure}[18]{r}{0.48\textwidth} 
                \centering
                \vspace{-15pt}
                \pmoo{3.0cm}{from-above}
                \pmoo{3.0cm}{from-side}
                \caption{
                    \textbf{\Archimedes.}
                    A \textbf{green} level surface of $l$ twists around a valley of
                    minima ($z$ axis) at its center; $l$ is large outside this
                    surface.  Due to anisotropic noise, $\theta$ scatters away
                    from the $z$ axis toward the \textbf{purple} tubes.
                    %
                    SGD pushes the scattered $\theta$s toward lower loss, i.e.\
                    toward the level surface, and so toward larger $z$.
                    %
                    The $z$ axis points into the page (\textbf{left}) or upward
                    (\textbf{right}).
                }
                \label{fig:archimedes}
            \end{wrapfigure}
            %
            \label{subsect:entropic}
            To test the claimed dependence on $C$, \S\ref{appendix:artificial}
            constructs a landscape, \Archimedes, with non-constant $C$ throughout
            its valley of global minima.  Figure \ref{fig:archimedes} 
            depicts \Archimedes' chiral shape.\footnote{
                We made these plots with
                the help of Paul Seeburger's online applet,
                \href{https://www.monroecc.edu/faculty/paulseeburger/calcnsf/CalcPlot3D/}{CalcPlot3D}.
            }  As in
            Archimedes' screw or Rock-Paper-Scissors, each point $\theta$ has a
            neighbor that, from $C(\theta)$'s perspective but not absolutely, is
            flatter.  This permits eternal motion despite the landscape's symmetry.
            %
            Indeed, Corollary \ref{cor:entropic} predicts 
            a $z$-velocity of $+\eta^2/6$ per timestep, while \cite{ch18}'s
            SDE-based analysis predicts a constant velocity of $0$.\footnote{
                Indeed, \Archimedes' velocity is $\eta$-perpendicular to the image
                of $(\eta C)^\mu_\nu$ in tangent space.
            }
            Our prediction agrees with experiment (Figure
            \ref{fig:vanilla}\ofsix{4}).
            %
            Because SGD's motion depends smoothly on the landscape, the special
            case of \Archimedes\ implies that non-conservativity is typical.
            %
            One may have sought an ``effective loss'' $\tilde{l}$ such that, up to
            $\sqrt{T}$ diffusion terms, SGD on $l$ matches ODE on $\tilde{l}$.  The
            non-conservativity of SGD's velocity shows that no such $\tilde{l}$
            exists.
            %
    
    
        %--------------------------------------------------------------------------
        %           Sharp vs Flat Minima                        
        %--------------------------------------------------------------------------
    
        \subsection{Sharp and flat minima both overfit less than medium minima} \label{subsect:overfit}
    
            Prior work (\S\ref{sect:related}) finds both that \emph{sharp} minima
            overfit less (for, $l^2$ regularization sharpens minima) or that
            \emph{flat} minima overfit less (for, flat minima are robust to small
            displacements).  In fact, both phenomena occur, and noise structure
            determines which dominates (Corollary \ref{cor:overfit}).  This effect
            appears even in \MeanEstimation\, (\S\ref{appendix:artificial}): Figure
            \ref{fig:vanilla}\ofsix{5}.
            %
            To combat overfitting, we may add Corollary \ref{cor:overfit}'s
            expression for gen.\ gap to $l$.  By descending on this regularized
            loss, we may tune smooth hyperparameters such as $l_2$ regularization
            coefficients for small datasets ($H \ll C/N$)
            (\S\ref{appendix:figures}).  Since matrix exponentiation takes time
            cubic in dimension, this regularizer is most useful for small models.



    \section{Conclusion}
    
        %~~~~~~~~~~~~~~~~~~~~~~~~~~~~~~~~~~~~~~~~~~~~~~~~~~~~~~~~~~~~~~~~~~~~~~~~~~
        %~~~~~~~~~  Summarize Contributions  ~~~~~~~~~~~~~~~~~~~~~~~~~~~~~~~~~~~~~~
    
        We presented a diagram-based method for studying stochastic optimization on
        short timescales or near minima.
            Corollaries \ref{cor:entropic} and \ref{cor:overfit} together offer
            insight into SGD's success in training deep networks: SGD avoids
            curvature and noise, and curvature and noise control generalization.
    
        Analyzing $\sdia{c(01-2)(02-12)}$, we proved that \textbf{flat and sharp
        minima both overfit less} than medium minima.  Intuitively, flat minima are
        robust to vector noise, sharp minima are robust to covector noise, and
        medium minima robust to neither.  We thus proposed a regularizer enabling
        gradient-based hyperparameter tuning.
        %
        Inspecting $\sdia{c(01-2-3)(02-12-23)}$, we extended \cite{we19b} to
        nonconstant, nonisotropic covariance to reveal that \textbf{SGD descends on
        a landscape smoothed by the current covariance $C$}.
        As $C$ evolves, the
        smoothed landscape evolves, resulting in non-conservative dynamics.
        %
        Examining $\sdia{c(01-2)(01-12)}$, we showed that \textbf{GD may emulate
        SGD}, as conjectured by \cite{ro18}.  This is significant because, while
        small batch sizes can lead to better generalization \citep{bo91}, modern
        infrastructure increasingly rewards large batch sizes \citep{go18}.  
    
        %~~~~~~~~~~~~~~~~~~~~~~~~~~~~~~~~~~~~~~~~~~~~~~~~~~~~~~~~~~~~~~~~~~~~~~~~~~
        %~~~~~~~~~  Anticipate Criticism of Limitations  ~~~~~~~~~~~~~~~~~~~~~~~~~~
    
            Since our predictions depend only on loss data near initialization,
            they break down after the weight moves far from initialization.
            Our theory thus best applies to small-movement contexts, whether
            for long times (large $\eta T$) near an isolated minimum or for
            short times (small $\eta T$) in general.
            %Thus, the theory might
            %help to analyze fine-tuning (e.g.\ \cite{fi17}'s MAML).
    
            Much as meteorologists understand how warm and cold fronts interact
            despite long-term forecasting's intractability, we quantify how
            curvature and noise contribute to counter-intuitive dynamics
            governing each short-term interval of SGD's trajectory.  Equipped
            with our theory, practitioners may now refine intuitions --- e.g.\
            that SGD descends on the training loss --- to account for noise.
    
        %~~~~~~~~~~~~~~~~~~~~~~~~~~~~~~~~~~~~~~~~~~~~~~~~~~~~~~~~~~~~~~~~~~~~~~~~~~
        %~~~~~~~~~  Future Work  ~~~~~~~~~~~~~~~~~~~~~~~~~~~~~~~~~~~~~~~~~~~~~~~~~~
    
        \subsection{Future work}
    
    \acks{
        We are deeply grateful to Sho Yaida, Joshua B.\ Tenenbaum, Dan A.\
        Roberts, and Greg Wornell for generous mentorship.  Dan introduced us
        to the SGD literature and taught us about Taylor series.  Sho suggested
        the question of how to resum and taught us thermodynamics.  We
        appreciate writerly feedback from Ben R.\ Bray, Wenli Zhao, and Chloe
        Kleinfeldt.
        %
        We thank our anonymous reviewers' for incisive feedback toward our
        writing's clarity.
    }
    
    \bibliography{perturb}
    
    \appendix
    
        \section{My Proof of Theorem 1}
        
            This is a boring technical proof.
        
        \section{My Proof of Theorem 2}
        
            This is a complete version of a proof sketched in the main text.
    
\end{document}
